\documentclass[main.tex]{subfiles}

\begin{document}

\textcolor{red}{Лекция 26.02.2022.}

Систематизируем полученные ранее замыкающие соотношения в таблицу.

\begin{table}[h]
\centering
\caption{Замыкающие соотношения}
\label{table:con}
{\renewcommand{\arraystretch}{3}
\begin{tabular}{ | c | c | c | c | }
\hline
Для скелета& $\displaystyle{}\varphi=-r_s\frac{\partial\Phi_s}{\partial p}$& $\displaystyle{}\eta_s=-\frac{\partial\Phi_s}{\partial\theta}$& $\displaystyle{}T_{ij}=r_s\frac{\partial\Phi_s}{\partial\varepsilon_{ij}}$ \\
\hline
Для флюида& $\displaystyle{}\frac{1}{\rho_f}=\frac{\partial\Phi_f}{\partial p}$& $\displaystyle{}\eta_f=-\frac{\partial\Phi_f}{\partial\theta}$ &  \\
\hline
\end{tabular}}
\end{table}

\section{Линейное приближение}

\subsection{Для флюида}

Представим $\Phi_f$ в виде частичной суммы ряда Тейлора по всем переменным.
\begin{multline}\label{PhiFTaylor}
\Phi_f\approx\Phi_f^0+\frac{\partial\Phi_f}{\partial p}\bigg|_{\substack{p=p_0\\ \theta=\theta_0}}\left(p-p_0\right)+\frac{\partial\Phi_f}{\partial\theta}\bigg|_{\substack{p=p_0\\ \theta=\theta_0}}\left(\theta-\theta_0\right)+\\+\frac{1}{2}\left(\frac{\partial^2\Phi_f}{\partial p^2}\bigg|_{\substack{p=p_0\\ \theta=\theta_0}}\left(p-p_0\right)^2+2\frac{\partial^2\Phi_f}{\partial\theta\partial p}\bigg|_{\substack{p=p_0\\ \theta=\theta_0}}\left(p-p_0\right)\left(\theta-\theta_0\right)+\frac{\partial^2\Phi_f}{\partial\theta^2}\bigg|_{\substack{p=p_0\\ \theta=\theta_0}}\left(\theta-\theta_0\right)^2\right)
\end{multline}

Условие существования такого разложения: $\Phi_f\in C_2$ (вторые производные $\Phi_f$ не только существуют, но и непрерывны)

Заметим, что во всех замыкающих соотношениях важны только производные, поэтому значение первого слагаемого в разложении может быть любым (другими словами, потенциальную и внутреннюю энергии можно отмерять от любого уровня)

Перепишем замыкающее соотношение из таблицы \ref{table:con}, подставив разложение \ref{PhiFTaylor}:
\beq\label{LinCon}
\eta_f=-\frac{\partial\Phi_f}{\partial\theta}\approx -\frac{\partial\Phi_f}{\partial\theta}\bigg|_{\substack{p=p_0\\ \theta=\theta_0}}-\frac{\partial^2\Phi_f}{\partial\theta\partial p}\bigg|_{\substack{p=p_0\\ \theta=\theta_0}}\left(p-p_0\right)-\frac{\partial^2\Phi_f}{\partial\theta^2}\bigg|_{\substack{p=p_0\\ \theta=\theta_0}}\left(\theta-\theta_0\right)
\eeq
(видим, что получили линейное приближение)

Из термодинамики помним, что $dQ=TdS$ и теплоёмкость
\beq
C=\frac{dQ}{dt}=T\frac{dS}{dT}
\eeq

Таким образом, из \eqref{LinCon} удельная теплоёмкость флюида
\beq\label{CF}
c_f^\theta=\theta\frac{\partial\eta_f}{\partial\theta}\approx-\theta\frac{\partial^2\Phi_f}{\partial\theta^2}\bigg|_{\substack{p=p_0\\ \theta=\theta_0}}
\eeq
(получили первое замыкающее соотношение на удельную теплоёмкость флюида; видим, что в первом приближении теплоёмкость является постоянной величиной; ранее использовали удельную теплоёмкость при выводе уравнения теплопроводности \eqref{HT})

Перепишем ещё одно замыкающее соотношение для флюида из таблицы \ref{table:con}:
\beq
\frac{1}{\rho_f}=\frac{\partial\Phi_f}{\partial p}\approx\frac{\partial\Phi_f}{\partial p}\bigg|_{\substack{p=p_0\\ \theta=\theta_0}}+\frac{\partial^2\Phi_f}{\partial p^2}\bigg|_{\substack{p=p_0\\ \theta=\theta_0}}\left(p-p_0\right)+\frac{\partial^2\Phi_f}{\partial p\partial\theta}\bigg|_{\substack{p=p_0\\ \theta=\theta_0}}\left(\theta-\theta_0\right)
\eeq
Обозначив $\displaystyle{}\frac{1}{\rho_{f0}}=\frac{\partial\Phi_f}{\partial p}\bigg|_{\substack{p=p_0\\ \theta=\theta_0}}$, перепишем в следующем виде
\beq\label{RhoF}
\frac{1}{\rho_f}\approx\frac{1}{\rho_{f0}}\left(1+\rho_{f0}\frac{\partial^2\Phi_f}{\partial p^2}\bigg|_{\substack{p=p_0\\ \theta=\theta_0}}\left(p-p_0\right)+\rho_{f0}\frac{\partial^2\Phi_f}{\partial p\partial\theta}\bigg|_{\substack{p=p_0\\ \theta=\theta_0}}\left(\theta-\theta_0\right)\right)
\eeq

Введём сжимаемость флюида
\beq\label{Compress}
c_f=-\rho_{f0}\frac{\partial^2\Phi_f}{\partial p^2}\bigg|_{\substack{p=p_0\\ \theta=\theta_0}}
\eeq
% из второго начала термодинамики можно доказать, что производная в последнем равенстве отрицательна
и коэффициент объёмного термического расширения флюида
\beq\label{Expansion}
\alpha_f^\theta=\rho_{f0}\frac{\partial^2\Phi_f}{\partial p\partial\theta}\bigg|_{\substack{p=p_0\\ \theta=\theta_0}}
\eeq
Тогда оборачивая обе части \eqref{RhoF} и вспоминая, что $\displaystyle{}\frac{1}{1+x}\approx 1-x$, получаем линейное замыкающее соотношение на плотность флюида:
\beq\label{RhoF2}
\rho_f\approx\rho_{f0}\left(1+c_f\left(p-p_0\right)-\alpha_f^\theta\left(\theta-\theta_0\right)\right)
\eeq
(заметим, что $c_f$ в \eqref{RhoF2} и $c_f^\theta$ в \eqref{CF} обозначают разные физические величины -- просто букв не хватает).

Знаки для сжимаемости флюида в \eqref{Compress} и коэффициента объёмного термического расширения флюида в \eqref{Expansion} выбрали исходя из физических соображений: плотность флюида увеличивается при увеличении давления и уменьшается при увеличении температуры.

\subsection{Для скелета}

Предполагаем, что в начальном (базовом) состоянии скелет не деформирован ($\varepsilon_{ij}=0$). Но напряжения в начальном состоянии могут быть (напряжения, присутствующие в материале при условии отсутствия деформаций -- могут быть при наличии включений или, например, при заморозке льда). Выпишем частичную сумму ряда Тейлора для $\Phi_s$:
\begin{multline}\label{PhiSTaylor}
\Phi_s\left(\theta,\varepsilon_{ij},p\right)\approx\Phi_{s}^0+\frac{\partial\Phi_s}{\partial\theta}\bigg|_{\substack{p_0,\theta_0\\\varepsilon_{ij}=0}}\left(\theta-\theta_0\right)+\frac{\partial\Phi_s}{\partial p}\bigg|_{\substack{p_0,\theta_0\\\varepsilon_{ij}=0}}\left(p-p_0\right)+\frac{\partial\Phi_s}{\partial\varepsilon_{ij}}\bigg|_{\substack{p_0,\theta_0\\\varepsilon_{ij}=0}}\varepsilon_{ij}+\\+\frac{1}{2}\frac{\partial^2\Phi_s}{\partial\theta^2}\bigg|_{\substack{p_0,\theta_0\\\varepsilon_{ij}=0}}\left(\theta-\theta_0\right)^2+\frac{1}{2}\frac{\partial^2\Phi_s}{\partial p^2}\bigg|_{\substack{p_0,\theta_0\\\varepsilon_{ij}=0}}\left(p-p_0\right)^2+\frac{1}{2}\frac{\partial^2\Phi_s}{\partial\varepsilon_{ij}\partial\varepsilon_{kl}}\bigg|_{\substack{p_0,\theta_0\\\varepsilon_{ij}=0}}\varepsilon_{ij}\varepsilon_{kl}+\\+\frac{\partial^2\Phi_s}{\partial\theta\partial p}\bigg|_{\substack{p_0,\theta_0\\\varepsilon_{ij}=0}}\left(\theta-\theta_0\right)\left(p-p_0\right)+\frac{\partial^2\Phi_s}{\partial\theta\partial\varepsilon_{ij}}\bigg|_{\substack{p_0,\theta_0\\\varepsilon_{ij}=0}}\left(\theta-\theta_0\right)\varepsilon_{ij}+\frac{\partial^2\Phi_s}{\partial p\partial\varepsilon_{ij}}\bigg|_{\substack{p_0,\theta_0\\\varepsilon_{ij}=0}}\left(p-p_0\right)\varepsilon_{ij}
\end{multline}

Если $\varphi=0$, то все слагаемые с давлением $p$ флюида обнуляются и получаем уравнения термоупругости.

При ненулевой пористости можем получить уравнения термопороупругости (есть вклад связанный с порами и давлением флюида, которое передаётся только через поры).

Перепишем скелетное замыкающее соотношение на энтропию из таблицы \ref{table:con}, подставив разложение \ref{PhiSTaylor}:
\begin{multline}
\eta_s=-\frac{\partial\Phi_s}{\partial\theta}=-\frac{\partial\Phi_s}{\partial\theta}\bigg|_{\substack{p_0,\theta_0\\\varepsilon_{ij}=0}}-\frac{\partial^2\Phi_s}{\partial p\partial\theta}\bigg|_{\substack{p_0,\theta_0\\\varepsilon_{ij}=0}}\left(p-p_0\right)-\frac{\partial^2\Phi_s}{\partial\theta\partial\varepsilon_{ij}}\bigg|_{\substack{p_0,\theta_0\\\varepsilon_{ij}=0}}\varepsilon_{ij}-\\-\frac{\partial^2\Phi_s}{\partial\theta^2}\bigg|_{\substack{p_0,\theta_0\\\varepsilon_{ij}=0}}\left(\theta-\theta_0\right)
\end{multline}

Удельная теплоёмкость скелета:
\beq
c_s^\theta=\theta\frac{\partial\eta_s}{\partial\theta}\approx -\theta\frac{\partial^2\Phi_s}{\partial\theta^2}\bigg|_{\substack{p_0,\theta_0\\\varepsilon_{ij}=0}}
\eeq
Видим, что в первом приближении теплоёмкость постоянна (не зависит от температуры и давления).

Перепишем скелетное замыкающее соотношение на тензор эффективных напряжений из таблицы \ref{table:con}, подставив разложение \ref{PhiSTaylor}:
\begin{multline}\label{TIJ}
T_{ij}=r_s\frac{\partial\Phi_s}{\partial\varepsilon_{ij}}=r_s\left(\frac{\partial\Phi_s}{\partial\varepsilon_{ij}}\bigg|_{\substack{p_0,\theta_0\\\varepsilon_{ij}=0}}+\frac{\partial^2\Phi_s}{\partial\varepsilon_{ij}\partial\varepsilon_{kl}}\bigg|_{\substack{p_0,\theta_0\\\varepsilon_{ij}=0}}\varepsilon_{kl}+\frac{\partial^2\Phi_s}{\partial\varepsilon_{ij}\partial\theta}\bigg|_{\substack{p_0,\theta_0\\\varepsilon_{ij}=0}}\left(\theta-\theta_0\right)+\right. \\ \left.+\frac{\partial^2\Phi_s}{\partial\varepsilon_{ij}\partial p}\bigg|_{\substack{p_0,\theta_0\\\varepsilon_{ij}=0}}\left(p-p_0\right)\right)
\end{multline}
Первое и второе слагаемые отвечают за упругость. Третье слагаемое -- за термоупругость. Четвёртое слагаемое -- за пороупругость.

Посмотрим на физический смысл каждого из слагаемых в \eqref{TIJ}.

В первом слагаемом "<зашито"> начальное напряжение:
\beq\label{1}
T_{ij}^0=r_s\frac{\partial\Phi_s}{\partial\varepsilon_{ij}}\bigg|_{\substack{p_0,\theta_0\\\varepsilon_{ij}=0}}
\eeq

Из второго слагаемого введём тензор упругих коэффициентов (тензор жёсткости):
\beq\label{2}
L_{ijkl}=r_s\frac{\partial^2\Phi_s}{\partial\varepsilon_{ij}\partial\varepsilon_{kl}}\bigg|_{\substack{p_0,\theta_0\\\varepsilon_{ij}=0}}
\eeq

Из четвёртого слагаемого введём тензор коэффициентов Био (показывает влияние давления флюида на полное напряжение в среде; например, влияние пластового давления на полное напряжение в породе):
\beq\label{3}
\alpha_{ij}=-r_s\frac{\partial^2\Phi_s}{\partial\varepsilon_{ij}\partial p}\bigg|_{\substack{p_0,\theta_0\\\varepsilon_{ij}=0}}
\eeq

С третьим слагаемым
\beq
X_{ij}=r_s\frac{\partial^2\Phi_s}{\partial\varepsilon_{ij}\partial\theta}\bigg|_{\substack{p_0,\theta_0\\\varepsilon_{ij}=0}}
\eeq
разобраться сложнее всего, так как обычно записывают зависимость от температуры в терминах деформаций:
\beq
\varepsilon_{ij}=\varepsilon_{ij}^0+\alpha_{ij}^\theta\left(\theta-\theta_0\right)
\eeq
Но в \eqref{TIJ} у нас влияние температуры на напряжение при условии сохранения деформации.

Можем выразить $\varepsilon_{kl}$ из \ref{TIJ}:
\beq
L_{ijkl}\varepsilon_{kl}=T_{ij}-T_{ij}^0-X_{ij}\left(\theta-\theta_0\right)+\alpha_{ij}\left(p-p_0\right)
\eeq
тогда
\beq
\varepsilon_{kl}=L_{ijkl}^{-1}\left(T_{ij}-T_{ij}^0-X_{ij}\left(\theta-\theta_0\right)+\alpha_{ij}\left(p-p_0\right)\right)
\eeq
и из последнего равенства тензор термического расширения (не путать с тензором коэффициентов Био \eqref{3} -- букв не хватает):
\beq\label{4}
\alpha_{kl}^\theta=-X_{ij}L_{ijkl}^{-1}
\eeq

Таким образом, подставляя \eqref{1}, \eqref{2}, \eqref{3} и \eqref{4} в \ref{TIJ}, получаем следующее линейное замыкающее соотношение на тензор эффективных напряжений:
\beq
T_{ij}\approx T_{ij}^0+L_{ijkl}\varepsilon_{kl}-\alpha_{kl}^\theta L_{ijkl}\left(\theta-\theta_0\right)-\alpha_{ij}\left(p-p_0\right)
\eeq
Примечание. Везде в механике: положительные напряжения -- растягивающие, а отрицательные -- сжимающие. Но в геомеханике наоборот.

\begin{comment}
\begin{center}
	Ход чёрных
	\medskip
	
	\newgame
	\fenboard{r2qnrk1/p3p3/2p3p1/3pb2P/6p1/2N1B3/PPPQ4/2KR3R w - - 0 20}
	\showboard
\end{center}
\end{comment}

Перепишем скелетное замыкающее соотношение на пористость из таблицы \ref{table:con}, подставив разложение \ref{PhiSTaylor}:
\begin{multline}\label{LinPhi}
\varphi=-r_s\frac{\partial\Phi_s}{\partial p}=-r_s\left(\frac{\partial\Phi_s}{\partial p}\bigg|_{\substack{p_0,\theta_0\\\varepsilon_{ij}=0}}+\frac{\partial^2\Phi_s}{\partial p^2}\bigg|_{\substack{p_0,\theta_0\\\varepsilon_{ij}=0}}\left(p-p_0\right)+\frac{\partial^2\Phi_s}{\partial p\partial\theta}\bigg|_{\substack{p_0,\theta_0\\\varepsilon_{ij}=0}}\left(\theta-\theta_0\right)+\right.\\+\left.\frac{\partial^2\Phi_s}{\partial p\partial\varepsilon_{ij}}\bigg|_{\substack{p_0,\theta_0\\\varepsilon_{ij}=0}}\varepsilon_{ij}\right)
\end{multline}

Посмотрим на физический смысл каждого из слагаемых в \eqref{LinPhi}.

В первом слагаемом "<зашита"> начальная пористость:
\beq
\varphi_0=-r_s\frac{\partial\Phi_s}{\partial p}\bigg|_{\substack{p_0,\theta_0\\\varepsilon_{ij}=0}}
\eeq

Из второго слагаемого введём модуль Био $N$ так, чтобы его обратный
\beq
\frac{1}{N}=-r_s\frac{\partial^2\Phi_s}{\partial p^2}\bigg|_{\substack{p_0,\theta_0\\\varepsilon_{ij}=0}}
\eeq

Из третьего слагаемого введём коэффициент термического расширения пор (как правило им пренебрегают):
\beq
\alpha_\theta^\varphi=-r_s\frac{\partial^2\Phi_s}{\partial p\partial\theta}\bigg|_{\substack{p_0,\theta_0\\\varepsilon_{ij}=0}}
\eeq

В четвёртом слагаемом есть тензор коэффициентов Био $\alpha_{ij}$, который ввели ранее \eqref{3}.

Таким образом, из \eqref{LinPhi} получаем линейное замыкающее соотношение на пористость:
\beq
\varphi=\varphi_0+\frac{p-p_0}{N}+\alpha_\theta^\varphi\left(\theta-\theta_0\right)+\alpha_{ij}\varepsilon_{ij}
\eeq

\subsection{Общие рассуждения о введённых величинах}
Везде, где есть производная по $\varepsilon_{ij}$, симметричность относительно замены индексов $i$ и $j$ (например, тензор коэффициентов Био \eqref{3} симметричен относительно замены индексов)

Теперь самое интересное: тензор упругих коэффициентов $L_{ijkl}$. У тензора четвёртого ранга 81 компонента. Но мы знаем, что для тензора $L_{ijkl}$ по определению верно равенство \eqref{2}: есть симметрия относительно замены индексов $i$ и $j$, относительно замены индексов $k$ и $l$, а также относительно замены $ij$ и $kl$. Таким образом, у тензора упругих коэффициентов 3 симметрии и, следовательно, $6+30/2=21$ независимая компонента в самом анизотропном случае.\\

Модуль Био -- коэффициент передачи от давления в изменение пористости (величина, обратная сжимаемости пор).\\

Тензор коэффициентов Био -- \\


Мы разобрались с консервативными слагаемыми второго начала термодинамики в форме неравенства Клаузиуса-Дюгема \eqref{KlD}. Консервативные слагаемые работают в термодинамически обратимых процессах.

\subsection{Роль диссипативных слагаемых неравенства Клаузиуса-Дюгема}

Термическая диссипация (ввели при переписывании неравенства \eqref{KlD0}):
\beq\label{ThermoDis}
\delta_\theta=-\frac{q_i \partial_i\theta}{\theta}
\eeq

Механическая диссипация (ввели при переписывании неравенства \eqref{KlDFull}):
\beq\label{MechDis}
\delta_f=-b_i^{dis,f}W_i-T_{ij}^{dis,f}\partial_j W_i
\eeq

Из неравенства Клаузиуса-Дюгема:
\beq\label{DisKlD}
\delta_f+\delta_\theta\geqslant 0
\eeq

Диссипации \eqref{ThermoDis} и \eqref{MechDis} независимы (например, можем приложить градиент температуры, но флюид будет стоять на месте), следовательно, неравенство \eqref{DisKlD} распадается на 2 независимых неравенства:
\beq\label{2InEq}
\delta_f\geqslant0\text{ и }\delta_\theta\geqslant0
\eeq

Заметим, что при отсутствии градиента температуры, поток тепла равен нулю (из физических соображений) и термическая диссипация равна нулю:
\beq\label{ZERO}
\partial_i\theta=0\Rightarrow q_i=0\text{ и }\delta_\theta=0
\eeq

Представим термическую диссипацию в виде частичной суммы ряда Тейлора:
\beq
\delta_\theta\approx\delta_\theta^0\bigg|_{\nabla_i\theta=0}+\frac{\partial\delta_\theta}{\partial\left(\nabla_i\theta\right)}\bigg|_{\nabla_i\theta=0}\nabla_i\theta+\frac{1}{2}\frac{\partial^2\delta_\theta}{\partial\left(\nabla_i\theta\right)\partial\left(\nabla_j\theta\right)}\bigg|_{\nabla_i\theta=0}\nabla_i\theta\nabla_j\theta
\eeq

Из следствия \eqref{ZERO} первое слагаемое равно нулю.

Из следствия \eqref{ZERO} и неравенства \eqref{2InEq} производная во втором слагаемом тоже равна нулю (т.к. в этой точке достигается минимум $\delta_\theta$).

Таким образом,
\beq
\delta_\theta\approx\frac{1}{2}\frac{\partial^2\delta_\theta}{\partial\left(\nabla_i\theta\right)\partial\left(\nabla_j\theta\right)}\bigg|_{\nabla_i\theta=0}\nabla_i\theta\nabla_j\theta
\eeq

С другой стороны, раскрывая $\delta_\theta$ по определению \eqref{ThermoDis}, получаем:
\beq
-\frac{q_i\nabla_i\theta}{\theta}\approx\frac{1}{2}\frac{\partial^2\delta_\theta}{\partial\left(\nabla_i\theta\right)\partial\left(\nabla_j\theta\right)}\bigg|_{\nabla_i\theta=0}\nabla_i\theta\nabla_j\theta
\eeq

Следовательно,
\beq
q_i\approx-\theta\cdot\frac{1}{2}\frac{\partial^2\delta_\theta}{\partial\left(\nabla_i\theta\right)\partial\left(\nabla_j\theta\right)}\bigg|_{\nabla_i\theta=0}\nabla_j\theta
\eeq
Введём тензор теплопроводности:
\beq
\displaystyle{}\varkappa_{ij}=\theta\cdot\frac{1}{2}\frac{\partial^2\delta_\theta}{\partial\left(\nabla_i\theta\right)\partial\left(\nabla_j\theta\right)}\bigg|_{\nabla_i\theta=0}
\eeq
тогда
\beq
q_i\approx-\varkappa_{ij}\nabla_j\theta
\eeq
(получили закон Фурье, который использовали при выводе уравнения теплопроводности \eqref{HeatTransfer})

Но получили не только закон Фурье, но и симметричность, и положительную определённость тензора теплопроводности $\varkappa_{ij}$.

Далее будем работать с механической диссипацией \eqref{MechDis}, условие на которую распадается на 2 независимых условия:
\beq\label{MechFirst}
\delta_f^1=-b_i^{dis}W_i\geqslant0
\eeq
и
\beq\label{MechSecond}
\delta_f^2=-T_{ij}^{dis}\partial_jW_i\geqslant0
\eeq

Заметим, что верно следствие
\beq\label{Important}
W_i=0\Rightarrow\delta_f^1=0\text{ и }b_i^{dis}=0
\eeq

Представим $\delta_f^1$ в виде частичной суммы ряда Тейлора:
\begin{multline}
\delta_f^1\approx\delta_f^1\bigg|_{W_i=0}+\frac{\partial\delta_f^1}{\partial W_i}\bigg|_{W_i=0}W_i+\frac{1}{2}\frac{\partial^2\delta_f^1}{\partial W_i\partial W_j}\bigg|_{W_i=0}W_iW_j+\\+\frac{1}{6}\frac{\partial^3\delta_f^1}{\partial W_i\partial W_j \partial W_k}\bigg|_{W_i=0}W_iW_jW_k
\end{multline}

Из следствия \eqref{Important} первое слагаемое равно нулю.

Из следствия \eqref{Important} и неравенства \eqref{MechFirst} производная во втором слагаемом тоже равна нулю (т.к. в этой точке достигается минимум $\delta_f^1$)

Таким образом,
\beq
b_i^{dis}\approx-\frac{1}{2}\frac{\partial^2\delta_f^1}{\partial W_i\partial W_j}\bigg|_{W_i=0}W_j-\frac{1}{6}\frac{\partial^3\delta_f^1}{\partial W_i\partial W_j \partial W_k}\bigg|_{W_i=0}W_jW_k
\eeq

Обозначим
\beq
A_{ij}=-\frac{1}{2}\frac{\partial^2\delta_f^1}{\partial W_i\partial W_j}\bigg|_{W_i=0}
\eeq

Вспомним ЗСИ (в форме Лагранжа):
\beq
r_f\frac{d_fv_i^f}{dt}=f_i+b_i^{dis}+b_i^{eq}+\partial_jT_{ij}^f
\eeq
перепишем
\beq
r_f\frac{d_fv_i^f}{dt}=r_fg_i+A_{ij}W_j+p\partial_i\varphi-\varphi\partial_ip-p\partial_i\varphi
\eeq
и получаем уравнение Навье-Стокса для пористой среды:
\beq
\partial_ip=-\rho_f\frac{d_fv_i^f}{dt}+\rho_fg_i+A_{ij}W_j
\eeq

Далее рассматриваем стационарное течение $\displaystyle{}\left(\frac{d_fv_i^f}{dt}=0\right)$, тогда из предыдущего равенства
\beq
W_j=A_{ij}^{-1}\left(\partial_ip-\rho_fg_i\right)
\eeq
Получили закон Дарси.

Аналогичным образом из \eqref{MechSecond}, учитывая равенство $\partial_jW_i=\dot{\varepsilon}_{ij}^f$ (в предположении $v_s=0$) получаем
\beq
T_{ij}^{dis}\approx-\frac{1}{2}\frac{\partial^2\delta_f^2}{\partial\dot{\varepsilon}_{ij}\partial\dot{\varepsilon}_{kl}}\bigg|_{\dot{\varepsilon}_{ij}=0}\dot{\varepsilon}_{kl}
\eeq

\end{document}




















