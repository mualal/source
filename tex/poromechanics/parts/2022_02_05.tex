\documentclass[main.tex]{subfiles}

\begin{document}

\textcolor{red}{Лекция 05.02.2022.}

\section{Закон сохранения энергии}

При выводе закона сохранения энергии (ЗСЭ) будем использовать ранее полученные закон сохранения массы (ЗСМ) и закон сохранения импульса (ЗСИ).\\

ЗСМ:
\beq\label{ЗСМ}
\frac{\partial r_{a}}{\partial t}+\partial_i\left(v_i^a r_a\right)=j_{ab},
\eeq
где $j_{ab}$ -- обменное слагаемое между фазами $a$ и $b$\\

ЗСИ:
\beq
r_a\frac{\partial v_i^a}{\partial t}+r_a v_j^a\partial_j v_i^a=\partial_j T_{ij}^a+b_i^{int,a}+f_i^a
\eeq

\subsection{Вывод ЗСЭ одной фазы и общего ЗСЭ}

В законе сохранения энергии впервые появляются тепло и температура.
Приток тепла в некоторый объём:
\beq
\delta Q=\left[\iiint\limits_{V}{\left(r_a Q^a+Q^{int,a}\right)}dV-\oiint\limits_{\partial V} h_a dS\right]dt,
\eeq
где $Q^a$ -- теплота от массовых источников тепла (например, в ядерных реакциях), $Q^{int, a}$ -- обмен теплом между фазами, $h_a=n_iq_i^a$ -- поток тепла через границу

Выполняется первое начало термодинамики:
\beq
\delta Q=\delta A_{\text{внутр}}+dU\Leftrightarrow dU=\delta Q+\delta A_{\text{внеш}}
\eeq

Внутренняя энергия $\displaystyle{}U=\iiint\limits_{V}{r_ae_a}dV$, где $\displaystyle{}e_a=u_a+\frac{1}{2}v_i^av_i^a$ -- удельная плотность энергии (см. теорему Кёнига)

Полная производная внутренней энергии по времени:
\beq
\frac{dU}{dt}=\iiint\limits_{V}{\frac{\partial}{\partial t}\left(r_a\left(u_a+\frac{1}{2}v_i^av_i^a\right)\right)dV}+\oiint\limits_{\partial V}{r_a\left(u_a+\frac{1}{2} v_i^a v_i^a\right)v_j n_j dS}
\eeq

Небольшое напоминание о работе и мощности сил: $\delta A=\vec{F}\cdot\vec{dS}=\left(\vec{F}\cdot\vec{v}\right)dt$

Работа внешних сил (объёмных + поверхностных; так как работа именно внешних сил, то внутренние взаимодействия $b^{int}$ не учитываем):
\beq
\delta A_{\text{внеш}}=\left[\iiint\limits_{V}{f_i^a v_i^a dV}+\oiint\limits_{\partial V}{v_i^aT_{ij}^an_jdS}\right]dt
\eeq

Подставляем $\displaystyle{}\frac{dU}{dt}$, $\delta Q$ и $\delta A_{\text{внеш}}$ в первое начало термодинамики:
\begin{multline}\label{FirstThermodynLaw}
\iiint\limits_{V}{\frac{\partial}{\partial t}\left(r_a\left(u_a+\frac{1}{2}v_i^av_i^a\right)\right)dV}+\oiint\limits_{\partial V}{r_a\left(u_a+\frac{1}{2} v_i^a v_i^a\right)v_j n_j dS}=\\
=\iiint\limits_{V}{f_i^a v_i^a dV}+\oiint\limits_{\partial V}{v_i^aT_{ij}^an_jdS}+\iiint\limits_{V}{\left(r_aQ^a+Q^{int,a}\right)dV}-\oiint\limits_{\partial V}q_j^a n_j dS
\end{multline}

Воспользуемся теоремой Остроградского-Гаусса и преобразуем подынтегральное выражение в левой части равенства \eqref{FirstThermodynLaw}:
\beq
e_a\frac{\partial r_a}{\partial t}+r_a\left(\frac{\partial u_a}{\partial t}+v_i^a\frac{\partial v_i^a}{\partial t}\right)+\partial_j\left(r_ae_av_j^a\right)=
\eeq

раскрывая последнюю производную произведения и вынося $e_a$ за скобку
\beq
=e_a\left(\frac{\partial r_a}{\partial t}+\partial_j\left(r_av_j^a\right)\right)+r_a\left(\frac{\partial u_a}{\partial t}+v_i^a\frac{\partial v_i^a}{\partial t}\right)+r_av_j^a\partial_{j}e_a=
\eeq

учитывая ЗСМ \eqref{ЗСМ} и подставляя выражение для $e_a$
\beq
=r_a\frac{\partial u_a}{\partial t}+r_av_i^a\frac{\partial v_i^a}{\partial t}+r_av_j^a\partial_j u_a+r_av_j^a\partial_j\left(\frac{1}{2}v_i^av_i^a\right)=
\eeq

группируя первое и третье слагаемые в полную производную и вынося $v_i$ во втором и четвёртом слагаемых за скобку
\beq\label{FirstTransformedThermoPart}
=r_a\frac{d_au_a}{dt}+v_i^a\left(r_a\frac{\partial v_i^a}{\partial t}+r_av_j^a\partial_j v_i^a\right)
\eeq

Воспользуемся теоремой Остроградского-Гаусса и преобразуем подынтегральное выражение в правой части равенства \eqref{FirstThermodynLaw}:
\beq
v_i^af_i^a+\partial_j\left(T_{ij}^av_i^a\right)+r_a Q^a+Q^{int,a}-\partial_jq_j^a=
\eeq

раскрывая производную во втором слагаемом и вынося $v_i^a$ за скобку
\beq\label{SecondTransformedThermoPart}
=v_i^a\left(f_i^a+b_i^{int,a}-b_i^{int,a}+\partial_jT_{ij}^a\right)+T_{ij}^a\left(\partial_jv_i^a\right)+r_aQ^a+Q^{int,a}-\partial_jq_j^a
\eeq

Подставляем преобразованные части \eqref{FirstTransformedThermoPart} и \eqref{SecondTransformedThermoPart} в исходное равенство \eqref{FirstThermodynLaw}, опуская тройной интеграл (по лемме Римана, так как объём произвольный и подынтегральная ф-ия непрерывна) и вычленяя ЗСИ:
\beq
r_a\frac{d_au_a}{dt}=T_{ij}^a\partial_jv_i^a+r_aQ^a+Q^{int,a}-\partial_jq_j^a-b_i^{int,a}v_i^a
\eeq

Получили ЗСЭ одной фазы. Суммируя по всем фазам, получаем общий ЗСЭ:
\begin{multline}
\sum\limits_{a}{r_a\frac{d_au_a}{dt}}=\sum\limits_{a}{T_{ij}^a\partial_jv_i^a+rQ-\partial_jq_j-b_i^{int}W_i}\,\,\,;\,\,\,rQ=\sum\limits_{a}{r_aQ^a}\,\,\,;\\ \,\,\,\,\,\,\,\,\,\,Q^{int,f}=-Q^{int,s}\,\,\,;\,\,\,b^{int}=b^{int,f}=-b^{int,s},
\end{multline}
где $q_j$ -- общий поток тепла.

\subsection{Уравнение теплопроводности}

Из ЗСЭ получим уравнение теплопроводности. Воспульзуемся двумя замыкающими соотношениями: законом Фурье и связью между внутренней энергией и температурой (в дальнейшем мы выведем эти замыкающие соотношения \eqref{CF} и \eqref{FourierLow} из второго начала термодинамики).

Закон Фурье: $q_i=-\varkappa_{ij}\partial_j\theta$, где $\varkappa_{ij}$ -- тензор теплопроводности

Связь внутренней энергии и температуры: $u_a=c_a^\theta\theta$, где $c_a^\theta$ -- удельная теплоёмкость фазы (считаем, что $\theta=\theta^s=\theta^f$, т.е. на масштабе рассматриваемых сред (на микромасштабе) теплообмен мгновенный, происходит быстрее любых других процессов; в плазме это предположение не выполняется, но у нас не плазма)

Также считаем, что $v_i^a=0$ (т.е. происходит только теплообмен, среды находятся в механическом равновесии)

Учитывая замыкающие соотношения и поставленное условие, перепишем общий ЗСЭ:
\beq\label{HT}
\sum\limits_{a}{r_ac_a^\theta\frac{\partial\theta}{\partial t}}=rQ+\partial_i\left(\varkappa_{ij}\partial_j\theta\right)
\eeq

предполагая однородность $\left(\varkappa_{ij}=\text{const}\right)$
\beq\label{HeatTransfer}
\sum\limits_{a}{r_ac_a^\theta\frac{\partial\theta}{\partial t}}-\varkappa_{ij}\partial_i\partial_j\theta=rQ
\eeq

предполагая изотропность $\left(\varkappa_{ij}=\varkappa\delta_{ij}\right)$
\beq
\frac{\partial\theta}{\partial t}-\frac{\varkappa}{rc}\Delta\theta=\frac{Q}{c}\,\,\,;\,\,\, rc=\sum\limits_{a}{r_ac_a^\theta}
\eeq

Получили уравнение теплопроводности ($c$ -- удельная теплоёмкость смеси; $Q$ -- удельный источник тепла)

\newpage
\section{Второе начало термодинамики}

Допустим у нас одна среда, тогда

\begin{itemize}

\item 5 скалярных уравнений: 1 из ЗСМ, 3 из ЗСИ, 1 из ЗСЭ. Есть ещё ЗСМИ, но он просто симметризует тензор напряжений (учитывается, когда говорим, что тензор напряжений симметричен)

\item неизвесных: 3 смещения в пространстве от времени, плотность материала, температура, внутренняя энергия (её связь с температурой не очевидна), 6 компонент тензора напряжений.

\end{itemize}

Видим, что неизвестных существенно больше, поэтому необходимо выбрать (ввести) замыкающие соотношения. Но произвольно это сделать не можем, так как есть ещё одно начало термодинамики (энтропия замкнутой системы может только увеличиваться со временем).

Энтропия:
\beq
H=\iiint\limits_{V}{\sum\limits_{a}{r_a\eta_a}dV},
\eeq
где $\eta_a$ -- удельная плотность энтропии

Изменение энтропии в случае равновесного процесса:
\beq
\frac{\delta Q}{T}=\iiint\limits_{V}\frac{rQ}{\theta}dV-\oiint\limits_{\partial V}{\frac{hdS}{\theta}}
\eeq

Неравенство Клаузиуса-Дюгема (одна из форм второго начала термодинамики):
\beq
\iiint\limits_{V}{\frac{rQ}{\theta}dV}-\oiint\limits_{\partial V}\frac{hdS}{\theta}\leq\frac{\partial H}{\partial t}+\oiint\limits_{\partial V}\left(\sum\limits_{a}{r_a\eta_av_i^a}\right)n_idS,
\eeq

где $\displaystyle{}\frac{\partial H}{\partial t}$ -- изменение энтропии в объёме; $\displaystyle{}\oiint\limits_{\partial V}\left(\sum\limits_{a}{r_a\eta_av_i^a}\right)n_idS$ -- отток энтропии наружу

Подставляем выражение для энтропии и переносим всё в одну часть неравенства, учитывая, что $h=n_iq_i$, и опуская при этом тройной интеграл по лемме Римана:
\beq
\sum\limits_{a}{\frac{\partial\left(r_a\eta_a\right)}{\partial t}}+\partial_i\left(\sum\limits_{a}{r_a\eta_av_i^a}\right)-\frac{rQ}{\theta}+\partial_i\left(\frac{q_i}{\theta}\right)\geq 0
\eeq

раскрывая производные и вынося $\eta_a$ за скобку
\beq
\sum\limits_{a}{\left[\eta_a\left(\frac{\partial r_a}{\partial t}+\partial_i\left(r_av_i^a\right)\right)+r_a\frac{\partial \eta_a}{\partial t}+r_av_i^a\partial_i\eta_a\right]}-\frac{rQ}{\theta}+\frac{\partial_iq_i}{\theta}-q_i\frac{\partial_i\theta}{\theta^2}\geq 0
\eeq

учитывая ЗСМ \eqref{ЗСМ} и выделяя полную производную удельной плотности энтропии
\beq
\sum\limits_{a}{r_a\frac{d_a\eta_a}{dt}}-\frac{rQ}{\theta}+\frac{\partial_iq_i}{\theta}-q_i\frac{\partial_i\theta}{\theta^2}\geq 0
\eeq

Из ЗСЭ:
\beq
rQ-\partial_iq_i=\sum\limits_a{r_a\frac{d_au_a}{dt}}-\sum\limits_a{T_{ij}^a\partial_jv_i^a}+b_i^{int}W_i
\eeq

Подставляем в неравенство:
\beq
\sum\limits_{a}{r_a\frac{d_a\eta_a}{dt}}-\sum\limits_{a}{r_a\frac{1}{\theta}\frac{d_au_a}{dt}}+\sum\limits_a{\frac{1}{\theta}T_{ij}^a\left(\partial_jv_i^a\right)-\frac{1}{\theta}b_i^{int}W_i}-\frac{q_i\partial_i\theta}{\theta^2}\geq 0
\eeq

умножая обе части неравенства на $\theta\,\,\left(\theta>0\right)$
\beq\label{KlD0}
\sum\limits_a{r_a\left(\theta\frac{d_a\eta_a}{dt}-\frac{d_au_a}{dt}\right)}+\sum\limits_a{T_{ij}^a\left(\partial_j v_i^a\right)}-b_i^{int}W_i-\frac{q_i\partial_i\theta}{\theta}\geq 0
\eeq

В дальнейшем введём обозначение $\displaystyle{}\delta_\theta=-\frac{q_i\partial_i\theta}{\theta}$ -- термическая диссипация.


Энтропия и её частные производные физически не совсем понятны, поэтому проще ввести термодинамический потенциал, у которого частные производные имеют определённый физический смысл.

Далее переходим на школьные обозначения (для краткости).

В равновесном процессе:
\beq
\delta Q=TdS=dU+dA_{внутр}
\eeq

Введём свободную энергию (термодинамический потенциал):
\beq
\Psi=U-TS
\eeq

Тогда
\beq
d\Psi=dU-TdS-SdT=-pdV-SdT
\eeq

Следовательно, $\Psi=\Psi(V,T)$ (т.е. $\Psi$ -- функция объёма и температуры, т.е. физически понятна)

Получили, что
\beq
TdS-dU=-d\Psi-SdT
\eeq

Тогда, возвращаясь к исходным обозначениям, получаем
\beq
\theta\frac{d_a\eta_a}{dt}-\frac{d_au_a}{dt}=-\left(\frac{d_a\Psi_a}{dt}+\eta_a\frac{d_a\theta}{dt}\right)
\eeq

Подставляем в неравенство:
\beq
-\sum\limits_a{r_a\left(\frac{d_a\Psi_a}{dt}+\eta_a\frac{d_a\theta}{dt}\right)}+\sum\limits_a{T_{ij}^a\left(\partial_jv_i^a\right)}-b_i^{int}W_i+\delta_\theta\geq 0
\eeq

\end{document}

