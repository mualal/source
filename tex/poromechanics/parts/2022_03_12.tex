\documentclass[main.tex]{subfiles}

\begin{document}

\textcolor{red}{Лекция 12.03.2022.}

Ранее ввели определение тензора вязкости \eqref{viscosity} и получили связь \eqref{SigmaViscEpsilon} диссипативного тензора истинных напряжений флюида и тензора скоростей малых деформаций флюида (при условии отсутствия скелета или нулевой скорости скелета во всех точках). Далее подробно разберёмся с тензором вязкости, учитывая изотропность флюида.

\section{Случай изотропной жидкости}

Изотропность = инвариантность физических свойств во всех направлениях. Математически означает неизменность представления тензора при повороте системы координат.

\end{document}
