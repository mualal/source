\documentclass[main.tex]{subfiles}

\begin{document}

\textcolor{red}{Лекция 12.03.2022.}

Ранее ввели определение тензора вязкости \eqref{viscosity} и получили связь \eqref{SigmaViscEpsilon} диссипативного тензора истинных напряжений флюида и тензора скоростей малых деформаций флюида (при условии отсутствия скелета или нулевой скорости скелета во всех точках). Далее подробно разберёмся с тензором вязкости, учитывая изотропность флюида.

\section{Случай изотропной жидкости}

Введём тензор поворота $\symbf{O}$ так, чтобы:
\beq
\symbf{a}=\symbf{O}\cdot\symbf{a}'
\eeq
в координатном виде:
\beq
a_i=O_{ik}a_k'
\eeq

Распишем тензор $\symbf{\sigma}$ в штрихованном базисе и базисе без штриха и приравняем полученные выражения:
\beq
n_m'n_r'\sigma_{mr}'=n_in_j\sigma_{ij}
\eeq
учитывая связь штрихованного базиса и базиса без штриха, получаем:
\beq\label{TensorRotInv}
n_m'n_r'\sigma_{mr}'=O_{im}n_m'O_{jr}n_r'\sigma_{ij}\Rightarrow \sigma_{mr}'=O_{im}\sigma_{ij}O_{jr}
\eeq
Вспоминая, что умножение на транспонированный тензор реализуется в случае сворачивания первого индекса с первым или второго индекса со вторым; а просто скалярное умножение на тензор реализуется в случае сворачивания первого индекса со вторым или второго индекса с первым, перепишем равенство \eqref{TensorRotInv} в тензорном виде:
\beq
\symbf{\sigma}'=\symbf{O}^{T}\cdot\symbf{\sigma}\cdot\symbf{O}
\eeq

Далее перепишем равенство \eqref{SigmaViscEpsilon} в тензорном виде в штрихованном базисе (в дальнейшем обозначения $dis, f$ опущены для краткости):
\beq
\symbf{O}^{T}\cdot\symbf{\sigma}\cdot\symbf{O}=-\symbf{\mu}':\left(\symbf{O}^T\cdot\symbf{\dot{\varepsilon}}\cdot\symbf{O}\right)
\eeq
преобразуем
\beq
\symbf{\sigma}=-\symbf{O}\cdot\left(\symbf{\mu}':\left(\symbf{O}^T\cdot\symbf{\dot{\varepsilon}}\cdot\symbf{O}\right)\right)\cdot\symbf{O}^T
\eeq

С другой стороны, в базисе без штриха:
\beq
\symbf{\sigma}=-\symbf{\mu}:\symbf{\dot{\varepsilon}}
\eeq

Таким образом,
\beq
\symbf{\mu}:\symbf{\dot{\varepsilon}}=\symbf{O}\cdot\left(\symbf{\mu}':\left(\symbf{O}^T\cdot\symbf{\dot{\varepsilon}}\cdot\symbf{O}\right)\right)\cdot\symbf{O}^T
\eeq

Перепишем в координатном виде:
\beq
\mu_{ijkl}\dot{\varepsilon}_{kl}=O_{ia}\mu_{abcd}'\dot{\varepsilon}_{cd}'O_{jb}=\mu_{abcd}'\dot{\varepsilon}_{kl}O_{kc}O_{ld}O_{ia}O_{jb}
\eeq
сокращая $\dot{\varepsilon}_{kl}$:
\beq
\mu_{ijkl}=\mu_{abcd}'O_{kc}O_{ld}O_{ia}O_{jb}
\eeq


Изотропность = инвариантность физических свойств во всех направлениях. Математически означает неизменность представления тензора при повороте системы координат. Далее будем считать тензор $\symbf{\mu}$ изотропным ($\symbf{\mu}'=\symbf{\mu}$). Тогда
\beq\label{IsotropicTensor}
\mu_{ijkl}'=\mu_{abcd}'O_{kc}O_{ld}O_{ia}O_{jb}
\eeq

Далее получим инварианты тензора $\symbf{\sigma}$. Скалярная функция 

Вспомним характеристическое уравнение для симметричного тензора второго ранга $\symbf{\sigma}$:
\beq\label{Eigen}
\det{\left(\symbf{\sigma}-\lambda\symbf{E}\right)}=0
\eeq

Вид уравнения \eqref{Eigen} не изменяется при повороте. Действительно,
\begin{multline}\label{StableProve}
\det{\left(\symbf{\sigma}-\lambda\symbf{E}\right)}=\det{\symbf{O}^T}\cdot\det{\left(\symbf{\sigma}-\lambda\symbf{E}\right)}\cdot\det{\symbf{O}}=\det{\left(\symbf{O}^T\cdot\left(\symbf{\sigma}-\lambda\symbf{E}\right)\cdot\symbf{O}\right)}=\\=\det{\left(\symbf{O}^T\cdot\symbf{\sigma}\cdot\symbf{O}-\lambda\symbf{O}^T\cdot\symbf{O}\right)}=\det{\left(\symbf{\sigma}'-\lambda\symbf{E}\right)}
\end{multline}

Раскроем определитель в уравнении \eqref{Eigen}:
\beq
\det{\left(\symbf{\sigma}-\lambda\symbf{E}\right)}=\frac{1}{3!}\varepsilon_{ikm}\varepsilon_{jln}\left(\sigma_{ij}-\lambda\delta_{ij}\right)\left(\sigma_{kl}-\lambda\delta_{kl}\right)\left(\sigma_{mn}-\lambda\delta_{mn}\right)=
\eeq
раскрывая скобки
\begin{multline}
=\frac{1}{3!}\varepsilon_{ikm}\varepsilon_{jln}\left(\sigma_{ij}\sigma_{kl}\sigma_{mn}-\lambda\left(\delta_{ij}\sigma_{kl}\sigma_{mn}+\delta_{kl}\sigma_{ij}\sigma_{mn}+\delta_{mn}\sigma_{ij}\sigma_{kl}\right)+\right.\\\left.+\lambda^2\left(\sigma_{mn}\delta_{kl}\delta_{ij}+\sigma_{kl}\delta_{ij}\delta_{mn}+\sigma_{ij}\delta_{kl}\delta_{mn}\right)-\lambda^3\delta_{ij}\delta_{kl}\delta_{mn}\right)=
\end{multline}
обнаруживая определитель и применяя Кронекеры
\begin{multline}
=\det{\symbf{\sigma}}-\frac{\lambda}{3!}\left(\varepsilon_{ikm}\varepsilon_{iln}\sigma_{kl}\sigma_{mn}+\varepsilon_{kmi}\varepsilon_{knj}\sigma_{ij}\sigma_{mn}+\varepsilon_{mik}\varepsilon_{mjl}\sigma_{ij}\sigma_{kl}\right)+\\+\frac{\lambda^2}{3!}\left(\varepsilon_{ikm}\varepsilon_{ikn}\sigma_{mn}+\varepsilon_{mik}\varepsilon_{min}\sigma_{kl}+\varepsilon_{kmi}\varepsilon_{kmj}\sigma_{ij}\right)-\frac{\lambda^3}{3!}\varepsilon_{ikm}\varepsilon_{ikm}=
\end{multline}
вспоминая свойства символов Леви-Чивиты и символов Кронекера

(а именно $\varepsilon_{ikm}\varepsilon_{iln}=\delta_{kl}\delta_{mn}-\delta_{kn}\delta_{ml}$ и $\varepsilon_{ikm}\varepsilon_{ikn}=2\delta_{mn}$, и $\delta_{mm}=3$)
\beq
=\det{\symbf{\sigma}}-\frac{3\lambda}{6}\left(\sigma_{kk}\sigma_{mm}-\sigma_{nm}\sigma_{mn}\right)+\frac{3\lambda^2}{6}\cdot 2\sigma_{mm}-\frac{2\cdot 3\cdot\lambda^3}{6}=
\eeq
вспоминая определение следа
\beq
=\det{\symbf{\sigma}-\lambda\cdot\frac{\text{tr}^2\symbf{\sigma}-\text{tr}\,\symbf{\sigma}^2}{2}}+\lambda^2\cdot\text{tr}\,\symbf{\sigma}-\lambda^3.
\eeq

Так как вид характеристического уравнения не изменяется при повороте (см. \eqref{StableProve}), то и коэффициенты этого уравнения инвариантны относительно поворота. Получаем 3 функционально независимых главных инварианта симметричного тензора второго ранга $\symbf{\sigma}$:
\begin{itemize}
	\item первый: $I_1(\symbf{\sigma})=\text{tr}\,\symbf{\sigma}$
	\item второй: $\displaystyle{}I_2(\symbf{\sigma})=\frac{\text{tr}\,\symbf{\sigma^2}-\text{tr}^2\symbf{\sigma}}{2}$
	\item третий: $I_3(\symbf{\sigma})=\det{\symbf{\sigma}}$
\end{itemize}

И уравнение \eqref{Eigen} перепишется в виде
\beq
-\lambda^3+\lambda^2\cdot I_1(\symbf{\sigma})-\lambda\cdot I_2(\symbf{\sigma})+I_3(\symbf{\sigma})=0
\eeq

Аналогичные выкладки с характеристическим уравнением верны для любого симметричного тензора второго ранга (в частности, для $\symbf{\sigma}$ и $\symbf{\dot{\varepsilon}}$).

Из изотропности тензора $\symbf{\mu}$ \eqref{IsotropicTensor} следует, что $\symbf{\mu}$ зависит только от инвариантов тензоров $\symbf{\sigma}$ и $\symbf{\dot{\varepsilon}}$. И данная зависимость в первом приближении линейна.

Тогда
\beq
I_1(\symbf{\sigma})=K^1I_1(\symbf{\dot{\varepsilon}})+K^2I_2(\symbf{\dot{\varepsilon}})+K^3I_3(\symbf{\dot{\varepsilon}})
\eeq

Но связь линейна, поэтому след тензора напряжений не может зависеть от квадрата и от куба скорости изменения тензора малых деформаций, т.е. $K^2=0$, $K^3=0$ и
\beq\label{TraceConnect}
I_1(\symbf{\sigma})=K^1I_1(\symbf{\dot{\varepsilon}})
\eeq

Выделим девиаторную и шаровую части тензора $\symbf{\sigma}$:
\beq\label{SpherePart}
\sigma_{ij}=\left(\sigma_{ij}-\frac{1}{3}\sigma_{kk}\delta_{ij}\right)+\frac{1}{3}\sigma_{kk}\delta_{ij}
\eeq

След девиатоной части равен нулю:
\beq
\text{tr}\left(\sigma_{ij}-\frac{1}{3}\sigma_{kk}\delta_{ij}\right)=0,
\eeq
поэтому у девиаторной части есть только второй и третий инварианты и девиаторная часть тензора $\symbf{\sigma}$ не зависит от первого инварианта девиаторной части тензора $\symbf{\dot{\varepsilon}}$. 

Следовательно, с учётом линейности второй инвариант девиаторной части тензора $\symbf{\sigma}$ зависит только от второго инварианта девиаторной части тензора $\symbf{\dot{\varepsilon}}$:
\beq\label{DevConnect}
\sigma_{ij}-\frac{1}{3}\sigma_{kk}\delta_{ij}=K^2\left(\dot{\varepsilon}_{ij}-\frac{1}{3}\dot{\varepsilon}_{kk}\delta_{ij}\right)
\eeq

Таким образом, из \eqref{TraceConnect}, \eqref{SpherePart} и \eqref{DevConnect}
\beq
\sigma_{ij}=K^2\dot{\varepsilon}_{ij}-\frac{K^2}{3}\dot{\varepsilon}_{kk}\delta_{ij}+\frac{K^1}{3}\dot{\varepsilon}_{kk}\delta_{ij}
\eeq

Введём вязкость $\displaystyle{}\mu=-\frac{K^2}{2}$ и вторую вязкость $\displaystyle{}\eta=\frac{K^1}{3}$, тогда
\beq
\sigma_{ij}^{dis,f}=-2\mu\dot{\varepsilon}_{ij}+\left(\frac{2\mu}{3}+\eta\right)\delta_{ij}\dot{\varepsilon}_{kk}
\eeq
Получили закон Ньютона для изотропной вязкой жидкости.

Обычно $\eta\ll\mu$ (для воды примерно на 2 порядка).

Также получили изотропный тензор вязкости:
\beq
\mu_{ijkl}=-\mu\left(\delta_{ik}\delta_{jl}+\delta_{il}\delta_{jk}\right)+\left(\eta+\frac{2\mu}{3}\right)\delta_{ij}\delta_{kl}
\eeq

Аналогично можем получить изотропный тензор жёсткости \eqref{2}:
\beq
L_{ijkl}=\left(K-\frac{2}{3}G\right)\delta_{ij}\delta_{kl}+G\left(\delta_{ik}\delta_{jl}+\delta_{il}\delta_{jk}\right)
\eeq
и закон Гука в изотропном случае:
\beq
\sigma_{ij}^s=K\delta_{ij}\varepsilon_{kk}+2G\left(\varepsilon_{ij}-\frac{1}{3}\delta_{ij}\varepsilon_{kk}\right),
\eeq
где $K$ -- модуль всестороннего сжатия и $G$ -- модуль сдвига.

И закон Гука в изотропном случае в перегруппированной форме:
\beq
\sigma_{ij}^s=\lambda\delta_{ij}\varepsilon_{kk}+2\mu\varepsilon_{ij},
\eeq
где $\displaystyle{}\lambda=K-\frac{2}{3}G$ и $\mu=G$ -- константы (параметры) Ляме.

\section{Соотношения Онзагера}

\end{document}
