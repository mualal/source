\documentclass[main.tex]{subfiles}

\begin{document}

\textcolor{red}{Лекция 19.02.2022.}
\beq
-\sum\limits_a{r_a\left(\frac{d_a\Psi_a}{dt}+\eta_a\frac{d_a\theta}{dt}\right)}+T_{ij}^s\partial_jv_i^s+T_{ij}^f\partial_jv_i^f+b_i^{eq}W_i+b_i^{dis}W_i+\delta_\theta\geq 0
\eeq

вспоминая, что $W_i=v_i^f-v_i^s\Leftrightarrow v_i^f=W_i+v_i^s$
\beq
-\sum\limits_a{r_a\left(\frac{d_a\Psi_a}{dt}+\eta_a\frac{d_a\theta}{dt}\right)}+T_{ij}^s\partial_jv_i^s+T_{ij}^f\partial_jv_i^s+T_{ij}^f\partial_j W_i+b_i^{eq}W_i+b_i^{dis}W_i+\delta_\theta\geq 0
\eeq

вынося $\partial_jv_i^s$ за скобку, учитывая $T_{ij}^s+T_{ij}^f=T_{ij}$ и разбивая $T_{ij}^f=T_{ij}^{eq,f}+T_{ij}^{dis,f}$
\beq
-\sum\limits_a{r_a\left(\frac{d_a\Psi_a}{dt}+\eta_a\frac{d_a\theta}{dt}\right)}+T_{ij}\partial_jv_i^s+T_{ij}^{eq,f}\partial_j W_i+b_i^{eq}W_i+T_{ij}^{dis,f}\partial_j W_i+b_i^{dis}W_i+\delta_\theta\geq 0
\eeq

вводя механическую диссипацию $\delta_f=T_{ij}^{dis,f}\partial_j W_i+b_i^{dis}W_i$
\beq
-\sum\limits_a{r_a\left(\frac{d_a\Psi_a}{dt}+\eta_a\frac{d_a\theta}{dt}\right)}+T_{ij}\partial_jv_i^s+T_{ij}^{eq,f}\partial_j W_i+b_i^{eq}W_i+\delta_f+\delta_\theta\geq 0
\eeq

Напоминание: закон Дарси можно напрямую получить из сил вязкого трения (вводя силу вязкого трения по закону Ньютона, автоматически получаем закон Дарси в пористых средах)

Вспомним тензор малых деформаций:
\beq
\varepsilon_{ij}=\frac{1}{2}\left(\frac{\partial u_i}{\partial x_j}+\frac{\partial u_j}{\partial x_i}\right)\Rightarrow \dot{\varepsilon}_{ij}^s=\frac{1}{2}\left(\frac{\partial\dot{u}_i}{\partial x_j}+\frac{\partial\dot{u}_j}{\partial x_i}\right)=\frac{1}{2}\left(\partial_jv_i^s+\partial_iv_j^s\right)
\eeq

учитывая симметричность $\varepsilon_{ij}$ преобразуем неравенство к следующему виду
\beq
-\sum\limits_a{r_a\left(\frac{d_a\Psi_a}{dt}+\eta_a\frac{d_a\theta}{dt}\right)}+T_{ij}\dot{\varepsilon}_{ij}^s+T_{ij}^{eq,f}\partial_j W_i+b_i^{eq}W_i+\delta_f+\delta_\theta\geq 0
\eeq

Далее $T_{ij}^{eq,f}=-\varphi p\delta_{ij}$ и $b_i^{eq}=-p\partial_i\varphi$
\beq
T_{ij}^{eq,f}\partial_j W_i+b_i^{eq}W_i=-p\left(\varphi\delta_{ij}\partial_j W_i+W_i\partial_i\varphi\right)=-p\left(\varphi\partial_i W_i+W_i\left(\partial_i\varphi\right)\right)=-p\,\partial_i\!\left(\varphi W_i\right)
\eeq

Подставляем в неравенство:
\beq
-\sum\limits_a{r_a\left(\frac{d_a\Psi_a}{dt}+\eta_a\frac{d_a\theta}{dt}\right)}+T_{ij}\dot{\varepsilon}_{ij}^s-p\,\partial_i\!\left(\varphi W_i\right)+\delta_f+\delta_\theta\geq 0
\eeq

Вспомним ЗСМ:
\beq
\frac{\partial r_f}{\partial t}+\partial_i\left(r_f v_i^f\right)=\rho_f\frac{\partial\varphi}{\partial t}+\varphi\frac{\partial\rho_f}{\partial t}+\rho_f\partial_i\left(\varphi v_i^f\right)+\varphi v_i^f\partial_i\rho_f=0
\eeq

вспоминая $v_i^f=W_i+v_i^s$ и выделяя полную производную
\beq
\varphi\frac{d_f\rho_f}{dt}+\rho_f\left(\frac{\partial\varphi}{\partial t}+\partial_i\left(\varphi W_i\right)+\partial_i\left(\varphi v_i^s\right)\right)=0
\eeq

раскрывая производную произведения
\beq
\varphi\frac{d_f\rho_f}{dt}+\rho_f\partial_i\left(\varphi W_i\right)+\rho_f\left(\frac{\partial\varphi}{\partial t}+v_i^s\partial_i\varphi+\varphi\partial_i v_i^s\right)=0
\eeq

вспоминая, что $\displaystyle{}\partial_i v_i^s=\dot{\varepsilon}_{ii}^s=\delta_{ij}\dot{\varepsilon}_{ij}^s=\delta_{ij}\frac{\partial\varepsilon_{ij}^s}{\partial t}=\delta_{ij}\frac{d_s\varepsilon_{ij}^s}{dt}$ (частная производная равна полной, т.к. тензор МАЛЫХ деформаций), выделяя ещё одну полную производную $\left(\frac{d_s\varphi}{dt}=\frac{\partial\varphi}{\partial t}+v_i^s\partial_i\varphi\right)$ и поделив обе части равенства на $\rho_f$
\beq
\partial_i\left(\varphi W_i\right)=-\frac{\varphi}{\rho_f}\frac{d_f\rho_f}{dt}-\frac{d_s\varphi}{dt}-\varphi\delta_{ij}\dot{\varepsilon}_{ij}^s
\eeq

Видим, что $\partial_i\left(\varphi W_i\right)$ распадается на сумму трёх полных производных. И именно из этой составляющей "вылезут" все законы пороупругости.

Подставляем в неравенство:
\beq
-\sum\limits_a{r_a\left(\frac{d_a\Psi_a}{dt}+\eta_a\frac{d_a\theta}{dt}\right)}+T_{ij}\dot{\varepsilon}_{ij}^s+p\frac{\varphi}{\rho_f}\frac{d_f\rho_f}{dt}+p\frac{d_s\varphi}{dt}+p\varphi\delta_{ij}\dot{\varepsilon}_{ij}^s+\delta_f+\delta_\theta\geq 0
\eeq

Достигли успеха: есть полная производная температуры, полная производная тензора малых деформаций (можно расписать как относительную производную плотностей скелета, но тензор  деформаций более общий вид), полная производная плотности флюида, полная производная пористости.

Далее будем получать замыкающие соотношения.
\beq
-\sum\limits_a{r_a\left(\frac{d_a\Psi_a}{dt}+\eta_a\frac{d_a\theta}{dt}\right)}+\left(T_{ij}+p\varphi\delta_{ij}\right)\dot{\varepsilon}_{ij}^s+\frac{p\varphi}{\rho_f}\frac{d_f\rho_f}{dt}+p\frac{d_s\varphi}{dt}+\delta_f+\delta_\theta\geq 0
\eeq

Мы знаем, что пористость и свободная энергия должны зависеть от температуры, от давления флюида (связано "намертво" неким замыкающим соотношением с плотностью флюида) и от деформации скелета:
\beq
\Psi_a=\Psi_a\left(\theta,\varepsilon_{ij},\rho_f\right)
\eeq

\beq
\varphi=\varphi\left(\theta,\varepsilon_{ij},\rho_f\right)
\eeq

Тогда
\beq
\frac{d_f\Psi_f}{dt}=\frac{\partial\Psi_f}{\partial\theta}\frac{d_f\theta}{dt}+\frac{\partial\Psi_f}{\partial \varepsilon_{ij}}\frac{d_f\varepsilon_{ij}}{dt}+\frac{\partial\Psi_f}{\partial\rho_f}\frac{d_f\rho_f}{dt}
\eeq

\beq
\frac{d_s\Psi_s}{dt}=\frac{\partial\Psi_s}{\partial\theta}\frac{d_s\theta}{dt}+\frac{\partial\Psi_s}{\partial \varepsilon_{ij}}\frac{d_s\varepsilon_{ij}}{dt}+\frac{\partial\Psi_s}{\partial\rho_f}\frac{d_s\rho_f}{dt}
\eeq

\beq
\frac{d_s\varphi}{dt}=\frac{\partial\varphi}{\partial\theta}\frac{d_s\theta}{dt}+\frac{\partial\varphi}{\partial \varepsilon_{ij}}\frac{d_s\varepsilon_{ij}}{dt}+\frac{\partial\varphi}{\partial\rho_f}\frac{d_s\rho_f}{dt}
\eeq

Замыкающие соотношения для флюида.

\begin{itemize}
\item Предполагаем, что производная $\displaystyle{}\frac{d_f\varepsilon_{ij}}{dt}$ может быть любой, тогда для выполнения неравенства необходимо
\beq\label{FirstFluid1}
\frac{\partial\Psi_f}{\partial\varepsilon_{ij}}=0
\eeq

\item Предполагаем, что производная $\displaystyle{}\frac{d_f\theta}{dt}$ может быть любой, тогда для выполнения неравенства необходимо 
\beq\label{FirstFluid2}
-r_f\frac{\partial\Psi_f}{\partial\theta}-r_f\eta_f=0\Rightarrow\eta_f=-\frac{\partial\Psi_f}{\partial\theta}
\eeq

(получили факт из школьного курса: энтропия есть минус производная свободной энергии по температуре)

\item Предполагаем, что производная $\displaystyle{}\frac{d_f\rho_f}{dt}$ может быть любой, тогда для выполнения неравенства необходимо 
\beq\label{FirstFluid3}
-r_f\frac{\partial\Psi_f}{\partial\rho_f}+\frac{p\varphi}{\rho_f}=0\Rightarrow p=\rho_f^2\frac{\partial\Psi_f}{\partial\rho_f}
\eeq

\end{itemize}

Далее скелетные замыкающие соотношения.

\begin{itemize}
\item Предполагаем, что производная $\displaystyle{}\frac{d_s\theta}{dt}$ может быть любой, тогда для выполнения неравенства необходимо 
\beq
-r_s\frac{\partial\Psi_s}{\partial\theta}-r_s\eta_s+p\frac{\partial\varphi}{\partial\theta}=0\Rightarrow\eta_s=-\frac{\partial\Psi_s}{\partial\theta}+\frac{p}{\rho_s\varphi}\frac{\partial\varphi}{\partial\theta}
\eeq

(видим, что энтропия скелета уже зависит от пористости)

\item Предполагаем, что производная $\displaystyle{}\frac{d_s\varepsilon_{ij}}{dt}$ может быть любой, тогда для выполнения неравенства необходимо 
\beq
-r_s\frac{\partial\Psi_s}{\partial\varepsilon_{ij}}+T_{ij}+p\varphi\delta_{ij}+p\frac{\partial\varphi}{\partial\varepsilon_{ij}}=0
\eeq

(если линеаризовать последнее равенство, то получим закон Гука)

\item Предполагаем, что производная $\displaystyle{}\frac{d_s\rho_f}{dt}$ може быть любой, тогда для выполнения неравенства необходимо 
\beq
-r_s\frac{\partial\Psi_s}{\partial\rho_f}+p\frac{\partial\varphi}{\partial\rho_f}=0
\eeq
\end{itemize}

Нам не нравятся полученные замыкающие соотношения. Мы реально можем управлять температурой и давлением. А в полученных замыкающих соотношениях есть, например, плотность флюида, значениями которой напрямую мы не управляем. Нам нужно перейти к замыкающим соотношениям на давление. Но напрямую это сделать нельзя; нужно будет сделать 2 промежуточных трюка:

\textbullet\ скажем, что пористость независимая переменная, а плотность флюида - зависимая от неё термодинамическая величина (можем сделать, так как есть связь между плотностью флюида и пористостью)

\textbullet\ перейдём к другому термодинамическому потенциалу

Сделаем замену переменных: 
\beq
f\left(\theta,\varepsilon_{ij},\rho_f\right)\rightarrow \tilde{f}\left(\theta,\varepsilon_{ij},\varphi(\theta,\varepsilon_{ij},\rho_f)\right)
\eeq

Тогда
\beq
r_s\frac{\partial\Psi_s}{\partial\rho_f}=r_s\frac{\partial\tilde{\Psi}_s}{\partial\varphi}\frac{\partial\varphi}{\partial\rho_f}
\eeq

и последнее замыкающее соотношение
\beq
-r_s\frac{\partial\Psi_s}{\partial\rho_f}+p\frac{\partial\varphi}{\partial\rho_f}=0\Leftrightarrow \left(-r_s\frac{\partial\tilde{\Psi}_s}{\partial\varphi}+p\right)\frac{\partial\varphi}{\partial\rho_f}=0
\eeq

Следовательно,
\beq
p=r_s\frac{\partial\tilde{\Psi}_s}{\partial\varphi}
\eeq

С другой стороны, ранее получили
\beq
p=\rho_f^2\frac{\partial\Psi_f}{\partial\rho_f}=\rho_f^2\frac{\partial\tilde{\Psi}_f}{\partial\varphi}\frac{\partial\varphi}{\partial\rho_f}
\eeq

Таким образом,
\beq
\varphi\rho_s\frac{\partial\tilde{\Psi}_s}{\partial\varphi}=\rho_f^2\frac{\partial\tilde{\Psi}_f}{\partial\varphi}\frac{\partial\varphi}{\partial\rho_f}
\eeq
(вывели просто так, не запоминаем)

После замены переменных распишем 
\beq
r_s\frac{\partial\Psi_s}{\partial\varepsilon_{ij}}=r_s\frac{\partial\tilde{\Psi}_s}{\partial\varepsilon_{ij}}+r_s\frac{\partial\tilde{\Psi}_s}{\partial\varphi}\frac{\partial\varphi}{\partial\varepsilon_{ij}}
\eeq

(так как $\tilde{\Psi}_s$ двояко зависит от деформаций: напрямую и через пористость)

и предпоследнее замыкающее соотношение
\beq
-r_s\frac{\partial\Psi_s}{\partial\varepsilon_{ij}}+T_{ij}+p\varphi\delta_{ij}+p\frac{\partial\varphi}{\partial\varepsilon_{ij}}=0
\eeq

равносильно
\beq
-r_s\frac{\partial\tilde{\Psi}_s}{\partial\varepsilon_{ij}}-r_s\frac{\partial\tilde{\Psi}_s}{\partial\varphi}\frac{\partial\varphi}{\partial\varepsilon_{ij}}+T_{ij}+p\varphi\delta_{ij}+p\frac{\partial\varphi}{\partial\varepsilon_{ij}}=0
\eeq

вынося $\displaystyle{}\frac{\partial\varphi}{\partial\varepsilon_{ij}}$ за скобку
\beq
-r_s\frac{\partial\tilde{\Psi}_s}{\partial\varepsilon_{ij}}+\left(p-r_s\frac{\partial\tilde{\Psi}_s}{\partial\varphi}\right)\frac{\partial\varphi}{\partial\varepsilon_{ij}}+T_{ij}+p\varphi\delta_{ij}=0
\eeq

Ранее получали, что
\beq
p=r_s\frac{\partial\tilde{\Psi}_s}{\partial\varphi}
\eeq

поэтому замыкающее соотношение перепишется в виде
\beq
-r_s\frac{\partial\tilde{\Psi}_s}{\partial\varepsilon_{ij}}+T_{ij}+p\varphi\delta_{ij}=0
\eeq

Ранее получали первое скелетное замыкающее соотношение:
\beq
\eta_s=-\frac{\partial\Psi_s}{\partial\theta}+\frac{p}{\rho_s\varphi}\frac{\partial\varphi}{\partial\theta}
\eeq

В новых переменных
\beq
\frac{\partial\Psi_s}{\partial\theta}=\frac{\partial\tilde{\Psi}_s}{\partial\theta}+\frac{\partial\tilde{\Psi}_s}{\partial\varphi}\frac{\partial\varphi}{\partial\theta}
\eeq

и первое скелетное замыкающее соотношение перепишется в виде
\beq
\eta_s=-\frac{\partial\tilde{\Psi}_s}{\partial\theta}-\frac{\partial\tilde{\Psi}_s}{\partial\varphi}\frac{\partial\varphi}{\partial\theta}+\frac{p}{r_s}\frac{\partial\varphi}{\partial\theta}
\eeq

Ранее получали, что 
\beq
p=r_s\frac{\partial\tilde{\Psi}_s}{\partial\varphi}
\eeq
поэтому замыкающее соотношение перепишется в виде
\beq
\eta_s=-\frac{\partial\tilde{\Psi}_s}{\partial\theta}
\eeq

Теперь осталось сделать последний шаг. Сделать так, чтобы это всё было в нормальных переменных: избавиться от пористости и получить давление.
Для этого перейдём к потенциалу Гиббса:
\beq
G=U-TS+pV=\Psi+pV
\eeq

Рассмотрим его дифференциал:
\beq
dG=dU-TdS-SdT+pdV+Vdp
\eeq

Из первого начала термодинамики знаем, что
\beq
dU-TdS=-pdV
\eeq

поэтому
\beq
dG=-SdT+Vdp
\eeq

(таким образом, термодинамический потенциал Гиббса меняется/зависит только от температуры и давления -- то, что нам нужно; а свободная энергия зависит только от температуры и объёма: из-за зависимости от объёма у нас и появлялась пористость)

Лирическое отступление: потенциал Гиббса сохраняется при фазовых переходах.

Но сделаем не совсем потенциал Гиббса (а потенциал Гиббса с точки зрения флюида):
\beq
\Phi_s=\tilde{\Psi}_s-\frac{p\varphi}{r_s}
\eeq

Распишем дифференциал свободной энергии скелета:
\beq
d\tilde{\Psi}_s=\frac{\partial\tilde{\Psi}_s}{\partial\theta}d\theta+\frac{\partial\tilde{\Psi}_s}{\partial\varepsilon_{ij}}d\varepsilon_{ij}+\frac{\partial\tilde{\Psi}_s}{\partial\varphi}d\varphi
\eeq

учитывая полученные ранее замыкающие соотношения
\beq
d\tilde{\Psi}_s=-\eta_s d\theta+\frac{T_{ij}+p\varphi\delta_{ij}}{r_s}d\varepsilon_{ij}+\frac{p}{r_s}d\varphi
\eeq

Распишем дифференциал
\beq
-d\left(\frac{p\varphi}{r_s}\right)=-\frac{p d\varphi}{r_s}-\frac{\varphi dp}{r_s}+\frac{\varphi p}{r_s}\frac{dr_s}{r_s}
\eeq

а относительное изменение плотности есть минус относительное изменение (деформация) объёма:
\beq
\frac{dr_s}{r_s}=-\delta_{ij}d\varepsilon_{ij}
\eeq

Распишем дифференциал введённого потенциала
\beq
d\Phi_s=d\tilde{\Psi}_s-d\left(\frac{p\varphi}{r_s}\right)=-\eta_s d\theta+\frac{T_{ij}+p\varphi\delta_{ij}}{r_s}d\varepsilon_{ij}+\frac{p}{r_s}d\varphi-\frac{p d\varphi}{r_s}-\frac{\varphi dp}{r_s}-\frac{\varphi p}{r_s}\delta_{ij}d\varepsilon_{ij}
\eeq

приводя подобные
\beq
d\Phi_s=-\eta_s d\theta+\frac{T_{ij}}{r_s}d\varepsilon_{ij}-\frac{\varphi}{r_s}dp
\eeq

(видим, что потенциал $\Phi$ зависит от температуры, деформации и давления)

Тогда получаем следующие замыкающие соотношения:

\begin{itemize}
\item на пористость
\beq
\varphi=-r_s\frac{\partial\Phi_s}{\partial p}
\eeq

\item на энтропию скелета
\beq
\eta_s=-\frac{\partial\Phi_s}{\partial\theta}
\eeq

\item на напряжение
\beq
T_{ij}=r_s\frac{\partial\Phi_s}{\partial\varepsilon_{ij}}
\eeq
\end{itemize}

Ещё в подобной форме необходимо замыкающее соотношение на энтропию флюида и замыкающее соотношение, которое покажет связь между давлением и плотностью флюида.
Введём потенциал для флюида:
\beq
\Phi_f=\Psi_f+\frac{p}{\rho_f}
\eeq

Запишем его дифференциал:
\beq\label{FullDif1}
d\Phi_f=d\Psi_f+\frac{dp}{\rho_f}-p\frac{d\rho_f}{\rho_f}\frac{1}{\rho_f}
\eeq

Перепишем первое слагаемое
\beq
d\Psi_f=\frac{\partial\Psi_f}{\partial\theta}d\theta+\frac{\partial\Psi_f}{\partial\varepsilon_{ij}}d\varepsilon_{ij}+\frac{\partial\Psi_f}{\partial\rho_f}d\rho_f
\eeq

учитывая ранее полученные замыкающие соотношения \eqref{FirstFluid1}, \eqref{FirstFluid2} и \eqref{FirstFluid3} для флюида получаем
\beq
d\Psi_f=-\eta_f d\theta+\frac{p}{\rho_f^2}d\rho_f
\eeq

подставляем в выражение для дифференциала \eqref{FullDif1}:
\beq
d\Phi_f=-\eta_f d\theta+\frac{p}{\rho_f^2}d\rho_f+\frac{dp}{\rho_f}-p\frac{d\rho_f}{\rho_f}\frac{1}{\rho_f}
\eeq

приводя подобные
\beq
d\Phi_f=-\eta_f d\theta+\frac{dp}{\rho_f}
\eeq

Тогда получаем следующие замыкающие соотношения:

\begin{itemize}
\item на энтропию флюида
\beq
\eta_f=-\frac{\partial\Phi_f}{\partial\theta}
\eeq

\item на плотность флюида
\beq
\frac{1}{\rho_f}=\frac{\partial\Phi_f}{\partial p}
\eeq
\end{itemize}

\end{document}

