\documentclass[main.tex]{subfiles}

\begin{document}

\section{Лекция 24.10.2022 (Брусиловский А.И.)}

\begin{center}
\includegraphics[width=\textwidth, page=51]{Брусиловский.pdf}
\end{center}

Дальше у нас важная тема промысловых и лабораторных исследований пластовых флюидов.

\begin{center}
\includegraphics[width=\textwidth, page=52]{Брусиловский.pdf}
\end{center}

Вот анализ исследований пластовых флюидов, иначе называется PVT-анализ, и физико-химические исследования пластовых флюидов, это ваш курс, включает в себя изучение и поверхностных, и пластовых флюидов, и исследования физико-химических свойств.
Аналог в практике это PVT-исследования, но физико-химические исследования даже скорее чаще говорят о поверхностных пробах, а вот когда говорят PVT-анализ, то это однозначно совершенно говорит о термодинамических исследованиях для пластовых флюидов.
И мы с вами вот видим перечисление важнейших свойств для нефтяной системы, для газоконтенсантной системы, для сухого газа, для жирного газа, для всех типов пластовых флюидов.
Для нефтяной системы давление насыщения, объемный коэффициент, газосодержание, относительная плотность нефти, относительная плотность газа, динамическая вязкость нефти, компонентный состав.
От компонентного состава, собственно говоря, зависит, определяется компонентным составом, как я говорил, и термобарическими условиями все свойства.
Теперь газокондензантная система тоже, здесь уже компонентный состав на первом месте, ввиду его чрезвычайной важности.
Для нефтяной системы компонентный состав тоже можно поставить на первое место, потому что им определяется всё, и давление насыщения, и объемный коэффициент, и газосодержание, и так далее, динамическая вязкость.
Значит, я даже себе, пожалуй, отмечу, чтобы переставить.
Значит, это слайд 52, нефтяная система -- компонентный состав.
Все, пошли дальше.
Это я просто пытаюсь, чтобы восприятие было легче и более логичным.
Но нет предела совершенству.
Значит, теперь по сухому газу и по жирному газу, где мы не имеем фазовых превращений, в пластовых условиях, в общем, везде компонентный состав, им определяется все.
Что это система какая -- сухой газ, жирный газ, все определяется компонентным составом.
Я не устаю повторять это, потому что, в общем, не все раньше правильно понимали.
И, кстати говоря, вот нефтяники, значит, они так немножко пренебрежительно к технологии газовых месторождений относились.
Это раньше было, до того, как столкнулись нефтяники с колоссальными проблемами при разработке двухфазных залежей и при повышении эффективности разработки газоконденсатных объектов, которые сейчас, в том числе Газпром, передал, Газпром Нефти.
Но это не только, и в Роснефти тоже газовая, газоконденсатная, и уже стали понимать, нефтяники, насколько непроста технология разработки газовых месторождений, свои проблемы.
Давления высокие, отрицательные температуры, гидратообразование, вопросы планирования добычи конденсата и так далее.
Это все сложный вопрос.

\begin{center}
\includegraphics[width=\textwidth, page=53]{Брусиловский.pdf}
\end{center}

Итак, начнем мы с исследований пластовых нефтей.
Мы с вами говорили, что мы комплексно рассматриваем исследования нефтей промысловые, термодинамические, и затем математическое моделирование.
Я очень кратко обзор сделаю, буквально введение, о современных подходах к математическому моделированию природных углеводородных систем.
Так вот, что касается исследований пластовой нефти, промысловые исследования пластовой нефти, отбирают глубинные пробы, а если невозможно отобрать глубинные пробы, представительные глубинные пробы, тогда отбирают пробы из сепаратора и их рекомбинируют в соответствии с измеренным газовым фактором.
Но по отраслевым стандартам у нас первичным является отбор представительных глубинных проб нефти.
Если можно такие пробы отобрать, то это имеет несомненный приоритет перед рекомбинацией сепараторных проб.

\begin{center}
\includegraphics[width=\textwidth, page=54]{Брусиловский.pdf}
\end{center}

Какое необходимое условие для достоверности параметров?
Это равномерный охват (охарактеризованность) пластов кондиционными пробами пластовой нефти.
В последнее время я почему тут кондиционными подчеркнуты, потому что обычно говорили представительные и репрезентативные.
На самом деле, в зарубежных источниках, в стандартах SPE, там говорится о репрезентативных пробах, то есть представительных.
Но у нас стали кондиционные пробы термин употреблять.
В общем, это аналог.
Для меня это аналог, совершенно равнозначный.
Может быть, те, кто говорит про кондиционные, имеют в виду какие-то особенности, я их не вижу.
Это реальный рисунок, схема одного из пластов крупного месторождения, где достаточно хороший охват по отбору глубинных проб.
Вы видите, что не идеальный охват, идеального и не бывает на практике.
Но достаточно широкий, достаточно равномерный охват.
Очень часто, кстати, в силу объективных причин, не удается этот равномерный охват на разведочном этапе осуществить, но просто в силу больших сложностей исследований в условиях Крайнего Севера, поэтому тут не может быть каких-то негативного отношения.
Но только можно посочувствовать, насколько сложные условия для разведчиков недр.
А исследуем мы, прежде всего, разведочные скважины.
То есть на практически всех разведочных скважинах мы должны отбирать глубинные пробы для того, чтобы понять, что за пластовая нефть в данной залежи, в данном пласте.

\begin{center}
\includegraphics[width=\textwidth, page=55]{Брусиловский.pdf}
\end{center}

Какие ключевые проблемы?
Значит, вот схема (слева) того, что происходит с пластовой нефтью.
Вот она притекла на забой, ствол скважины на забой, стрелочкой, через интервалы перфорации.
И у этой нефти давление насыщения, оно близко к забойному давлению.
То есть, забойное давление, оно ниже, вернее, выше.
В общем, давление пластовое, давление притекающего флюида, пластовой нефти, давление притекающего флюида, оно выше, чем давление насыщения.
То есть, в однофазном состоянии флюид у нас попал в ствол скважины.
А дальше, дальше, значит, мы видим, достаточно быстро начинается разгазирование.
Сначала отдельные пузырьки газа, потом их всё больше и больше, а потом этот газ, если выше, вот эти пузырьки, они смыкаются в глобулы и меняется режим течения смеси в скважине, и всё это усложняет расчёты термо-гидро-динамические течения газо-жидкостной смеси в скважине.
Ну, это, значит, нас интересует в данном случае то, что пробоотборник находится уже в зоне двухфазного газожидкостного течения.
И вот на той глубине, где указан пробоотборник, мы отобрать глубинную пробу не можем, не можем.
Нам нужно спускать пробоотборник ниже, значит, нам нужно снижать давление, забойное, для того, чтобы в зоне пробоотборника было однофазное течение нашей пластовой смеси, которая поступит в этот пробоотборник.
Представительную пробу можно получить только при однофазном течении нашей пластовой нефти в зоне отбора.
Это совершенно обязательное условие, еще раз, если мы не уверены в однофазности течения, бессмысленно, просто не надо отбирать пробу, не надо отбирать пробу глубинную, потому что мы обязаны это делать только в области однофазного состояния.
Значит, давление у нас в стволе скважины при этом меняется, меняется достаточно.
Вы видите, вот по той зависимости, которая нарисована.
И мы в случае, значит, вот когда у нас в случае, давайте вспомним PV-зависимости, PV-зависимости, которые мы, значит, вот смотрели на прошлой лекции, в начале нашей презентации.
Так вот, для нефти с повышенным газосодержанием у нас заметного излома давления при переходе из гомогенного состояния в двухфазное не происходит заметно в данном случае.
В данном случае, то есть это достаточно высокое газосодержание.
И мы, если бы у нас было низкое газосодержание, ну или средненькое, там до 100 кубометров на кубометр, мы бы профиль давления имели четкий излом, четкий излом по глубине и в зоне излома, значит, это вот соответствует началу разгазирования в стволе нашей скважины.
И мы должны отбирать пробу ниже, ниже, чем та отметка, где начинается разгазирование.
Но мы видим четкого излома нет, значит, в этом случае, вот сейчас, чтобы понять в какой зоне однофазного течения или гетерогенного течения в современных, значит, пробоотборниках используют оптические анализаторы.
И это совершенно стало большим прогрессом, потому что до этого понять при повышенном газосодержании было непонятно, ну непонятно было.
Поэтому, значит, вот согласно отраслевому стандарту есть требование как можно ниже спускать пробоотборник, ну в зону над перфорацией, над интервалом перфорации, чтобы по крайней мере, если в скважину нефть притекает в однофазном состоянии, мы могли отобрать пробу нефти в однофазном состоянии, максимально близко к интервалу перфорации.

Теперь какие ключевые проблемы?
Забор пробы выше допустимой глубины.
Вот как раз тот случай, который здесь указан.
У нас пробоотборник находится уже в зоне двухфазного течения, и мы не можем получить представительную пробу.
Дальше вторая проблема.
Если мы, и с ней приходилось сталкиваться неоднократно, мы отбираем представительную пробу, кондиционную, но из-за недостатков пробоотборника, ну из-за его негерметичности, в процессе отбора пробоотборника осуществляется разгазирование, температура падает, давление падает, по мере подъема пробоотборника температура.
Внутри пробоотборника выделяется газ, давление спускается ниже, чем давление насыщения, и из-за его негерметичности этот газ начинает уходить из пробоотборника, и когда мы уже достаем этот пробоотборник, газосодержание нефти в нем, в итоге оказывается, газосодержание той смеси, которая находится в пробоотборнике, после того как мы ее приведем к пластовым условиям, газосодержание этой смеси будет меньше, чем в пластовой нефти, из-за негерметичности.

Хотел это сказать, что вот данная проблема, она сейчас уже решается путем использования пробоотборников с принудительным закрытием, ну то есть, чтобы они закрывались на забое и впредь не открывались.
Я это знаю, я просто говорю о проблемах, ключевые проблемы.
Она решается, но она не решена повсеместно.
То есть, смотрите, вы говорите о том, она решается.
Я согласен.
Это отдельные решения, например, в нашей компании, да?
Но это большая проблема в целом для индустрии.
И это, поскольку вы эстафету молодое поколение у нас берете, конечно же, вы будете уже, так сказать, совсем на другом технологическом уровне работать, да?
И слава богу, что она решается.
Но она есть, она есть.
И очень...
Я за последние 20 лет, 10 лет я активно занимался анализом исследований пластовых нефтей.
И то, что творила, например, очень известная лаборатория геоэкологии в Тюмени, это Тюменская центральная лаборатория вообще, бывшая.
Это же уму непостижимо, потому что там проблемы с негерметичностью приемной камеры, там проблемы с отбором проб не в однофазном состоянии приводили к тому, что и подсчет запасов неправильно осуществлялся, и проектирование разработки.
Вот следующий слайд будет.
Поэтому я рассказываю о ключевых проблемах, и вы говорите, что одна из ключевых проблем, она сейчас решается.
Она еще не решена до конца, но она решается.
Конечно, она должна решаться.
Мало того, и сами пробоотборники, конечно, совершенствуются.
Конечно.
Спасибо.
И отбор проб из неподготовленной скважины.
То есть суть такова, у нас скважина работает с такой депрессией, что у нас в призабойной зоне выделяется газ.
Выделяется газ.
И система попадает в скважину уже в двухфазном состоянии.
Так вот, прежде чем... А мы не можем отобрать пробу при таком ее двухфазном течении.
Поэтому еще в отраслевом стандарте первом, это 1980 год, указывалось о том, что скважину нужно готовить к отбору проб.
Каким образом?
Значит, нужно добиться того, чтобы нефть притекала в однофазном состоянии, а весь газ, который был в призабойной зоне, выделился из нефти, чтобы он весь ушел.
Поэтому в отраслевом стандарте первом 1980 года, и он был повторен в 2003 году, там говорится о том, что нужно уменьшать депрессию в скважине при подготовке к отбору глубинных проб.
И делается оценка по времени, сколько нужно времени для того, чтобы выделившийся до того в призабойной зоне газ, он весь ушел из призабойной зоны.
И мы имели в стволе скважины, в зоне отбора, в зоне перфорации и чуть выше, однофазное состояние пластового флюида.
Тогда мы можем отобрать пробу в однофазном состоянии.
Вот это вот ключевые такие моменты.
Вот мне просто интересно, очень правильное вы замечание, очень хорошо, что это все делается.
Закрытие и так далее.
А теперь вот, если вы имеете к этому отношение, те пробоотборники, которыми, у нас же масса месторождений, компаний, это не одно, не два, это десятки месторождений и пластов.
И мы привлекаем к отбору проб разные компаний.
Так вот, какие пробоотборники у них?
И привлекают ли к отбору тоже Геоэкологию или подобные компании, которых категорически нельзя привлекать к исследованиям месторождений с повышенным газосодержанием?
Ну вот, я вам рассказываю из своего опыта, а то, что разрабатывается, очень хорошо.

\begin{center}
\includegraphics[width=\textwidth, page=56]{Брусиловский.pdf}
\end{center}

Требования к идеальному пробоотборнику, то, что надежная герметичность, я так понимаю, именно об этом речь идет, то, что сейчас делают, замечательно.
Обеспечение еще заданной глубины отбора проб.
Следующий слайд даст вам больше представления.
Фиксирование термобарических условий отбора проба, простота и надежность в эксплуатации, безопасность в эксплуатации.

\begin{center}
\includegraphics[width=\textwidth, page=57]{Брусиловский.pdf}
\end{center}

Вот сравнение того, что было и то, что удалось добиться 20 лет назад.
А сейчас-то, конечно, еще больше прогресс.
Это очень широко распространенный пробоотборник ВПП-300, разработан в 1968-м году, и предельное давление было 300, предельная температура 100 градусов Цельсия, и почти полное отсутствие всякой аппаратуры, позволяющей контролировать и термобарические условия по стволу скважины, и фиксирование момента открытия камеры, и фиксирование глубины спуска, и контроль герметичности камеры вообще отсутствовало.
И до сих пор эти пробоотборники используются, и в частности той фирмой, которую я упомянул.
Вот просто ради любопытства можно поинтересоваться, особенно на старых месторождениях, где берутся контрольные пробы, по крайней мере несколько лет назад ВПП-300 еще использовали, и индукционный магнитный серийный пробоотборник 22 (ИМСП-22).
Значит, это 2000 года разработка, и мы видим уже предельное давление до 600 бар, температура выше, возможно есть электронный таймер, встроенный термометр-манометр, электромагнитный датчик, магнитный локатор муфт, электромагнитный датчик тоже для контроля герметичности камеры.
Но вот это совсем другое дело.
И то, и то, значит, все это с большими трудностями внедрялось, и в итоге... Это отечественный.
А существует много импортных пробоотборников, которые используются в разных компаниях, в том числе в нашей.
Ну, вот это есть курс по отбору, хороший курс, который наши специалисты из Тюмени подготовили, и там уже приводятся самые различные пробоотборники, и там, кстати говоря, то, о чем вы говорите, наверняка тоже говорится о достижениях современных.
А это вот просто для того, чтобы показать, что есть прогресс и прогресс именно в наших разработках, отечественных.

\begin{center}
\includegraphics[width=\textwidth, page=58]{Брусиловский.pdf}
\end{center}

Ну, и то, что используется для глубокопогруженных залежей с низкопроницаемыми коллекторами, это модульный динамический пластоиспытатель MDT компании Schlumberger.
Там есть и оптические анализаторы, они разрабатывали этот динамический пластоиспытатель очень долгое время.
Проба отбирается, флюиды, с сохранением пластовых условий, то есть если в обычных пробоотборниках наших, там по мере подъема пробоотборника у нас уменьшается температура и проба переходит в однофазное состояние, то при использовании пластоиспытателей у нас осуществляется поджатие, после отбора пробы осуществляется поджатие, значит давление резко возрастает, у жидкости же она быстро, при изменении объема у жидкости, в следствии ее малой сжимаемости, давайте вспоминать PV-соотношение, значит у нас давление резко растет, и вот это вот давление выше пластового, оно сохраняется в пробоотборнике и поступает в итоге, в итоге вот это проба поступает в лабораторию и там уже выгружается в однофазном состоянии.
Это вот большой прогресс, большой прогресс в области, по сравнению с тем, что было там 15-20 лет назад, но это такие пластоиспытатели MDT, это разработки Schlumberger, нужно иметь в виду, что стоимость вот всех процедур, связанных с отбором проб, она многократно выше, чем при использовании обычных пробоотборников.
И тут понятно, что если мы имеем дело с низкопроницаемыми коллекторами, сложными условиями и пластовыми флюидами с повышенным газосодержанием, то есть когда у нас мы не можем отобрать, мы не можем отобрать представительные пробы с использованием обычных пробоотборников, вот тогда, конечно же, несмотря на высокую стоимость, мы должны привлекать и модульные динамические пластоиспытатели компании Schlumberger и другие фирмы тоже, но они отстают от Schlumberger в этом плане, потому что получение представительных проб, то есть получение правильной информации о пластовом флюиде -- это условие абсолютно необходимое для последующего и подсчета запасов правильного, и грамотного проектирования разработки, абсолютно это все нужно.
Но я хочу сказать тем, кто будет на практике заниматься, этим занимаются геологические отделы соответствующих НГДУ и деньги, значит, у главного геолога и нужно, чтобы совершенно ясно было обоснование применения вот этих современных пробоотборников, пластоиспытателей, потому что денег часто не хватает и нужно очень хорошо обосновывать, это вот я из своей практики, понимаете, из теории совершенно ясно, что нужно лучшее использовать, но это лучшее стоит деньги.
А если мы с зарубежными фирмами имеем дело, то это стоит очень большие деньги, и поэтому геологи, специалисты должны иметь четкую информацию, четкое объяснение, вот как раз задача специалистов в НТЦ, я этим занимался достаточно много, чтобы обосновывать и разъяснять особенности свойств пластовых флюидов, когда нужны и какие последствия могут быть в случае неправильного понимания PVT-свойств и пластовых нефтей, и это же относится к газоконденсатным системам, но газоконденсатными системами мы начали заниматься сравнительно недавно, а вот проблема обоснования отбора проб пластовых нефтей, она давно стоит, и она будет все острее и острее.
Так что это очень важная проблема, и здорово, что у нас этим занимаются серьезно.

\begin{center}
\includegraphics[width=\textwidth, page=59]{Брусиловский.pdf}
\end{center}

Вот технические характеристики прибора MDT, вы видите, максимальные температуры высокие, очень высокое максимальное давление, то есть все на высоком уровне.
И отбор пластовых флюидов с контролем качества отбираемых флюидов, имеется в виду контроль качества, потому что отбор с применением MDT, он обычно на разведочных скважинах, либо на эксплуатационных, только вышедших из бурения, и нужно, вот в чем еще особенность MDT, этому Schlumberger, я знаю, большое внимание уделяло, это как нам все-таки понять, как нам понять, что мы отбираем пластовый флюид, а не отбираем этот флюид в смеси с буровым раствором.
Такие проблемы были, вот они были решены в результате комплексного использования специалистов разных направлений, и физхимики, и механика, и так далее, имеется в виду механика грунтов, и специалисты по флюидам, специалисты по автоматике и прочее.
То есть, большие команды, большие команды классных специалистов решали эту проблему, и в итоге вот был создан MDT.
И то, даже с применением MDT мы далеко не всегда можем решить проблему отбора представительных проб пластовых флюидов.
Почему?
Потому что если мы имеем дело, например, с двухфазными залежами, то у нас, несмотря на небольшую депрессию, минимально возможную депрессию, а для низкопроницаемых коллекторов она все равно может составлять и 10, и 20, и даже больше бар.
То есть, тогда у нас может быть двухфазный приток, то есть, то о чем мы... и не только, потому что у нас в двухфазных залежах, вблизи газонефтяного контакта, у нас нефть, значит, она предельно насыщена практически, газонефтяного контакта.
И даже если мы создаем небольшую депрессию, все равно уже начинается разгазирование, и мы получаем в пробоотборнике можем получить уже в неправильном соответствии...
Значит, вот...
Ну, то есть, когда есть свободная газовая фаза, то мы представительную пробу пластовой нефти уже по сути получить нам не удается.
Неконтролируемое соотношение газовой фазы и нефтяной фазы.
Ну, а если еще вот вблизи газонефтяного контакта, если есть прорыв из газовой шапки, ну, это вот совсем, совсем плохо.
И вот можно вспомнить пример, который мы час назад посмотрели, когда из-за прорыва газовой шапки мы получили...
Ну, не можем получить представительную пробу и правильно ее представить.
Вот, но тем не менее, значит, вот типа MDT пластовые испытатели -- это, конечно, большой прогресс, большой прогресс.
Ну, здесь, слава Богу, что работают квалифицированные люди, в том числе и в НТЦ нашей компании "<Газпромнефти">, и в частности в Тюмени.

\begin{center}
\includegraphics[width=\textwidth, page=60]{Брусиловский.pdf}
\end{center}



\begin{center}
\includegraphics[width=\textwidth, page=61]{Брусиловский.pdf}
\end{center}

Теперь дальше, ну, так сказать, записываются условия пробоотбора, я должен сказать, что и здесь, и здесь уж что-что, ну, это-то нужно соблюдать, зафиксировать.
И давление в точке отбора, и температуру в точке отбора, и депрессию на пласт, и все, что здесь написано.
Да не делается это на практике, далеко...
Ну, в общем, далеко не всегда это делается.
Это просто, ну, поскольку кто-то из вас придет работать в нефтяные компании и поедет на промысел, значит, набираться опыта и будет там расти, пожалуйста, соблюдайте регламент и следите за тем, чтобы ваши подчиненные соблюдали регламенты.
Потому что мне пришлось столкнуться с ...
Когда, значит, вот, ведь смотрели результаты исследований 80-х годов, особенно 90-х годов, когда у нас, Бог знает, что творилось.
Так вот, там непонятно, какие же термобарические условия в месте отбора пробы, какая депрессия на пласт.
Значит, подготавливалась ли скважина, но ничего, такой информации не было.
Поэтому не удивительно, что, значит, много-много всякой неправильной информации в итоге приходилось, ну, так сказать, несмотря на достаточно высокую квалификацию интерпретатору, все равно в качестве исходной информации она играет определяющую роль.
И она часто бывает просто плохой.
Сейчас на новых месторождениях с новой техникой, технологией, новым пониманием в случае ответственности высокой можно получить достаточно хорошую информацию.
А то, что касается исследований, я повторяю, 20-30 летней давности, все было очень печально зачастую.

\begin{center}
\includegraphics[width=\textwidth, page=62]{Брусиловский.pdf}
\end{center}

Теперь, каковы же наиболее частые следствия использования непредставительных глубинных проб пластовой нефти?
Это занижение ее газосодержания.
Ну, вот в частности, значит, из-за негерметичности пробоотборников.
А если мы занижаем газосодержание, то мы занижаем объемный коэффициент.
А в подсчете запасов используется величина, обратная объемному коэффициенту, это пересчетный коэффициент.
Для тех, кто с этим не сталкивался, многие, конечно же, в дальнейшем все будет показано, но предварительно просто обратите внимание, что если у нас непредставительные глубинные пробы, то занижается газосодержание, занижается объемный коэффициент.
В результате мы завышаем геологические запасы нефти, занижаем геологические запасы растворенного газа и занижаем КИН по фактическим данным добычи.
Потому что если мы завышаем геологические запасы нефти, а коэффициент излечения нефти по фактическим данным добычи, значит, это что такое?
КИН -- отношение добытой нефти к ее геологическим запасам.
Если мы завысили геологические запасы из-за того, что мы занизили газосодержание, значит, соответственно, текущий КИН мы занижаем.
Недопустимые ошибки при оценке PVT-свойств в технологических расчетах и проектировании системы добычи.
Это то, о чем я уже говорил, вы это уже не раз слышали.
И приходится сталкиваться, приходилось и сейчас приходится сталкиваться с проблемой, когда оказывается, что через несколько лет, через какое-то время после подсчета запасов, когда происходит пересчет запасов, или же катастрофически не хватает газа.
То есть газа в подсчете запасов меньше было записано, ну подсчитано, чем по результатам фактической добычи.
И, значит, вот это та проблема, это неправильные исходные первичные данные.
Занижение газосодержания, занижение геологических запасов растворенного газа.
А потом вот приходится волевым методом завышать газосодержание.
При этом те, кто не понимает физику флюидов, зачастую говорят, что завышение, значит, вот там в геологическом отделе для подсчета, для того, чтобы в ГКЗ приняли пересчет запасов и так далее, там вот волевым образом завышают начальное газосодержание, но при этом же автоматически снижаются начальные запасы нефти, а это не рассматривается; в отрыве, а так не бывает.
Если у вас газосодержание нефти выше, у вас объёмный коэффициент становится меньше.
Пардон, значит, если у вас завышается, то есть повышаете вы газосодержание, повышаете газосодержание, у вас автоматически повышается величина объёмного коэффициента, а пересчётный коэффициент становится меньше.
Пересчётный коэффициент повторяю это обратная величина объёмного коэффициента.
Вот формулу я потом приведу, и, значит, запасы начальные нефти меньше становятся, а на это закрывают глаза.
Ну, вот это вот из-за... в общем много всякой ерунды творится, а вы должны просто физику понимать.
Вот для этого и я стараюсь вам рассказывать о PVT-свойствах так, как их нужно понимать, потому что зачастую в учебниках недостаточно, конечно же.
Это всё объясняется.

\begin{center}
\includegraphics[width=\textwidth, page=63]{Брусиловский.pdf}
\end{center}

Теперь, ну, требования к отбору проб.
А теперь, значит, смотрите, это мы про глубинные пробы говорили, да?
А в стандарте говорится, что в случае невозможности отбора глубинных проб, представительных глубинных проб, отбираются, значит, пробы на сепараторе, пробы газа и пробы нефти на сепараторе.
И они впоследствии в лаборатории рекомбинируются, то есть воссоединяются в соответствии с измеренным газовым фактором.
А газовый фактор равен отношению дебита газа к дебиту, значит, нефти в условиях сепарации.
И вот сепаратор при этом, вот требования к нему достаточно серьёзные должны, значит, они тут описаны.
То есть, нужно все эти требования соблюдать.

\begin{center}
\includegraphics[width=\textwidth, page=64]{Брусиловский.pdf}
\end{center}

Теперь, значит, у нас с вами ещё есть сегодня чуть-чуть времени, и мы рассмотрим процедуру принятия решения для оценки PVT-свойств в случае, когда у нас много уже проведено исследований, они разные, у нас очень широкий разброс.

\begin{center}
\includegraphics[width=\textwidth, page=65]{Брусиловский.pdf}
\end{center}

У нас очень широкий разброс, вы видите, значит, отобранные пробы были исследованы в лаборатории, и широкий диапазон изменения газосодержания и давления насыщения, которое зависит от газосодержания.
Значит, вот в результате, когда мы учитывали практически, ну, не отбраковывали пробу, не проводили анализ, у нас с вами оценка давления насыщения 10.8 мегапаскалей.
Невысокая, да и само газосодержание, оно тут невысокое, но достаточно большой разброс по газосодержанию отобранных проб.
Теперь вот что интересно, опять по терминологии, значит, я говорю газосодержание, но дело в том, что газосодержание используют, это терминология применяемая, по данным лабораторных исследований проб, то, что измеряется на промысле, это газовый фактор.
Поэтому, если строго, мы же с вами оцениваем давление насыщения по результатам лабораторных исследований, поэтому более правильно, я это себе тоже сейчас запишу, если смогу, то поправлю, на оси абсцисс у нас не ГФ (газовый фактор), а более правильно это газосодержание.
Ну, это я вам так, в качестве замечания, в качестве замечания, такой комментарий, в качестве комментария я вам рассказываю, терминология имеет значение.
Давление насыщения, повторяю, оно оценивается по результатам лабораторных исследований.
И только предварительно, может быть, по данным первых разведочных скважин, пока лабораторные исследования не проведены, можно, можно, да, можно, по данным газового фактора оценивать с помощью корреляций, о которых я тоже затем, впоследствии скажу.
Когда у нас ещё нет результатов лабораторных исследований представительных проб, но есть измерение газового фактора на скважине, значит, в самом начале, когда у нас давление ещё не упало ниже давления насыщения, то газовый фактор, он соответствует начальному газосодержанию.
И тут есть такая, значит, вот, чему можно следовать, что если измерять величину газового фактора с начала эксплуатации скважины, то можно что заметить?
Значит, у нас этот газовый фактор, он имеет постоянное значение, у нас давление насыщения ещё не упало, давление не упало в пласте ниже давления насыщения, газовый фактор должен быть постоянным.
А затем, вдруг он чуть, ну, в общем, немного падает, этот газовый фактор, это сигнал того, что мы, наше пластовое давление достигло давления насыщения.
И очень быстро, то есть это короткий период падения величины газового фактора, через которое короткое время, у нас он начнет сильно возрастать.
Почему он падает?
Потому что у нас вот давление насыщения мы достигли, значит, начался газ выделяться из пластовой нефти, но, и это зависит от величины газосодержания нефти, в начале газовая фаза не фильтруется.
Как любая фаза, для неё существует критическая насыщенность.
Это величина, ниже которой данная фаза не течёт в пласте.
Значит, для газа это величина меньше, чем для жидкости, чем для нефти.
Но, тем не менее, это не ноль.
Значит, поначалу, когда газ выделился из нефти, он не фильтруется, пока его доля в объёме пор, то есть насыщенность не достигнет так называемой критической насыщенности.
И затем, как только достигнет, очень быстро газовые фазы начинают фильтроваться к забоям, и газовый фактор начинает быстро возрастать.
Это, значит, зависит, вот эта вот критическая насыщенность, она зависит от характеристик пористой среды.
И её нужно определять в лаборатории для пористой среды данного месторождения, пласта.
Ну, повторяю, обычно насыщенность газовой фазой возрастает достаточно быстро, и поэтому газовый фактор...
Чем выше газосодержание нефти, тем меньше этот период.
Его можно вообще пропустить.
Ну, вот это очень полезная зависимость.
Если следить за газовым фактором, мониторить его постоянно, тогда можно чётко определить время достижения пластовым давлением давления насыщения.

Теперь ясно, что при таком разбросе, который показан на рисунке, нельзя точно определить давление насыщения.
Значит, нужно удалять разгазированные пробы.
Ну, вот все нижние, все то, что внизу, всё это, конечно, удаляется.
Это уже происходит...
Это уже совершенно не соответствует пластовым пробам.

\begin{center}
\includegraphics[width=\textwidth, page=66]{Брусиловский.pdf}
\end{center}

Значит, вот после удаления смотрят на величину плотности нефти, дегазированной нефти.
И нефти с ошибочной плотностью, с повышенной плотностью или с заниженной плотностью нужно удалить.

\begin{center}
\includegraphics[width=\textwidth, page=67]{Брусиловский.pdf}
\end{center}

Вот затем нужно рассмотреть свойства для каждой пробы.
И тут показан в углу 5\% по критерию Стьюдента, по статистике.
Вот давайте внизу посмотрим.
Значит, у нас внизу есть отрезок для всех оставшихся проб.
Мы видим, что одна из проб явно совершенно в стороне от общего массива.
Значит, она удаляется, но удаляется и две пробы тоже просто визуально.
Явно совершенно давление насыщения двух проб в районе 8 мегапаскалей, они, так сказать, выпадают из общего диапазона.

\begin{center}
\includegraphics[width=\textwidth, page=68]{Брусиловский.pdf}
\end{center}

Вот в результате анализа и по плотностям, и по другим критериям.
Значит, вот если ошибка превышает 5\% из общего массива, значит, их удалили, эти пробы.
Это нужно смотреть просто статистику, теорию статистики и вот эти критерии Стьюдента.
Ну, мало это кто делает.
Но теоретически, да, теоретически так правильно.
Но ни в отраслевых стандартах об этом не сказано, ни в курсах обычно не говорится.
Но это вот теоретически, да, нужно поступать так.
В результате из общего массива остаётся гораздо меньше проб, на которые нужно ориентироваться.
И после осреднения давление насыщения оставшихся проб, давление насыщения оценивается величиной 12.4 мегапаскалей, то есть выше значительно, чем первоначально.
А чем это чревато?
Мы же ориентируемся на давление насыщения, что мы можем, это та величина, ниже которой нежелательно спускать пластовое давление, за исключением случаев, когда низкое газосодержание нефти.
Ну, вот в данном случае, кстати, можно при разработке чуть-чуть ниже 12.4 снижать давление, потому что выделившийся газ, он фильтроваться не будет из-за его низкой насыщенности, и это можно допустить.
Но это надо специально исследовать.
В данном случае оценили давление насыщения 12.4 мегапаскалей.
Значит, значительно выше, чем до этого.
То есть была неправильная оценка.
Значит, ориентировались, что можно снижать давление существенно ниже, чем 12.4, и было бы уже приличное разгазирование и всякие проблемы.

Вот на этом я на сегодня закончу лекцию, и, значит, завтра мы начнем с исследований пластовой нефти, в которых определяются базовые параметры.
Спасибо за внимание!

\end{document}
