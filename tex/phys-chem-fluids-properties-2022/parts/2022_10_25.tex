\documentclass[main.tex]{subfiles}

\begin{document}

\section{Лекция 25.10.2022 (Брусиловский А.И.)}

\subsection{Исследования пластовой нефти, в которых определяются базовые параметры}

\begin{center}
\includegraphics[width=\textwidth, page=69]{Брусиловский.pdf}
\end{center}

Итак, давайте начнем наш учебный день.
Значит, очень важная тема.
Темы все важные, в том числе и та, которую сейчас мы будем рассматривать.
Это исследование пластовой нефти.
Исследования проводятся в лабораториях, и можно проводить математическое моделирование, создав модель пластовой нефти многокомпонентной, и с
применением уравнения моделировать те процессы, которые осуществляются при лабораторных исследованиях.
При этом, зная результаты лабораторных исследований, можно адаптировать математическую модель под результаты экспериментальных исследований.
Это очень хорошая будет модель для дальнейшего использования при проектировании, разработке, эксплуатации, расчете различных процессов, течения в скважине, на промысле, с сепарацией и так далее.
И в том числе при композиционном моделировании.
И при моделировании процессов с помощью моделей типа Black Oil.

Итак, разделяют различные виды разгазирования.
Контактное разгазирование -- это процесс, когда выделившийся газ не удаляется, а находится в равновесии с жидкой фазой.
Состав смеси неизменен, постоянен.
Частным случаем контактного разгазирования является так называемая стандартная сепарация, когда контактное разгазирование осуществляется при температуре 20 градусов Цельсия и давлении 1 физическая атмосфера.
Допускается текущее атмосферное давление в лаборатории.
Когда мы проводим математическое моделирование, мы задаем 20 градусов Цельсия и указанное давление 1 физическая атмосфера.
За рубежом, я уже говорил, температура стандартная является 60 градусов Фаренгейт или 15,56 градуса Цельсия.
Почему я параллельно говорю об условиях за рубежом?
Потому что практически во всех лабораториях сейчас оборудование поступившее из-за рубежа, оно более совершенное, чем разработанное у нас ранее.
Затем методик много, которые опробированы за рубежом крупными компаниями типа Schlumberger, Halliburton и других.
И они используются у нас в лабораториях.
И программные комплексы тоже в основном используются те, которые разработаны зарубежными специалистами, это Eclipse и другие.
Eclipse, потому что в компании Газпромнефть этот комплекс давно используется.
Он включает в себя модуль PVTI, в котором моделируются процессы исследования природных углеводородных систем.
И можно рассчитывать фазовые равновесия природных углеводородных
систем, задав их состав и соответствующие термобарические условия, выбрав соответствующий тип экспериментов.
Причём стандартная сепарация, это однократное разгазирование при стандартных термобарических условиях, я немножко разжёвываю, чтобы вам лучше было понятно, является базовым экспериментом для последующего расчёта состава пластовой нефти (компонентного состава пластовой нефти).
Формулы я вам покажу, и вам будет всё понятно.

Дифференциальное разгазирование, которое по самому своему названию, по терминологии предопределяет, что эксперимент проводится не при постоянном составе смеси и вот, значит, осуществляется разгазирование, отводится газ и на практике это осуществляется ступенчато; мы более подробно рассмотрим.
А описывается система обыкновенных дифференциальных уравнений, я более подробно об этом скажу, когда буду рассказывать про газоконденсатную систему.
Итак, стандартные.

И третий тип, это ступенчатая сепарация, которая моделируется в условиях промысловой сепарации нефти с применением нескольких ступеней сепарации, поэтому называется ступенчатая сепарация, при термобарических условиях, соответствующих системе промысловой сепарации на промысле.

Итак, стандартная сепарация, дифференциальное разгазирование при пластовой температуре и ступенчатая сепарация, моделирующая промысловую подготовку.
Ну и это основные эксперименты, мы рассмотрим их результаты.

\begin{center}
\includegraphics[width=\textwidth, page=70]{Брусиловский.pdf}
\end{center}

Следующий слайд, вот слайд с системой добычи и подготовки нефти.
Значит, в случае, когда у нас давление в пласте превышает давление насыщения пластовой нефти, то есть нефть, как углеводородная система, находится в однофазном состоянии, разгазирование еще не происходит, давление выше давления насыщения.
Вода всегда есть в пористой среде, она может быть неподвижной, а может в результате осуществления процесса поддержания пластового давления быть подвижной, поступать на забой скважины.
Но нас интересует в данном случае, как ведет себя углеводородная система.
Повторяю, это случай, когда у нас в пласте углеводородная система имеет давление выше давления перехода в гетерогенное состояние, то есть выше давления насыщения.
В стволе скважины уже осуществляется разделение на газ и жидкость, потому что давление в стволе скважины уменьшается, и по достижении давления насыщения и дальнейшем уменьшении давления из пластовой нефти выделяется газ.
Происходит это в стволе скважины -- контактное разгазирование.
Состав текущей смеси неизменен, что поступило на забой скважины, то и выносится на устье.
То есть для текущей смеси в стволе скважины состав не меняется.
Поэтому, когда мы моделируем этот процесс, то мы используем процесс контактной конденсации, то есть при неизменном составе нашей смеси.
А давление изменяется за счет изменения объема, рабочего объема сосуда.
Поступающая с устья скважины смесь течет в сепаратор.
Здесь показана двухступенчатая сепарация.
Первая ступень -- это сепаратор, а вторая ступень -- это так называемый стоктанк или нефтехранилище.
В общем, в сепараторе поддерживается давление, $P_{\text{сеп}}$ и температура, выделяется газ, голубеньким –- это вода и зелёным –- это нефть.
В сепараторе в соответствии с заданным давлением сепарации осуществляется контактное разгазирование из многокомпонентной системы.
Она разделяется уже окончательно в сепараторе.
Считается, что соответствует состоянию равновесия термодинамического.
Достаточно большой по объему аппарат.
Выделившийся газ, газовая фаза, направляется на установки подготовки газа, а вода выделившаяся должна поступать в систему подготовки воды, так называемой подтоварной воды, и для дальнейшего её использования в системе поддержания пластового давления, если осуществляется заводение.
А нефть поступает в стоктанк.
В стоктанке поддерживается давление одна физическая атмосфера и температура 20 градусов Цельсия, когда мы моделируем или в лаборатории.
Стандартные термобарические условия.
Вот это двухступенчатая система.
Ну вот, и вот вы видите, что все стрелочки, уже вам всё понятно.
Из стоктанка, поскольку у нас давление снизилось до стандартного одной атмосферы, выделяются остатки газа, который либо сжигается, либо добавляется к газу, который на собственные нужды и так далее.
На практике.

\begin{center}
\includegraphics[width=\textwidth, page=71]{Брусиловский.pdf}
\end{center}

Следующий вариант -- это когда у нас давление в пласте ниже давления насыщения.
Всё похоже, но в пласте у нас, поскольку давление ниже, чем давление насыщения пластовой нефти, углеводородная система уже не гомогенная, не однофазная, а двухфазная.
Одна фаза жидко-углеводородная, а вторая газовая фаза, которая выделилась из пластовой нефти при снижении давления ниже давления насыщения.
Вот всё это поступает в ствол скважины.
У нас уже в стволе скважины, начиная с забоя, углеводородная система находится в двухфазном состоянии.
С точки зрения математического моделирования, поскольку это контактная конденсация, контактная конденсация моделируется, то есть фиксируется состав пластовой нефти, который неизменен, ещё раз, при течении от забоя до устья, но поскольку меняются термобарические условия, давление падает и температура тоже изменяется, у нас соотношение между газовой и жидко-углеводородными фазами непрерывно меняется.
Это приводит к изменению, значит, вот это соотношение между фазами к изменению структур течения.
Я здесь не рассматриваю, но вы уже наверняка знаете, что при течении в стволе скважины, значит, при изменении соотношения объёмов жидкой фазы и газовой фазы у нас структуры течения меняются.
Я показывал вам, давайте вспомним слайд, когда мы говорили об условиях отбора проб.
И вот там, значит, на забой поступала нефть в однофазном состоянии, пробоотборник у нас уже был в условиях пузырькового режима, когда у нас пузырьки газа уже выделились из нефти, а по мере снижения давления при движении к устью скважины у нас консолидируются пузырьки газа, они становятся, значит, уже образуются глобулы газа достаточно большие и режимы течения меняются.
Каждому режиму течения соответствуют свои гидродинамические критерии.
И это предмет самостоятельного изучения течения газожидкостных смесей в скважинах и трубопроводах.
Это самостоятельная тема, весьма ёмкая и десятилетиями изучается.
Особенности, критерии и так далее.
Всё это реализуется потом в программных комплексах, результаты экспериментальных исследований, что позволяет достаточно точно моделировать условия течения углеводородных смесей в системах трубопроводов при добыче и подготовке нефти.
И это же относится к газоконденсатным системам.
Всё это единая наука, просто соотношение между фазами в нефтяных системах и в газоконденсатных, оно различное, естественно.
В газоконденсатных системах значительно больше объём газовой фазы, соответственно, превалируют другие структуры течения и так далее.
А так наука гидродинамика газожидкостных потоков этим занимается.
А с точки зрения PVT-свойств мы должны уметь обеспечить исходной информацией, необходимой информацией вот эти расчёты гидродинамические.
Это осуществляется, раньше это осуществлялось экспериментальными исследованиями при температурах отличных от пластовой, от 20 градусов Цельсия до пластовой и между ними всё это экспериментально исследовалось, процесс вопроса контактных конденсаций.
А сейчас экспериментальные исследования проводятся наиболее часто при пластовой температуре, чтобы обеспечить информацией о том, каковы свойства фаз в пласте.
А мы создаём, повторяю, математические модели, основанные на применении уравнений состояния и уже создав математическую модель пластовой нефти, мы можем при различных термобарических условиях рассчитывать контактную конденсацию.
Это в программах зарубежных называется FLASH эти процессы.
Значит, контактная конденсация.
И в профессиональных программах, я понимаю, HI-SIM, HI-SYS и так далее, значит, там осуществляется расчёт фазовых превращений углеводородных смесей на основе заданного состава пластовой смеси, которая должна быть адаптирована к имеющимся экспериментальным данным.
Вот такая, значит, логика.
Ну вот, я вам устно рассказал, теперь вам совершенно очевидно должно быть.

\begin{center}
\includegraphics[width=\textwidth, page=72]{Брусиловский.pdf}
\end{center}

То, что вы прочитали, ещё раз уже тут написано на слайде, значит, это как повторение.
Теперь важно, что, ещё раз хочу сказать, что результат дифференциального разгазирования используется для идентификации изменения свойств углеводородных фаз в пласте при изменении давлений, при, значит, давлении ниже давления насыщения, то есть, когда у нас в пласте находятся две фазы.
Почему я вам вот так тщательно это дело объясняю?

Значит, всё уже понятно, что контактное разгазирование осуществляется, считается, что в стволе скважины, в трубопроводах промысловых, в сепараторах и моделируется математически контактным разгазированием.
То есть, мы задаём компонентный состав смеси и, значит, в результате получаем компонентный состав и относительное количество равновесных фаз.
Соответствующую постановку математическую я расскажу, когда буду делать обзор, краткий обзор по применению уравнения состояния.
Значит, а дифференциальное разгазирование, о котором сейчас пойдёт речь, когда я буду рассказывать тоже о сути экспериментальных исследований, значит, оно позволяет нам, как вы видите написано, вот это вот нужно чётко понимать, определить свойства равновесных фаз паровой-углеводородной и жидко-углеводородной сосуществующих в пласте при давлении меньше, чем давление насыщения.
И тоже будет пример, значит, результат дифференциального разгазирования и видно будет, как же меняется с давлением свойства углеводородных фаз в пласте при постоянной пластовой температуре и изменении давления.
Вот это вот суть процесса дифференциального разгазирования контактного.

\begin{center}
\includegraphics[width=\textwidth, page=73]{Брусиловский.pdf}
\end{center}

Теперь, значит, контактное разгазирование, англоязычный термин Flash и пояснение, у нас есть мольный состав смеси, который обозначается $Z_i$, и давление, при котором эта смесь загружена в сосуд PVT, часто говорят бомба PVT, сосуд высокого давления, значит, предположим, что мы загрузили 1 моль в смеси, находится наша смесь в однофазном состоянии, дальше мы за счёт изменения объёма, рабочего объёма этого сосуда уменьшаем давление и появились первые, видите, вот 3 точечки, 3 жёлтенькие, это первые пузырьки газа, это давление, значит, мы достигли давления насыщения и при дальнейшем увеличении объёма у нас выделяется из жидкого раствора, выделяется газ.

Значит, при давлении $P^2$, которое меньше, чем давление $P^1$, равного давлению насыщения, на $\Delta P^2$, вот мы снизили, у нас жёлтая, это выделившийся газ, поскольку он находится в равновесии с жидкостью, обозначается $V$ (от вейпа, это общепринятое обозначение), значит, и равновесный наш раствор, газированная жидкость $L$ (от liquid).
Состав жидкости обычно обозначается $x_i$, а газовая фаза или паровая -- $y_i$, а суммарный состав у нас $Z_i$.
Мы из сосуда PVT ничего не удаляли, мы давление изменяли за счёт изменения объёма, поэтому контактная конденсация у нас осуществлялась, значит, вот контактное разгазирование в данном случае.
Для нефти разгазировали.
Если бы у нас была, заодно комментарий, если бы у нас была исходная система газоконденсатной находилась в газообразном агрегатном состоянии, то в результате уменьшения давления при увеличении объёма, уменьшение давления ниже давления начала ретроградной конденсации.
Этот процесс назывался бы контактная конденсация, в зависимости от того, с какой смесью мы имеем дело.
И хотя нам в этом курсе понятие коэффициента распределения или константа равновесия не понадобится, но чтобы вы знали, что отношение мольной доли $i$-ого компонента в равновесной паровой фазе к отношению мольной доли этого компонента в жидкой фазе, называется константой равновесия, но более правильно коэффициент распределения.

\begin{center}
\includegraphics[width=\textwidth, page=74]{Брусиловский.pdf}
\end{center}

Итак, контактное разгазирование.
Компонентный состав системы не изменяется во время эксперимента, газ остаётся в равновесии с жидкостью на протяжении всего эксперимента и определяется зависимость между давлением и объёмом при пластовой температуре.
Эта зависимость, ну это вы уже знаете.
PV-зависимость, PV-соотношение, которое мы с вами подробно рассмотрели, суть, или я уже путаю, мы только будем рассматривать, в общем, PV-соотношение, нет, мы с вами уже об этом говорили, да, позволяет определить давление насыщения, а также объёмную упругость пластовой нефти.
Конечно, мы об этом.
Или я на прошлой неделе курс читал для сотрудников компании, и поэтому у меня уже в голове мешанина.
Значит, итак, прошу прощения за такую вот небольшую путаницу, но вас не должно это в смущение вводить.

\begin{center}
\includegraphics[width=\textwidth, page=75]{Брусиловский.pdf}
\end{center}

Давление насыщения нефти газом и PV-зависимость.
Итак, на основе проведения эксперимента контактной конденсации мы имеем возможность построить так называемые PV-зависимости.
То есть, зависимость между давлением и объёмом при постоянной температуре.
И этот эксперимент, который происходит в рамках контактной конденсации, значит, он что позволяет нам?
Он позволяет нам определить давление насыщения нашей смеси.
Предположим, мы загрузили наш образец пластовой нефти в сосуд высокого давления.
И объём таков, что давление выше давления насыщения, весьма существенно выше.
Нам давление насыщения неизвестно, но мы создали давление выше пластового.
И это может быть на несколько десятков бар выше.
Важно, что выше пластового.
Потому что максимальная оценка давления насыщения -- это величина пластового давления.

И создаём давление выше пластового нашей системы, наша проба, загруженная в сосуд PVT, в однофазном состоянии.
И мы увеличиваем контактным образом, не удаляя ничего из системы увеличиваем объём нашего сосуда PVT.
И в результате этого давление снижается.
Причём, чем меньше растворено газа в пластовой нефти, тем быстрее будет снижаться давление при увеличении объёма сосуда PVT.
Это физически совершенно понятно.
То есть, при увеличении объёма наша система должна расшириться до этого объёма моментально.
Но это легко сделать газу, совсем легко, но это очень сложно сделать жидкой фазе.
И чем меньше её газосодержание, чем меньше газа растворено в жидкости, тем сильнее будет падать давление при увеличении рабочего объёма.
И вот вы видите, что мы...
Вот несколько точек, соответствующих этапам увеличения рабочего объёма нашего сосуда.
Значит, тут при достижении давления насыщения выделившийся газ, он же расширяется, и уже упругость двухфазной системы совершенно другая, чем у однофазной жидкой системы.
И поэтому зависимость между давлением и объёмом при увеличении объёма, она уже совершенно иная.
Совершенно иная.
И вы видите, что при выделении газа у нас эта зависимость ну просто другая, значит, мы можем определить при каком же давлении у нас произошло выделение газа, каково давление, при котором начался процесс разгазирования.
Значит, у нас в начале, вот написано, ветвь двухфазного состояния при...
Вот третье, при некорректном проведении опыта.
Третье, при некорректном.
То есть, когда мы идём первоначально с большими шагами, и мы пропустили на самом деле давление насыщения, и излом у нас первоначально показан при давлении $p_b'$.
Но затем мы можем более точно определить величину давления насыщения, значит, мы возвращаемся, то есть объём опять уменьшаем, уменьшаем объём, перемешиваем нашу смесь в однофазном гомогенном состоянии при давлении выше, чем… ну, то есть, уже создали условия, уменьшив объём, при котором давление поднялось.
И уже с меньшим шагом по изменению объёма мы идём и фиксируем зависимость давления от объёма.
И это вот ветвь 2.
И мы видим, что при более высоком давлении, несколько более высоком давлении у нас излом осуществляется.
Значит, мы можем таким образом с меньшим шагом осуществить эксперимент в области перехода из однофазного в двухфазное состояние, определить достаточно точно давление насыщения.
Вот таким образом на практике это и делается в сосудах высокого давления при экспериментальном определении.

\begin{center}
\includegraphics[width=\textwidth, page=76]{Брусиловский.pdf}
\end{center}

Вот в виде упражнения показана как бы запись экспериментальных данных.
Таким образом записывают текущее давление и какой объём системы при текущем давлении.
Вот такой табличкой.
И по этим данным можно определить, при каком давлении будет излом PV-зависимости, то есть определить давление насыщения.

\begin{center}
\includegraphics[width=\textwidth, page=77]{Брусиловский.pdf}
\end{center}

Вот мы, например, в Excel ввели указанные данные и получили такой график.
И мы видим чётко совершенно, где у нас излом PV-зависимости, какому давлению соответствует.
И определили таким образом, что давление насыщения равно 87 бар.
И ещё раз, это излом резкий, резкий.
Это значит, что газодержание нашей нефти невысокое, не выше чем 200, ну вот для данной системы это порядка наверное 100 м$^3$ на м$^3$ по резкому излому.

\begin{center}
\includegraphics[width=\textwidth, page=78]{Брусиловский.pdf}
\end{center}

Это из моей монографии 2002 года.
Значит, для реальных, по реальным данным для месторождений различных.
И я, ну, наверное, не буду вам говорить, какие месторождения, это просто смысла нет.
Значит, это различные месторождения, характеризующиеся различным газодержанием, то есть количеством растворённого газа в пластовой нефти.
И из того, что я уже рассказал, вы поняли, что там, где чётко совершенно, вот на кривой один, чётко виден излом, это давление насыщение, там газодержание меньше, чем в тех случаях, когда резкого излома нет.
Значит, последовательно показаны зависимости с увеличением газосодержания в пластовой нефти.
Это разные месторождения.
И мы видим, что если по зависимости 1 чётко совершенно можно определить графически давление насыщения, но менее чётко, но всё-таки достаточно для практических целей можно определить давление насыщения по зависимости 2, просто нужно с более мелким шагом по давлению идти и определяется давление насыщения, то уже зависимость 3 резкого такого изменения направления в PV-зависимости нет, плавная, а по графику 4 мы вообще не можем определить по зависимости давление насыщения.
И та точка, которая показана на зависимости 4, это Карачаганакское месторождение, там газодержание около 500 м$^3$ на м$^3$, это давление насыщения, то есть начало выделения газовой фазы определено оптическим методом, не по PV-зависимости, а это можно определить визуально, либо с помощью оптических анализаторов, которые позволяют определить появление новой фазы.
Но в современной аппаратуре у нас, конечно же, оптические анализаторы присутствуют, и это позволяет более точно, более точно, ну то есть это нормальный метод определения перехода из гомогенного в гетерогенное состояние.
И в отчетах и зарубежных компаний, и наших лабораторий, которые сейчас оснащены современными установками, мы уже не визуально, а с помощью приборов определяем появление новой фазы, то есть начало разгазирования нефти.

\begin{center}
\includegraphics[width=\textwidth, page=79]{Брусиловский.pdf}
\end{center}

Теперь по результатам PV-соотношений мы можем определить изотермический коэффициент сжимаемости или объемную упругость пластовой нефти.
Значит мы это делаем по данным PV-соотношений можем сделать, заменяя производные на приращения, и используется эта величина изотермический коэффициент сжимаемости или объемная упругость в частности можно оценить совершенно простым способом нефтеотдачу пласта при работе на упругом режиме без поддержания давления, зная объемную упругость пластовой нефти.
Вот совершенно простой формулой.
Я не буду сейчас приводить эти соотношения, это можно даже и самим в виде упражнения сделать.
Еще раз, нефтеотдачу на упругом режиме можно оценить зная изотермический коэффициент сжимаемости пластовой нефти, именно за счет упругости нефти.
За счет упругости нефти конечно мы высокой нефтеотдачи добиться не можем, но если у нас большая разница между пластовым давлением и давлением насыщения (она бывает десятки мегапаскалей), то значительная часть добычи нефти формируется, когда нефть находится в однофазном состоянии
в пласте, она может быть сформирована на упругомрежиме.
И чем выше газосодержание пластовой нефти начальной, тем выше будет величина нефтеотдачи при разработке на упругом режиме, хотя конечно высокого значения мы добиться не можем.
Все равно нам нужно будет в итоге создавать систему и вводить в действие поддержание пластового давления.
Я говорю о физическом смысле, то есть изотермического коэффициента сжимаемости, где он может применяться.
Он раньше в явном виде применялся в формулах материального баланса при проектировании разработки, при оценке на упругом режиме и так далее.

\begin{center}
\includegraphics[width=\textwidth, page=80]{Брусиловский.pdf}
\end{center}

Также по результатам исследований пластовых нефтей определяют температурный коэффициент объемного расширения.
Его определение вы видите на слайде.
Также приведен диапазон его величины для большинства пластовых нефтей.
Используется эта величина при проектировании термического воздействия на нефтяные пласты.

\begin{center}
\includegraphics[width=\textwidth, page=81]{Брусиловский.pdf}
\end{center}

Теперь переходим к стандартной сепарации.
Тут немножко съехала схемочка.
Ну, ничего, я ее поправлял, она опять съехала, надо было закреплять.
Итак, вот у вас глубинные пробы.
Мы разгазируем однократно и определяем объем газа и его компонентный состав, которые выделились при однократном разгазировании глубинной пробы при 20 градусах Цельсия допускается температура окружающей среды в лаборатории и одна физическая атмосфера.
Также измеряется объем дегазированной жидкости, её компонентный состав, хроматография, и измеряется плотность и молекулярная масса дегазированной жидкости.

\begin{center}
\includegraphics[width=\textwidth, page=82]{Брусиловский.pdf}
\end{center}

Перед лекцией я еще раз посмотрел, что пишут по поводу стандартной сепарации, что написано было в монографии Абрама Юдельевича Намиота, о которой я говорил, это замечательная книга "<Фазовые равновесия в добыче нефти">, где человек просто описывает основные моменты, очень хорошо понимая и физику, и особенности физико-химических свойств.
И вот я вам могу просто зачитать, сейчас я взял в руки.
Так вот, эта стандартная сепарация, однократное разгазирование, она осуществляется при дросселировании.
Пластовая нефть, находящаяся в сосуде, это я зачитываю его текст, просто чтобы кому-то будет полезно.
Пластовая нефть, находящаяся в сосуде, которая была загружена в сосуд высокого давления, при давлении выше давления насыщения, через слегка открытый вентиль, дроссель, выпускают стеклянный сепаратор, находящийся под атмосферным давлением.
Вот каким образом это осуществляется, понимаете, разгазирование.
То есть, не увеличивают, ведь вот вопрос, который мог возникнуть у вас, то есть у нас давление может быть в сотни атмосфер или бар, если при высоком газосодержании нефти, ее давление насыщения может составлять и 300 и 400 бар.
И каким образом осуществляется однократное разгазирование?
Ведь мы же не будем в десятки раз увеличивать объем сосуда PVT, или даже больше.
Вот осуществляется это при дросселировании.
При выпуске, значит, вот в стеклянных сепараторах, находящихся под атмосферным давлением, при выпуске пластовой нефти давление в сосуде поддерживают выше давление насыщения путем перемещения плунжера -- поршня.
В процессе сепарации нефть стекает в нижнюю часть сепаратора, а газ подается из сепаратора в измерительную бутыль -- газометр, или на газовый счетчик.
Все просто.
Но более подробно вот просто найдите это издание, очень хорошо там все написано.
Я учился по книгам Намиота, с ним контактировал.
И, значит, просто настоящий классический профессор, который работал, который хорошо чувствовал и суть экспериментальных исследований, и физику явлений.
При этом он на компьютере не умел, тогда рассчитывали, это делали первоначально сотрудники, просто помощники, ну как классическая лаборатория.
А уже когда я с ним взаимодействовал, просто, так сказать, не работая с ним, а просто взаимодействовал, это со второй половины 80-х годов и до середины 90-х годов я по его просьбе проводил математическое, компьютерное моделирование.
И мне доставляло большое удовольствие взаимодействие с ним, я гораздо лучше стал понимать отдельные особенности физических явлений, когда стал заниматься пластовыми нефтями.
Конечно, он был специалист по исследованию пластовых нефтей и еще проводил исследования по растворимости газа в воде, ну это в 60-е годы.
Это я вас ориентирую просто.
И этот специалист, он ни в коем случае не уступал по своим знаниям и глубине понимания зарубежным специалистам.
А в комплексе, когда мы читаем книжки и зарубежных специалистов, и наших классиков, вот тогда мы становимся настоящими специалистами.
И в любом деле, что гидродинамическое моделирование разработки, что физическая химия, исследование пластовых флюидов, то есть это очень полезно.
Если кому-то ссылки понадобятся точные, я могу дать.

Теперь дальше.
Понятие объемного коэффициента и газосодержания, которые при проектировании разработки в подсчете запасов используются.
Объемный коэффициент пластовой нефти равен отношению объема, занимаемого углеводородной жидкой фазой пластовой смеси при пластовых условиях к объему дегазированной нефти при стандартных условиях.
Аббревиатура $V_r$, индекс $r$ -- это reservoir, то есть пласт, залежь при условиях пласта, объем занимаемой пробы, отнесенный к объему занимаемой пробы нефти при стандартных условиях.
Этот объем, объемный коэффициент, он зависит, его величина зависит от того, проводим мы стандартную сепарацию, однократное разгазирование, либо мы проводим дифференциальное разгазирование при пластовой температуре, либо мы осуществляем ступенчатую сепарацию, моделирующую промысловую подготовку нашей нефти.
Вот мы получаем разные величины, причем, чем больше газа растворено в пластовой нефти, тем сильнее будут отличаться величины объемного коэффициента, полученные при разных видах разгазирования.
Газоодержание пластовой нефти, количество газа, выделившееся из растворенного состояния при изменении условий от пластовых до стандартных и отнесенного к объему или массе дегазированной нефти при стандартных условиях.
Значит, это газосодержание пластовой нефти.
Мы встречаем на практике газосодержание и метр куб на метр куб, и метр куб на тонну.
Вот понятие объемного коэффициента и газосодержания.

\begin{center}
\includegraphics[width=\textwidth, page=83]{Брусиловский.pdf}
\end{center}

Компонентный состав и относительная плотность дегазированной нефти.
Здесь просто обозначения показаны.
У нас есть проба дегазированной нефти массой $m_o$, объемом $V_o$ и компонентным составом $X_i$, который определяется хроматографией.
В результате стандартной сепарации определяют и молекулярную массу дегазированной нефти.
Почему индекс STO? Это Stock Tank Oil.
Дегазированная нефть -- это стандартное ее обозначение во всех зарубежных источниках и в отчетах наших компаний тоже часто применяется, тем более что они все, повторяю, развивались под влиянием компаний Schlumberger, Halliburton или других, в основном эти.
Соответствующие обозначения автоматом перешли; это нормально совершенно.
Сейчас аббревиатура используется та, что в англоязычной литературе, и это нормально совершенно.
Вообще аббревиатура, я хочу сказать, в SPE (Society of Petroleum Engineers) общество, оно издает журналы, оно издает монографии и стремится к единообразию обозначений (также как у нас есть).
У нас есть справочник, просто не думаю, что вы об этом знаете, например, справочник по терминологии и обозначениям, изданный в Академии наук СССР.
Академия наук СССР это не РАН, это была вообще могучая абсолютно организация выдающихся, самых выдающихся ученых и они в том числе, значит, издательство Наука, это вот как раз издательство РАН, оно издаёт очень солидные монографии.
Есть в том числе справочник по терминологии в области разных наук, и механика, и физическая химия, ну все отрасли абсолютно.
И там чётко и обозначения рекомендуются, и четко говорится о том, что имеется в виду под этой терминологией.
И если вы заметили или заметите, вот то же самое в изданиях под эгидой SPE, там единая терминология применяется.
И зная эту терминологию, ты уже можешь источники (книги), написанные разными авторами, читать совершенно свободно из разных стран, потому что терминология и обозначения общепринятые.
Вот такая вещь.
У нас такого единообразия до сих пор нет, и только-только как-то приводится, стараются это единообразие создать.
Но неприятность, может быть, кому-то я впервые сообщу, ну, кого-то это заинтересует, кого-то не заинтересует.
Ну, те, кто в нефтяной области будет работать.
К сожалению, на днях был закрыт в России Московский офис SPE.
Он существовал с начала 90-х годов, и мы...
Затем, кстати, в Санкт-Петербурге секция была образована.
SPE закрыла в России офисы, и, к сожалению, президент SPE всем членам SPE написал, что очень сожалеет и будут...
Буквально это произошло пару дней назад.
Но, тем не менее, значит, возможность читать литературу, ценные монографии и статьи SPE, надеюсь, у нас останется.
Надеюсь, у нас останется и через какое-то время деятельность SPE в России возобновится.
Это в качестве информации.
Потому что сыграло очень большую положительную роль то, что Society of Petroleum Engineers, значит, объединяло специалистов из разных организаций в нашей стране тоже.
И были ежегодные конференции, но, надеюсь, через какое-то время они возобновят свою работу.
Значит, теперь, возвращаясь к слайду, ну, плотность очевидная.
Плотность определяется отношением массы к объему.
Значит, а вот используются в инженерной практике относительные плотности, понятие относительной плотности.
И относительная плотность нефти -- это ее отношение к плотности дистиллированной воды.
Я об этом говорил и повторяю, чтобы вы уже четко совершенно это усвоили.
Что когда обозначение $\gamma_o$ -- это отношение плотности нефти к плотности воды, естественно, в одних и тех же единицах, и плотности именно дистиллированной воды.
Если хотим мы плотность воды узнать абсолютно, то мы должны относительную плотность...
Ну, вернее, так.
Это я уже...
В общем, если мы знаем относительную плотность нефти, а хотим узнать ее абсолютную плотность, то нам нужно относительную плотность нефти умножить на 1000, потому что используется дистиллированная вода плотностью 1000 кг на метр кубический.

\begin{center}
\includegraphics[width=\textwidth, page=84]{Брусиловский.pdf}
\end{center}

Дальше пошли компонентные составы и относительная плотность газа.
Не все это знают.
Поэтому что получается у нас?
Вот у нас есть газ.
Эта терминология относится и к растворенным газам, и к свободным газам.
Мы сейчас рассматриваем однократное разгазирование, стандартную сепарацию, имея в виду выделившийся из пластовой нефти газ.
Компонентный состав его $y_i$ в сумме равняется единице, и молекулярную массу газа мы можем рассчитать, зная компонентный состав и молекулярную массу отдельных компонентов, просто аддитивно, складывая произведение мольной доли этого компонента на молекулярную массу, и можем рассчитать с вами относительную плотность газа.
Если для нефти относительная плотность была безразмерной величиной, и мы относили к плотности дистиллированной воды, то относительная плотность газа это безразмерная величина отношения плотности газа размерной к плотности воздуха.
И она равна отношению, просто это из-за уравнения состояния идеального газа.
Молекулярная масса газа, разделённая на молекулярную массу воздуха.
Молекулярная масса воздуха 28.96, поэтому вот эта формула.
Теперь вас не должно смущать, вот здесь 28.96, где-то написано 28.97, я уже об этом вам говорил, просто в разных источниках разные величины.
В инструкциях у нас 28.96.
У Кёртиса Витсена в монографии [не распознано] %ФСБ Хиви
28.97.
И я в свое время делал разные программки и использовал 28.97, потому что хотел, чтобы все было в соответствии с правилами SPE.
Ну, там 28.97 -- это в официальной монографии SPE.
На самом деле такая точность, конечно, совершенно не нужна, потому что используются эти величины для оценочных расчетов, для корреляций.
Это с точки зрения инженерных расчетов абсолютно одно и то же, 28.96 или 28.97.
И кроме всего прочего, мы знаем, что состав воздуха несколько меняется, и поэтому, в общем, можно для инженерных целей использовать 29, даже, что и делает Билл Маккейн, о котором я говорил.

\begin{center}
\includegraphics[width=\textwidth, page=85]{Брусиловский.pdf}
\end{center}

Повторение той таблицы, которую я уже показывал 10 октября и вчера тоже, когда мы проходили сначала быстренько, я тоже показывал.
Это, значит, свойства индивидуальных компонентов.
И я просто единственное хочу сказать, что вчера был вопрос по ацентрическому фактору, что я говорил, что с увеличением числа атомов углерода, это видно здесь для углеводородов, величина ацентрического
фактора увеличивается.
И я смотрел вчера, да, и был вопрос, а если у нас симметричные молекулы, есть изомеры, то как там.
Значит, этот вывод неизменен.
Значит, я посмотрел очень известный справочник, свойства газовой жидкости, причем несколько изданий у меня в домашней библиотеке.
Значит, это Рид Шерват, Рид Шерват-Прауснец, значит, это все очень известные ученые американские, разныхгодов.
И последнее издание это Рид Прауснец-Поллинг.
Значит, там до $C_{20}$ приведены углеводороды самого различного строения и максимальное количество атомов углерода до 20.
Вот и там приведены вот сведения по критическим молярной массе, критическому давлению, температуре, объему и ацентрическому фактору.
Значит, как я вчера и говорил, ацентрический фактор равный нулю, он соответствует неону (компонент).
В этом справочнике там не только углеводороды, там и другие компоненты.
Вот неон, для него ноль, ацентрический фактор, упругость паров его соответствует вот этой классической кривой, которая была в 1955 году Питцером предложена.
Александр Иосифович, а можно еще вопрос?
В книге у вас написано, что ацентрический фактор определяется как разница десятичных логарифмов.
Это же то, что вы видите на слайде.
Да, да.
При чем здесь книга?
Книгу же не все читали.
Так, так, я слушаю вас.
Вопрос, а что, какое вещество берется в качестве такого, которое подчиняется универсальной зависимости?
Вот, значит...
Кроме неона.
Нет, нет, об этом не сказано.
Понимаете, в этих монографиях об этом… Это нужно первоиспытать.
Это нужно в первоисточнике статью Питцера поднимать.
Но вы посмотрите, пожалуйста, Рид Шерват-Прауснец, да, классика, свойства газовой жидкости.
И там просто этому вопросу достаточно много внимания уделено.
И даже Абрам Юдельевич Намиот, на которого я все время ссылаюсь, он ссылается в своей монографии на этот справочник, ну, одно из первых изданий.
Вот.
Поэтому, значит, тут нам не важно, понимаете, важно суть.
То есть во всех программах, вот вы которые используете, там есть величина ацентрического фактора в библиотеках.
Вот я просто рассказывал о смысле его.
Что он, значит...
И главное, что была вот эта вот...
Когда был сформулирован принцип соответственных состояний, то, значит, на основе уравнения Ван-дер-Ваальса это было сделано, и было показано, что существует однозначное соответствие теоретически строго между приведенным давлением, приведенной температурой, удельным объемом и другими веществами.
Значит, вывод такой, что зная приведенное давление, приведенную температуру, вы можете определить приведенную величину любого свойства, ну, в частности, удельный объем, это из уравнения Ван-дер-Ваальса.
И там же из этих рассуждений и из уравнения состояния Ван-дер-Ваальса следует универсальное…
Следует универсальное…
Это не то, что какое-то вещество.
Вот смотрите, всё, я вспомнил.
Значит, вот из уравнения, из принципа соответственных состояний следует однозначная зависимость между приведенным давлением и приведенной температурой.
Это для веществ, обладающих шаровой симметрией, и вот приведен такой график.
Но более подробно вы просто читайте монографию.
И первоисточники просто можно почитать.
Сейчас у нас нет такой возможности.
Ну, вы поняли, да, Дэн?
Да, понял.
Большое спасибо.
Ну вот, да-да.
Я также и дальше, я же буду делать обзор по уравнениям состояния, у меня не будет возможности подробно обсуждать.
Это всё такие специальные вопросы.
Специальные вопросы, и нужно просто углубляться, уже читая монографии, статьи.
Ну, естественно, если специалисты этим занимались, конкретно этим вопросом, то и со специалистами беседовать.
Но они всё равно черпают свои знания из монографий и статьей, из первоисточников.
Вот Рид Шерват-Прауснец, свойство газовой жидкости, вот так вот.
Это первоисточник для нас, если не статья 1955 года и так далее.
Давайте дальше пойдём.
Я, значит, хотел просто вам сказать, что у нас ацентрический фактор, его величина растёт с увеличением числа атомов углерода по мере отклонения молекул от шаровой симметрии.
Там не только шаровая симметрия и всякие дипольные моменты и прочее-прочее, но это специальные вопросы.

\begin{center}
\includegraphics[width=\textwidth, page=86]{Брусиловский.pdf}
\end{center}

Значит, вот пример простой.
Из глубинной пробы, полученной при стандартной сепарации, такой-то объём нефти, объём газа, масса нефти, компонентный состава выделившегося газа, какие базовые параметры можно определить, рассчитать их значения?
И вот сейчас мы быстренько посмотрим.
И подобная задача в сборнике, задача, которую я переслал.
Ну, простейшая задача, понимаете?
Просто для закрепления материала; вряд ли вы будете определять по имеющимся данным газосодержание или же объёмный коэффициент.
Но для закрепления материала по имеющимся данным это всё полезно.
Это просто учёба.

\begin{center}
\includegraphics[width=\textwidth, page=87]{Брусиловский.pdf}
\end{center}

Значит, вот мы в соответствии с данными определили газосодержание, определили объёмный коэффициент, определили молекулярную массу газа.
Вообще, вот сейчас, кстати, заодно скажу, молекулярный вес, не применяю терминологию, это всё раньше было.
Сейчас нужно массу.
Значит, ну, а численно это то же самое.
Числено то же самое, молекулярная масса газа.
И, как я говорил, значит, опускают у.е. (углеродные
единицы).
Просто пишут безразмерной величины.
Относительная плотность газа определяется отношением молекулярной массы газа к молекулярной массе воздуха и относительная плотность нефти определяется, значит, вот, отношением плотности нефти к плотности дистиллированной воды.
Всё.
Вот мы всё определили.
И пойдём дальше.

\begin{center}
\includegraphics[width=\textwidth, page=88]{Брусиловский.pdf}
\end{center}

Упражнение 5, подобное только что разобранному.

\begin{center}
\includegraphics[width=\textwidth, page=89]{Брусиловский.pdf}
\end{center}

Упражнение 6.

\begin{center}
\includegraphics[width=\textwidth, page=90]{Брусиловский.pdf}
\end{center}

Вот, теперь вот важно очень.
Мы, значит, рассматриваем с вами стандартное однократное разгазирование, стандартную сепарацию.
И по результатам этого эксперимента определяется компонентный состав пластовой нефти, мольная доля.
И вот я добавил в этот слайд, значит, для вашей, для обучения, значит, формулу, что ж такое компонентный состав, мольная доля $i$-ого компонента.
Мольная доля $i$-ого компонента равняется, вот когда у нас две составляющих, растворённый газ и дегазированный нефть, да, вот в результате стандартной сепарации.
Значит, мольная доля.
Мы должны знать число молей $i$-ого компонента в газовой фазе.
Мы должны знать число молей $i$-ого компонента в нефти.
Значит, по каждому компоненту мы это знаем.
Мы суммируем по всем компонентам, получаем общее число молей в газовой фазе.
По всем компонентам в нефтяной фазе дегазированной нефти,
получаем $n^o$.
И, значит, вот таким образом мы получаем мольную долю $i$-ого компонента пластовой нефти для всех компонентов.
В сумме это будет, как вы понимаете, единица.
А вот ниже уже показана формула для расчёта мольной доли $i$-ого компонента по результатам стандартной сепарации, то есть, вот однократного разгазирования, которая выведена из верхней формулы.
Когда мы...
Вот это можно самим сделать.
Это можно в качестве упражнения вы можете попробовать вывести эту формулу.
То есть определить чему равно число молей $i$-ого компонента в газовой фазе, чему равно число молей $i$-ого компонента в нефтяной дегазированной фазе, просуммировать по всем компонентам, и вы получите эту формулу.
В предположении, что у нас при одной атмосфере и 20 градусах Цельсия газ ведёт себя как идеальный, а для идеального газа 1 моль занимает при стандартных термобарических условиях объём 24.04 литра или 0.0244 метра кубических.
По другим данным это не 24.04, а 24.05.
Это опять же с инженерной точки зрения одно и то же.
В книжке Намиота вообще написано 24.06.
Он просто взял из какого-то источника.
Если совсем точно считать, будет 24.05.
Но во многих книгах я всегда... откуда-то когда-то взял 24.04?
Поэтому я не стал исправлять.
Если взять универсальную газовую постоянную с большим числом знаков, то мы получим 24.05 на самом деле.
Не точно, а с округлением.
Суть в том, что здесь обозначения все даны.
Если ещё раз вы число молей $i$-ого компонента просуммируете и пронормируете, поделив на сумму числа молей всех компонентов в смеси, то вы получите формулу, которая $Z_i$ равняется мольная доля $i$-ого компонента на газосодержание.
Обратите внимание, что газосодержание метр куб на метр куб.
Здесь все размерности соблюдены.
Плюс мольная доля $i$-ого компонента дегазированной нефти на 24.04 на относительную плотность нефти по воде.
И численно она равна плотности дегазированной нефти в граммах на сантиметр куб или в тоннах на метр куб.
Делённая на молекулярную массу этой дегазированной нефти, и вы получаете компонентный состав пластовой нефти, который вы в программу свою вводите.
И в лабораториях, когда приводится компонентный состав пластовой нефти, повторяю, это расчётная величина, полученная на основе замеров экспериментальных значений растворённого газа, дегазированной нефти, молекулярной массы нефти и газосодержания.

\begin{center}
\includegraphics[width=\textwidth, page=91]{Брусиловский.pdf}
\end{center}

Теперь тоже пример, это вот если мы не в лаборатории.
Ну, этот пример, значит, что здесь написано?
Из суточного замера в тестовом сепараторе и так далее.
А можно было так не писать?
Это из, опять же, опыта, то, что преподавал Александр Иванович Адегов в Роснефти.
А в лаборатории, можно сказать, что полученный компонентный состав газа, компонентный состав жидкости, да?
Значит, газосодержание, метр куб на метр куб.
Но если на промысле делаются измерения, то там получают объём нефти, там получают объём газа.
Газ всегда при стандартных условиях, как я говорил, объём, да?
Значит, мы получаем в лаборатории относительную плотность дегазированной нефти и составы.
Вот составы на промысле, всё-таки составы -- это прерогатива лабораторных исследований, хроматографии.
И редко когда на промысле компонентный состав нефти дегазированной или растворённого газа определяют.
Ну, где-то это и есть.
В таком случае вас не должно смущать, что такое суточный тест, если он в сепараторе или не важно.
Поделим, значит, мы дебит газа на дебит нефти, получим газосодержание.
Это можно делать в лаборатории, это можно делать где-то там в промысловых исследованиях, не важно.
Вот нам даны составы, нам дана молекулярная масса и относительная плотность группы $C_{7+}$ для нефти.

\begin{center}
\includegraphics[width=\textwidth, page=92]{Брусиловский.pdf}
\end{center}

И, значит, мы по данным, которые приведены, можем определить газосодержание, поскольку здесь говорится про промысловые исследования, значит, это ГФ -- газовый фактор.
Когда мы говорим газовый фактор, это означает, ну в подавляющем большинстве случаев, что речь идёт о промысловых исследованиях.
Потому что, когда мы говорим о лабораторных исследованиях, там всё-таки применяется терминология газосодержания.
Если возьмёте технические отчёты, там газосодержание.
Если возьмёте отчёты по промыслу, будет газовый фактор.
Но, я уже говорил, заодно скажу, что пока давление в пласте не снизилось ниже давления насыщения, газовый фактор, он равен газосодержанию.
Численно равен газосодержанию.
И какая здесь особенность?
Значит, у нас было дано газосодержание метр куб...
Так, значит, здесь мы определяем газовый фактор метр куб на тонну.
Но у нас даны были объёмы добычи нефти, объём добычи газа и относительная плотность нефти.
Поэтому мы здесь используем эту относительную плотность нефти и получаем, что газовый фактор равен 268 метр куб на тонну.

\begin{center}
\includegraphics[width=\textwidth, page=93]{Брусиловский.pdf}
\end{center}

И дальше, хочу обратить ваше внимание, вот дальше табличка, где даны наши исходные данные по составу дегазированной нефти, молекулярная масса каждого компонента и в том числе группа $C_{7+}$, в лаборатории определенная.
Значит, и предварительно рассчитанный состав пластовой нефти.
Вот, получается, тут есть отличие от... и это специально сделано, чтобы вы обратили внимание.
Смотрите, у нас, значит, газовый фактор метр куб на тонну в формуле, да?
А до этого в формуле газосодержание было метр куб на метр куб.
Вот в лаборатории, в лаборатории, да, вот если бы мы считали по этой формуле и использовали метр куб на метр куб, то у нас мы бы использовали эту формулу.
А если у нас где-то на промысле там привычно использовать газовый фактор метр куб на тонну, значит, поэтому формула несколько меняется.
Видите, тут в этой формуле нет относительной плотности нефти в отличие от той формулы, где использовалось газосодержание.
Просто обращаю Ваше внимание.
Потому что мы использовали относительную плотность нефти, когда мы уже газовый фактор вычисляли метр куб на тонну.
Вот таким образом можно определить, что в лаборатории, это та формула, что была использована.
Или по промысловым данным и данным состава, полученных промысловой лабораторией, можно оценить начальный состав пластовой нефти.

\begin{center}
\includegraphics[width=\textwidth, page=94]{Брусиловский.pdf}
\end{center}

Соответственно упражнение.

\begin{center}
\includegraphics[width=\textwidth, page=95]{Брусиловский.pdf}
\end{center}

Переходим к следующему.
Итак, еще раз, стандартная сепарация.
Основной смысл этого эксперимента, во-первых, это определение газосодержания, оценка газосодержания.
И главное, это определение компонентного состава пластовой нефти.
Мы определили компонентный состав пластовой нефти.
Переходим к эксперименту дифференциального разгазирования пластовой нефти при пластовой температуре.
Значит, физический смысл, еще раз, получение данных зависимости свойств нефти и выделившегося газа от давления при пластовой температуре.
Значит, раньше всё это были рекомендации в отраслевом стандарте о том, что необходимо сделать в этом эксперименте 8-10 ступеней снижения давления при пластовой температуре.
Причем эти значения давления, это ниже, чем давление насыщения.
Этот эксперимент проводится, начиная от давления насыщения пластовой нефти, которая была определена по данным контактной конденсации, по данным PV-соотношений.
Или оптическим образом, оптический анализатор.
Использовался, ну, в общем, данные контактной конденсации.
Мы знаем давление насыщения, привели нашу, загрузили бомбу PVT, привели, значит, объем таким образом, чтобы давление стало равным давлению насыщения.
У нас однофазное состояние нашей пластовой нефти при давлении, равном давлению насыщения.
И начинаем эксперимент по дифференциальному разгазированию.
Делается поэтапно, ступенчато снижая давление.
Что сначала мы делаем?
Увеличиваем рабочий объем в нашем сосуде PVT.
И в результате увеличения объема мы достигаем давления $P_2$ меньше, чем первоначально, чем давление насыщения.
Поскольку давление меньше, чем давление насыщения, из раствора нефтяного выделился газ.
Вот у нас уже двухфазное состояние нашей системы при давлении $P_2$.
Мы измеряем объем, занимаемый газовой частью и нефтью.
То есть, знаем суммарный объем.
И что мы делаем?
Мы открываем вентиль и выпускаем газовую фазу при постоянном давлении, равном $P_2$.
И весь газ у нас при этом выпускается.
Как только газа уже нет, мы закрываем вентиль.
Измеряется компонентный состав выпущенного газа.
Измеряется объем из стандартных условий газа, который был выпущен.
Измеряется объем нефти, которая частично разгазирована, из которой был выпущен газ при давлении $P_2$.
И дальше осуществляется вторая ступень разгазирования.
Всё то же самое.
Мы увеличиваем объем контактным образом, то есть, не изменяя компонентный состав нашей пластовой нефти.
Мы снизили давление до $P_3$ меньше $P_2$, выделился газ.
И дальше мы его опять выпускаем.
Измеряем объем выделившегося газа, его компонентный состав и так далее.
Вот это всё делается в 8-10 ступеней.
Есть рекомендации, которые я вам покажу.
Они были в отраслевом стандарте 1980 года и 2003 года без изменений.
А как делается сейчас, я вам потом скажу.
Сейчас делается под влиянием зарубежных методик несколько иначе.
Просто даже не то, что методик, а то, что усовершенствована аппаратура.
И сейчас просто нет регламента по числу ступеней.
Значит, при атмосферном давлении температура снижается.
Вот мы эксперимент проводим при пластовой температуре.
Достигли одной атмосферы, выделился газ.
Дальше мы снижаем температуру до 20 градусов или повышаем ее до 20 градусов.
Это всё зависит от пластовой температуры, при которой мы проводили эксперимент.
И оставшаяся нефть называется разгазированной.

\begin{center}
\includegraphics[width=\textwidth, page=96]{Брусиловский.pdf}
\end{center}

Теперь следующее.
Вот сколько же нам ступеней осуществить?
Вот в отраслевом стандарте приводилась такая номограмма (график) для определения ступеней.
Мы ведь эксперимент должны проводить, начиная с давления, равного давлению насыщения.
Вот в данном случае, в данном примере, давление насыщения равно 16.9 мегапаскалей.
Мы проводим от оси абсцисс, горизонтальную линию, до пересечения с осью...
От катета до гипотенузы проводим горизонтальную линию до пересечения с гипотенузой.
И опускаем вертикальную линию до нижнего катета.
Что мы...
Мы пересекаем линии с надписями, соответствующими давлению в мегапаскалях.
Если мы пересекаем пунктирную линию, это необязательная ступень.
Если мы пересекаем линию сплошную, это обязательная ступень.
Ну, вот это такой график из опыта был.
Тут никаких специальных обоснований нигде нет.
Вот из опыта исследований в лабораториях такие рекомендации были даны.
Значит, в данном случае что мы видим?
Какие обязательные ступени при снижении давления с 16,9 мегапаскалей?
Значит, мы видим линия обязательно 11 мегапаскалей, 7 мегапаскалей, 3 мегапаскалей и 1 мегапаскалей.
Ну, естественно, потом атмосферное давление.
Вот.
А необязательные пунктирные линии, ну, тоже можно.
Можно проводить исследование.
Вот так.

\begin{center}
\includegraphics[width=\textwidth, page=97]{Брусиловский.pdf}
\end{center}

И приведена таблица результатов.
Пример выдачи результатов при дифференциальном разгазировании пластовой нефти.
Вот рассмотрим с вами, как же ведут себя переменные.
Значит, у нас начальное давление 213.7 бар.
И при этом мы видим, что начальное газосодержание 184.8 м$^3$ на м$^3$.
Как определяется начальное газосодержание?
Мы же его вначале не знаем.
Проводится эксперимент дифференциального разгазирования.
Фиксируется выпуск газа на каждой ступени дифференциального разгазирования.
Потом объем газа по всем ступеням суммируется, и мы получаем величину начального газосодержания пластовой нефти, который здесь и приведен 184.8 м$^3$ на м$^3$.
По результатам дифференциального разгазирования приводятся стандартные таблицы.
И там, в частности, такие данные, как газ, выделившийся по ступеням и суммарно, и газ в растворе.
И вот это абсолютно симметричные данные.
Значит, сначала фиксируется, сколько же газа выделяется, потом уже таким образом определяется, повторяя суммарное количество газа и его количество, остающееся в нефти по ступеням.
Теперь на каждой ступени измеряется объем, занимаемый пробой нефти, начиная от давления насыщения и по мере разгазирования.
Причем объем этот, который занимает на ступени нефть газированная, он измеряется после выпуска газа, остаётся нефть и вот её объем, он измеряется в сосуде PVT.
И дальше мы вспоминаем определение объемного коэффициента, и мы получаем с вами величину объемного коэффициента, который вы видите, который уменьшается при снижении давления ниже давления насыщения.
Объемный коэффициент уменьшается, уменьшается.
На что хочу обратить внимание.
Вот у нас при давлении 1 бар объемный коэффициент равен 1.33 в данном случае.
То есть, значит, объем дегазированной нефти, отнесенный при пластовой температуре, при температуре эксперимента, отнесенный к объему при стандартных условиях, он больше.
Значит, этот объем нефти при пластовой температуре дегазированной больше, чем объем нефти при температуре 20 градусов Цельсия.
Это вот типичная такая величина.
Может быть 1.04, может быть 1.02.
Что это значит?
Сразу можно сказать.
Температура в пласте или нашего эксперимента больше, чем 20 градусов Цельсия.
Потому что объем дегазированной нефти при нашей рабочей температуре отнесенной к объему при 20 градусах Цельсия больше единиц.
Значит, температура больше чем 20 градусов Цельсия.
Если бы температура пластовой была меньше чем 20 градусов Цельсия, то объемный коэффициент при одной атмосфере и пластовой температуре был бы меньше единицы.
Вот я с такими месторождениями сталкивался.
Ну это обычно все-таки пластовая температура большинства месторождений больше 20 градусов Цельсия.
Но есть и пласты с температурой меньше 20 градусов Цельсия.
Вот в Восточной Сибири такие месторождения есть, и в других местах это уже конкретно.
По крайней мере, если вы, то есть зная определение объемного коэффициента, вы сразу можете сказать, у вас пластовая температура меньше или больше чем 20 градусов Цельсия.
По данным разгазирования при одной атмосфере, для одной атмосферы.
И последнее значение это при 20 градусах Цельсия, естественно, что раз мы относим к объему нефти при 20 градусах Цельсия, то есть то объемный коэффициент, если у нас рабочая температура 20 градусов Цельсия, ну она единица.
Теперь газосодержание всегда при одной атмосфере равно нулю, потому что мы... ну прекращается, считаем, что весь растворенный газ вышел.
Значит, еще раз хочу сказать, мы суммируем газ, выделившись на всех ступенях, получаем начальное газосодержание, затем вычитая из этой суммы объем газа, который выделился на ступенях, мы получаем объем газа, который остался в растворе.
И вот так вот вы видите сформированы данные по газосодержанию в зависимости от давления, от начального до нуля.
Теперь плотность... объемный коэффициент газа, что по определению вот вы видите наверху, по определению объемный коэффициент газа так же, как и объемный коэффициент нефти.
То есть объем, занимаемый в пластовых условиях, отнесенный к объему, занимаемый при стандартных термобарических условиях.
И на поведение объемного коэффициента газа в зависимости от давления, оно совершенно иное, чем то, которое демонстрирует объемный коэффициент нефти.
Если объемный коэффициент нефти у нас монотонно уменьшается при снижении давления от давления насыщения до атмосферного, и газ в итоге у нас выходит.
Значит, объемный коэффициент газа, ведь в пластовых условиях его объем значительно меньше, чем при стандартных условиях.
И если у нас газ подчиняется законам идеальных газов, то просто пропорционально давлению, то есть газ идеальный при давлении 100 бар занимает объем в 100 раз меньше, чем при одной физической атмосфере.
Это для того, чтобы порядок понять.
В общем, наши реальные газы, они, конечно, не подчиняются законам идеальных газов, но для оценки вы можете понять.
Поэтому, что объем газа при пластовых условиях, значит, вот может быть на два порядка меньше, чем объем, занимаемый при стандартных условиях.
Ну, короче, пропорционально давлению.
И поэтому при зависимости, значит, вот объемного коэффициента газа, он при низких давлениях имеет значительно больше величину, чем при высоких давлениях.
И пластовая нефть имеет один вид зависимости объемного коэффициента от давления, а газ совершенно другую.
Теперь об объемном коэффициенте газа; можно для него формулу показать, куда входит Z-фактор газа.
Вот крайне правая колонка, все свойства газов, они однозначно связаны с Z-фактором газа или коэффициентом сверхсжимаемости, который характеризует нам отклонение поведения газа от идеального газа.
Я дальше этого коснусь.
И будут даны формулы/корреляции для оценки Z-фактора газа, которые применяются в инженерных расчетах, а также Z-фактор газа, он в исключительных случаях может определяться экспериментально.
Это в том случае, когда у нас газ содержит значительное количество кислых компонентов, сероводорода, диоксида углерода, когда у нас высокие давления, тогда можно экспериментально определять, но это делается, экспериментально определяют Z-фактор редко, а большей частью по корреляциям или по уравнению состояния, когда мы знаем компонентный состав газа.
Так вот, и плотность нефти.
Плотность нефти, она может измеряться денситометром, а может и применяться, применяться формулой для расчета, для расчета плотности нефти, это материальный баланс, тоже формула будет потом показана, которая связывает плотность нефти при текущих термобарических условиях с газосодержанием нефти, плотностью растворенного газа, значит, плотностью сепарированной нефти и объемным коэффициентом нефти.
Здесь показаны экспериментальные значения величины для какого-то из примеров, а дальше, когда мы будем рассматривать расчет свойств нефти по корреляциям и так далее, там будет и про Z-фактор газа более подробно, и про плотность нефти, и там эта формула будет показана, эта формула это материальный баланс.

\begin{center}
\includegraphics[width=\textwidth, page=98]{Брусиловский.pdf}
\end{center}

Типичные графики, они относятся не к тому примеру, но это типичные, которые строятся по результатам дифференциального разгазирования, строятся 5 графиков, 3 графика для пластовой нефти, это зависимость газосодержания пластовой нефти от давления, зависимость объемного коэффициента пластовой нефти от давления, а также зависимость динамической вязкости пластовой нефти от давления, про динамическую вязкость чуть дальше.
И 2 графика строятся для растворенного газа, это зависимость объемного коэффициента газа от давления и зависимость динамической вязкости газа от давления.
Вот здесь приведены графики зависимости газосодержания пластовой нефти от давления, из которого видно, что начальное пластовое давление у нас 270 бар, а давление насыщения у нас значительно ниже, значит в начале у нас газосодержание не меняется, достигли давления насыщения и осуществляется в результате дифференциального разгазирования уменьшение газосодержания пластовой нефти, в данном случае у нас практически прямолинейная зависимость, прямолинейная зависимость.
И надо сказать, что форма зависимости газосодержания при снижении давления ниже давления насыщения, она нам показывает, какой газ растворен, какой газ, не знаем компонент состава, скажем, но по форме этой зависимости мы можем сказать.
В данном случае это сухой газ и у нас в растворённом газе в основном метан, метан содержится.
Если бы у нас было много этана, пропана, бутанов, то у нас не прямолинейная была бы зависимость, а у нас в начале, поскольку эти компоненты, метан легко выходит из раствора и линейно подчиняется так называемому закону Генри.
Значит, а если у нас такие компоненты, как этан, пропан в нефти содержатся, то они с неохотой выходят из раствора, и поэтому в начале вот эта зависимость, она вверх прижимается, а затем уже, когда при более низких давлениях у нас волей-неволей выходят этан, пропан, бутаны, у нас резкое уменьшение газосодержания происходит.
Ну, я вам просто на словах это сказал, то есть не прямолинейная, а несколько криволинейная зависимость.
А если у нас околокритические системы, вот давайте вспомним, там где бурное разгазирование осуществляется, в околокритических системах, то рисуночек для пласта Юрского вспомним, у нас будет резкое выделение газа, а затем оно будет выполаживаться, уже приближаться к нулю, то есть криволинейная зависимость будет.
Но в итоге при снижении давления до атмосферного у нас газосодержание равно нулю.
Поэтому вот такие особенности.
Такие особенности.

\begin{center}
\includegraphics[width=\textwidth, page=99]{Брусиловский.pdf}
\end{center}

Теперь о динамической вязкости пластовой нефти.
Ну, вы видите определение, которое вы знаете с курса физики, хорошо, да?
Значит, динамическая вязкость, значит, очень важная характеристика, которая используется при моделировании гидродинамическом.
И раз мы говорили про дифференциальное разгазирование, значит, экспериментальные данные, которые обязательно измеряются при условиях, соответствующих ступеням дифференциального разгазирования и пластовой температуре, они используются при моделировании разработки месторождений.
Моделирование разработки месторождения осуществляется в предположении того, что фильтрация в пласте осуществляется по закону Дарси.
А в законе Дарси у нас скорость фильтрации обратно пропорциональна динамической вязкости пластового флюида.
И поэтому знание динамической вязкости пластового флюида обязательно для достоверного проектирования разработки.
Вот типичный график того, как ведет себя динамическая вязкость пластовой нефти по результатам дифференциального разгазирования.
Значит, это данные те же, что для двух предыдущих иллюстраций.
Давление начальное у нас 270 бар и давление насыщения значительно ниже, оно порядка 120 бар.
При снижении давления от начального до давления насыщения динамическая вязкость уменьшается, поскольку давление уменьшается.
И практически это происходит линейно.
А затем мы наблюдаем излом, который для всех типов флюидов осуществляется; и при снижении давления ниже давления насыщения у нас происходит разгазирование нашей пластовой нефти.
Газосодержание нефти, нефтяной фазы при снижении давления ниже давления насыщения, оно уменьшается и следовательно увеличивается динамическая вязкость нефтяной фазы.
Вот поэтому это характерная зависимость для поведения динамической вязкости в зависимости от давления для нефтяной фазы.

\begin{center}
\includegraphics[width=\textwidth, page=100]{Брусиловский.pdf}
\end{center}

По величине динамической вязкости пластовую нефть при начальном термобарическом условии классифицируют на несколько типов.
Если у нас динамическая вязкость пластовой нефти при начальных условиях меньше 5 мПа$\cdot$с (мПа$\cdot$с, если вы будете встречать, он равняется сантипуазу).
мПа$\cdot$с в системе СИ, а сантипуазы это в инженерной практике раньше использовали.
В общем, мПа$\cdot$с и сантипуаз это одно и то же, слава Богу.
Незначительной вязкости -- вязкость меньше или равно 5, маловязкая нефть в диапазоне от 5 до 10, повышенная вязкость или просто вязкая от 10 до
30, высоковязкая от 30 до 200 и сверхвязкая более 200.
Теперь чем больше вязкость пластовой нефти, тем меньше скорость ее фильтрации к забоям скважин при той же депрессии и соответственно меньше дебит.
То есть с увеличением вязкости нефти уменьшается дебит ее при той же депрессии.
Теперь для сведения существуют льготы для разработки сверхвязких нефтей и высоковязких нефтей, потому что для того, чтобы компании прилагали усилия для разработки этих активов и для повышения нефтеотдачи.
Какие сверхвязкие нефти, я экспертировал работы Татнефти, там есть большое количество Татбитум, это как раз огромные запасы сверхвязких нефтей.
И там это нефти по плотности битуминозная, давайте вспомним, свяжем вязкость с плотностью битуминозная, с очень маленьким газосодержанием, практически нулевым, поэтому они сверхвязкие.
И для того, чтобы нефть притекала все-таки, применяются термические методы воздействия.
При увеличении температуры вязкость уменьшается и начинается фильтрация такой нефти.
Вот так без применения термических методов воздействия мы практически не можем извлечь эту нефть сверхвязкую.
И очень трудно, естественно, высоковязкую нефть.
В общем, чем выше вязкость нефти, тем сложнее ее разработка и меньше нефтеотдача.

\begin{center}
\includegraphics[width=\textwidth, page=101]{Брусиловский.pdf}
\end{center}

Теперь следующее, это очень важный тоже материал, посвященный ступенчатой сепарации пластовой нефти.

\begin{center}
\includegraphics[width=\textwidth, page=102]{Брусиловский.pdf}
\end{center}

Перед вами принципиальная технологическая схема сбора и подготовки нефти и нефтяного газа.
Они могут быть разными на разных промыслах.
Это одна из возможных технологических схем.
Но смысл всех сводится к чему?
Что нефть добывается, она сепарируется, из нее извлекается газ.
Мы должны достигнуть товарной кондиции.
Вспомним условия по содержанию воды, солей и так далее.
Давление насыщенных паров нефти перед тем, как эта нефть будет сдана в систему Транснефти, то есть, нефть должна быть доведена до товарной кондиции в соответствии с требованиями отраслевого стандарта.
Тут нет стандарта в предприятии.
Тут это даже не ОСТ, а ГОСТ, государственный стандарт.
И мы акцентировали на давлении насыщенных паров.
Я рассказывал об этом достаточно подробно.
Меньше 0.667 бар и температуре 37.8 градусов Цельсия.
И для особо легких нефтей, я напоминаю, это большие легкие фракции, нефти, они улетучиваются.
И должны либо смешивать с более тяжелой нефтью для удовлетворения этих товарных кондиций.
И практически воды не должно быть, меньше 0.5\% допускается в нефти товарной кондиции.
Ну и по сероводороду, по содержанию хлористых солей там серьезные требования.
То есть в результате нефть мы добываем и она проходит через автоматизированную групповую замерную установку ОГЗУ и поступает на дожимную станцию, на ДНС, где производится/осуществляется первая ступень сепарации.
Давление на первой ступени сепарации от 8, может быть от 6, до 10 атмосфер обычно.
Чем выше газосодержание пластовой нефти добываемой, тем выше давление первой ступени на ДНС.
Но как правило оно не превышает 10 бар.
Выделившийся газ уходит на установку подготовки газа и дальше на газоперерабатывающий завод.
Так происходит сейчас систему газопроводов.
Ну то есть газ сейчас запрещено сжигать.
Это очень важное обстоятельство.
Раньше просто было видно, как огромное количество факелов и не технологических, а именно огромное количество вот этих факелов, где сжигали попутный газ.
Ну это давно уже было, но это было.
Просто вот и по цвету факелов можно было сказать.
Растворенный газ сухой метановый, это голубой цвет, значит метан в основном растворен в пластовой нефти.
Если желтый, значит там уже этан, пропан, если еще и красноватый, значит есть бутаны, жирный газ.

Так вот, первая ступень.
И вот я сказал, по цвету газа, который выделяется на первой ступени сепарации можно определить, какой же растворенный газ в пластовой нефти.
Затем нефть уже, она насосами перекачивается на установку подготовки нефти, где осуществляется вторая ступень сепарации, и давление на второй ступени сепарации, оно порядка 3-4 атмосферы.
Всё по-разному может быть, но я порядок говорю.
То есть, давление снижается, газосодержание этой нефти, естественно, оно уже гораздо меньше, чем газосодержание нефти пластовой.
Потому что давление-то низкое уже, низкое давление, и основная часть газа уже выделилась на первой ступени сепарации.
И, кроме того, на ДНС уже нефть пришла в двухфазном состоянии, естественно.
Это газожидкостной поток, и там на первой ступени из нефти выделяется то, что можно было выделить при давлении на этой ступени, и уже свободный газ этого двухфазного потока.
Вот это уходит на установки подготовки газа.
Значит, на второй ступени у нас газ выделился, и в лаборатории обычно трёхступенчатую сепарацию моделируют.
Третья ступень -- это считается уже стандартное условие.
На практике, что происходит после второй ступени сепарации?
У нас происходит стабилизация нефти.
Поскольку итог должен быть, что нефть должна быть сдана в систему Транснефти, то должны добиться того, чтобы давление насыщенных паров было не больше 500 миллиметров ртутного столба при температуре 37.8 градуса Цельсия.
Для этого, вот если кто на промысле на практике был или будет потом и работает, обратите внимание, значит, нефть поднимается на эстакады где-то метров 5, и из неё выделяется газ, и она, кроме того, происходит ещё и подогрев нефти, и в результате давление насыщенных паров становится равным требованиям.
Не превышает 500 миллиметров ртутного столба.
Дальше нефть уже подготовленная, значит, она насосами перекачивается и уже готова к сдаче в систему Транснефти.
Значит, нам важно что?
Для моделирования, для понимания того, что в лаборатории делается, вот обычно там делается трёхступенчатая сепарация, которая схематично передаёт нам подготовку нефти на промысле.

\begin{center}
\includegraphics[width=\textwidth, page=103]{Брусиловский.pdf}
\end{center}

Значит, чем отличается от дифференциального разгазирования?
Ну, принципиально отличается, потому что при дифференциальном разгазировании мы проводим эксперимент при пластовой температуре, и наша задача получить зависимость свойств нефти от давления при пластовой температуре.
А задача моделирования ступенчатой сепарации -- это моделирование при термобарических условиях, соответствующих подготовке нефти на промысле.
Поэтому, что делается в лаборатории?
Вот вначале загружается в сосуд PVT, нефть, создаётся температура равная пластовой, и нефть приводится к давлению насыщения.
Можно приводить к начальному пластовому давлению, можно к давлению насыщения.
Если мы приводим к давлению насыщения, мы затем можем воспользоваться результатами PV-соотношений для того, чтобы уточнить объёмный коэффициент нефти при пластовых условиях по результатам ступенчатой сепарации.
У нас, повторяю, объёмный коэффициент нефти и газосодержание по результатам однократного разгазирования, дифференциального разгазирования при пластовой температуре и ступенчатой сепарации будут разными.
Чем меньше газосодержание пластовой нефти, тем меньше будут различаться результаты.
Но при повышенных газосодержаниях заметно различаются величины, полученные по результатам различных экспериментов.
И результаты экспериментов ступенчатой сепарации где используются?
Используются в подсчёте запасов.
Мы привели, загрузили нашу нефть в сосуд PVT, привели к термобарическим условиям, соответствующим пластовым, температура пластовая и либо давление насыщения, либо пластовое давлению.
Дальше.
А дальше следующая ступень.
Мы создаём условия, соответствующие условиям на ДНС, на первой ступени сепарации.
И там температура уже отличается от пластовой.
И давление то, о котором я говорил, оно значительно меньше, чем пластовое, естественно.
Она соответствует обычно где-то 6-10 атмосфер.
Но бывает и выше, это в случае повышенного газосодержания.
Теперь на первой ступени ДНС, значит, когда мы моделируем, у нас газ выделился из пластовой нефти.
И мы при давлении, соответствующем давлении на ДНС, весь газ, который выделился, выпускаем из сосуда PVT.
Измеряется его компонентный состав, измеряется его объём при стандартных условиях.
И дальше мы поджимаем, значит, вот мы поджимаем поршнем, это рабочая жидкость, я не буду акцентировать, но она разная, значит.
Совсем давно ртуть использовали, потом запретили её использовать, использовали жидкометаллический сплав, разные могут быть составы.
Важно, чтобы мениск между газом и жидкостью был минимальным.
И чтобы это не вредные были условия для проводящих эксперимент.
Это очень важно.
Значит, теперь выпустили газ, и что?
Нефть идёт на вторую ступень сепарации.
Создаём условия, соответствующие второй ступени сепарации, то есть давление ниже, температура, обычно в экспериментах она постоянная на различных ступенях, это проще.
Но если нужно какую-то температуру создать, это всё делается.
Значит, часто при 20 градусах Цельсия, но вообще может быть температура и выше, и выше, чем 20 градусов Цельсия и на первой, и на второй ступени сепарации.
Значит, при снижении давления на второй ступени у нас из той нефти, которая с ДНС пришла, выделяется газ.
Этот газ, я хочу отметить, гораздо более жирный, чем газ первой ступени.
Я на примере это покажу только в том случае, когда у нас тяжёлые нефти, битуминозные, по типу битуминозной нефти, в них всегда растворён газ чисто метановый, или очень-очень похожий на метановый.
Вспомните таблицу, я вам показывал состав газа свободного Сеномана, вот он метановый.
Вот такой газ растворён в пластовой нефти, тяжёлых нефтей и битуминозных.
В битуминозных совсем мало там газа растворено, в общем для тяжёлых нефтей газ сухой растворён практически всегда.
И на второй ступени газ выделился, мы его выпустили, и дальше газ, создаются условия, приводится к стандартным условиям, приводится к стандартным условиям уже, то есть условиям стоктанка.
И выделяется газ, если там не сухой газ был изначально, то выделяется газ, он самый жирный, но его совсем немного, его совсем немного, потому что давление-то на второй ступени невысокое, и мы приводим к одной атмосфере, но там не было высокого газосодержания, поэтому газа не может выделиться много, вот так.
Теперь на каждой ступени у нас объём газа выделившегося фиксируется, определяется его компонентный состав, определяются его свойства, определяются на последней ступени объём и характеристики дегазированной нефти, плотность, молярная масса, объёмы, деление объёма газа выделившегося на ступенях, рассчитывается газосодержание на каждой ступени, газосодержание на ступени -- это отношение газа выделившегося на данной ступени, вернее, я неправильно говорю, отношение газа, который, значит, газ выделившийся и газ в растворе, вот так, газ выделившийся при суммировании по всем ступеням равен, значит, сумме растворённой, в общем, величине растворённого газа в пластовой нефти, вот, и объём, суммарный объём выделившегося газа на всех ступенях, отнесённый к объёму или массе дегазированной нефти даёт нам газосодержание по результатам ступенчатой сепарации.
Значит, отношение, значит, объёма образца пластовой нефти при пластовых термобарических условиях, то есть начальное, отнесённое к объёму дегазированной нефти, даёт нам величину объёмного коэффициента, обратная величина, пересчётный коэффициент используется в подсчёте запасов и делается это по результатам именно эксперимента ступенчатой сепарации.

\begin{center}
\includegraphics[width=\textwidth, page=104]{Брусиловский.pdf}
\end{center}

Ну, приведена схема, но тут я вам практически всё уже рассказал, то есть, пробы из сепараторов, газа определяется компонентный состав, зная компонентный состав, мы можем определить и молекулярную массу, и относительную плотность газа, значит, и пробы жидкости разгазируются, определяется объём выделившегося газа и его компонентный состав, объём нефти и, значит, её компонентный состав, плотность и молекулярный вес нефти.

\begin{center}
\includegraphics[width=\textwidth, page=105]{Брусиловский.pdf}
\end{center}

И вот формула, как, значит, по результатам ступенчатой сепарации также можно определить компонентный состав пластовой нефти, но формула более ёмкая, но в технических отчётах по исследованию пластовой нефти приводится, значит, компонентный состав пластовой нефти, полученный как по результатам однократного разгазирования, так и по результатам ступенчатой промысловой сепарации.
И они отличаются мало.
Они не должны в идеале отличаться, потому что это компонентный состав одной и той же пластовой нефти.
Но, естественно, есть всегда погрешность в измерениях, и, значит, меньшая погрешность имеет место при меньшем числе измерений, значит, по результатам однократного разгазирования и стандартной сепарации мы имеем более точный компонентный состав пластовой нефти, мы на него обычно ориентируемся.
Но имеет смысл сравнивать компонентный состав при однократном разгазировании и ступенчатой сепарации для того, чтобы убедиться, что не было каких-то грубых ошибок.
Они при нормальном проведении эксперимента всегда очень близки.
Так вот, здесь мы видим, это по результатам двухступенчатой сепарации, не трехступенчатой, а двухступенчатой сепарации.
И, сейчас, одну секунду, просто вспоминаю.
Мне казалось, что я формулу расчёта состава, компонентного состава пластовой нефти, я тоже показал, чтобы было понятно, с числом молей.
Но что-то я не вижу, но не важно.
Значит, опять же, как получают эту формулу?
Определяя число молей, получая число молей выделившегося газа на первой ступени сепарации, число молей выделившегося газа на второй ступени сепарации и число молей дегазированной нефти соответствующей.
И по компонентному составу газосепарации первой ступени, второй ступени и состава дегазированной нефти определяют число молей по компонентам.
Кроме того, определяют молекулярную массу дегазированной нефти, определяют её относительную плотность, это $\gamma_o$.
И вот таким образом определяют по результатам ступенчатой сепарации компонентный состав пластовой нефти.
Причём здесь предусмотрено даже то, что у нас температура Stock Tank Oil в этой формуле.
$T_s$ -- это температура на первой ступени сепарации.
$T_{sc}$ -- это температура при стандартных условиях, standard condition.
А возможность ситуации, когда у нас температура Stock Tank Oil отличается от температуры при стандартных условиях, ну вот это тогда корректируется, это учтено, значит формула соответствующая выведена мной с учётом того, что температура Stock Tank Oil может отличаться от стандартной.
И в этом случае, да.
А когда они не отличаются, ну значит у нас этот соответствующий сомножитель во втором члене, сумма, что числитель и знаменатель равен 1.
Ну, то есть это наиболее общий случай.
Если кого-то заинтересует вывод этой формулы, я могу прислать рабочий листок, где просто это всё выведено.
Ну это такая специфическая вещь.

\begin{center}
\includegraphics[width=\textwidth, page=106]{Брусиловский.pdf}
\end{center}

Теперь, вот пример расчёта состава пластовой нефти по данным двухступенчатой сепарации.
Это вот, значит, что у нас дано.
У нас даны условия на ступенях сепарации, у нас дан состав газа, выделившегося на первой ступени сепарации, у нас есть состав газа, который выделен на второй ступени сепарации, то есть в стоктанке, и состав дегазированной в стоктанке нефти.
Значит, и нам нужно найти плотность, спрашивается, найти плотность газа сепарации первой и второй ступени сепарации и состав пластовой нефти.

\begin{center}
\includegraphics[width=\textwidth, page=107]{Брусиловский.pdf}
\end{center}

Значит, мы с вами определяем по исходным данным, $R_1$ -- это
газосодержание, это отношение объёма газа, выделившейся на первой ступени сепарации, к объёму нефти, выделившейся в стоктанке.
Значит, по исходным данным, это затем мы определяем с вами по объёму газа, который выделился в стоктанке, газосодержание нефти, поступившей с первой ступени сепарации на стоктанк.
Просуммировав $R_1$ и $R_2$, мы получаем газосодержание пластовой нефти.
Теперь как мы, значит, вот внизу, внизу показано, как мы определяем плотность газа, выделившегося в сепараторе первой ступени.
Это мы по компонентному составу определили молекулярную массу газа, выделившегося на первой ступени, отнесли к молекулярной массе воздуха, получили относительную плотность газа, который выделился на первой ступени.
И умножив, если мы хотим знать абсолютную плотность газа, выделившегося на первой ступени, мы должны умножить на абсолютную плотность воздуха -- это 1.205 кг/м$^3$.
Мы получаем, что на первой ступени у нас выделился газ с плотностью 0.848 кг/м$^3$.
Это газ такой полужирненький.
И давайте посмотрим на его состав.
Там почти 78\% мольных метана, достаточно много этана, что характерно для газа, растворённого в нефти.
В нефти лёгкой, либо средней, ну вот так вот.
В общем, вот такая концентрация этана, она характерна для газов нефтяных месторождений, растворённых в пластовой нефти.
В свободных газовых месторождениях этана с концентрацией 15\% практически не наблюдается.
Значит, это на первой ступени выделился газ.
А на второй, из стоктанка, вот соседняя колонка, газ гораздо более жирный, он просто жирный, 34\% всего-навсего метана, зато этана 32\%, пропана 25\%.
То есть всегда, повторяю, для обычных нефтей, где растворён газ, содержащий этан, пропан, бутаны, газ, выделяющийся в стоктанке гораздо более жирный, чем в первой ступени сепарации.
Ну и если это трёхступенчатая сепарация, значит, всегда более сухой газ на первой ступени, затем более жирный и самый жирный.
Но другое дело, что наибольший объём газа выделяется всегда на первой ступени сепарации, и он определяет тот компонентный состав, который идёт на переработку и который вот экономически оценивается, экономической целесообразности его использования.
Это первая ступень сепарации, потому что на второй ступени сепарации давление значительно меньше, и там уже просто по объёму этого газа небольшой, со стоктанка вообще незначительный, просто раньше его на факел или в технологических целях каких-то использовали.
Значит, вот так, вот таким образом.
И по формуле, которую я показывал, зная газосодержание, значит, вот газы $R_1$, $R_2$, зная компонентный состав газа, выделевшегося на первой ступени $y_i^{(1)}$, на второй ступени $y_i^{(2)}$, компонентный состав дегазированной нефти, её молекулярная масса замерена, мы определяем компонентный состав пластовой нефти.
Вот он, крайняя правая колонка, состав пластовой нефти, вот мы таким образом определили.

\begin{center}
\includegraphics[width=\textwidth, page=108]{Брусиловский.pdf}
\end{center}

Бывают случаи, и их немало, когда у нас объём газа, выделяющегося из стока танка, не замеряется.
И вот представлены рекомендации для инженерных расчётов, каким образом оценить газосодержание пластовой нефти, если у нас нет данных по газу выделяющемуся в стоктанке.
Это для случаев двухступенчатой сепарации.
Значит, у нас есть данные по газосодержанию на первой ступени сепарации, обычно, это объём газа, выделевшегося на первой ступени, отнесённый к объёму дегазированной нефти, а газосодержание на второй ступени, оно оценивается по корреляции, причём есть разные корреляции.
Но более ранние из них, это корреляции Стендинга-Каца.
Стендинга-Каца, они активно работали в 40-х, 50-х годах
вместе.
И очень много было сделано этими специалистами, это ведущие американские специалисты в области Кац, газовой технологии, исследований различных аспектов, касающихся природных газов.
Стендинг -- это исследование пластовых нефтей, классик по исследованию пластовых нефтей и газоконденсатных систем.
Это я заодно вам говорю, что Кац -- это выдающийся американский учёный-инженер в области технологии разработки месторождений и исследования природных газов.
В 60-е годы был переведён справочник американский толстенная книга под редакцией Егора Павловича Каратаева, он был заведующий кафедрой разработки газово-газоконденсатных месторождений в Губкинском институте и трижды лауреат государственной премии.
Известный учёный, значит, был моим научным руководителем, когда я в аспирантуре учился.
И вот Катц, его руководство по всем аспектам исследования, разработки и так далее, природных газов, толстенная книга, сейчас это, конечно, библиографическая редкость, её сейчас нет, но это вот оттуда-оттуда взяты многие зависимости, графики, методы и так далее.
И Стендинг, автор монографии по исследованию природных углеводородных систем, замечательная монография, которую мне, слава Богу, досталось, тоже я её читаю и перечитываю, не перестаю удивляться глубине и знания, также, как и монография на Намиота.
В общем, это классики, уже ушедшие от нас, но это вот выдающиеся специалисты.
Так вот, Стендинг-Кац, корреляция, она позволяет оценить газосодержание, то есть объём газа, выделившийся на стоктанке, отнесённый к объёму дегазированной нефти.
И здесь, что нам нужно знать?
Давление и температуру на первой ступени сепарации, потому что от них зависит объём нефти, который ушёл в стоктанк, и, соответственно, газосодержание этой нефти, естественно.

\begin{center}
\includegraphics[width=\textwidth, page=109]{Брусиловский.pdf}
\end{center}

Плюс, значит, относительную плотность этой нефти, в стоктанке.
И вот и всё.
Значит, это вот $R_2$ рассчитали и прибавили к $R_1$, получили полное газосодержание пластовой нефти.
Теперь, но нам нужно, если вы обратите внимание, вот в формулах, сейчас, здесь вот относительная плотность газа $A_1$, $A_2$, $A_3$.
И, в общем, нам нужно ещё с вами оценить плотность газа, плотность газа, который выделился в стоктанке, это вот $\gamma_{gST}$, значит, по формуле $A_2$ плюс $A_3R_2$.
Значит, мы сначала $R_2$ определили, а потом уже плотность этого газа.
И после этого мы можем определить плотность растворённого газа в пластовой нефти.
Это просто осреднение в соответствии с газосодержанием на первой и второй степенях сепарации.

\begin{center}
\includegraphics[width=\textwidth, page=110]{Брусиловский.pdf}
\end{center}

Мы с вами, задачка, которая по следам этой корреляции вот оценить, сейчас.
Значит, в каких случаях вот эти вот корреляции используются?
Это уже на промысле, не в лаборатории.
В лаборатории там просто экспериментом определяют точно газосодержание.
А вот когда нам нужно на промысле определить газосодержание пластовой нефти, мы замерили вот те исходные данные, которые приведены.
На промысле получен за одно и то же время объём нефти дегазированной 58.83 кубометра.
Объём газа соответствующий.
И объём дегазированной нефти, то есть из стоктанка, а объём газа сепарации, это газ именно первой ступени сепарации.
Значит, измерили в лаборатории плотность нефти, поделили на плотность воды, получили относительную плотность дегазированной нефти, молекулярную массу её, измерили криоскопическим методом и измерили компонентный состав газа, выделившегося на первой ступени сепарации.
И нам даны условия по условиям сепарации первой ступени.
И мы можем воспользоваться формулой, которая для того, чтобы рассчитать $R_2$, ну $R_1$, значит, $R_1$, рассчитаем $R_2$ по корреляциям и так далее.

\begin{center}
\includegraphics[width=\textwidth, page=111]{Брусиловский.pdf}
\end{center}

Мы получим с вами газосодержания.
Ну вот всё вам понятно.
$R_1$ -- объём газа сепарации к объёму нефти.
Определяем молекулярную массу газа, выделевшегося на первой ступени.
По компонентному составу определяем относительную плотность этого газа.
Затем мы должны с вами $R_2$ оценить с учётом давления $P_s$ первой ступени сепарации и температуры на первой ступени сепарации.

\begin{center}
\includegraphics[width=\textwidth, page=112]{Брусиловский.pdf}
\end{center}

Мы рассчитываем эти эмпирические коэффициенты $A_1$, $A_2$, $A_3$.
Определяем, значит, сколько выделилось у нас газа на второй ступени, $R_2$, удельное количество, на метр куб дегазированной нефти.
Вы суммировали, получили газосодержание пластовой нефти.
То есть, это всё можно оперативно делать, не имея результатов исследований глубинных проб.
Ну и вот, когда мы хотим понять, что за пластовая нефть.
Что за пластовая нефть у нас.
И вот так мы поступаем.

\begin{center}
\includegraphics[width=\textwidth, page=113]{Брусиловский.pdf}
\end{center}

\begin{center}
\includegraphics[width=\textwidth, page=114]{Брусиловский.pdf}
\end{center}

Ну вот, у нас практически сейчас 25 минут первого, и у нас следующая тема -- это исследование газоконденсатных систем.
Я завтра с утра начну эту тему, сегодня смысла нет уже начинать.
Вот это важная тема.
Мы с вами завтра разберём, значит, тему по исследованию газоконденсатных характеристик.
Дальше мы с вами рассмотрим коротко обзор по теоретическим методам исследования природных углеводородных систем.
Это что лежит в основе, значит, расчёта фазовых равновесий.
Очень коротко.
Исходя из фундаментальных основ термодинамики многокомпонентных систем.
Значит, какие уравнения состояния сейчас применяются для расчёта фазового равновесия, и почему, собственно говоря, мы их используем, можем использовать.
Ну и дальше, в соответствии с содержанием нашего курса, уже мы рассмотрим расчёт свойств нефтей.
Да, будут примеры даны, значит, моделирование свойств нефти с применением уравнений состояния.
То есть мы с вами рассмотрели экспериментальные исследования, и мы с вами уже знаем терминологию.
Вот теперь, значит, мы перейдём к тому, какие же расчёты проводятся, основные моменты.
И с вами интересно будет посмотреть на характеристики нефтей с повышенным газосодержанием и с низким газосодержанием по результатам моделирования свойств этих нефтей, PVT-свойств.
Затем мы перейдём к применению корреляций для нефти.
Тут тоже краткий обзор будет.
И я хочу обратить ваше внимание, что дан файл экселевский по оценке свойств нефти, газа и воды с применением корреляций.
Я знаю, что программу делал Александр Иванович Адегов, он был вообще в своей деятельности в Роснефти увлечён применением корреляций.
Я занимался использованием уравнений состояния и в Академии наук работая, и здесь всегда использовал уравнение состояния.
А он в Роснефти, потом он перешёл к нам, несколько лет работал, он акцентировал своё внимание на использовании корреляций.
Значит, материал по корреляциям для нефтей рассмотрим, затем мы рассмотрим особенности, которые присущи пластовым флюидам газоконденсатных месторождений, они очень сильно отличаются от свойств пластовых нефтей.
Ну просто принципиально, совершенно я на этом внимание ваше акцентирую.
Не на теории, а на то, на что нужно вам на практике обращать внимание как инженеру, те, кто будут заниматься проблематикой разработки газоконденсатных месторождений, и затем какие корреляции используются для оценки свойств природных газов, и в конце очень совсем коротко о корреляциях, используемых для оценки свойств пластовой воды.
Вот так.
И сейчас мы на 115 слайде, вас не должно смущать, что осталось два дня и нам ещё много-много слайдов.
Я в последний день обычно рассказываю про корреляции, и это много слайдов, но это всё мы успеваем, потому что там особенностей не так много.
В общем, всё мы успеем, и давайте, до завтра, до завтра, и желаю вам всего хорошего.
И обращайте внимание на те особенности, на которые я акцентирую внимание, потому что всё это через себя пропущено, и опыт, и так далее, и в учебниках вы это не прочитаете.
Всего хорошего.

\end{document}
