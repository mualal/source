\documentclass[main.tex]{subfiles}

\begin{document}

\section{Лекция 25.10.2022 (Брусиловский А.И.)}

\subsection{Исследования пластовой нефти, в которых определяются базовые параметры}

\begin{center}
\includegraphics[width=\textwidth, page=69]{Брусиловский.pdf}
\end{center}

Итак, давайте начнем наш учебный день.
Значит, очень важная тема.
Темы все важные, в том числе и та, которую сейчас мы будем рассматривать.
Это исследование пластовой нефти.
Исследования проводятся в лабораториях, и можно проводить математическое моделирование, создав модель пластовой нефти многокомпонентной, и с
применением уравнения моделировать те процессы, которые осуществляются при лабораторных исследованиях.
При этом, зная результаты лабораторных исследований, можно адаптировать математическую модель под результаты экспериментальных исследований.
Это очень хорошая будет модель для дальнейшего использования при проектировании, разработке, эксплуатации, расчете различных процессов, течения в скважине, на промысле, с сепарацией и так далее.
И в том числе при композиционном моделировании.
И при моделировании процессов с помощью моделей типа Black Oil.

Итак, разделяют различные виды разгазирования.
Контактное разгазирование -- это процесс, когда выделившийся газ не удаляется, а находится в равновесии с жидкой фазой.
Состав смеси неизменен, постоянен.
Частным случаем контактного разгазирования является так называемая стандартная сепарация, когда контактное разгазирование осуществляется при температуре 20 градусов Цельсия и давлении 1 физическая атмосфера.
Допускается текущее атмосферное давление в лаборатории.
Когда мы проводим математическое моделирование, мы задаем 20 градусов Цельсия и указанное давление 1 физическая атмосфера.
За рубежом, я уже говорил, температура стандартная является 60 градусов Фаренгейт или 15,56 градуса Цельсия.
Почему я параллельно говорю об условиях за рубежом?
Потому что практически во всех лабораториях сейчас оборудование поступившее из-за рубежа, оно более совершенное, чем разработанное у нас ранее.
Затем методик много, которые опробированы за рубежом крупными компаниями типа Schlumberger, Halliburton и других.
И они используются у нас в лабораториях.
И программные комплексы тоже в основном используются те, которые разработаны зарубежными специалистами, это Eclipse и другие.
Eclipse, потому что в компании Газпромнефть этот комплекс давно используется.
Он включает в себя модуль PVTI, в котором моделируются процессы исследования природных углеводородных систем.
И можно рассчитывать фазовые равновесия природных углеводородных
систем, задав их состав и соответствующие термобарические условия, выбрав соответствующий тип экспериментов.
Причём стандартная сепарация, это однократное разгазирование при стандартных термобарических условиях, я немножко разжёвываю, чтобы вам лучше было понятно, является базовым экспериментом для последующего расчёта состава пластовой нефти (компонентного состава пластовой нефти).
Формулы я вам покажу, и вам будет всё понятно.

Дифференциальное разгазирование, которое по самому своему названию, по терминологии предопределяет, что эксперимент проводится не при постоянном составе смеси и вот, значит, осуществляется разгазирование, отводится газ и на практике это осуществляется ступенчато; мы более подробно рассмотрим.
А описывается система обыкновенных дифференциальных уравнений, я более подробно об этом скажу, когда буду рассказывать про газоконденсатную систему.
Итак, стандартные.

И третий тип, это ступенчатая сепарация, которая моделируется в условиях промысловой сепарации нефти с применением нескольких ступеней сепарации, поэтому называется ступенчатая сепарация, при термобарических условиях, соответствующих системе промысловой сепарации на промысле.

Итак, стандартная сепарация, дифференциальное разгазирование при пластовой температуре и ступенчатая сепарация, моделирующая промысловую подготовку.
Ну и это основные эксперименты, мы рассмотрим их результаты.

\begin{center}
\includegraphics[width=\textwidth, page=70]{Брусиловский.pdf}
\end{center}

Следующий слайд, вот слайд с системой добычи и подготовки нефти.
Значит, в случае, когда у нас давление в пласте превышает давление насыщения пластовой нефти, то есть нефть, как углеводородная система, находится в однофазном состоянии, разгазирование еще не происходит, давление выше давления насыщения.
Вода всегда есть в пористой среде, она может быть неподвижной, а может в результате осуществления процесса поддержания пластового давления быть подвижной, поступать на забой скважины.
Но нас интересует в данном случае, как ведет себя углеводородная система.
Повторяю, это случай, когда у нас в пласте углеводородная система имеет давление выше давления перехода в гетерогенное состояние, то есть выше давления насыщения.
В стволе скважины уже осуществляется разделение на газ и жидкость, потому что давление в стволе скважины уменьшается, и по достижении давления насыщения и дальнейшем уменьшении давления из пластовой нефти выделяется газ.
Происходит это в стволе скважины -- контактное разгазирование.
Состав текущей смеси неизменен, что поступило на забой скважины, то и выносится на устье.
То есть для текущей смеси в стволе скважины состав не меняется.
Поэтому, когда мы моделируем этот процесс, то мы используем процесс контактной конденсации, то есть при неизменном составе нашей смеси.
А давление изменяется за счет изменения объема, рабочего объема сосуда.
Поступающая с устья скважины смесь течет в сепаратор.
Здесь показана двухступенчатая сепарация.
Первая ступень -- это сепаратор, а вторая ступень -- это так называемый стоктанк или нефтехранилище.
В общем, в сепараторе поддерживается давление, $P_{\text{сеп}}$ и температура, выделяется газ, голубеньким –- это вода и зелёным –- это нефть.
В сепараторе в соответствии с заданным давлением сепарации осуществляется контактное разгазирование из многокомпонентной системы.
Она разделяется уже окончательно в сепараторе.
Считается, что соответствует состоянию равновесия термодинамического.
Достаточно большой по объему аппарат.
Выделившийся газ, газовая фаза, направляется на установки подготовки газа, а вода выделившаяся должна поступать в систему подготовки воды, так называемой подтоварной воды, и для дальнейшего её использования в системе поддержания пластового давления, если осуществляется заводение.
А нефть поступает в стоктанк.
В стоктанке поддерживается давление одна физическая атмосфера и температура 20 градусов Цельсия, когда мы моделируем или в лаборатории.
Стандартные термобарические условия.
Вот это двухступенчатая система.
Ну вот, и вот вы видите, что все стрелочки, уже вам всё понятно.
Из стоктанка, поскольку у нас давление снизилось до стандартного одной атмосферы, выделяются остатки газа, который либо сжигается, либо добавляется к газу, который на собственные нужды и так далее.
На практике.

\begin{center}
\includegraphics[width=\textwidth, page=71]{Брусиловский.pdf}
\end{center}

Следующий вариант -- это когда у нас давление в пласте ниже давления насыщения.
Всё похоже, но в пласте у нас, поскольку давление ниже, чем давление насыщения пластовой нефти, углеводородная система уже не гомогенная, не однофазная, а двухфазная.
Одна фаза жидко-углеводородная, а вторая газовая фаза, которая выделилась из пластовой нефти при снижении давления ниже давления насыщения.
Вот всё это поступает в ствол скважины.
У нас уже в стволе скважины, начиная с забоя, углеводородная система находится в двухфазном состоянии.
С точки зрения математического моделирования, поскольку это контактная конденсация, контактная конденсация моделируется, то есть фиксируется состав пластовой нефти, который неизменен, ещё раз, при течении от забоя до устья, но поскольку меняются термобарические условия, давление падает и температура тоже изменяется, у нас соотношение между газовой и жидко-углеводородными фазами непрерывно меняется.
Это приводит к изменению, значит, вот это соотношение между фазами к изменению структур течения.
Я здесь не рассматриваю, но вы уже наверняка знаете, что при течении в стволе скважины, значит, при изменении соотношения объёмов жидкой фазы и газовой фазы у нас структуры течения меняются.
Я показывал вам, давайте вспомним слайд, когда мы говорили об условиях отбора проб.
И вот там, значит, на забой поступала нефть в однофазном состоянии, пробоотборник у нас уже был в условиях пузырькового режима, когда у нас пузырьки газа уже выделились из нефти, а по мере снижения давления при движении к устью скважины у нас консолидируются пузырьки газа, они становятся, значит, уже образуются глобулы газа достаточно большие и режимы течения меняются.
Каждому режиму течения соответствуют свои гидродинамические критерии.
И это предмет самостоятельного изучения течения газожидкостных смесей в скважинах и трубопроводах.
Это самостоятельная тема, весьма ёмкая и десятилетиями изучается.
Особенности, критерии и так далее.
Всё это реализуется потом в программных комплексах, результаты экспериментальных исследований, что позволяет достаточно точно моделировать условия течения углеводородных смесей в системах трубопроводов при добыче и подготовке нефти.
И это же относится к газоконденсатным системам.
Всё это единая наука, просто соотношение между фазами в нефтяных системах и в газоконденсатных, оно различное, естественно.
В газоконденсатных системах значительно больше объём газовой фазы, соответственно, превалируют другие структуры течения и так далее.
А так наука гидродинамика газожидкостных потоков этим занимается.
А с точки зрения PVT-свойств мы должны уметь обеспечить исходной информацией, необходимой информацией вот эти расчёты гидродинамические.
Это осуществляется, раньше это осуществлялось экспериментальными исследованиями при температурах отличных от пластовой, от 20 градусов Цельсия до пластовой и между ними всё это экспериментально исследовалось, процесс вопроса контактных конденсаций.
А сейчас экспериментальные исследования проводятся наиболее часто при пластовой температуре, чтобы обеспечить информацией о том, каковы свойства фаз в пласте.
А мы создаём, повторяю, математические модели, основанные на применении уравнений состояния и уже создав математическую модель пластовой нефти, мы можем при различных термобарических условиях рассчитывать контактную конденсацию.
Это в программах зарубежных называется FLASH эти процессы.
Значит, контактная конденсация.
И в профессиональных программах, я понимаю, HI-SIM, HI-SYS и так далее, значит, там осуществляется расчёт фазовых превращений углеводородных смесей на основе заданного состава пластовой смеси, которая должна быть адаптирована к имеющимся экспериментальным данным.
Вот такая, значит, логика.
Ну вот, я вам устно рассказал, теперь вам совершенно очевидно должно быть.

\begin{center}
\includegraphics[width=\textwidth, page=72]{Брусиловский.pdf}
\end{center}

То, что вы прочитали, ещё раз уже тут написано на слайде, значит, это как повторение.
Теперь важно, что, ещё раз хочу сказать, что результат дифференциального разгазирования используется для идентификации изменения свойств углеводородных фаз в пласте при изменении давлений, при, значит, давлении ниже давления насыщения, то есть, когда у нас в пласте находятся две фазы.
Почему я вам вот так тщательно это дело объясняю?

Значит, всё уже понятно, что контактное разгазирование осуществляется, считается, что в стволе скважины, в трубопроводах промысловых, в сепараторах и моделируется математически контактным разгазированием.
То есть, мы задаём компонентный состав смеси и, значит, в результате получаем компонентный состав и относительное количество равновесных фаз.
Соответствующую постановку математическую я расскажу, когда буду делать обзор, краткий обзор по применению уравнения состояния.
Значит, а дифференциальное разгазирование, о котором сейчас пойдёт речь, когда я буду рассказывать тоже о сути экспериментальных исследований, значит, оно позволяет нам, как вы видите написано, вот это вот нужно чётко понимать, определить свойства равновесных фаз паровой-углеводородной и жидко-углеводородной сосуществующих в пласте при давлении меньше, чем давление насыщения.
И тоже будет пример, значит, результат дифференциального разгазирования и видно будет, как же меняется с давлением свойства углеводородных фаз в пласте при постоянной пластовой температуре и изменении давления.
Вот это вот суть процесса дифференциального разгазирования контактного.

\begin{center}
\includegraphics[width=\textwidth, page=73]{Брусиловский.pdf}
\end{center}

Теперь, значит, контактное разгазирование, англоязычный термин Flash и пояснение, у нас есть мольный состав смеси, который обозначается $Z_i$, и давление, при котором эта смесь загружена в сосуд PVT, часто говорят бомба PVT, сосуд высокого давления, значит, предположим, что мы загрузили 1 моль в смеси, находится наша смесь в однофазном состоянии, дальше мы за счёт изменения объёма, рабочего объёма этого сосуда уменьшаем давление и появились первые, видите, вот 3 точечки, 3 жёлтенькие, это первые пузырьки газа, это давление, значит, мы достигли давления насыщения и при дальнейшем увеличении объёма у нас выделяется из жидкого раствора, выделяется газ.

Значит, при давлении $P^2$, которое меньше, чем давление $P^1$, равного давлению насыщения, на $\Delta P^2$, вот мы снизили, у нас жёлтая, это выделившийся газ, поскольку он находится в равновесии с жидкостью, обозначается $V$ (от вейпа, это общепринятое обозначение), значит, и равновесный наш раствор, газированная жидкость $L$ (от liquid).
Состав жидкости обычно обозначается $x_i$, а газовая фаза или паровая -- $y_i$, а суммарный состав у нас $Z_i$.
Мы из сосуда PVT ничего не удаляли, мы давление изменяли за счёт изменения объёма, поэтому контактная конденсация у нас осуществлялась, значит, вот контактное разгазирование в данном случае.
Для нефти разгазировали.
Если бы у нас была, заодно комментарий, если бы у нас была исходная система газоконденсатной находилась в газообразном агрегатном состоянии, то в результате уменьшения давления при увеличении объёма, уменьшение давления ниже давления начала ретроградной конденсации.
Этот процесс назывался бы контактная конденсация, в зависимости от того, с какой смесью мы имеем дело.
И хотя нам в этом курсе понятие коэффициента распределения или константа равновесия не понадобится, но чтобы вы знали, что отношение мольной доли $i$-ого компонента в равновесной паровой фазе к отношению мольной доли этого компонента в жидкой фазе, называется константой равновесия, но более правильно коэффициент распределения.

\begin{center}
\includegraphics[width=\textwidth, page=74]{Брусиловский.pdf}
\end{center}

Итак, контактное разгазирование.
Компонентный состав системы не изменяется во время эксперимента, газ остаётся в равновесии с жидкостью на протяжении всего эксперимента и определяется зависимость между давлением и объёмом при пластовой температуре.
Эта зависимость, ну это вы уже знаете.
PV-зависимость, PV-соотношение, которое мы с вами подробно рассмотрели, суть, или я уже путаю, мы только будем рассматривать, в общем, PV-соотношение, нет, мы с вами уже об этом говорили, да, позволяет определить давление насыщения, а также объёмную упругость пластовой нефти.
Конечно, мы об этом.
Или я на прошлой неделе курс читал для сотрудников компании, и поэтому у меня уже в голове мешанина.
Значит, итак, прошу прощения за такую вот небольшую путаницу, но вас не должно это в смущение вводить.

\begin{center}
\includegraphics[width=\textwidth, page=75]{Брусиловский.pdf}
\end{center}

Давление насыщения нефти газом и PV-зависимость.
Итак, на основе проведения эксперимента контактной конденсации мы имеем возможность построить так называемые PV-зависимости.
То есть, зависимость между давлением и объёмом при постоянной температуре.
И этот эксперимент, который происходит в рамках контактной конденсации, значит, он что позволяет нам?
Он позволяет нам определить давление насыщения нашей смеси.
Предположим, мы загрузили наш образец пластовой нефти в сосуд высокого давления.
И объём таков, что давление выше давления насыщения, весьма существенно выше.
Нам давление насыщения неизвестно, но мы создали давление выше пластового.
И это может быть на несколько десятков бар выше.
Важно, что выше пластового.
Потому что максимальная оценка давления насыщения -- это величина пластового давления.

И создаём давление выше пластового нашей системы, наша проба, загруженная в сосуд PVT, в однофазном состоянии.
И мы увеличиваем контактным образом, не удаляя ничего из системы увеличиваем объём нашего сосуда PVT.
И в результате этого давление снижается.
Причём, чем меньше растворено газа в пластовой нефти, тем быстрее будет снижаться давление при увеличении объёма сосуда PVT.
Это физически совершенно понятно.
То есть, при увеличении объёма наша система должна расшириться до этого объёма моментально.
Но это легко сделать газу, совсем легко, но это очень сложно сделать жидкой фазе.
И чем меньше её газосодержание, чем меньше газа растворено в жидкости, тем сильнее будет падать давление при увеличении рабочего объёма.
И вот вы видите, что мы...
Вот несколько точек, соответствующих этапам увеличения рабочего объёма нашего сосуда.
Значит, тут при достижении давления насыщения выделившийся газ, он же расширяется, и уже упругость двухфазной системы совершенно другая, чем у однофазной жидкой системы.
И поэтому зависимость между давлением и объёмом при увеличении объёма, она уже совершенно иная.
Совершенно иная.
И вы видите, что при выделении газа у нас эта зависимость ну просто другая, значит, мы можем определить при каком же давлении у нас произошло выделение газа, каково давление, при котором начался процесс разгазирования.
Значит, у нас в начале, вот написано, ветвь двухфазного состояния при...
Вот третье, при некорректном проведении опыта.
Третье, при некорректном.
То есть, когда мы идём первоначально с большими шагами, и мы пропустили на самом деле давление насыщения, и излом у нас первоначально показан при давлении $p_b'$.
Но затем мы можем более точно определить величину давления насыщения, значит, мы возвращаемся, то есть объём опять уменьшаем, уменьшаем объём, перемешиваем нашу смесь в однофазном гомогенном состоянии при давлении выше, чем… ну, то есть, уже создали условия, уменьшив объём, при котором давление поднялось.
И уже с меньшим шагом по изменению объёма мы идём и фиксируем зависимость давления от объёма.
И это вот ветвь 2.
И мы видим, что при более высоком давлении, несколько более высоком давлении у нас излом осуществляется.
Значит, мы можем таким образом с меньшим шагом осуществить эксперимент в области перехода из однофазного в двухфазное состояние, определить достаточно точно давление насыщения.
Вот таким образом на практике это и делается в сосудах высокого давления при экспериментальном определении.

\begin{center}
\includegraphics[width=\textwidth, page=76]{Брусиловский.pdf}
\end{center}

Вот в виде упражнения показана как бы запись экспериментальных данных.
Таким образом записывают текущее давление и какой объём системы при текущем давлении.
Вот такой табличкой.
И по этим данным можно определить, при каком давлении будет излом PV-зависимости, то есть определить давление насыщения.

\begin{center}
\includegraphics[width=\textwidth, page=77]{Брусиловский.pdf}
\end{center}

Вот мы, например, в Excel ввели указанные данные и получили такой график.
И мы видим чётко совершенно, где у нас излом PV-зависимости, какому давлению соответствует.
И определили таким образом, что давление насыщения равно 87 бар.
И ещё раз, это излом резкий, резкий.
Это значит, что газодержание нашей нефти невысокое, не выше чем 200, ну вот для данной системы это порядка наверное 100 м$^3$ на м$^3$ по резкому излому.

\begin{center}
\includegraphics[width=\textwidth, page=78]{Брусиловский.pdf}
\end{center}

Это из моей монографии 2002 года.
Значит, для реальных, по реальным данным для месторождений различных.
И я, ну, наверное, не буду вам говорить, какие месторождения, это просто смысла нет.
Значит, это различные месторождения, характеризующиеся различным газодержанием, то есть количеством растворённого газа в пластовой нефти.
И из того, что я уже рассказал, вы поняли, что там, где чётко совершенно, вот на кривой один, чётко виден излом, это давление насыщение, там газодержание меньше, чем в тех случаях, когда резкого излома нет.
Значит, последовательно показаны зависимости с увеличением газосодержания в пластовой нефти.
Это разные месторождения.
И мы видим, что если по зависимости 1 чётко совершенно можно определить графически давление насыщения, но менее чётко, но всё-таки достаточно для практических целей можно определить давление насыщения по зависимости 2, просто нужно с более мелким шагом по давлению идти и определяется давление насыщения, то уже зависимость 3 резкого такого изменения направления в PV-зависимости нет, плавная, а по графику 4 мы вообще не можем определить по зависимости давление насыщения.
И та точка, которая показана на зависимости 4, это Карачаганакское месторождение, там газодержание около 500 м$^3$ на м$^3$, это давление насыщения, то есть начало выделения газовой фазы определено оптическим методом, не по PV-зависимости, а это можно определить визуально, либо с помощью оптических анализаторов, которые позволяют определить появление новой фазы.
Но в современной аппаратуре у нас, конечно же, оптические анализаторы присутствуют, и это позволяет более точно, более точно, ну то есть это нормальный метод определения перехода из гомогенного в гетерогенное состояние.
И в отчетах и зарубежных компаний, и наших лабораторий, которые сейчас оснащены современными установками, мы уже не визуально, а с помощью приборов определяем появление новой фазы, то есть начало разгазирования нефти.

\begin{center}
\includegraphics[width=\textwidth, page=79]{Брусиловский.pdf}
\end{center}

Теперь по результатам PV-соотношений мы можем определить изотермический коэффициент сжимаемости или объемную упругость пластовой нефти.
Значит мы это делаем по данным PV-соотношений можем сделать, заменяя производные на приращения, и используется эта величина изотермический коэффициент сжимаемости или объемная упругость в частности можно оценить совершенно простым способом нефтеотдачу пласта при работе на упругом режиме без поддержания давления, зная объемную упругость пластовой нефти.
Вот совершенно простой формулой.
Я не буду сейчас приводить эти соотношения, это можно даже и самим в виде упражнения сделать.
Еще раз, нефтеотдачу на упругом режиме можно оценить зная изотермический коэффициент сжимаемости пластовой нефти, именно за счет упругости нефти.
За счет упругости нефти конечно мы высокой нефтеотдачи добиться не можем, но если у нас большая разница между пластовым давлением и давлением насыщения (она бывает десятки мегапаскалей), то значительная часть добычи нефти формируется, когда нефть находится в однофазном состоянии
в пласте, она может быть сформирована на упругомрежиме.
И чем выше газосодержание пластовой нефти начальной, тем выше будет величина нефтеотдачи при разработке на упругом режиме, хотя конечно высокого значения мы добиться не можем.
Все равно нам нужно будет в итоге создавать систему и вводить в действие поддержание пластового давления.
Я говорю о физическом смысле, то есть изотермического коэффициента сжимаемости, где он может применяться.
Он раньше в явном виде применялся в формулах материального баланса при проектировании разработки, при оценке на упругом режиме и так далее.

\begin{center}
\includegraphics[width=\textwidth, page=80]{Брусиловский.pdf}
\end{center}

Также по результатам исследований пластовых нефтей определяют температурный коэффициент объемного расширения.
Его определение вы видите на слайде.
Также приведен диапазон его величины для большинства пластовых нефтей.
Используется эта величина при проектировании термического воздействия на нефтяные пласты.

\begin{center}
\includegraphics[width=\textwidth, page=81]{Брусиловский.pdf}
\end{center}

Теперь переходим к стандартной сепарации.
Тут немножко съехала схемочка.
Ну, ничего, я ее поправлял, она опять съехала, надо было закреплять.
Итак, вот у вас глубинные пробы.
Мы разгазируем однократно и определяем объем газа и его компонентный состав, которые выделились при однократном разгазировании глубинной пробы при 20 градусах Цельсия допускается температура окружающей среды в лаборатории и одна физическая атмосфера.
Также измеряется объем дегазированной жидкости, её компонентный состав, хроматография, и измеряется плотность и молекулярная масса дегазированной жидкости.

\begin{center}
\includegraphics[width=\textwidth, page=82]{Брусиловский.pdf}
\end{center}

Перед лекцией я еще раз посмотрел, что пишут по поводу стандартной сепарации, что написано было в монографии Абрама Юдельевича Намиота, о которой я говорил, это замечательная книга "<Фазовые равновесия в добыче нефти">, где человек просто описывает основные моменты, очень хорошо понимая и физику, и особенности физико-химических свойств.
И вот я вам могу просто зачитать, сейчас я взял в руки.
Так вот, эта стандартная сепарация, однократное разгазирование, она осуществляется при дросселировании.
Пластовая нефть, находящаяся в сосуде, это я зачитываю его текст, просто чтобы кому-то будет полезно.
Пластовая нефть, находящаяся в сосуде, которая была загружена в сосуд высокого давления, при давлении выше давления насыщения, через слегка открытый вентиль, дроссель, выпускают стеклянный сепаратор, находящийся под атмосферным давлением.
Вот каким образом это осуществляется, понимаете, разгазирование.
То есть, не увеличивают, ведь вот вопрос, который мог возникнуть у вас, то есть у нас давление может быть в сотни атмосфер или бар, если при высоком газосодержании нефти, ее давление насыщения может составлять и 300 и 400 бар.
И каким образом осуществляется однократное разгазирование?
Ведь мы же не будем в десятки раз увеличивать объем сосуда PVT, или даже больше.
Вот осуществляется это при дросселировании.
При выпуске, значит, вот в стеклянных сепараторах, находящихся под атмосферным давлением, при выпуске пластовой нефти давление в сосуде поддерживают выше давление насыщения путем перемещения плунжера -- поршня.
В процессе сепарации нефть стекает в нижнюю часть сепаратора, а газ подается из сепаратора в измерительную бутыль -- газометр, или на газовый счетчик.
Все просто.
Но более подробно вот просто найдите это издание, очень хорошо там все написано.
Я учился по книгам Намиота, с ним контактировал.
И, значит, просто настоящий классический профессор, который работал, который хорошо чувствовал и суть экспериментальных исследований, и физику явлений.
При этом он на компьютере не умел, тогда рассчитывали, это делали первоначально сотрудники, просто помощники, ну как классическая лаборатория.
А уже когда я с ним взаимодействовал, просто, так сказать, не работая с ним, а просто взаимодействовал, это со второй половины 80-х годов и до середины 90-х годов я по его просьбе проводил математическое, компьютерное моделирование.
И мне доставляло большое удовольствие взаимодействие с ним, я гораздо лучше стал понимать отдельные особенности физических явлений, когда стал заниматься пластовыми нефтями.
Конечно, он был специалист по исследованию пластовых нефтей и еще проводил исследования по растворимости газа в воде, ну это в 60-е годы.
Это я вас ориентирую просто.
И этот специалист, он ни в коем случае не уступал по своим знаниям и глубине понимания зарубежным специалистам.
А в комплексе, когда мы читаем книжки и зарубежных специалистов, и наших классиков, вот тогда мы становимся настоящими специалистами.
И в любом деле, что гидродинамическое моделирование разработки, что физическая химия, исследование пластовых флюидов, то есть это очень полезно.
Если кому-то ссылки понадобятся точные, я могу дать.

Теперь дальше.
Понятие объемного коэффициента и газосодержания, которые при проектировании разработки в подсчете запасов используются.
Объемный коэффициент пластовой нефти равен отношению объема, занимаемого углеводородной жидкой фазой пластовой смеси при пластовых условиях к объему дегазированной нефти при стандартных условиях.
Аббревиатура $V_r$, индекс $r$ -- это reservoir, то есть пласт, залежь при условиях пласта, объем занимаемой пробы, отнесенный к объему занимаемой пробы нефти при стандартных условиях.
Этот объем, объемный коэффициент, он зависит, его величина зависит от того, проводим мы стандартную сепарацию, однократное разгазирование, либо мы проводим дифференциальное разгазирование при пластовой температуре, либо мы осуществляем ступенчатую сепарацию, моделирующую промысловую подготовку нашей нефти.
Вот мы получаем разные величины, причем, чем больше газа растворено в пластовой нефти, тем сильнее будут отличаться величины объемного коэффициента, полученные при разных видах разгазирования.
Газоодержание пластовой нефти, количество газа, выделившееся из растворенного состояния при изменении условий от пластовых до стандартных и отнесенного к объему или массе дегазированной нефти при стандартных условиях.
Значит, это газосодержание пластовой нефти.
Мы встречаем на практике газосодержание и метр куб на метр куб, и метр куб на тонну.
Вот понятие объемного коэффициента и газосодержания.

\begin{center}
\includegraphics[width=\textwidth, page=83]{Брусиловский.pdf}
\end{center}



\begin{center}
\includegraphics[width=\textwidth, page=84]{Брусиловский.pdf}
\end{center}

\begin{center}
\includegraphics[width=\textwidth, page=85]{Брусиловский.pdf}
\end{center}

\begin{center}
\includegraphics[width=\textwidth, page=86]{Брусиловский.pdf}
\end{center}

\begin{center}
\includegraphics[width=\textwidth, page=87]{Брусиловский.pdf}
\end{center}

\begin{center}
\includegraphics[width=\textwidth, page=88]{Брусиловский.pdf}
\end{center}

\begin{center}
\includegraphics[width=\textwidth, page=89]{Брусиловский.pdf}
\end{center}

\begin{center}
\includegraphics[width=\textwidth, page=90]{Брусиловский.pdf}
\end{center}

\begin{center}
\includegraphics[width=\textwidth, page=91]{Брусиловский.pdf}
\end{center}

\begin{center}
\includegraphics[width=\textwidth, page=92]{Брусиловский.pdf}
\end{center}

\begin{center}
\includegraphics[width=\textwidth, page=93]{Брусиловский.pdf}
\end{center}

\begin{center}
\includegraphics[width=\textwidth, page=94]{Брусиловский.pdf}
\end{center}

\begin{center}
\includegraphics[width=\textwidth, page=95]{Брусиловский.pdf}
\end{center}

\begin{center}
\includegraphics[width=\textwidth, page=96]{Брусиловский.pdf}
\end{center}

\begin{center}
\includegraphics[width=\textwidth, page=97]{Брусиловский.pdf}
\end{center}

\begin{center}
\includegraphics[width=\textwidth, page=98]{Брусиловский.pdf}
\end{center}

\begin{center}
\includegraphics[width=\textwidth, page=99]{Брусиловский.pdf}
\end{center}

\begin{center}
\includegraphics[width=\textwidth, page=100]{Брусиловский.pdf}
\end{center}

\begin{center}
\includegraphics[width=\textwidth, page=101]{Брусиловский.pdf}
\end{center}

\begin{center}
\includegraphics[width=\textwidth, page=102]{Брусиловский.pdf}
\end{center}

\begin{center}
\includegraphics[width=\textwidth, page=103]{Брусиловский.pdf}
\end{center}

\begin{center}
\includegraphics[width=\textwidth, page=104]{Брусиловский.pdf}
\end{center}

\begin{center}
\includegraphics[width=\textwidth, page=105]{Брусиловский.pdf}
\end{center}

\begin{center}
\includegraphics[width=\textwidth, page=106]{Брусиловский.pdf}
\end{center}

\begin{center}
\includegraphics[width=\textwidth, page=107]{Брусиловский.pdf}
\end{center}

\begin{center}
\includegraphics[width=\textwidth, page=108]{Брусиловский.pdf}
\end{center}

\begin{center}
\includegraphics[width=\textwidth, page=109]{Брусиловский.pdf}
\end{center}

\begin{center}
\includegraphics[width=\textwidth, page=110]{Брусиловский.pdf}
\end{center}

\begin{center}
\includegraphics[width=\textwidth, page=111]{Брусиловский.pdf}
\end{center}

\begin{center}
\includegraphics[width=\textwidth, page=112]{Брусиловский.pdf}
\end{center}

\begin{center}
\includegraphics[width=\textwidth, page=113]{Брусиловский.pdf}
\end{center}

\begin{center}
\includegraphics[width=\textwidth, page=114]{Брусиловский.pdf}
\end{center}

\end{document}