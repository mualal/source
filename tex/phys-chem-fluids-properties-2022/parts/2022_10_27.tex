\documentclass[main.tex]{subfiles}

\begin{document}

\section{Лекция 27.10.2022 (Брусиловский А.И.)}

\begin{center}
\includegraphics[width=\textwidth, page=146]{Брусиловский.pdf}
\end{center}

Сегодня последний день курса.
И я буду очень интенсивно рассказывать.
Если вы обратите внимание, мы на 146 слайде, у нас их около 300.
Вас это не должно смущать, я все успею вам рассказать, потому что значительная часть из того, что осталось, это посвящено корреляциям.
Просто я покажу вам наборы известных корреляций для моделирования свойств нефти, природного газа, пластовой воды.
Есть в прилагаемом материале программа, написанная на Excel, по которой можно посчитать по корреляциям различные свойства нефти, газа и воды, в том числе и те, которые в презентации корреляции указаны.
Это я буду делать в виде очень быстрого обзора, так что мы все с вами успеем.
Итак, за прошедшие 4 дня (сегодня у нас пятый день) мы с вами познакомились с, во-первых, особенностями компонентного состава природных углеводородных систем и умеем с вами отличить состав системы, находящейся в жидком агрегатном состоянии, то есть пластовой нефти, от многокомпонентной природной системы, находящейся в газовом агрегатном состоянии, то есть это природные газы.
Когда я говорю газовое агрегатное состояние, естественно, я имею в виду и свободные газы, и газовые шапки двухфазных залежей.
Также мы с вами рассмотрели методы получения информации с помощью глубинных и поверхностных проб о компонентном составе пластовых нефтей и газоконденсатных систем.
И мы с вами послушали о том, каковы особенности и требования к отбору проб.
И самое главное, что в каких случаях отбирают глубинные пробы, в каких поверхностные.
Напомню, что для нефтей основным способом получения информации о составе и свойствах пластового флюида является отбор глубинных проб.
Для газоконденсатных систем основным способом является рекомбинация поверхностных проб, полученных в сепараторах большого объема.
Это то, что предпочитают в промышленности.
Мы с вами узнали, какие эксперименты проводятся при исследовании пластовых нефтей и газоконденсатных систем, и какой физический смысл заложен в проведении этих экспериментов и в получении тех или иных результатов.

\begin{center}
\includegraphics[width=\textwidth, page=147]{Брусиловский.pdf}
\end{center}

И вот теперь мы перейдем к тому, каким...
Ну, фактически к методам математического моделирования свойств природных углеводородных систем и пластовых нефтей, и газоконденсатных систем сухих и жирных газов.
И два направления здесь существует.
Это применение уравнений состояния, основанное на фундаментальных положениях термодинамики многокомпонентных систем, и второе — применение корреляций, когда обработкой большого объема статистического и фактического материала, лабораторных и промысловых исследований, получают зависимости, позволяющие при отсутствии компонентного состава, почти всегда, тем не менее, оценивать так называемые PVT-свойства, необходимые для инженерных расчетов.
Начнем мы наше изложение с тех основ, которые используются при моделировании с использованием уравнений состояния.
Итак, это очень короткий обзор, но он дает абсолютно правильное направление вам для понимания того, что делается.

\begin{center}
\includegraphics[width=\textwidth, page=148]{Брусиловский.pdf}
\end{center}

Итак, необходимые условия фазового равновесия $N$-компонентной системы.
Это напоминание того, что я уже вам говорил.
В общем случае, если сосуществуют $N$-фаз $N$-компонентной система, то необходимым условием того, чтобы гетерогенная система, то есть находящаяся в многофазном состоянии, и при этом в состоянии равновесия.
Чтобы фазы были в равновесии, нам нужно соблюдение условий, которые вы видите на слайде.
Равенство температур во всех фазах, равенство давлений во всех фазах, равенство летучестей каждого компонента во всех фазах.
Первичным в термодинамике является равенство химических потенциалов компонента во всех фазах.
В термодинамике для практического использования вводится понятие летучести.
Летучесть –- это fugitiveness, по-английски fugacity.
Летучесть связана с химическим потенциалом однозначно.
На практике используют равенство летучестей.
Итак, как в частном случае, то что у нас в нашей практике инженерной, мы рассчитываем двухфазное равновесие пар-жидкость.
В большинстве случаев, в подавляющем большинстве случаев для $N$-компонентной системы.
И наша задача заключается в том, чтобы вычислить летучесть.
Так вот, из термодинамики многокомпонентных систем следует нижняя формула, что для многокомпонентной системы натуральный логарифм летучести $i$-ого компонента, он равен интеграл от $V$ до бесконечности.
Вот интегральное выражение.
То, что вы видите.
Смысла нет, это все озвучивать.
И вопрос заключается в том, а вообще как нам эту формулу использовать?
Нужно же знать частную производную давления по числу молей $i$-ого компонента при постоянной температуре, объеме и числах молей остальных компонентов.
Это основа.
Для этого используется уравнение состояния.
И уравнение состояния, чем точнее уравнение состояния описывает свойства вещества, тем точнее мы будем рассчитывать значения летучести компонентов и рассчитывать термодинамическое равновесие, в частности парожидкостное равновесие наших систем.

\begin{center}
\includegraphics[width=\textwidth, page=149]{Брусиловский.pdf}
\end{center}

В практике расчета парожидкостного равновесия последовательно на протяжении нескольких десятилетий использовали разные методы.
В начале так называемое давление схождения, использующее константы фазового равновесия.
И я об этом говорить не буду.
Раньше это было в 50-х, 60-х вплоть до 70-х годов.
Это был основной способ определения состава равновесных фаз углеводородных систем.
Применялись также комбинированные методы, использующие методы физической химии.
И основной, повторяю, комбинированный метод, там свойства газовой фазы рассчитывали с применением уравнения состояния, а свойства жидкой фазы рассчитывались с использованием теории регулярных растворов.
И с середины 70-х годов, когда появились уравнения состояния, более-менее адекватно описывающие парожидкостное равновесие многокомпонентных углеводородных систем, в том числе включающих азот, диоксид углерода и сероводород, стали использовать единые уравнения состояния, которые используют для описания и паровой фазы, или свободной газовой фазы, и жидкой фазы, находящейся…
Значит, рассчитывают условия…
Из заданных условий термобарических и суммарного компонентного состава рассчитывают, на фазы какого состава расслоится система и свойства этих фаз.
Нужно сказать, что используя уравнения состояния, мы можем рассчитывать не только компонентные равновесные составы фаз, сосуществующих, но и согласованные с ними термодинамические свойства.
Об этом, значит, я не думаю, что все слышали, а, по крайней мере, в учебниках об этом особенно не пишут.
Но если вы знаете компонентный состав фазы, то вы можете рассчитать все термодинамические свойства, в том числе энтальпию, энтропию, изобарно-изотермические и другие потенциалы, теплоемкость при постоянном давлении, постоянном объеме и интегральные дифференциальные эффекты Джоуля-Томсона.
Изоэнтальпийный эффект, это вот Джоуля-Томсона, изоэнтропийный эффект.
И также, зная, значит, уже определив компонентный состав и плотности фазы из термодинамических расчетов, вы можете рассчитать теплофизические свойства.
Главное теплофизическое свойство, которое нам необходимо для гидродинамических расчетов, как в пласте, так и в стволе скважины, в трубопроводах, это динамическая вязкость.
И я вам покажу, уже когда будем обсуждать, я буду делать обзор применения корреляций, там будут корреляции для оценки динамической вязкости газов и нефтей.

Напоминаю правила фаз Гиббса, классику термодинамики многокомпонентных систем, что мы можем определить с помощью правила фаз Гиббса, какое максимальное количество фаз в нашей системе.
И если у нас N компонентов, то максимальное количество фаз теоретически равно $N+2$, и мы имеем дело в основном, в подавляющем большинстве случаев, только с двумя фазами, это особенность наших природных углеводородных систем, что они в термобарических условиях, которые нас интересуют, пласты, значит, это температура где-то до 150 градусов Цельсия, давление до 1000 атмосфер, выше давления крайне редко, только в очень глубокопогруженных залежах, это исключение.
Значит, у нас существуют две фазы, расслаивается наша система на две фазы, паровую и жидкую, несмотря на то, что системы состоят из очень большого количества компонентов обычных (кроме когда мы имеем дело с сухими газами, там подавляющая концентрация метана, и, собственно говоря, там и фазовых превращений не происходит, если отсутствует конденсат).
Я всё время возвращаюсь к практическим вопросам, потому что они, ради понимания того, как на практике нам нужно поступать, и понимание того, что может быть, собственно, и этот курс читается.
И нас интересует, какое число независимых переменных системы нужно задать для однозначного определения значений остальных переменных, и вот скоро мы увидим формулировку задач расчёта парожидкостного равновесия для природных систем, которая вытекает из базовых положений фазового равновесия термодинамики многокомпонентных систем.
Формулировки будут.
Я вам напомню, когда мы будем рассматривать систему уравнений, очень скоро, что правило фаз Гиббса, оно говорит о том, какое количество независимых переменных нам нужно задать для однозначного определения значений остальных переменных, которыми являются составы фаз.
В правиле фаз Гиббса не говорится об относительном количестве фаз в системе, а только о их компонентном составе.
На практике нам нужно знать не только на фазы какого состава расслоится система, но и каково относительное количество фаз, в частности, при моделировании процесса контактной конденсации, когда мы моделируем сепарацию нашей смеси, промысловую сепарацию.
Для этого случая будет дана формулировка, соответствующая инженерной задаче.

\begin{center}
\includegraphics[width=\textwidth, page=150]{Брусиловский.pdf}
\end{center}

Итак, основное направление для моделирования свойств природных углеводородных систем -- это применение кубических уравнений состояния, которые к настоящему времени достаточно точно описывают и фазовое равновесие, и PVT-свойства фаз.
И началось это с уравнения состояния Ван-дер-Ваальса.
В 1873 году в своей докторской диссертации о непрерывности газового и жидкого состояния он предложил модификацию уравнения состояния идеального газа для моделирования свойств уже не идеальных газов, а реальных флюидов.
Речь шла о газах, а потом уже стали применять это уравнение состояния Ван-дер-Ваальсового типа и для моделирования свойств жидкой фазы потихонечку научились.
Итак, что сделал Ван-дер-Ваальдс, модернизируя уравнение идеального газа?
Он сделал такие предположения, что в идеальном газе, представим себе бильярдный стол и бильярдные шары, там нет сил межмолекулярного взаимодействия.
То есть, вот эти шары бильярдные, у них нет сил ни отталкивания, ни притяжения.
Мы гоняем эти шары, они соударяются и это простейший случай.
Вот два сомножителя вы видите, $pV$ было в уравнении идеального газа, а стало $p$ плюс некая добавка, $n$ квадрат $a$ делить на $V$ квадрат и $V$ минус $nb$.
Было учтено, что между молекулами существует притяжение, и поэтому внешнее давление, оно еще и корректируется на внутреннее давление, которое в итоге уменьшает наше давление.
Эта корректировка осуществляется членом $n$ квадрат $A$ делить на $V$ квадрат, что это взаимодействие, оно обратно пропорционально объему, занимаемому молекулами ($n$ молями вещества).
А кроме того, оно значит, вот обратно пропорционально расстоянию между молекулами.
В итоге у нас формируется вот такой член по отношению к внешнему давлению.
А второй сомножитель учитывает, что у молекул существует некий эффективный объем молекул, он приблизительно равен четырехкратному, это по оценкам Ван-дер-Вавльса, четырехкратному объему молекул, в который молекула не пускает другие молекулы как бы они ни старались проникнуть, соудариться с данной молекулой, ничего не получится.
Вот этот эффективный объем это учитывает.
В результате вот он сформулировал уравнение, которое стало основой крупного направления в развитии моделирования свойств реальных веществ.
И он понятия не имел о нефти и природном газе -- 1873 год, физик.
Но плодотворная такая вот идея, она, как и в других областях науки, плодотворная идея, обладающая самыми разными, с точки зрения численного моделирования, недостатками, недостаточно точно описывающими какие-то явления, тем не менее, является определяющей, вот идея определяет направление.
И уравнение Ван-дер-Ваальса позволило развиться крупному направлению в промышленности, и нефтехимической, и в области разработки, эксплуатации месторождения, оценки свойств самых различных природных углеводородных систем.
Хотя в численном отношении уравнение Ван-дер-Ваальса обладает разными недостатками, они перечислены в учебниках, в монографии, в частности, в тех, которые я писал.
И я сейчас не буду это делать.
Это те, кто заинтересуется более подробно этим направлением, будут читать специальные книги, обзоры, посвященные этому вопросу.
А мы перейдем к следующему слайду.

\begin{center}
\includegraphics[width=\textwidth, page=151]{Брусиловский.pdf}
\end{center}

Так вот, прошло 100 лет, и итальянский ученый Джордж Соаве, который занимался вопросами химической технологии, он на основе уравнения типа Ван-дер-Ваальса, но на самом деле он предложил модификацию, основанную на той, которую чуть ранее, в 1949 году осуществил немецкий ученый Редлих.
На слайде видите Соаве-Редлих-Квонг.
Редлих в 1949 году, профессор Редлих со своим аспирантом Квонгом, они предложили модифицировать уравнение Ван-дер-Ваальса в том виде, который вы видите в первой строчке этого слайда.
Но еще там в знаменателе стояла абсолютная температура в степени 0.5.
Собственно, у Ван-дер-Вальдса не было зависимости от температуры в…
Я вам сейчас еще раз покажу.
Видите, вот на предыдущем слайде во множителе первом $a$ -- это константа, которая, как и константа $b$, их величины определяются в условиях критической точки.
Подробно я не буду рассказывать, это долгая история.
Это все подробно описано.
А вот Редлих с Квонгом в 1949 году сделали модификацию формы уравнения Ван-дер-Вальса и стали точнее вычисляться, моделироваться свойства газовой фазы.
И где-то до давлений порядка 100 атмосфер.
Этого достаточно для того, чтобы свойства газовой фазы в технологических процессах, где давление несколько десятков атмосфер, уже моделировались с использованием уравнения состояния.
И плотность, Z-фактор, естественно, и другие свойства.
А что предложил Соаве в 1972 году?
И это послужило очень значительным достижением для уточнения расчета парожидкостного равновесия.
Для возможности не просто моделирования газовой фазы, а парожидкостного равновесия.
Он предложил, во-первых, Соаве я имею ввиду, а реализовал Иваскин, модифицировал знаменатель в правой части уравнения состояния.
Как вы видите оно уже переписано относительно давления.
Мы переписали форму, которая была на предыдущем слайде, в несколько другом виде.
Ну, просто преобразование сделали легкое алгебраическое.
У нас в левой части только давление оказалось.
И знаменатель у Ван-дер-Ваальса был $V$ квадрат просто, а здесь $V$ на ($V$ плюс $B$).
То есть, кроме того, предложил Соаве температурную зависимость при коэффициенте $A$ определять из условия точного моделирования упругости паров компонентов.
Первоначально уравнения состояния всегда предлагаются для чистых веществ.
А уже потом вводятся правила для расчёта коэффициента уравнения состояния для смесей.
И это следующий шаг.
А первоначально для чистых веществ.
И вот для десятка компонентов углеводородов плюс ещё азот, диоксид углерода, были определены коэффициенты $\alpha$.
Видите, сомножитель $\alpha$ при расчёте коэффициента $A$.
Из того, что в диапазоне температур от тройной точки до критической, вспомним про чистое вещество, уравнение состояния точно описывает упругость паров вещества.
Вот около десятка веществ были сделаны, были получены соответствующие коэффициенты.
Они были аппроксимированы.
От величины приведённой температуры, напомню, температура делённая на критическую температуру,
и от значения центрического фактора, о котором мы говорили, когда мы рассматривали таблицы свойств чистых веществ.
И говорилось, что такое ацентрический фактор, я уже сейчас этому время не буду уделять.
Но величина ацентрического фактора приводится как справочные данные наряду с критическими температурами, давлением, молекулярной массой веществ.
И затем был аппроксимирован коэффициент в выражении.
В общем, было предложено выражение для вычисления коэффициента $\alpha$.
То есть сначала были разрозненные, полученные значения $\alpha$, а потом это $\alpha$ было аппроксимировано тем выражением, которое вы видите.
И коэффициент $m$ зависит от ацентрического фактора.
Теперь, коэффициент $b$, и параметр $b$, и параметр $a_c$, они определяются из условий в критической точке, так же, как это предложил делать Ван-дер-Ваальс.
А что, там же было тоже $a$ и $b$, и вот что он впервые предложил, и Соаве тоже использовал это.
А то, что в критической точке у нас первая и вторая частные производные давления по объему, они равны нулю.
Вот это, если смотреть на…
То есть это точка перегиба для функции $p$ от $V$.
Точка перегиба для любого чистого вещества общее свойства в критической точке.
Точка перегиба.
Имея эти два уравнения, и уравнение состояния у нас два параметра, $a$ и $b$, мы получаем вот эти выражения, да, и задав структуру заранее, что у нас $b$ равняется некий коэффициент умноженный на универсальную газовую постоянную, на критическую температуру, деленную на критический объем, мы получаем коэффициент для этого вида уравнения состояния, как то, что вы видите, 0.08654.
А для коэффициента $a_c$, который имеет другую структуру, значит, $R$ квадрат, $t_c$ квадрат делить на $P_c$, коэффициент 0.42747.
Это просто получается из применения для уравнения состояния конкретного условия в критической точке.
Итак, главное, что уравнение состояния стало точно описывать упругость паров чистых компонентов.
То, что не могли делать с применением именно самого уравнения Ван-дер-Ваальса.
И, кроме того, уравнения Ван-дер-Ваальса температурные зависимости различных веществ, они описывались плохо.
А уравнения в той форме, которую предложил Соаве, гораздо точнее.
И многие термодинамические свойства в зависимости от температуры, особенно для газовой фазы, стали описываться.
Второй момент, это уже при переходе к смесям, что очень важно для нас.
Это было предложено правило вычисления коэффициента $a_m$ ($m$ -- это от mixture, смесь).
Вот ту, которую вы видите, которая отличается от того, что использовал Ван-дер-Ваальс.
Использовалось уравнение Редлиха-Квонга для смеси, более примитивное правило.
Что здесь важно?
Для понимания.
Вот у нас смесь состава $y_i$.
$y$, потому что, прежде всего, я вам говорил, что $y$ обозначает состав газовой, паровой фазы, а $x$ обозначает состав жидкой фазы.
А для смеси $z_i$.
Вот ещё одно замечание.
Уравнение состояния, оно описывает не свойства гетерогенные двух или более фазной системы сразу.
Оно описывает свойства отдельных фаз.
Прежде всего, уравнение Ван-дер-Ваальсового типа были предназначены для описания свойств газовой фазы.
Поэтому, здесь $y$ показан.
Но при расчёте парожидкостного равновесия, при расчёте летучести компонентов правила вычисления коэффициентов $a$ и $b$, они одинаковы для сосуществующих фаз.
То есть, для определённости здесь $y$ показан.
$y_i$ -- это мольная доля $i$-ого компонента в фазе.
Это первое.
Второе.
Появился сомножитель  с так называемыми коэффициентами парного взаимодействия.
Потому что без...
Но это поправочные коэффициенты.
Это эмпирические поправочные коэффициенты, предназначенные для того, чтобы уточнить расчёт свойств фазы многокомпонентной системы.
От двухкомпонентной до $N$-компонентной, $N$ -- произвольное число, какое вам нужно.
И были для ограниченного числа веществ приведены значения этих коэффициентов по оценкам Соаве и Редлиха.
Вот это основные модификации.
И надо сказать, что во всех программных комплексах, когда вы откроете, вы увидите, какие вам уравнения состояния предлагают использовать.
В этом перечне есть уравнение Соаве-Редлиха-Квонга.
Оно очень неплохо описывает (повторяю) свойства газовой фазы и вполне удовлетворительно рассчитывает парожидкостное равновесия при умеренных давлениях в несколько десятков атмосфер.
Прежде всего, Соаве занимался вопросами химической технологии.
То есть, это вопросы заводской переработки и так далее.
Это вот несколько десятков, но может быть до 150 бар.

\begin{center}
\includegraphics[width=\textwidth, page=152]{Брусиловский.pdf}
\end{center}

\begin{center}
\includegraphics[width=\textwidth, page=153]{Брусиловский.pdf}
\end{center}

\begin{center}
\includegraphics[width=\textwidth, page=154]{Брусиловский.pdf}
\end{center}

\begin{center}
\includegraphics[width=\textwidth, page=155]{Брусиловский.pdf}
\end{center}

\begin{center}
\includegraphics[width=\textwidth, page=156]{Брусиловский.pdf}
\end{center}

\begin{center}
\includegraphics[width=\textwidth, page=157]{Брусиловский.pdf}
\end{center}

\begin{center}
\includegraphics[width=\textwidth, page=158]{Брусиловский.pdf}
\end{center}

\begin{center}
\includegraphics[width=\textwidth, page=159]{Брусиловский.pdf}
\end{center}

\begin{center}
\includegraphics[width=\textwidth, page=160]{Брусиловский.pdf}
\end{center}

\begin{center}
\includegraphics[width=\textwidth, page=161]{Брусиловский.pdf}
\end{center}

\begin{center}
\includegraphics[width=\textwidth, page=162]{Брусиловский.pdf}
\end{center}

\begin{center}
\includegraphics[width=\textwidth, page=163]{Брусиловский.pdf}
\end{center}

\begin{center}
\includegraphics[width=\textwidth, page=164]{Брусиловский.pdf}
\end{center}

\begin{center}
\includegraphics[width=\textwidth, page=165]{Брусиловский.pdf}
\end{center}

\begin{center}
\includegraphics[width=\textwidth, page=166]{Брусиловский.pdf}
\end{center}

\begin{center}
\includegraphics[width=\textwidth, page=167]{Брусиловский.pdf}
\end{center}

\begin{center}
\includegraphics[width=\textwidth, page=168]{Брусиловский.pdf}
\end{center}

\begin{center}
\includegraphics[width=\textwidth, page=169]{Брусиловский.pdf}
\end{center}

\begin{center}
\includegraphics[width=\textwidth, page=170]{Брусиловский.pdf}
\end{center}

\begin{center}
\includegraphics[width=\textwidth, page=171]{Брусиловский.pdf}
\end{center}

\begin{center}
\includegraphics[width=\textwidth, page=172]{Брусиловский.pdf}
\end{center}

\begin{center}
\includegraphics[width=\textwidth, page=173]{Брусиловский.pdf}
\end{center}

\begin{center}
\includegraphics[width=\textwidth, page=174]{Брусиловский.pdf}
\end{center}

\begin{center}
\includegraphics[width=\textwidth, page=175]{Брусиловский.pdf}
\end{center}

\begin{center}
\includegraphics[width=\textwidth, page=176]{Брусиловский.pdf}
\end{center}

\begin{center}
\includegraphics[width=\textwidth, page=177]{Брусиловский.pdf}
\end{center}

\begin{center}
\includegraphics[width=\textwidth, page=178]{Брусиловский.pdf}
\end{center}

\begin{center}
\includegraphics[width=\textwidth, page=179]{Брусиловский.pdf}
\end{center}

\begin{center}
\includegraphics[width=\textwidth, page=180]{Брусиловский.pdf}
\end{center}

\begin{center}
\includegraphics[width=\textwidth, page=181]{Брусиловский.pdf}
\end{center}

\begin{center}
\includegraphics[width=\textwidth, page=182]{Брусиловский.pdf}
\end{center}

\begin{center}
\includegraphics[width=\textwidth, page=183]{Брусиловский.pdf}
\end{center}

\begin{center}
\includegraphics[width=\textwidth, page=184]{Брусиловский.pdf}
\end{center}

\begin{center}
\includegraphics[width=\textwidth, page=185]{Брусиловский.pdf}
\end{center}

\begin{center}
\includegraphics[width=\textwidth, page=186]{Брусиловский.pdf}
\end{center}

\begin{center}
\includegraphics[width=\textwidth, page=187]{Брусиловский.pdf}
\end{center}

\begin{center}
\includegraphics[width=\textwidth, page=188]{Брусиловский.pdf}
\end{center}

\begin{center}
\includegraphics[width=\textwidth, page=189]{Брусиловский.pdf}
\end{center}

\begin{center}
\includegraphics[width=\textwidth, page=190]{Брусиловский.pdf}
\end{center}

\begin{center}
\includegraphics[width=\textwidth, page=191]{Брусиловский.pdf}
\end{center}

\begin{center}
\includegraphics[width=\textwidth, page=192]{Брусиловский.pdf}
\end{center}

\begin{center}
\includegraphics[width=\textwidth, page=193]{Брусиловский.pdf}
\end{center}

\begin{center}
\includegraphics[width=\textwidth, page=194]{Брусиловский.pdf}
\end{center}

\begin{center}
\includegraphics[width=\textwidth, page=195]{Брусиловский.pdf}
\end{center}

\begin{center}
\includegraphics[width=\textwidth, page=196]{Брусиловский.pdf}
\end{center}

\begin{center}
\includegraphics[width=\textwidth, page=197]{Брусиловский.pdf}
\end{center}

\begin{center}
\includegraphics[width=\textwidth, page=198]{Брусиловский.pdf}
\end{center}

\begin{center}
\includegraphics[width=\textwidth, page=199]{Брусиловский.pdf}
\end{center}

\begin{center}
\includegraphics[width=\textwidth, page=200]{Брусиловский.pdf}
\end{center}

\begin{center}
\includegraphics[width=\textwidth, page=201]{Брусиловский.pdf}
\end{center}

\begin{center}
\includegraphics[width=\textwidth, page=202]{Брусиловский.pdf}
\end{center}

\begin{center}
\includegraphics[width=\textwidth, page=203]{Брусиловский.pdf}
\end{center}

\begin{center}
\includegraphics[width=\textwidth, page=204]{Брусиловский.pdf}
\end{center}

\begin{center}
\includegraphics[width=\textwidth, page=205]{Брусиловский.pdf}
\end{center}

\begin{center}
\includegraphics[width=\textwidth, page=206]{Брусиловский.pdf}
\end{center}

\begin{center}
\includegraphics[width=\textwidth, page=207]{Брусиловский.pdf}
\end{center}

\begin{center}
\includegraphics[width=\textwidth, page=208]{Брусиловский.pdf}
\end{center}

\begin{center}
\includegraphics[width=\textwidth, page=209]{Брусиловский.pdf}
\end{center}

\begin{center}
\includegraphics[width=\textwidth, page=210]{Брусиловский.pdf}
\end{center}

\begin{center}
\includegraphics[width=\textwidth, page=211]{Брусиловский.pdf}
\end{center}

\begin{center}
\includegraphics[width=\textwidth, page=212]{Брусиловский.pdf}
\end{center}

\begin{center}
\includegraphics[width=\textwidth, page=213]{Брусиловский.pdf}
\end{center}

\begin{center}
\includegraphics[width=\textwidth, page=214]{Брусиловский.pdf}
\end{center}

\begin{center}
\includegraphics[width=\textwidth, page=215]{Брусиловский.pdf}
\end{center}

\begin{center}
\includegraphics[width=\textwidth, page=216]{Брусиловский.pdf}
\end{center}

\begin{center}
\includegraphics[width=\textwidth, page=217]{Брусиловский.pdf}
\end{center}

\begin{center}
\includegraphics[width=\textwidth, page=218]{Брусиловский.pdf}
\end{center}

\begin{center}
\includegraphics[width=\textwidth, page=219]{Брусиловский.pdf}
\end{center}

\begin{center}
\includegraphics[width=\textwidth, page=220]{Брусиловский.pdf}
\end{center}

\begin{center}
\includegraphics[width=\textwidth, page=221]{Брусиловский.pdf}
\end{center}

\begin{center}
\includegraphics[width=\textwidth, page=222]{Брусиловский.pdf}
\end{center}

\begin{center}
\includegraphics[width=\textwidth, page=223]{Брусиловский.pdf}
\end{center}

\begin{center}
\includegraphics[width=\textwidth, page=224]{Брусиловский.pdf}
\end{center}

\begin{center}
\includegraphics[width=\textwidth, page=225]{Брусиловский.pdf}
\end{center}

\begin{center}
\includegraphics[width=\textwidth, page=226]{Брусиловский.pdf}
\end{center}

\begin{center}
\includegraphics[width=\textwidth, page=227]{Брусиловский.pdf}
\end{center}

\begin{center}
\includegraphics[width=\textwidth, page=228]{Брусиловский.pdf}
\end{center}

\begin{center}
\includegraphics[width=\textwidth, page=229]{Брусиловский.pdf}
\end{center}

\begin{center}
\includegraphics[width=\textwidth, page=230]{Брусиловский.pdf}
\end{center}

\begin{center}
\includegraphics[width=\textwidth, page=231]{Брусиловский.pdf}
\end{center}

\begin{center}
\includegraphics[width=\textwidth, page=232]{Брусиловский.pdf}
\end{center}

\begin{center}
\includegraphics[width=\textwidth, page=233]{Брусиловский.pdf}
\end{center}

\begin{center}
\includegraphics[width=\textwidth, page=234]{Брусиловский.pdf}
\end{center}

\begin{center}
\includegraphics[width=\textwidth, page=235]{Брусиловский.pdf}
\end{center}

\begin{center}
\includegraphics[width=\textwidth, page=236]{Брусиловский.pdf}
\end{center}

\begin{center}
\includegraphics[width=\textwidth, page=237]{Брусиловский.pdf}
\end{center}

\begin{center}
\includegraphics[width=\textwidth, page=238]{Брусиловский.pdf}
\end{center}

\begin{center}
\includegraphics[width=\textwidth, page=239]{Брусиловский.pdf}
\end{center}

\begin{center}
\includegraphics[width=\textwidth, page=240]{Брусиловский.pdf}
\end{center}

\begin{center}
\includegraphics[width=\textwidth, page=241]{Брусиловский.pdf}
\end{center}

\begin{center}
\includegraphics[width=\textwidth, page=242]{Брусиловский.pdf}
\end{center}

\begin{center}
\includegraphics[width=\textwidth, page=243]{Брусиловский.pdf}
\end{center}

\begin{center}
\includegraphics[width=\textwidth, page=244]{Брусиловский.pdf}
\end{center}

\begin{center}
\includegraphics[width=\textwidth, page=245]{Брусиловский.pdf}
\end{center}

\begin{center}
\includegraphics[width=\textwidth, page=246]{Брусиловский.pdf}
\end{center}

\begin{center}
\includegraphics[width=\textwidth, page=247]{Брусиловский.pdf}
\end{center}

\begin{center}
\includegraphics[width=\textwidth, page=248]{Брусиловский.pdf}
\end{center}

\begin{center}
\includegraphics[width=\textwidth, page=249]{Брусиловский.pdf}
\end{center}

\begin{center}
\includegraphics[width=\textwidth, page=250]{Брусиловский.pdf}
\end{center}

\begin{center}
\includegraphics[width=\textwidth, page=251]{Брусиловский.pdf}
\end{center}

\begin{center}
\includegraphics[width=\textwidth, page=252]{Брусиловский.pdf}
\end{center}

\begin{center}
\includegraphics[width=\textwidth, page=253]{Брусиловский.pdf}
\end{center}

\begin{center}
\includegraphics[width=\textwidth, page=254]{Брусиловский.pdf}
\end{center}

\begin{center}
\includegraphics[width=\textwidth, page=255]{Брусиловский.pdf}
\end{center}

\begin{center}
\includegraphics[width=\textwidth, page=256]{Брусиловский.pdf}
\end{center}

\begin{center}
\includegraphics[width=\textwidth, page=257]{Брусиловский.pdf}
\end{center}

\begin{center}
\includegraphics[width=\textwidth, page=258]{Брусиловский.pdf}
\end{center}

\begin{center}
\includegraphics[width=\textwidth, page=259]{Брусиловский.pdf}
\end{center}

\begin{center}
\includegraphics[width=\textwidth, page=260]{Брусиловский.pdf}
\end{center}

\begin{center}
\includegraphics[width=\textwidth, page=261]{Брусиловский.pdf}
\end{center}

\begin{center}
\includegraphics[width=\textwidth, page=262]{Брусиловский.pdf}
\end{center}

\begin{center}
\includegraphics[width=\textwidth, page=263]{Брусиловский.pdf}
\end{center}

\begin{center}
\includegraphics[width=\textwidth, page=264]{Брусиловский.pdf}
\end{center}

\begin{center}
\includegraphics[width=\textwidth, page=265]{Брусиловский.pdf}
\end{center}

\begin{center}
\includegraphics[width=\textwidth, page=266]{Брусиловский.pdf}
\end{center}

\begin{center}
\includegraphics[width=\textwidth, page=267]{Брусиловский.pdf}
\end{center}

\begin{center}
\includegraphics[width=\textwidth, page=268]{Брусиловский.pdf}
\end{center}

\begin{center}
\includegraphics[width=\textwidth, page=269]{Брусиловский.pdf}
\end{center}

\begin{center}
\includegraphics[width=\textwidth, page=270]{Брусиловский.pdf}
\end{center}

\begin{center}
\includegraphics[width=\textwidth, page=271]{Брусиловский.pdf}
\end{center}

\begin{center}
\includegraphics[width=\textwidth, page=272]{Брусиловский.pdf}
\end{center}

\begin{center}
\includegraphics[width=\textwidth, page=273]{Брусиловский.pdf}
\end{center}

\begin{center}
\includegraphics[width=\textwidth, page=274]{Брусиловский.pdf}
\end{center}

\begin{center}
\includegraphics[width=\textwidth, page=275]{Брусиловский.pdf}
\end{center}

\begin{center}
\includegraphics[width=\textwidth, page=276]{Брусиловский.pdf}
\end{center}

\begin{center}
\includegraphics[width=\textwidth, page=277]{Брусиловский.pdf}
\end{center}

\begin{center}
\includegraphics[width=\textwidth, page=278]{Брусиловский.pdf}
\end{center}

\begin{center}
\includegraphics[width=\textwidth, page=279]{Брусиловский.pdf}
\end{center}

\begin{center}
\includegraphics[width=\textwidth, page=280]{Брусиловский.pdf}
\end{center}

\begin{center}
\includegraphics[width=\textwidth, page=281]{Брусиловский.pdf}
\end{center}

\begin{center}
\includegraphics[width=\textwidth, page=282]{Брусиловский.pdf}
\end{center}

\begin{center}
\includegraphics[width=\textwidth, page=283]{Брусиловский.pdf}
\end{center}

\begin{center}
\includegraphics[width=\textwidth, page=284]{Брусиловский.pdf}
\end{center}

\begin{center}
\includegraphics[width=\textwidth, page=285]{Брусиловский.pdf}
\end{center}

\begin{center}
\includegraphics[width=\textwidth, page=286]{Брусиловский.pdf}
\end{center}

\begin{center}
\includegraphics[width=\textwidth, page=287]{Брусиловский.pdf}
\end{center}

\begin{center}
\includegraphics[width=\textwidth, page=288]{Брусиловский.pdf}
\end{center}

\begin{center}
\includegraphics[width=\textwidth, page=289]{Брусиловский.pdf}
\end{center}

\begin{center}
\includegraphics[width=\textwidth, page=290]{Брусиловский.pdf}
\end{center}

\begin{center}
\includegraphics[width=\textwidth, page=291]{Брусиловский.pdf}
\end{center}

\begin{center}
\includegraphics[width=\textwidth, page=292]{Брусиловский.pdf}
\end{center}

\begin{center}
\includegraphics[width=\textwidth, page=293]{Брусиловский.pdf}
\end{center}

\begin{center}
\includegraphics[width=\textwidth, page=294]{Брусиловский.pdf}
\end{center}

\begin{center}
\includegraphics[width=\textwidth, page=295]{Брусиловский.pdf}
\end{center}

\begin{center}
\includegraphics[width=\textwidth, page=296]{Брусиловский.pdf}
\end{center}

\begin{center}
\includegraphics[width=\textwidth, page=297]{Брусиловский.pdf}
\end{center}

\begin{center}
\includegraphics[width=\textwidth, page=298]{Брусиловский.pdf}
\end{center}

\begin{center}
\includegraphics[width=\textwidth, page=299]{Брусиловский.pdf}
\end{center}

\begin{center}
\includegraphics[width=\textwidth, page=300]{Брусиловский.pdf}
\end{center}

\begin{center}
\includegraphics[width=\textwidth, page=301]{Брусиловский.pdf}
\end{center}

\begin{center}
\includegraphics[width=\textwidth, page=302]{Брусиловский.pdf}
\end{center}

\begin{center}
\includegraphics[width=\textwidth, page=303]{Брусиловский.pdf}
\end{center}

\begin{center}
\includegraphics[width=\textwidth, page=304]{Брусиловский.pdf}
\end{center}

\end{document}