\documentclass[main.tex]{subfiles}

\begin{document}

\section{Лекция 27.10.2022 (Брусиловский А.И.)}

\begin{center}
\includegraphics[width=\textwidth, page=146]{Брусиловский.pdf}
\end{center}

Сегодня последний день курса.
И я буду очень интенсивно рассказывать.
Если вы обратите внимание, мы на 146 слайде, у нас их около 300.
Вас это не должно смущать, я все успею вам рассказать, потому что значительная часть из того, что осталось, это посвящено корреляциям.
Просто я покажу вам наборы известных корреляций для моделирования свойств нефти, природного газа, пластовой воды.
Есть в прилагаемом материале программа, написанная на Excel, по которой можно посчитать по корреляциям различные свойства нефти, газа и воды, в том числе и те, которые в презентации корреляции указаны.
Это я буду делать в виде очень быстрого обзора, так что мы все с вами успеем.
Итак, за прошедшие 4 дня (сегодня у нас пятый день) мы с вами познакомились с, во-первых, особенностями компонентного состава природных углеводородных систем и умеем с вами отличить состав системы, находящейся в жидком агрегатном состоянии, то есть пластовой нефти, от многокомпонентной природной системы, находящейся в газовом агрегатном состоянии, то есть это природные газы.
Когда я говорю газовое агрегатное состояние, естественно, я имею в виду и свободные газы, и газовые шапки двухфазных залежей.
Также мы с вами рассмотрели методы получения информации с помощью глубинных и поверхностных проб о компонентном составе пластовых нефтей и газоконденсатных систем.
И мы с вами послушали о том, каковы особенности и требования к отбору проб.
И самое главное, что в каких случаях отбирают глубинные пробы, в каких поверхностные.
Напомню, что для нефтей основным способом получения информации о составе и свойствах пластового флюида является отбор глубинных проб.
Для газоконденсатных систем основным способом является рекомбинация поверхностных проб, полученных в сепараторах большого объема.
Это то, что предпочитают в промышленности.
Мы с вами узнали, какие эксперименты проводятся при исследовании пластовых нефтей и газоконденсатных систем, и какой физический смысл заложен в проведении этих экспериментов и в получении тех или иных результатов.

\begin{center}
\includegraphics[width=\textwidth, page=147]{Брусиловский.pdf}
\end{center}

И вот теперь мы перейдем к тому, каким...
Ну, фактически к методам математического моделирования свойств природных углеводородных систем и пластовых нефтей, и газоконденсатных систем сухих и жирных газов.
И два направления здесь существует.
Это применение уравнений состояния, основанное на фундаментальных положениях термодинамики многокомпонентных систем, и второе — применение корреляций, когда обработкой большого объема статистического и фактического материала, лабораторных и промысловых исследований, получают зависимости, позволяющие при отсутствии компонентного состава, почти всегда, тем не менее, оценивать так называемые PVT-свойства, необходимые для инженерных расчетов.
Начнем мы наше изложение с тех основ, которые используются при моделировании с использованием уравнений состояния.
Итак, это очень короткий обзор, но он дает абсолютно правильное направление вам для понимания того, что делается.

\begin{center}
\includegraphics[width=\textwidth, page=148]{Брусиловский.pdf}
\end{center}

Итак, необходимые условия фазового равновесия $N$-компонентной системы.
Это напоминание того, что я уже вам говорил.
В общем случае, если сосуществуют $N$-фаз $N$-компонентной система, то необходимым условием того, чтобы гетерогенная система, то есть находящаяся в многофазном состоянии, и при этом в состоянии равновесия.
Чтобы фазы были в равновесии, нам нужно соблюдение условий, которые вы видите на слайде.
Равенство температур во всех фазах, равенство давлений во всех фазах, равенство летучестей каждого компонента во всех фазах.
Первичным в термодинамике является равенство химических потенциалов компонента во всех фазах.
В термодинамике для практического использования вводится понятие летучести.
Летучесть –- это fugitiveness, по-английски fugacity.
Летучесть связана с химическим потенциалом однозначно.
На практике используют равенство летучестей.
Итак, как в частном случае, то что у нас в нашей практике инженерной, мы рассчитываем двухфазное равновесие пар-жидкость.
В большинстве случаев, в подавляющем большинстве случаев для $N$-компонентной системы.
И наша задача заключается в том, чтобы вычислить летучесть.
Так вот, из термодинамики многокомпонентных систем следует нижняя формула, что для многокомпонентной системы натуральный логарифм летучести $i$-ого компонента, он равен интеграл от $V$ до бесконечности.
Вот интегральное выражение.
То, что вы видите.
Смысла нет, это все озвучивать.
И вопрос заключается в том, а вообще как нам эту формулу использовать?
Нужно же знать частную производную давления по числу молей $i$-ого компонента при постоянной температуре, объеме и числах молей остальных компонентов.
Это основа.
Для этого используется уравнение состояния.
И уравнение состояния, чем точнее уравнение состояния описывает свойства вещества, тем точнее мы будем рассчитывать значения летучести компонентов и рассчитывать термодинамическое равновесие, в частности парожидкостное равновесие наших систем.

\begin{center}
\includegraphics[width=\textwidth, page=149]{Брусиловский.pdf}
\end{center}

В практике расчета парожидкостного равновесия последовательно на протяжении нескольких десятилетий использовали разные методы.
В начале так называемое давление схождения, использующее константы фазового равновесия.
И я об этом говорить не буду.
Раньше это было в 50-х, 60-х вплоть до 70-х годов.
Это был основной способ определения состава равновесных фаз углеводородных систем.
Применялись также комбинированные методы, использующие методы физической химии.
И основной, повторяю, комбинированный метод, там свойства газовой фазы рассчитывали с применением уравнения состояния, а свойства жидкой фазы рассчитывались с использованием теории регулярных растворов.
И с середины 70-х годов, когда появились уравнения состояния, более-менее адекватно описывающие парожидкостное равновесие многокомпонентных углеводородных систем, в том числе включающих азот, диоксид углерода и сероводород, стали использовать единые уравнения состояния, которые используют для описания и паровой фазы, или свободной газовой фазы, и жидкой фазы, находящейся…
Значит, рассчитывают условия…
Из заданных условий термобарических и суммарного компонентного состава рассчитывают, на фазы какого состава расслоится система и свойства этих фаз.
Нужно сказать, что используя уравнения состояния, мы можем рассчитывать не только компонентные равновесные составы фаз, сосуществующих, но и согласованные с ними термодинамические свойства.
Об этом, значит, я не думаю, что все слышали, а, по крайней мере, в учебниках об этом особенно не пишут.
Но если вы знаете компонентный состав фазы, то вы можете рассчитать все термодинамические свойства, в том числе энтальпию, энтропию, изобарно-изотермические и другие потенциалы, теплоемкость при постоянном давлении, постоянном объеме и интегральные дифференциальные эффекты Джоуля-Томсона.
Изоэнтальпийный эффект, это вот Джоуля-Томсона, изоэнтропийный эффект.
И также, зная, значит, уже определив компонентный состав и плотности фазы из термодинамических расчетов, вы можете рассчитать теплофизические свойства.
Главное теплофизическое свойство, которое нам необходимо для гидродинамических расчетов, как в пласте, так и в стволе скважины, в трубопроводах, это динамическая вязкость.
И я вам покажу, уже когда будем обсуждать, я буду делать обзор применения корреляций, там будут корреляции для оценки динамической вязкости газов и нефтей.

Напоминаю правила фаз Гиббса, классику термодинамики многокомпонентных систем, что мы можем определить с помощью правила фаз Гиббса, какое максимальное количество фаз в нашей системе.
И если у нас N компонентов, то максимальное количество фаз теоретически равно $N+2$, и мы имеем дело в основном, в подавляющем большинстве случаев, только с двумя фазами, это особенность наших природных углеводородных систем, что они в термобарических условиях, которые нас интересуют, пласты, значит, это температура где-то до 150 градусов Цельсия, давление до 1000 атмосфер, выше давления крайне редко, только в очень глубокопогруженных залежах, это исключение.
Значит, у нас существуют две фазы, расслаивается наша система на две фазы, паровую и жидкую, несмотря на то, что системы состоят из очень большого количества компонентов обычных (кроме когда мы имеем дело с сухими газами, там подавляющая концентрация метана, и, собственно говоря, там и фазовых превращений не происходит, если отсутствует конденсат).
Я всё время возвращаюсь к практическим вопросам, потому что они, ради понимания того, как на практике нам нужно поступать, и понимание того, что может быть, собственно, и этот курс читается.
И нас интересует, какое число независимых переменных системы нужно задать для однозначного определения значений остальных переменных, и вот скоро мы увидим формулировку задач расчёта парожидкостного равновесия для природных систем, которая вытекает из базовых положений фазового равновесия термодинамики многокомпонентных систем.
Формулировки будут.
Я вам напомню, когда мы будем рассматривать систему уравнений, очень скоро, что правило фаз Гиббса, оно говорит о том, какое количество независимых переменных нам нужно задать для однозначного определения значений остальных переменных, которыми являются составы фаз.
В правиле фаз Гиббса не говорится об относительном количестве фаз в системе, а только о их компонентном составе.
На практике нам нужно знать не только на фазы какого состава расслоится система, но и каково относительное количество фаз, в частности, при моделировании процесса контактной конденсации, когда мы моделируем сепарацию нашей смеси, промысловую сепарацию.
Для этого случая будет дана формулировка, соответствующая инженерной задаче.

\begin{center}
\includegraphics[width=\textwidth, page=150]{Брусиловский.pdf}
\end{center}

Итак, основное направление для моделирования свойств природных углеводородных систем -- это применение кубических уравнений состояния, которые к настоящему времени достаточно точно описывают и фазовое равновесие, и PVT-свойства фаз.
И началось это с уравнения состояния Ван-дер-Ваальса.
В 1873 году в своей докторской диссертации о непрерывности газового и жидкого состояния он предложил модификацию уравнения состояния идеального газа для моделирования свойств уже не идеальных газов, а реальных флюидов.
Речь шла о газах, а потом уже стали применять это уравнение состояния Ван-дер-Ваальсового типа и для моделирования свойств жидкой фазы потихонечку научились.
Итак, что сделал Ван-дер-Ваальдс, модернизируя уравнение идеального газа?
Он сделал такие предположения, что в идеальном газе, представим себе бильярдный стол и бильярдные шары, там нет сил межмолекулярного взаимодействия.
То есть, вот эти шары бильярдные, у них нет сил ни отталкивания, ни притяжения.
Мы гоняем эти шары, они соударяются и это простейший случай.
Вот два сомножителя вы видите, $pV$ было в уравнении идеального газа, а стало $p$ плюс некая добавка, $n$ квадрат $a$ делить на $V$ квадрат и $V$ минус $nb$.
Было учтено, что между молекулами существует притяжение, и поэтому внешнее давление, оно еще и корректируется на внутреннее давление, которое в итоге уменьшает наше давление.
Эта корректировка осуществляется членом $n$ квадрат $A$ делить на $V$ квадрат, что это взаимодействие, оно обратно пропорционально объему, занимаемому молекулами ($n$ молями вещества).
А кроме того, оно значит, вот обратно пропорционально расстоянию между молекулами.
В итоге у нас формируется вот такой член по отношению к внешнему давлению.
А второй сомножитель учитывает, что у молекул существует некий эффективный объем молекул, он приблизительно равен четырехкратному, это по оценкам Ван-дер-Вавльса, четырехкратному объему молекул, в который молекула не пускает другие молекулы как бы они ни старались проникнуть, соудариться с данной молекулой, ничего не получится.
Вот этот эффективный объем это учитывает.
В результате вот он сформулировал уравнение, которое стало основой крупного направления в развитии моделирования свойств реальных веществ.
И он понятия не имел о нефти и природном газе -- 1873 год, физик.
Но плодотворная такая вот идея, она, как и в других областях науки, плодотворная идея, обладающая самыми разными, с точки зрения численного моделирования, недостатками, недостаточно точно описывающими какие-то явления, тем не менее, является определяющей, вот идея определяет направление.
И уравнение Ван-дер-Ваальса позволило развиться крупному направлению в промышленности, и нефтехимической, и в области разработки, эксплуатации месторождения, оценки свойств самых различных природных углеводородных систем.
Хотя в численном отношении уравнение Ван-дер-Ваальса обладает разными недостатками, они перечислены в учебниках, в монографии, в частности, в тех, которые я писал.
И я сейчас не буду это делать.
Это те, кто заинтересуется более подробно этим направлением, будут читать специальные книги, обзоры, посвященные этому вопросу.
А мы перейдем к следующему слайду.

\begin{center}
\includegraphics[width=\textwidth, page=151]{Брусиловский.pdf}
\end{center}

Так вот, прошло 100 лет, и итальянский ученый Джордж Соаве, который занимался вопросами химической технологии, он на основе уравнения типа Ван-дер-Ваальса, но на самом деле он предложил модификацию, основанную на той, которую чуть ранее, в 1949 году осуществил немецкий ученый Редлих.
На слайде видите Соаве-Редлих-Квонг.
Редлих в 1949 году, профессор Редлих со своим аспирантом Квонгом, они предложили модифицировать уравнение Ван-дер-Ваальса в том виде, который вы видите в первой строчке этого слайда.
Но еще там в знаменателе стояла абсолютная температура в степени 0.5.
Собственно, у Ван-дер-Вальдса не было зависимости от температуры в…
Я вам сейчас еще раз покажу.
Видите, вот на предыдущем слайде во множителе первом $a$ -- это константа, которая, как и константа $b$, их величины определяются в условиях критической точки.
Подробно я не буду рассказывать, это долгая история.
Это все подробно описано.
А вот Редлих с Квонгом в 1949 году сделали модификацию формы уравнения Ван-дер-Вальса и стали точнее вычисляться, моделироваться свойства газовой фазы.
И где-то до давлений порядка 100 атмосфер.
Этого достаточно для того, чтобы свойства газовой фазы в технологических процессах, где давление несколько десятков атмосфер, уже моделировались с использованием уравнения состояния.
И плотность, Z-фактор, естественно, и другие свойства.
А что предложил Соаве в 1972 году?
И это послужило очень значительным достижением для уточнения расчета парожидкостного равновесия.
Для возможности не просто моделирования газовой фазы, а парожидкостного равновесия.
Он предложил, во-первых, Соаве я имею ввиду, а реализовал Иваскин, модифицировал знаменатель в правой части уравнения состояния.
Как вы видите оно уже переписано относительно давления.
Мы переписали форму, которая была на предыдущем слайде, в несколько другом виде.
Ну, просто преобразование сделали легкое алгебраическое.
У нас в левой части только давление оказалось.
И знаменатель у Ван-дер-Ваальса был $V$ квадрат просто, а здесь $V$ на ($V$ плюс $B$).
То есть, кроме того, предложил Соаве температурную зависимость при коэффициенте $A$ определять из условия точного моделирования упругости паров компонентов.
Первоначально уравнения состояния всегда предлагаются для чистых веществ.
А уже потом вводятся правила для расчёта коэффициента уравнения состояния для смесей.
И это следующий шаг.
А первоначально для чистых веществ.
И вот для десятка компонентов углеводородов плюс ещё азот, диоксид углерода, были определены коэффициенты $\alpha$.
Видите, сомножитель $\alpha$ при расчёте коэффициента $A$.
Из того, что в диапазоне температур от тройной точки до критической, вспомним про чистое вещество, уравнение состояния точно описывает упругость паров вещества.
Вот около десятка веществ были сделаны, были получены соответствующие коэффициенты.
Они были аппроксимированы.
От величины приведённой температуры, напомню, температура делённая на критическую температуру,
и от значения центрического фактора, о котором мы говорили, когда мы рассматривали таблицы свойств чистых веществ.
И говорилось, что такое ацентрический фактор, я уже сейчас этому время не буду уделять.
Но величина ацентрического фактора приводится как справочные данные наряду с критическими температурами, давлением, молекулярной массой веществ.
И затем был аппроксимирован коэффициент в выражении.
В общем, было предложено выражение для вычисления коэффициента $\alpha$.
То есть сначала были разрозненные, полученные значения $\alpha$, а потом это $\alpha$ было аппроксимировано тем выражением, которое вы видите.
И коэффициент $m$ зависит от ацентрического фактора.
Теперь, коэффициент $b$, и параметр $b$, и параметр $a_c$, они определяются из условий в критической точке, так же, как это предложил делать Ван-дер-Ваальс.
А что, там же было тоже $a$ и $b$, и вот что он впервые предложил, и Соаве тоже использовал это.
А то, что в критической точке у нас первая и вторая частные производные давления по объему, они равны нулю.
Вот это, если смотреть на…
То есть это точка перегиба для функции $p$ от $V$.
Точка перегиба для любого чистого вещества общее свойства в критической точке.
Точка перегиба.
Имея эти два уравнения, и уравнение состояния у нас два параметра, $a$ и $b$, мы получаем вот эти выражения, да, и задав структуру заранее, что у нас $b$ равняется некий коэффициент умноженный на универсальную газовую постоянную, на критическую температуру, деленную на критический объем, мы получаем коэффициент для этого вида уравнения состояния, как то, что вы видите, 0.08654.
А для коэффициента $a_c$, который имеет другую структуру, значит, $R$ квадрат, $t_c$ квадрат делить на $P_c$, коэффициент 0.42747.
Это просто получается из применения для уравнения состояния конкретного условия в критической точке.
Итак, главное, что уравнение состояния стало точно описывать упругость паров чистых компонентов.
То, что не могли делать с применением именно самого уравнения Ван-дер-Ваальса.
И, кроме того, уравнения Ван-дер-Ваальса температурные зависимости различных веществ, они описывались плохо.
А уравнения в той форме, которую предложил Соаве, гораздо точнее.
И многие термодинамические свойства в зависимости от температуры, особенно для газовой фазы, стали описываться.
Второй момент, это уже при переходе к смесям, что очень важно для нас.
Это было предложено правило вычисления коэффициента $a_m$ ($m$ -- это от mixture, смесь).
Вот ту, которую вы видите, которая отличается от того, что использовал Ван-дер-Ваальс.
Использовалось уравнение Редлиха-Квонга для смеси, более примитивное правило.
Что здесь важно?
Для понимания.
Вот у нас смесь состава $y_i$.
$y$, потому что, прежде всего, я вам говорил, что $y$ обозначает состав газовой, паровой фазы, а $x$ обозначает состав жидкой фазы.
А для смеси $z_i$.
Вот ещё одно замечание.
Уравнение состояния, оно описывает не свойства гетерогенные двух или более фазной системы сразу.
Оно описывает свойства отдельных фаз.
Прежде всего, уравнение Ван-дер-Ваальсового типа были предназначены для описания свойств газовой фазы.
Поэтому, здесь $y$ показан.
Но при расчёте парожидкостного равновесия, при расчёте летучести компонентов правила вычисления коэффициентов $a$ и $b$, они одинаковы для сосуществующих фаз.
То есть, для определённости здесь $y$ показан.
$y_i$ -- это мольная доля $i$-ого компонента в фазе.
Это первое.
Второе.
Появился сомножитель  с так называемыми коэффициентами парного взаимодействия.
Потому что без...
Но это поправочные коэффициенты.
Это эмпирические поправочные коэффициенты, предназначенные для того, чтобы уточнить расчёт свойств фазы многокомпонентной системы.
От двухкомпонентной до $N$-компонентной, $N$ -- произвольное число, какое вам нужно.
И были для ограниченного числа веществ приведены значения этих коэффициентов по оценкам Соаве и Редлиха.
Вот это основные модификации.
И надо сказать, что во всех программных комплексах, когда вы откроете, вы увидите, какие вам уравнения состояния предлагают использовать.
В этом перечне есть уравнение Соаве-Редлиха-Квонга.
Оно очень неплохо описывает (повторяю) свойства газовой фазы и вполне удовлетворительно рассчитывает парожидкостное равновесия при умеренных давлениях в несколько десятков атмосфер.
Прежде всего, Соаве занимался вопросами химической технологии.
То есть, это вопросы заводской переработки и так далее.
Это вот несколько десятков, но может быть до 150 бар.

\begin{center}
\includegraphics[width=\textwidth, page=152]{Брусиловский.pdf}
\end{center}

Теперь переходим дальше в нашем обзоре.
И вот в 1976 году канадский ученый Дональд Бишней-Робинсон, это выдающийся профессор, именем которого назван Центр по изучению теплофизических свойств веществ в провинции Альберта.
Так вот, он со своим аспирантом Пенгом, они все из Юго-Восточной Азии, поэтому Редлих, Квонг, Пенг и так далее.
Это помощники, вот они из Юго-Восточной Азии.
В 1976 году профессор Робинсон со своим аспирантом тоже предложил модифицировать кубическое уравнение Соаве Ван-дер-Ваальсового типа, вот в том виде, в котором вы видите.
Что здесь?
Главное, что был модифицирован знаменатель второго члена в правой части уравнения.
И это повлекло за собой то, что формулы для расчета коэффициентов $b$ и коэффициентов $a_c$, они изменились.
Потому что форма знаменателя изменилась, и сразу изменились эти коэффициенты.
Но подход к получению этих коэффициентов, то есть условия в критической точке, то что я говорил, точный расчет упругости паров и отдельных компонентов, правила смешения для коэффициентов $a$ и $b$ смеси были такие, какие были предложены ранее Соаве.
А изменение в знаменателе, оно было продиктовано тем, что уравнение Соаве, оно достаточно не точно, вернее так, недостаточно точно описывает плотность, а следовательно и свойства жидкой фазы.
И с увеличением числа атомов углерода эта погрешность сильно возрастает.
Вот пытаясь этот недостаток исправить, была проведена модификация в знаменателе второго члена правой части этого уравнения.

И это уравнение Пенга-Робинсона, оно в основном сейчас является рабочим.
Рабочим при моделировании фазового равновесия и свойств природных углеводородных систем, когда вы выбираете среди предлагаемых уравнений состояния, какое же вам использовать, вы обычно выбираете уравнение Пенга-Робинсона.
Но не во всех компаниях и не во всех случаях выбирают Пенга-Робинсона.
Моделируют и с уравнением Соаве-Редлиха-Квонга.
Но Пенга-Робинсона -- это более приемлемый вариант.

\begin{center}
\includegraphics[width=\textwidth, page=153]{Брусиловский.pdf}
\end{center}

Когда я учился в аспирантуре, то я вообще занялся изучением этого уравнения Пенга-Робинсона.
Я поступил в аспирантуру в 1977 году, а до этого профессор той кафедры, куда я поступил в аспирантуру, он получил монографию написанную Кацом с учениками.
Дональд Кац это выдающийся инженер, я уже говорил об этом, американский классик технологии моделирования свойств, ну вообще технологии описания свойств природных газов.
Так вот, была издана монография, в которой даже до публикации статьи в солидном журнале было приведено уравнение Пенга-Робинсона.
И мне было предложено написать программу по моделированию парожидкостного равновесия и вообще изучить, что это за уравнение.
И это было связано с необходимостью моделирования свойств уже при пластовых давлениях и температурах для месторождений газоконденсатных.
То есть все исследования, которые проводились, это была кафедра разработки газоконденсатных месторождений, это естественно прикладные исследования, но основанные на передовых достижениях в области математического моделирования.
Тогда это так было.
Так же как сейчас вы в Высшей Школе Теоретической Механики и Математической Физики осваиваете и пытаетесь развить методы математического моделирования различных процессов, в том числе и для связанных с разработкой месторождений природных углеводородов, вот тогда и в ведущем вузе нашей страны, Губкинском университете, который сейчас не является ведущим, на мой взгляд, то есть уже много воды утекло, вот тогда проводились на кафедре разработки газовых месторождений многочисленные исследования, в том числе первое моделирование многомерных процессов, происходящих в пластах при разработке месторождений, первые результаты по трехмерному моделированию, однофазного, были на этой кафедре.
И вот мне поручено было, я очень рад, это была удача, изучать уравнение Пенга-Робинсона, совершенствовать, изучать методы расчета парожидкостного равновесия для природных газоконденсатных систем, тогда только газоконденсатных систем, уравнения Ван-дер-Ваальсового вида для нефтяных систем тогда не применялись, все было очень далеко.
И поскольку институт занимался научной работой, связанной с освоением месторождений Прикаспия, это Оренбургское месторождение, Астраханское месторождение, и я был вплотную связан с изучением особенностей, свойств пластового газа, газоконденсатной системы Оренбургского месторождения, а там был и сероводород, и азот, и конденсат.
И в составе этого месторождения я довольно подробно в первый день вам говорил об особенностях.
Я говорил о том, что на этом месторождении был построен первый, а их очень мало, первый гелевый завод.
Это я уже заодно.
В составе природного газа Оренбургского месторождения обнаружили гелий, и поэтому построили гелевый завод, его извлекали.
Сейчас на Ковыктинском месторождении, это я вам с практической точки зрения говорю, потому что гелий -- это стратегическое сырье, из природного газа его можно извлекать в том случае, когда концентрация компонента превышает минимальную промышленную концентрацию.
Были в результате обработки экспериментальных данных десятков бинарных систем получены коэффициенты парного взаимодействия между диоксидом углерода и алканами, диоксидом углерода и сероводородом, сероводородом и азотом, и также между сероводородом и различными алканами, потому что для углеводородов с различным числом атомов углерода оказались разные величины, и были получены зависимости, которые вы видите на графике.
Вот эти величины коэффициентов парного взаимодействия, они были впервые получены для большого числа веществ, фактически для того массива компонентов, который присутствует в пластовых газах и газоконденсатных системах, и эти коэффициенты парного взаимодействия, они включены в базы данных программных комплексов, в частности, Eclipse.
Чем отличается развитие научных исследований сейчас от того, что было раньше?
Сейчас с помощью интернета мы имеем возможность изучать результаты, полученные по всему миру, и в частности, этому способствует очень библиотека SPE, общество инженеров-нефтяников, всемирное общество, которое сейчас включает более 250 тысяч технических статей, которые были доложены на различных конференциях.
Началось это в самом начале 80-х годов, формирование базы данных SPE, и в одной из первых работ под номером 1000 с чем-то или что-то такое, были опубликованы вот эти коэффициенты парного взаимодействия по Брусиловскому.
То есть, это один из результатов моей кандидатской диссертации.
Один из достаточно хороших результатов.
Но главное, что мы смогли приспособить, поняли, как использовать кубические уравнения состояния для моделирования PVT-свойств природных газоконденсатных систем.

\begin{center}
\includegraphics[width=\textwidth, page=154]{Брусиловский.pdf}
\end{center}

Тем не менее, несмотря на усовершенствования, проведенные Робинсоном и Пенгом, плотность жидкой фазы недостаточно точно описывалась.
Только для легких углеводородов ($C_5$, $C_6$, $C_7$) нормально описывалась плотность жидкой фазы, а для более тяжелых углеводов плохо.
То есть, повторяю, тогда речи о моделировании свойств нефтей пластовых с применением уравнений состояния в конце 70-х и первой половине 80-х годов вообще не шла.
И на нефтяной кафедре у нас, во всех институтах нашей страны, и за рубежом тоже использовали и для парожидкостного равновесия, для расчета парожидкостного равновесия, разгазирования в нефтяных системах, использовали подходы либо комбинированные, где кубические уравнения состояния применялись только для моделирования свойств газовой фазы, либо просто вообще кубические уравнения состояния не использовались нефтяниками.
И вот в 1988-м году американцы Явер и Юнгрин, они предложили использовать для улучшения плотности, для расчета плотности жидкой фазы способ, который был до этого предложен французами, а которые использовали идею другого американца Мартина.
В общем, развитие, видите как, идеи предлагаются, потом они развиваются не обязательно тем, кто предложил эту идею.
Вот такое коллективное развитие прикладной науки.
Вот оно привело к тому, что стали, что делать?
Расщеплять расчет парожидкостного равновесия и моделирование свойств плотности фаз.
Но это касалось прежде всего из-за необходимости моделирования, улучшения моделирования свойств жидкой фазы.
Значит, вначале рассчитывается парожидкостное равновесие, а с применением уравнения Пенга-Робинсона достаточно хорошо, и коэффициентов парного взаимодействия, что важно, достаточно хорошо составы фаз определяются в том диапазоне термобарических условий, которые соответствуют разработке месторождения.
А уже потом, с использованием рассчитанных составов равновесных фаз, можно уточнять плотность.
Ну, здесь в данном случае удельный объем, первоначальный удельный объем, как в уравнении состояния.
Значит, с применением поправочных коэффициентов структура для каждого компонента, вот она $c_i$ равняется $s_i$ (это вот самый шифт-параметр; шифт -- это сдвиг, сдвиговый параметр) умножить на коэффициент $b$ уравнения состояния для $i$-ого компонента так делается, для смеси для фазы используют правило смешения, которое выше вы видите.
Значит, простое суммирование произведений $c_i$ на мольную долю фазы.
И полученный коэффициент $c$, это и есть...
Значит, вот на его величину поправляют удельный объем, который рассчитали по исходному уравнению состояния, в данном случае, Пенга-Робинсона.
Такой же подход можно использовать, когда вы применяете уравнение Соаве-Редлиха-Квонга, любого кубического уравнения состояния.
Но конкретные величины, вот видите, компоненты и значения $s_i$, это получены при использовании уравнения Пенга-Робинсона, и для каждого уравнения состояния они свои.
Вот такую работу провели Явер и Юнгрин.
Эти значения шифт-параметра, они тоже, так же как и коэффициенты парного взаимодействия, которые в том числе и я получал, они включены в базы данных для программных комплексов, и когда вы выбираете уравнение состояния и вводите состав смеси в программу, то автоматически компонентам присваиваются свойства, которые им присущи, ну и справочные данные в том числе, и коэффициенты парного взаимодействия, матрица коэффициентов парного взаимодействия и величины шифт-параметров.
Это для тех, кто уже знакомился и считал по программе PVTI, ну это в Эклипсе или другим программам, ну вот, это таким образом происходит.

\begin{center}
\includegraphics[width=\textwidth, page=155]{Брусиловский.pdf}
\end{center}

В этих программах используется обобщенный вид кубического уравнения состояния, который был впервые предложен Майклом Михельсеном и Хейдеманом.
Майкл Михельсен -- это датский ученый, очень видный теоретик в области фазовых равновесий, математического моделирования процессов фазового равновесия, очень-очень грамотный человек, и Хейдеман Роберт -- это австриец, вот они взаимодействовали, математик.
Значит, они предложили такую форму в 1981 году, обобщенную, которая заложена в программных вычислительных комплексах, и в зависимости от того, что вы там кликните, выберите Пенг-Робинсон или Соаве-Редлих-Квонг, значения коэффициентов $\delta_1$, $\delta_2$, $\Omega_a$, $\Omega_b$ будут иметь те или иные значения.
Так же, как и вы видите, $\alpha$, $m$, но это уже все, так сказать, я вам показывал формулу.
То есть вы выбираете уравнение состояния просто кликом, и автоматом выбираются величины коэффициентов, которые приводят обобщенную форму к конкретной форме.

\begin{center}
\includegraphics[width=\textwidth, page=156]{Брусиловский.pdf}
\end{center}

Теперь, до сих пор мы, как исходно было предложено, использовали форму уравнения состояния относительно $p$ (что $p$ является функцией температуры и удельного объема $V$).
Гораздо более удобно использовать значение Z-фактора или коэффициента сжимаемости, который связан с мольным объемом и давлением уравнением реального газа.
И после введения комплексов безразмерных $A$ и $B$, уравнение Ван-дер-Ваальсового типа можно переписать относительно Z-фактора в том виде, который вы видите.
Этот вид и используется в программном вычислительном комплексе в настоящее время.
Это, почему говорят, кубическое уравнение состояния?
Ну вот вы видите, относительно Z-фактора записано уравнение состояния в кубической форме.
Также можно было бы записать относительно удельного объема, но получение корней значительно удобнее для Z-фактора, потому что сам диапазон изменения Z-фактора значительно меньше.
И удобнее пользоваться.
Любое кубическое уравнение состояния имеет три или один действительных корня.
Если у нас один действительный корень для фазы, ну вот мы его и выбираем.
А если у нас три действительных корня, то для газовой фазы мы выбираем наибольшее из положительных действительных корней, а если мы моделируем свойства жидкой фазы, то наименьшие из положительных действительных корней.
Ну то есть вот таким образом это делается.

Александр Иосифович, а можно вопрос?
Да.
А там не надо такой корень выбирать, у которого энергия Гиббса наименьшая?
Ну подождите, ну я же лекцию читаю, ну это вы знаете.
Это следующее, о чем я хочу сказать.
Надо, конечно.
Значит, это делается, да, совершенно верно.
Значит, дальше, ну просто слушатели другие не поймут, о чем речь идет.
Если я буду... Зря вы так, Денис.

Значит, конечно, это самый простой подход.
Когда наибольшее из положительных действительных корней соответствует газовой фазе, а жидкой фазе соответствует минимальный из положительных корней, но их всего три или один.
Один, тогда вообще вариантов нет.
А вот когда три, тогда нужно выбирать.
Но всегда это либо наибольший, либо наименьший корень.
Теперь, значит, в середине 80-х годов были опубликованы, вот как раз Майклом Михельсеном, критерии стабильности фазы.
И вообще было предложено то, что потом реализовано.
И сейчас это реализовано.
И это было реализовано и в моих программах, значит, вот в 90-х годах, которые я писал.
Что, значит, физически, каково состояние фазы вообще?
Вот она газовая или жидкая?
Это определяется тем, при каком агрегатном состоянии наша многокомпонентная система и вообще, так сказать, будет ли она в двухфазном состоянии или в однофазном состоянии, значит, она стремится к экстремальному значению потенциала.
Для изобарно-изотермического потенциала Гиббса, значит, это минимум.
Также можно выбирать для свободной энергии Гельмгольца и других потенциалов.
О потенциалах написано 75 страниц в моей книге 2002 года.
Вообще вот я стремился к тому, чтобы последовательно от первого и второго законов термодинамики показать, как выводятся условия фазового равновесия, какие критерии.
Там же показаны, приведены, это обобщение термодинамических достижений.
Вот там показано, что состояние системы соответствует минимальному значению изобарно-изотермического потенциала Гиббса, когда у нас независимые переменные давление и температура.
Я просто не хочу у нас курс PVT-свойств, а не курс по термодинамике или же по моделированию фазового состояния с применением уравнения состояния.
Я сейчас могу очень много и долго на эту тему говорить, не успею.
О чем-то другом, о большинству инженеров нужно сориентироваться, узнать о многих свойствах и различных методах.
Вот почему я не обо всем буду говорить.
В частности, я не буду говорить о методах решения систем трансцендентных уравнений, понимаете?
И по анализе стабильности.
Но это не тема нашего курса.
Я делаю... Дальше иду.
Значит, вот такая обобщенная форма кубического уравнения состояния, она заложена в программном вычислительном комплексе.
И я очень попрошу в дальнейшем, профессора не перебивайте.
Я готов с вами потом говорить на разные темы, но вы начали только изучать эту науку, какие-то отдельные вопросы.
И каждый из вопросов требует глубокого изучения.
Вам даются основы.
Это касается и вопросов исследования скважин, это касается и вопросов математического моделирования и так далее.
Эти основы очень важны для дальнейшего понимания.
Дальше.

\begin{center}
\includegraphics[width=\textwidth, page=157]{Брусиловский.pdf}
\end{center}

Я повторяю, для того, чтобы окончательно усвоить, что условия фазового равновесия, необходимые условия, вот они написаны.
Это прежде всего равенство... Всегда равенство температур.
Если нет равенства температур, то равновесия не будет.
Это равенство давлений.
Это условие, когда мы пренебрегаем наличием капиллярного давления.
И равенство летучести компонентов во всех сосуществующих фазах.
И это условие, оно необходимое, но не достаточное.
А достаточным является, если у нас минимум имеет экстремум, значит, потенциал.
Для потенциала Гиббса нужно, чтобы система имела минимум из всех возможных значений.
Теперь не всегда, опять же, если у нас есть силовые поля, ну вот, гравитационное поле, то у нас не будет равенства летучести компонентов.
Но мы предполагаем, что в тех вопросах, которые мы сейчас рассматриваем, у нас локальное термодинамическое равновесие между фазами достигается.
Мы не рассматриваем влияние гравитационного поля в нашей конкретной точке.
Ни когда мы рассчитываем сепарацию, ни когда мы рассчитываем фазовое равновесие при течении смеси к скважине в стволе, ни когда мы рассчитываем, когда мы моделируем с применением многокомпонентной фильтрации.
У нас в каждой разностной точке пласта, у нас соблюдаются условия равенства летучести.
Вот что мы...
А, надо сказать, что я могу отметить и следующее, что с точки зрения решения задач, а будет ли в равновесии у нас паровая и жидкая фазы, значит, не факт, что у нас...
У нас есть состав смеси, но мы не знаем ее, будет ли она расслаиваться на фазы...
Эта смесь при заданном давлении температуре, будет ли она расслаиваться на фазы, скажем...
А для этого в современных программных комплексах проводится анализ стабильности, критерии которого были предложены Михельсеном и Хейдеманом.
Это в специальном курсе должно быть, это в книжках, и в частности есть в моей книге, в зарубежных книгах все эти критерии описаны.
Мы идем дальше, у нас вводный курс по свойствам пластовых флюидов.

\begin{center}
\includegraphics[width=\textwidth, page=158]{Брусиловский.pdf}
\end{center}

Почему раньше уравнения состояния вообще не очень-то и применялись?
Ну, потому что их стали применять только, когда появились первые компьютеры.
До этого использовать уравнение состояния для расчета фазового равновесия не представлялось возможным, потому что для той обобщенной формы уравнения, которая записана в программных комплексах, вот опять же приведена формула общая термодинамическая для расчета коэффициента летучести, коэффициент летучести написан, это летучесть, деленная на парциальное давление в фазе.
Получается достаточно громоздкое выражение, вот оно внизу, логарифм коэффициента летучести.
Почему логарифм?
Я просто не стал писать, что коэффициент летучести равен экспонента и так далее, чтобы не загромождать, чтобы не загромождать написание.
Так вот, без компьютеров совершенно невозможно было вычислять для многокомпонентных систем эти выражения совершенно.
А сейчас это абсолютно возможно.
И когда мы считаем сепарацию, мы считаем там летучесть, будет сформулирована постановка задачи.
Ну, доли секунды и мы получаем ответ.
А почему композиционное моделирование существенно дольше по времени, чем моделирование с применением модели Black-Oil?
Так вот, это связано как раз с тем, что на каждом шаге, в каждой разностной точке, на каждом шаге временном мы должны оценивать стабильность нашей системы.
Будет ли она расслаиваться на фазы.
Если будет расслаиваться, да, и тогда тоже вычисляются громоздкие выражения, которые вытекают из расчета потенциалов для многокомпонентных систем.
И композиционное моделирование, оно кратно, многократно более долговременное, чем моделирование с применением модели типа Black-Oil.
Это что касается пластовых процессов.
А что касается моделирования процессов заводской переработки или же промысловой обработки, там, конечно, это значительно меньше по времени, чем процессы, происходящие в пластах.
И там мы с применением и персональных компьютеров получаем очень быстро результаты, необходимые нам результаты, моделируя различные процессы и аппараты.

\begin{center}
\includegraphics[width=\textwidth, page=159]{Брусиловский.pdf}
\end{center}

Теперь о формулировке основных задач.
Двухфазное парожидкостное равновесие.
Вот нам даны давление и температура, дан суммарный состав смеси.
И мы в результате должны определить мольную долю каждой фазы.
Это fraction of vapour ($F_V$), fraction of liquid ($F_L$).
И состав паровой фазы $y$, жидкой фазы $x$.
Нам нужно $2N+2$ уравнения.
Первые $N$ уравнений -- это равенство летучести компонентов в сосуществующих фазах.
Следующие $N$ уравнений -- это уравнения материального баланса, связывающие состав компонентов смеси в целом с его долей в фазах и долей самих фаз в системе.
Это еще $N$ уравнений.
И замыкающие соотношения -- это сумма мольных долей в одной из фаз.
В данном случае паровой фазы равняется единице.
И сумма мольных долей фаз равняется единице.
Вот у нас $2N+2$ переменные и $2N+2$ уравнения, система трансцедентных алгебраических уравнений, которую мы решаем одним из эффективных математических способов.
И здесь уже тот способ, который вы видите.
Решение должно быть одно.
Для любых способов решения...
Я в свое время использовал метод Ньютона-Рафсона, самый обычный метод.
Потом использовал методы другие для убыстрения расчета.
Для решения этой задачи используют также так называемые коэффициенты распределения или константы равновесия.
И последовательное их уточнение.
Это еще один способ.
Там не требуется вычисление частных производных, как в методе Ньютона.
Гораздо быстрее.
Но в околокритической области он не сходится.
То есть нельзя в околокритической области, где составы фаз становятся близкими, использовать метод последовательных приближений.
А там нужны уже методы типа метода Ньютона, где производные.
Это тонкость.
Это относится к вопросам решения задач нелинейных алгебраических уравнений.
Естественно, что в программных комплексах, которые пишут профессионалы, во всех областях профессионалы, от PVT-свойств, физики, математики, используются наилучшие методы.
Я вам должен сказать, что, если не помню, говорил об этом или нет.
Программу Eclipse писали математики и физики, которые работали над атомным проектом.
Они прямо группой перешли, когда решили основные проблемы, которые были.
И руководитель этой группы в начале 2000-х годов приезжал, еще не было Газпромнефти-НТЦ, это была Сибнефть, когда работала компания Инпетро.
Вот они приезжали в Москву.
Сейчас это уже исключено совершенно, визит таких гостей.
А тогда была оттепель.
И вот они посетили разные компании, в том числе НТЦ Сибнефти Инпетро.
И руководитель математиков, руководитель группы, он немножечко познакомился с тем, какие задачи стоят, какие используют.
И он сказал, я почему об этом вспомнил, когда сказал о том, что для промышленности работают лучшие специалисты.
Лучшие специалисты в области прикладных вычислений и программистов, и так далее, в разных областях.
Значит, он сказал, за ваши деньги любой каприз.
Это к вопросу о том, что если кто-то думает, что он лучше других, занимается математическим моделированием и так далее, нужно просто четко понимать, что лучшие работают на самом острие, а обычные специалисты, они эксплуатируют то, что сделали лучшие, рядовые, и пытаются что-то в отдельных направлениях улучшить.
Ну вот, значит, такое замечание.
Не случайно.
Schlumberger, скажем, ведь программа PVTI, она до сих пор уточняется.
Ну, я к примеру, или PVTSIM, это десятки лет уточнения, улучшения программных, улучшения метода, применения новых методов, расширения функциональных возможностей.
Это коллективы, работающие на постоянной основе ни месяц, ни полгода и ни год.
Это многие годы, постоянные коллективы, тогда добиваются успехов.
Тогда добиваются успехов, значимых.
И результатами этих исследований пользуются уже во многих компаниях, во многих местах.

\begin{center}
\includegraphics[width=\textwidth, page=160]{Брусиловский.pdf}
\end{center}

Следующая постановка задачи.
Значит, предыдущее, это вот когда мы в программе FLASH выбираем, это самое распространенное решение.
Теперь постановка задачи расчета давления начала конденсации.
У нас возникает пузырек газа первый, вспоминаем фазовую диаграмму в координатах давление-температура.
Значит, вспоминаем фазовую диаграмму, бесконечно маленький пузырек газа, и состав жидкой фазы.
Пардон, это я про нефть стал.
Просто меня немножко сбило вот это вот комментарий Хамедуллина, потому что я собирался об этом сказать.
Мы не на семинаре, еще раз, Денис.
Это очень плохое, ну как бы, нельзя так.
Это не семинар, это лекции, причем с заочным исполнением, большой курс.
Итак, значит, у нас возникает пузырек газа, неизвестного, пардон, давление начала конденсации.
Капля жидкости ретроградной, неизвестного состава.
Состав газа мы знаем, поскольку капля жидкости бесконечно маленькая, бесконечно маленькая.
Значит, это ретроградная фаза возникает с этого бесконечно маленькой капельки жидкости.
И давление мы не знаем, при котором эта капелька возникнет.
Нам задана температура, задан суммарный состав смеси.
Наша задача определить состав капельки.
И главное, значит, нам нужно определить давление начала конденсации, при котором возникнет эта ретроградная капелька жидкости.
Значит, и у нас $N+1$ неизвестная, и мы должны написать $N+1$ уравнение.
Вы уже видите, вы уже поняли в чем дело.
Вот это первое, это уравнение, которое я, значит, описываю, равенство летучестей компонентов в сосуществующих фазах.
Значит, между нашим газом, газоконденсатом, и этой вот капелькой жидкости.
И мы имеем $n$ уравнений.
Написали еще одно уравнение, замыкающее соотношение, что сумма концентраций компонентов в капле жидкости равняется единице.
И все этого нам достаточно для идентификации давления, при котором возникнет эта капелька жидкости, и ее состав.
И это давление начала ретроградной конденсации.
Причем, для особо интересующихся, я могу сказать, у нас две точки росы (в ретроградной области и при низком давлении).
Если мы вспоминаем вот эту фазовую диаграмму.
И таким образом, при этой формулировке нет различия в том, какое давление мы получим в ретроградной области или в области низкого давления, в области прямого испарения.
То есть если мы получим низкое давление, это нужно проверять, а в какую же сторону, вообще говоря, пошел алгоритм, и не получили ли мы давление не в ретроградной области, а в области прямой точки росы.
Раз интересуют подробности.

\begin{center}
\includegraphics[width=\textwidth, page=161]{Брусиловский.pdf}
\end{center}

Так вот, следующая формулировка, очень аналогичная.
Это давление начала кипения для пластовой нефти.
Нам задан состав пластовой нефти, нам задана температура.
Значит, начало кипения нефти соответствует давлению насыщения образования первого пузырька газа.
Значит, мы не знаем его состав.
Бесконечно маленький пузырек газа.
И мы пишем, что должно быть равенство летучестей всех компонентов, входящих в этот пузырёк газа, и нашей нефтяной системы при искомом давлении начала кипения (давление насыщения; saturation pressure; bubble point pressure; по-разному, по-разному пишут).
Значит, и сумма концентраций мольных долей компонентов в этом пузырьке газа должна быть равна единице.
Вот это похоже.
Вот формулировки трёх наиболее часто встречающихся задач.
Существует много других задач, которые, в частности, мне пришлось решать.
И которых я сейчас совершенно не касаюсь.
Но которые все основаны на использовании критериев, фундаментальных термодинамических критериев равенства летучестей компонентов при фазовом равновесии (необходимое условие).
И есть задачи самые разные.
Это я рассказал об основных задачах и принципе.
А методы решения этих систем алгебраических уравнений, я сказал, должны выбираться самые эффективные.
Собственно, они и используются в программных комплексах, которые пишут хорошие очень специалисты.

\begin{center}
\includegraphics[width=\textwidth, page=162]{Брусиловский.pdf}
\end{center}



\begin{center}
\includegraphics[width=\textwidth, page=163]{Брусиловский.pdf}
\end{center}

Теперь следующая тема.

\begin{center}
\includegraphics[width=\textwidth, page=164]{Брусиловский.pdf}
\end{center}

Следующая тема, это немножко мы подробнее.
Мы говорили с вами об экспериментальных исследованиях.
И обзор по уравнениям состояния был дан.
Постановка задач, расчёты фазового равновесия.
И очень коротко, штрих-пунктирно, о моделировании PVT-свойств пластовых нефтей с использованием уравнений состояния.
Для тех, кто более подробно хочет узнать, а как при использовании программных вычислительных комплексов строить PVT-модели и проводить вычисления, которые удовлетворяют данным экспериментальных исследований, в этом году утверждён нормативно-методический документ.
Но это, понимаете, вы студенты Политеха.
Просто вам хочу сказать, в каждой компании специалисты стремятся к созданию каких-то документов, которые облегчают проведение расчётов и понимания процессов для инженеров.
Вот и мы тоже создали нормативно-методический документ по формированию PVT-моделей природных углеводородных систем (пластовых нефтей и газоконденсатных систем) и их использованию в нашей программе.
У нас в компании используется не только, но в том числе и Eclipse, и комплекс PVTI программы.
Для использования этой программы мы прямо написали на 90 страницах подробные инструкции.
Это для тех, кто будет работать в нашей компании.
Ну и статьи тоже на эту тему публикуются.
Публикуется в журналах университетских, разных солидных журналах.
Мы тоже с Тарасом Ющенко в этом году опубликовали большую, такую фундаментальную статью по моделированию PVT-свойств.
Созданию моделей.
Тарас Сергеевич, кандидат физико-математических наук, выпускник Физтеха, который учился в аспирантуре Физтеха и одновременно стал работать в НТЦ.
Сейчас он руководитель направления в Сколтехе.
Это пример.
Я его привожу в пример, потому что очень целеустремленный, способный человек, который во взаимодействии с опытными специалистами добивается нормальных успехов.
И способствует успехам компании.
На передовых позициях.
Вот таким образом.

\begin{center}
\includegraphics[width=\textwidth, page=165]{Брусиловский.pdf}
\end{center}

Теперь сначала о моделировании с применением уравнения состояния пластовых нефтей.
Здесь общие вещи, которые вы уже знаете, я уже об этом говорил.
Идем дальше.

\begin{center}
\includegraphics[width=\textwidth, page=166]{Брусиловский.pdf}
\end{center}

Два основных направления, которые вы тоже уже понимаете.
Это использование применения уравнения состояния.
Для использования уравнения состояния нужно знать компонентный состав пластовой смеси.
И тогда мы можем использовать моделирование с применением уравнений состояния на основе фундаментальных термодинамических положений.
И использование корреляции.
Это более простая вещь, которая тоже очень полезна для инженерных расчетов.
Когда мы осуществляем не академические исследования, а должны проводить оценки инженерные, что и делается в нефтяных и газовых компаниях.
Не нужно забывать для каких целей.
Когда мы говорим про нефть и природные газы, это прежде всего прикладные цели.
И здесь разные методы используются.

\begin{center}
\includegraphics[width=\textwidth, page=167]{Брусиловский.pdf}
\end{center}

Первое, из чего мы.
Это основы применения уравнения состояния.
Вот вы видите уже знакомое вам уравнение Пенга-Робинсона с шифт параметром.
Тут уже объяснять нечего, только что мы прошли этот материал.
И в качестве основы мы используем экспериментальные исследования представительных проб, о которых я тоже много рассказывал.
И о том, какие проблемы с получением проб, и какие экспериментальные исследования проводят.
Только что сделан обзор по уравнениям состояния.
И вот как же на практике можно смоделировать, сделать модель пластовой нефти очень коротко.
И возможны варианты.
Вот то, как я делал с коллегами.
Когда мы используем уравнение состояния, помимо состава, вот если мы вводим состав, нам же нужно, чтобы каждый компонент был идентифицирован значением критической температуры, критического давления, ацентрического фактора.
Для чистых компонентов это справочные данные.
А для фракций $C_{7+}$, или если больше компонентов $C_{n+}$, где $n$ больше 7, значит нам нужно уметь вычислять эти параметры (критическая температура, критическое давление, ацентрический фактор).
Для вычислительной машины совершенно все равно, какой это, что это за компонент.
Фракция это или чистый компонент.
Ей важно для расчетов по алгоритму, который заложен, иметь некие константы.
Некие константы, которые характеризуют компоненты системы.

\begin{center}
\includegraphics[width=\textwidth, page=168]{Брусиловский.pdf}
\end{center}

Теперь мы используем для природных систем данные фракционной разгонки.
Если нам нужно разбить фракцию группы $C_{7+}$ пластовой нефти нужно разбить на фракции.
Это для особо легких нефтей.
О том, что такое эти нефти, вы уже отлично знаете.
И будет пример скоро показан особо легкой нефти с повышенным газосодержанием и обычной нефти, где мы можем вполне обойтись моделированием группы $C_{7+}$ без разбиения ее на фракции.
И в программных вычислительных комплексах для идентификации значений переменных фракций группы $C_{7+}$ или только этой фракции целиком используется регрессионный анализ.
Мы, столкнувшись с тем, что формальное использование регрессионного анализа нередко дает абсурдные результаты, абсурдные с физической точки зрения по значениям критических параметров и так далее, и о том, что это закрытая система, мы не можем повлиять на результаты, мы решили в свое время создать такой алгоритм последовательного воспроизведения ключевых осреднённых экспериментальных данных.
И это было одним из важных результатов в том числе кандидатской диссертации одной из наших сотрудниц.
И в результате в компании применяется уже много лет эффективный алгоритм создания PVT-моделей пластовых нефтей, где последовательно (повторяю, не как в черном ящике, а последовательно) определяются давление насыщения при пластовой температуре, плотность сепарированной нефти, объемный коэффициент при начальных термобарических условиях (с помощью, это делается с использованием уравнения состояния) и динамическая вязкость пластовой нефти при начальных термобарических условиях (это теплофизическое свойство; уравнение состояния тут ни при чем; значит, мы используем один из методов для моделирования динамической вязкости и идентифицируем, находим такое значение параметров в этой вот формуле, которая дает удовлетворительное значение динамической вязкости при пластовых термобарических условиях).

\begin{center}
\includegraphics[width=\textwidth, page=169]{Брусиловский.pdf}
\end{center}

Ну вот, давайте в той последовательности, о которой было сказано, что влияет на давление насыщения пластовой нефти?
Как можно, от чего зависит расчетная величина давления насыщения пластовой нефти?
Она зависит в сильной степени от коэффициента парного взаимодействия между метаном и группой $C_{7+}$.
Вот здесь показана возможная зависимость для реальной системы.
Почему именно между метаном и группой $C_{7+}$ мы выбрали?
Потому что метан всегда присутствует в пластовых нефтях.
Если у нас сухой газ растворён, его больше, и большое газосодержание его больше.
Если у нас несухой газ, и меньше его газосодержание, всё равно концентрация метана в пластовой нефти, за исключением битуминозных нефтей (это особый случай), она достаточная, она высокая.
И мы именно с использованием коэффициента парного взаимодействия метана с $C_{7+}$ эффективно можем подогнать, именно подогнать, более красивое слово, адаптировать туда-сюда.
Давление насыщения наше расчётное к экспериментальному значению.
Это первое.

\begin{center}
\includegraphics[width=\textwidth, page=170]{Брусиловский.pdf}
\end{center}

Затем плотность сепарированной нефти и затем объёмный коэффициент пластовой нефти.
При создании модели пластовой нефти возможны два типичных варианта идентификации объёмного коэффициента и плотности сепарированной нефти.
Вы видите на слайде, ну я просто для тех, кто хуже видит, говорю.
Первый вариант.
Лабораторные исследования ступенчатой сепарации проведены в соответствии с фактической системой промысловой сепарации.
И в этом случае значения плотности сепарированной нефти и объёмного коэффициента пластовой нефти воспроизводятся в соответствии с экспериментальными данными ступенчатой сепарации.
Почему именно ступенчатой?
Да потому что именно по результатам ступенчатой сепарации у нас определяются подсчётные параметры.
То есть запасы нашего флюида (нашей нефти) определяются параметрами ступенчатой сепарации.
Об этом я подробно говорил.
Но если у нас лабораторные исследования ступенчатой сепарации отсутствуют, бывают такие варианты, либо проведены при отсутствии данных о фактической системе промысловой сепарации, то тогда можно провести, воспроизвести значения по результатам однократного разгазирования (стандартной сепарации).
И это совершенно нормальные, хорошие результаты даёт, потому что и методики, которые мы развили, они очень-очень хорошо описывают свойства пластовых нефтей.

\begin{center}
\includegraphics[width=\textwidth, page=171]{Брусиловский.pdf}
\end{center}

Вот зависимость между...
После того, как мы определили параметр и воспроизвели давление насыщения, мы воспроизводим плотность сепарированной нефти от значения шифт-параметра группы $C_{7+}$ и выше.
Мы...
Дело в том, что именно группа $C_{7+}$ выше, потому что речь идет о сепарированной нефти, дегазированной, там практически нет метана, там основное на 90 с лишним процентов это группа $C_{7+}$ и выше.
Поэтому, значит, либо мы...
Если у нас $C_{7+}$ выше не разбита на фракции, тогда, вот как в данном случае, величина шифт-параметра группы $C_{7+}$ позволяет нам найти такое значение шифт-параметра, вот когда мы точно воспроизводим плотность сепарированной нефти.

И следующий вариант -- это зависимость объемного коэффициента пластовой нефти, следующий рисуночек, от значения шифт-параметра метана.
То есть, мы находим такое значение шифт-параметра метана, при котором объемный коэффициент по результатам...
Значит, вот объемный коэффициент пластовой нефти, он соответствует экспериментальному значению.
И это делается тоже быстро, эффективно.
И при этом, смотрите, у нас параметры, они почти...
Вот, коэффициент... до этого давление насыщения.
Коэффициент парного взаимодействия, он почти не влияет на плотность сепарированной нефти.
В то же время, коэффициент парного взаимодействия почти не влияет на величину объемного коэффициента.
То есть мы именно выбирали последовательность и такие параметры, которые не ухудшают ранее полученные результаты.
Именно такая последовательность.
Значит, и вот опять метан, как превалирующий в растворенном газе компонент, концентрация которого всегда значимая.
Вот, если мы варьируем его шифт-параметр, мы объемный коэффициент определяем...
Мы можем моделировать объемный коэффициент пластовой нефти достаточно...
Не достаточно, а просто точно.
Что мы и делаем на практике.
Мы именно таким образом поступаем.
Есть нюансы, о которых я сейчас не говорю, но эти нюансы, они описаны в нормативном методическом документе.
Там очень подробно все описано и так далее.
Ну да и в статьях, которые доступны и студентам, и сотрудникам других компаний и так далее.
В статьях все подробно описано.
Сейчас это обзор мы делаем.

\begin{center}
\includegraphics[width=\textwidth, page=172]{Брусиловский.pdf}
\end{center}

Вот, это вот, значит, и так еще раз.
Давление насыщения, плотность сепарированной нефти, объемный коэффициент.
То есть мы для подсчета запасов, для моделирования сепарации, все с вами определили и можем считать.
Но нам нужно также, когда мы выгружаем зависимости для гидродинамического моделирования, там, значит, нам требуются и зависимости динамической вязкости от давления получать.
И вопрос в том, опять же, чтобы это были реалистичные значения, мы имеем обычно зависимости экспериментальной и динамической вязкости от пластового давления.
И ориентируясь на них, мы должны...
Вот, расчетные зависимости, получаемые расчетом зависимости, должны быть близки к экспериментальным.
Вот для этого используется такая возможность, которая заложена в программных комплексах.
Это моделирование по методу Лоренца-Брэя-Кларка.
Вот написано, что это за метод.
Он очень давно был предложен.
Причем для моделирования динамической вязкости пластовых газов прежде всего.
Значит, это вообще 1964 год для интересующихся.
И речь шла о газах.
В статье было написано "<Газов и жидкостей">, но прежде всего для газов он используется.
Вот, а для жидкостей важен такой параметр, как критический объем группы $C_{7+}$.
Именно $C_{7+}$ в этом методе Лоренца-Брэя-Кларка.
Причем этот критический объем, это именно параметр, параметр для уточнения моделирования динамической вязкости.
Он не имеет отношения к критическому объему группы $C_{7+}$ или остатков.
Группа $C_{7+}$ -- это именно для расчета вязкости.
В программных комплексах, там тот, кто будет пользоваться или пользуется, обратите внимание, есть параметр, называется $V_{Crit,visc}$.
То есть для вязкости, критический объем для вязкости.
Вот вы можете его, изменять его значение и смотреть, как у вас будет меняться значительно весьма вязкость.
А если для нефтей, там группа $C_{7+}$.

\begin{center}
\includegraphics[width=\textwidth, page=173]{Брусиловский.pdf}
\end{center}

Вот, вот такой метод используется.
И здесь вот иллюстрация зависимости динамической вязкости пластовой нефти от значения псевдокритического мольного объема группы $C_{7+}$.
Причем вязкость зависит только от этого параметра.
Ну, параметры другие в методе, вот в формуле Лоренца-Брэя-Кларка.
Опыт показывает, нежелательно менять, потому что коэффициенты, потому что улучшив расчет плотности жидкой фазы, вы ухудшите расчет плотности газовой фазы, меняя значительно величины коэффициентов.
Ну, это так, это я заодно вам говорю, очень аккуратно нужно к изменению коэффициентов, если есть такая возможность, можно их тоже менять.
Но это может повлиять на ухудшение расчета вязкости другой фазы.
Значит, другой, ну, то есть газовой фазы, свойства которой тоже выгружаются и нужны для гидродинамических расчетов.
А вот, значит, мольная доля группы $C_{7+}$, когда вы ее по динамической вязкости пластовой нефти подбираете, она обычно даёт и возможность нормального воспроизведения динамической вязкости и газовой фазы.
Газовая фаза, как вы знаете, она порядка двух сотых, трёх сотых, ну, с повышением давления, естественно, её величина увеличивается и может быть весьма значительная, я пример вам приведу дальше.
Но нам очень важно при разработке для моделирования нефтяных месторождений правильно воспроизводить динамическую вязкость нефти.
Прежде всего, прежде всего.

\begin{center}
\includegraphics[width=\textwidth, page=174]{Брусиловский.pdf}
\end{center}



\begin{center}
\includegraphics[width=\textwidth, page=175]{Брусиловский.pdf}
\end{center}

Значит, следующее.
Вот хороший пример того, прежде всего, ну, во-первых, что наша методика даёт хорошие результаты.
Значит, вот пример.
И второе, второе, это то, что условия ступенчатой сепарации могут значительно влиять на величины нашей пластовой нефти, если её газосодержание высокое.
Вот в данном случае пример для пластовой нефти с повышенным газосодержанием.
Состав вы видите, это реальная система.
Вы видите сразу, что газосодержание пластовой нефти высокое, потому что метана 57\% мольных, то есть очень высокое газосодержание.
По однократному разгазированию экспериментальное значение у нас 2.05 объёмного коэффициента.
И по расчёту, значит, в данном случае мы нашу модель отрабатывали по результатам однократного разгазирования (стандартной сепарации).
Точно воспроизвели объём.
Здесь данные объёмного коэффициента, а и плотность, и другие параметры тоже точно воспроизвели.
Значит, здесь мы приводим данные по объёмному коэффициенту.
Точно, совершенно.
Затем мы созданную модель использовали для расчёта ступенчатой сепарации.
Прежде всего нас интересовала величина объёмного коэффициента, как различные условия ступенчатой сепарации влияют на величину объёмного коэффициента, а обратная величина объёмного коэффициента, это пересчётный коэффициент, используется в подсчёте запасов.
И два варианта промысловой сепарации, значит, это условия на ДНС, дожимную насосную станцию, вот вы видите, это реальные условия.
Значит, и второе, которые были воспроизведены в лаборатории, и получена экспериментальная величина объёмного коэффициента 1.909.
И наша модель, которая создана была по результатам однократного разгазирования, дала объёмный коэффициент равный 1.913.
То есть до двух знаков просто одинаковый с экспериментом результат, до трёх знаков есть расхождение 0.2\%, и это расхождение меньше, чем погрешность эксперимента.
И второй вариант, который воспроизводился, ступенчатая сепарация воспроизводилась в лабораторных условиях, это первая ступень 41 бар, нефть с повышенным газосодержанием, в лаборатории, значит, тогда не было возможности со столь высоким газосодержанием, то есть делать ниже давление на первой ступени ступенчатой сепарации, и сделали давление 4.1 мегапаскалей.
Объёмный, значит, экспериментальное значение тоже с высокой точностью было воспроизведено в результате моделирования ступенчатой сепарации.
Различие 0.3\% тоже в пределах погрешности эксперимента, но я хочу обратить ваше внимание на то, что различные условия ступенчатой сепарации значительно влияют на результаты идентификации объёмного коэффициента пластовой нефти с повышенным газосодержанием.
Вот, то есть, тут два заяца.
Первый заяц -- это то, что созданная модель по результатам стандартной сепарации отлично описывает результаты и ступенчатой сепарации, а второе -- это уже то, что иллюстрации того, что условия ступенчатой сепарации могут влиять существенно на идентификацию подсчётных параметров.

\begin{center}
\includegraphics[width=\textwidth, page=176]{Брусиловский.pdf}
\end{center}

Вот ещё раз, для чего применяются модели пластовых нефтей?
Очевидно, что для моделирования многокомпонентной фильтрации, для формирования зависимостей для Black Oil.
И третий пункт.
Я хочу обратить внимание.
Для реалистичного прогнозирования состава и свойств добываемой продукции, в частности состава растворённого газа.
Потому что, когда мы создали модель многокомпонентную и рассчитали модель многокомпонентной фильтрации и ступенчатую сепарацию с реальными условиями, мы прогнозируем компонентный состав добываемого растворённого газа, который выделяется на промысле.
И не только компонентный состав, но и объёмы самого газа, что очень важно для прогнозирования, что нужно, какие технологические решения для переработки или транспорта этого газа.
Это позволяет создание таких моделей, PVT-моделей адекватных для разных целей может использоваться, в том числе для прогнозирования состава и свойств добываемой продукции.

\begin{center}
\includegraphics[width=\textwidth, page=177]{Брусиловский.pdf}
\end{center}

Теперь о формировании зависимостей свойств фаз для проектирования разработки.
Три метода.
Первый -- использование результатов дифференциального разгазирования.
Использовался этот метод по данным экспериментальных исследований дифференциального разгазирования.
До того, как стали эти программные вычислительные комплексы современные, стали давать возможность и других методов получения искомых зависимости.
Комплексное использование данных дифференциального разгазирования при пластовой температуре и промысловой ступенчатой сепарации исходной пластовой нефти.
Этот метод был предложен еще в 1953 году одним из американских специалистов, очень хороших.
И описан в статьях.
По-моему, в этой презентации я не имел возможности, просто по времени и так далее.
Это описано в статьях, в обзоре.
И в современных программных комплексах что делается?
То, что предложил Кёртис Витсон.
Это профессор Тронхейминского университета.
И он в своей книге "<Phase Behavior"> об этом пишет.
И это реализовано.
Это моделируется дифференциальное разгазирование.
И на каждом шаге дифференциального разгазирования пластовой нефти моделируется промысловая ступенчатая сепарация.
И вот сейчас для выгрузки в гидродинамических симуляторах используется этот метод.

Все, опять же, основано на моделировании фазового равновесия.
Если вы создали адекватную, хорошую PVT-модель пластовой нефти, вы получите хорошие реалистичные зависимости.
PVT-зависимости для гидродинамического моделирования.
И уже то, как будет адекватно или нет, потом описывается добыча и так далее, это уже зависит не от PVT-модели непосредственно, а от тех исходных данных, которые были использованы для создания этой PVT-модели.
То есть были ли пробы представительными, была ли достаточно то, о чем я говорил,
достаточно охарактеризована площадь нефтеностности вот этими пробами, с тем, чтобы после их осреднения пластовая нефть адекватно описывала свойства пластовой нефти для данной залежи, для данного пласта.
Тогда и в случае, когда у вас, скажем, фазовые проницаемости будут нормально отражать физику коллекторов и сами свойства коллекторов, вы получите правильные прогнозные результаты.
От многих вещей зависит, в том числе от PVT, как правильно создавать PVT-модели,
какие возможны ошибки, вот об этом шла речь, это вы уже знаете.

\begin{center}
\includegraphics[width=\textwidth, page=178]{Брусиловский.pdf}
\end{center}

Теперь два очень значимых примера, полезных примера.
Первое -- это пластовая нефть Юрских отложений.
Смотрите на ее состав, на результат однократного разгазирования.
Значит, это по составу, по метану, это сразу видно, нефть с повышенным газосодержанием, плотность сепарированной нефти соответствует особо легкой нефти, и для нефтей с повышенным газосодержанием, таких как вот эта вот нефть юрской залежи, посмотрите на значение вязкости пластовой нефти - 0.12.
У нас температура 102, высокая температура, и высокое давление пластовое -- 42 МПа.
Несмотря на столь высокое давление, вследствие повышенного газосодержания прежде всего, у нас вязкость пластовой нефти 0,12 мПа*с -- это экспериментальное значение.

И справа -- пласт, значит, где у нас битуминозная нефть, посмотрите, плотность сепарированной нефти -- 915, по составу сразу видно, что газосодержание невысокое, и оно по однократному разгазированию -- 46, оно фактически в 12 раз меньше, в общем, на порядок меньше газосодержание, чем у нефти Юрской залежи.
И температура ниже значительно, и давление, то есть соответствует, всё соответствует.
И посмотрите на вязкость в пластовых условиях этой нефти, она в 100 раз выше.

Вот это замечательные примеры взаимосвязи компонентного состава пластовых нефтей, значит, с характеристиками, PVT-характеристиками, в том числе с динамической вязкостью пластовой нефти.
И в 100 раз для этих пластовых нефтей отличается величина вязкости в пластовых условиях.

\begin{center}
\includegraphics[width=\textwidth, page=179]{Брусиловский.pdf}
\end{center}

Это означает, что...
Да, ну вот вы видите иллюстрацию результатов слева для нефти с повышенным газосодержанием, и тут по результатам различных экспериментов видно чётко совершенно, как отличается газосодержание, объёмные коэффициенты, динамика.
А для нефти справа с низким газосодержанием результаты, полученные что дифференциальным разгазированием, что дифференциально-ступенчатым, значит, что однакратным...
Короче, они все сливаются и имеют практически одно и то же значение.
Значит, вот по этой причине в 70-х годах проводили практически только однократную сепарацию, поскольку нефти были с низким газосодержанием, и по плотности эти нефти были тяжёлые или средние, и по характеристикам результаты, что ступенчатой сепарации, что дифференциального разгазирования при пластовой температуре, пластовая температура была низкой, не были глубоко погружены залежи, они сливались, имели очень близкое значение.
Отсюда и формулировка о том, как осуществлять подсчёт запасов (с помощью каких параметров) в регламентах Государственной комиссии по запасам, там и говорилось, по результатам дифференциального разгазирования, и не акцентировалось внимание более точно.
Вот откуда, откуда это вот всё.

\begin{center}
\includegraphics[width=\textwidth, page=180]{Брусиловский.pdf}
\end{center}



\begin{center}
\includegraphics[width=\textwidth, page=181]{Брусиловский.pdf}
\end{center}

И, значит, да, дальше поехали.
Вот формула для подсчёта запасов пластовой нефти.
И первые четыре сомножителя характеризуют наш пласт нефтенасыщенный, а два последних характеризуют PVT-свойства пластовой нефти.
Значит, обозначение принятое в регламентах ГКЗ.
Пересчётный коэффициент величина обратная объёмному коэффициенту, и нужно использовать результаты ступенчатой сепарации, плотность нефти в поверхностных условиях $\sigma_{\text{н}}$ (не $\rho$, там, ну, в общем, тут все обозначения указаны).
Эти обозначения соответствуют, они, вот эта формула приведена методическими рекомендациями по подсчёту запасов, геологических запасов нефти.
Кто будет с этим связываться, имейте в виду, есть такие методические рекомендации подробные.
И с повышением пересчётного коэффициента увеличиваются геологические запасы нефти, и, значит, стрелочки тут указаны.

\begin{center}
\includegraphics[width=\textwidth, page=182]{Брусиловский.pdf}
\end{center}

И увеличение пересчётного коэффициента соответствует уменьшению объёмного коэффициента (величина обратная).

\begin{center}
\includegraphics[width=\textwidth, page=183]{Брусиловский.pdf}
\end{center}

Следующее, значит, это взаимосвязь между объёмным коэффициентом и газосодержанием.
Вот вы видите, что объёмный... Это просто из материального баланса, это никакая не эмпирика.
Вот эти формулы, значит, чем выше газосодержание, получается, тем выше объёмный коэффициент.
А если поменять местами в формуле, которые справа, знаменатель...
То есть можем формулу преобразовать, что плотность пластовой нефти равняется, числитель этой формулы делённый на объёмный коэффициент, мы получим...
Эта формула применяется, когда... Вот я её применял, чтобы проверять данные технических отчётов по исследованию пластовых нефтей.
Данные, которые были по экспериментальным данным.
Насколько экспериментальные данные не противоречат друг другу.
Значит, вот денситометром полученные плотность пластовой нефти и объёмный коэффициент в лаборатории и так далее, вот эта формула использовалась.
И, значит, вот запасы, геологические запасы растворённого газа определяются просто произведением запасов нефти на газосодержание.
С учётом того, что запасы нефти у нас в тоннах, а запасы газа в кубометрах стандартных, вот, в зависимости от того, газосодержание в каких единицах, в данном случае метр куб на метр куб, значит, мы должны поделить на плотности сепарированной нефти.
Если у нас газосодержание метр куб на тонну, то уже деления на плотность не требуется.
Ну вот, понятно вам.

\begin{center}
\includegraphics[width=\textwidth, page=184]{Брусиловский.pdf}
\end{center}



\begin{center}
\includegraphics[width=\textwidth, page=185]{Брусиловский.pdf}
\end{center}

И о влиянии условий промысловой сепарации на подсчётные параметры нефти и растворённого газа на примере вы уже видели.
Как могут параметры термобарические условия ступенчатой сепарации значимо влиять на подсчётные параметры нефти при повышенном газосодержании её.
И для нефтей особо лёгких нужно учитывать требования по упругости паров в сырой нефти, не более 500 мм ртутного столба при температуре 37.8, то, о чём я вам уже говорил и в первый день, и далее.
Когда напоминал, о чём была речь на первой лекции.

\begin{center}
\includegraphics[width=\textwidth, page=186]{Брусиловский.pdf}
\end{center}

И здесь приводится тот текст, который был в регламентах Государственной комиссии по запасам 1984 года.
Подсчёт запасов в нефти приводится по результатам дифференциального разгазирования глубинных, понимаете, дифференциального разгазирования.
С позицией для нефти с повышенным газосодержанием это совершенно неправильно.
И справа тот текст, который отражает правильную формулировку, я написал его, чётко ясно, как должны подсчётные параметры нефтей определяться.
И то же самое относится к растворённому газу.
Это в 1984 году.

\begin{center}
\includegraphics[width=\textwidth, page=187]{Брусиловский.pdf}
\end{center}

В последнем варианте 2013 года уже было дифференциального, или ступенчатого разгазирования.
Так вот, хорошо, что запятую не забыли.
Вот эта всё нечёткость этих формулировок, она привела к тому, что, скажем, в одной очень крупной нефтяной компании подсчёт запасов регламентировали делать по результатам дифференциального разгазирования.
Не ступенчатой сепарации, а дифференциального разгазирования.
И исходя из формулировок ГКЗ, якобы, и только после разъяснения, которые я давал, обратились ко мне из этой компании специалисты, которые просили объяснить руководству этой компании, как же надо делать.
Вот я написал записку, обратились ко мне просто коллеги, с которыми мы поддерживали хорошие отношения творческие.
Вот пришлось объяснять руководству, самому высокому руководству крупнейшей нефтяной компании, как же, в чём физический смысл, и на что нужно ориентироваться для подсчёта запасов пластовой нефти.
И это я просто говорю о важности правильных формулировок, правильного понимания.

\begin{center}
\includegraphics[width=\textwidth, page=188]{Брусиловский.pdf}
\end{center}

Теперь, переходим к корреляциям.
Про уравнение состояния для моделирования нефтей мы уже очень многое знаем.
А на практике используют корреляции, не требующие знания компонентного состава, что важно, для проведения различных инженерных расчётов и для оценок.

\begin{center}
\includegraphics[width=\textwidth, page=189]{Брусиловский.pdf}
\end{center}

Что такое корреляция, это понятно.
Взаимосвязь (функциональная зависимость) исследуемого параметра и так далее.
Это всё понятно, вы уже прочитали.

\begin{center}
\includegraphics[width=\textwidth, page=190]{Брусиловский.pdf}
\end{center}

Зачем они нужны?
Когда мы не имеем компонентного состава, когда нам нужно оценить свойства нефти, газа и пластовой воды при конкретных термобарических условиях без знания компонентного состава.
Бывает, что нет лабораторных данных, либо они неточны, противоречивы, но на раннем этапе, например, лабораторные данные просто отсутствуют.
Нам нужно оценить и давление насыщения нефти, и вязкость её в пластовых условий, для оценок первичных.
Кроме того, корреляции для оценки вязкости, для вязкости вообще корреляция используется, а для термодинамических свойств не всегда имеются результаты лабораторных исследований, и нет данных по компонентному составу.
Вот тогда используют корреляцию.

\begin{center}
\includegraphics[width=\textwidth, page=191]{Брусиловский.pdf}
\end{center}

Это приведен график по вязкости, вы уже знаете это все, излом при давлении насыщения.
Если нет прямых замеров, то применяют корреляции.

\begin{center}
\includegraphics[width=\textwidth, page=192]{Брусиловский.pdf}
\end{center}

Ну, все понятно, и вопрос в том, какие же корреляции?

\begin{center}
\includegraphics[width=\textwidth, page=193]{Брусиловский.pdf}
\end{center}

Вот еще, вот формула Дюпюи (оценка дебита нефти).
Вы видите в знаменателе нужно знать вязкость, динамическую вязкость в пластовых условиях пластовой нефти, объемный коэффициент пластовой нефти.
Значит, если нет экспериментальных данных, нет лабораторных данных, то применяют корреляции для оценки динамической вязкости и объемного коэффициента.

\begin{center}
\includegraphics[width=\textwidth, page=194]{Брусиловский.pdf}
\end{center}

И в качестве примера показано, как значение вязкости влияет на дебет.
Ну, очень значительно, просто обратно, ну, это в соответствии с формулой Дюпюи.
У нас же фильтрация происходит по закону Дарси, в который входит вязкость в знаменателе.
Поэтому при, вот из формулы Дюпюи следует, что при одной и той же депрессии, и, значит, у нас от вязкости пластовой нефти зависит обратно пропорционально дебит нефти, который мы получаем.

\begin{center}
\includegraphics[width=\textwidth, page=195]{Брусиловский.pdf}
\end{center}

И от объемного коэффициента, от величины объемного коэффициента тоже зависит и дебит, который мы оцениваем, и, естественно, что прогнозный объем геологических запасов нефти, вот, ну, это понятно, что.
А запасы нефти, они вообще по регламенту, обязательно, наличие экспериментальных данных, лабораторных данных по объемному коэффициенту, по растворенному газу, и только в исключительных случаях позволяют использовать корреляции, когда нет экспериментальных данных.
Но это не лучший вариант.
Обязательно нужно проводить эксперименты, обязательно.
Значит, прогнозный объем добычи нефти тоже, значит, вот показана формула, да, recovery factor - это коэффициент излечения нефти, он зависит от прогнозных объемов геологических запасов нефти,
которые, в свою очередь, зависят от величины, от применяемой величины объемного коэффициента при начальных пластовых условиях.
И еще раз я повторяю, что при повышенном газосодержании обязательно будет приличное очень расхождение между величиной объемного коэффициента, полученной по результатам ступенчатой сепарации и по результатам однократного разгазирования и дифференциального разгазирования.
Это нужно четко совершенно понимать.

\begin{center}
\includegraphics[width=\textwidth, page=196]{Брусиловский.pdf}
\end{center}

Ну, еще один пример, уже все понятно.

\begin{center}
\includegraphics[width=\textwidth, page=197]{Брусиловский.pdf}
\end{center}

Теперь, значит, вот базовые параметры, которые мы используем в корреляциях -- это газосодержание нефти при давлении насыщения, обозначение $R_{sb}$.
Это относительная плотность нефти, безразмерная величина, напоминаю, да, которая численно равна граммам на сантиметр куб или тоннам на метр куб, относительная плотность газа по воздуху.
Вы все это уже хорошо знаете.
В корреляциях используется, значит, вот обозначение, которое вы видите, и в некоторых корреляциях, не во всех, есть величины температуры и давления в сепараторе, первой ступени сепарации.
В некоторых корреляциях, вот они, значит, используют суммарное полное газосодержание, а некоторые только то газосодержание, которое следует из замеров на первой ступени сепарации, из того объема газа, который на первой ступени сепарации получен.

\begin{center}
\includegraphics[width=\textwidth, page=198]{Брусиловский.pdf}
\end{center}

Вот, теперь в модели Black Oil, они все требуют каких исходных данных?
Вот в первой строчке, когда вот исходные данные для модели Black Oil мы используем, это относительная плотность нефти, относительная плотность газа неизменные, и газосодержание, объемный коэффициент, динамическая вязкость изменяются под давлением.
При этом температура тоже фиксирована.
В моей практике мы фиксируем температуру, ну, пластовую температуру, при пластовой температуре, да, это вот эксперименты проводятся, значит, изменяется только давление.
Но, возможно, в случае, когда, значит, вот изменяется температура для каких-то участков пласта, ну, это просто разные зависимости.
Вот, и особенность модели Black Oil, еще раз, что у нас постоянным считается неизменным относительная плотность нефти и относительная плотность растворенного газа.
Это совершенно не соответствует тому, что происходит в газоконденсатных месторождениях.
Поэтому, значит, вот, использование модели Black Oil даже с учетом возможности растворимости нефтяного компонента в газовой фазе, для моделирования, значит, газоконденсатных месторождений нежелательно.
Нежелательно, то есть этой опции нежелательно.
Газоконденсатные месторождения нужно моделировать, значит, с применением композиционного моделирования.
Либо, второй вариант, это то, что мы тоже предложили, и теперь в нашей компании это, значит, уже рабочий инструмент.
Мы моделируем с применением модели Black Oil, это для газоконденсатной системы.
Но параллельно для, значит, вот, уже прогнозирования компонентного состава и свойств добываемого газа, мы используем результаты дифференциальной конденсации CVD, процесс, полученный при лабораторных исследованиях.
И, значит, при лабораторных исследованиях, либо, значит, на основе моделирования соответствующих процессов, математического моделирования с применением адекватных моделей пластовой газоконденсатной системы.
Это я вам заодно говорю, и дальше я про газоконденсатную систему вам тоже расскажу кое-что.
И второе, композиционная модель, модель многокомпонентной фильтрации.

\begin{center}
\includegraphics[width=\textwidth, page=199]{Брусиловский.pdf}
\end{center}

Теперь модели сырой нефти, значит, они в разных странах разными авторами предлагали, предлагали корреляционные модели.
И вот здесь приведены условия для нефтей с какой плотностью, вот, были получены различные корреляционные модели.

\begin{center}
\includegraphics[width=\textwidth, page=200]{Брусиловский.pdf}
\end{center}

И вот приводятся дальше корреляции наиболее популярные в инженерной практике для оценки давления насыщения.
Приводится, вот вы видите корреляцию Standing, полученную Майклом Стендингом, фамилия которого уже неоднократно звучала в моих лекциях.
Это если в газовой технологии это Кац, Дональд Кац, значит, лидер и научный руководитель был, то лидером в PVT-исследованиях пластовых нефтей в Америке долгие годы был Майкл Стендинг.
Проводил экспериментальные исследования и создавал корреляции.
Вот его корреляция по давлению насыщения, очень известная, рекомендуемая и в наших учебниках, и по разработке нефтяных месторождений.
Значит, сюда входит начальное газосодержание, то есть газосодержание при давлении насыщения $R_{sb}$, сюда входит относительная плотность растворенного газа и нужно еще знать пластовые температуры и относительную плотность нефти $\gamma_{o}$.

\begin{center}
\includegraphics[width=\textwidth, page=201]{Брусиловский.pdf}
\end{center}

Дальше.
Относительно недавно, ну, 20 лет назад уже прошло, время просто не течет, оно бежит со страшной силой.
Значит, вот McCain, Билл Маккейн, это очень известный специалист американский, которому сейчас 89 лет и который, ну, очень известен по всему миру своими рекомендациями по оценке свойств нефтей.
Прежде всего, нефтей пластовых.
И, значит, он трудился в различных организациях.
Любопытный факт, это написано, что он генерал, бригадный генерал был.
Он бригадный генерал, то есть до 40, там до 45 лет вообще служил в войсках, которые обслуживали горючие смазочные материалы, как я понимаю, и занимался физико-химическими свойствами и так далее.
А потом уже стал профессором известных университетов и самым известным на Западе специалистом по PVT-свойствам для оценки их, ну, его очень известные монографии.
Он даже приезжал в начале 2000-х в Москву и был в Ноябрьске.
И, значит, вот он проводил занятия и учил пользоваться корреляциями.
И мне удалось познакомиться с ним.
Значит, сейчас ему 89 лет, надеюсь, что он в добром здравии.
В 80 лет он еще издал монографию с другими специалистами, ну, то есть исключительный человек.
Компетенция хорошая и такое трудолюбие.

\begin{center}
\includegraphics[width=\textwidth, page=202]{Брусиловский.pdf}
\end{center}

Значит, вот он тоже опубликовал давление насыщения пластовой нефти.
И это уже модель другого типа.
Вот Стендинг и Маккейн самые известные спецы.
Но Стендинг, повторяю, все-таки основа его работы это экспериментальные исследования.
У Маккейна это обработка экспериментальных исследований, получение корреляций и объяснение, как ими пользоваться. Вот еще, ну, вот все тут написано.
И вот обратите внимание, значит, что здесь гамма газа это для газа первой ступени сепарации.
А $\gamma_{gc}$ это для газа с учетом и стоктанкоил (STO).
То есть, когда вы, плотность всего растворенного газа, относительную плотность и относительную плотность всего растворенного газа должны использовать.
А гамма газа это когда у нас это относительная плотность для газа первой ступени.
Вот чтобы нормально использовать корреляции и приведено тут нужно знать еще условия первой ступени сепарации это давление $P_s$.
Это на первой ступени сепарации, но там условные обозначения были до этого даны.

\begin{center}
\includegraphics[width=\textwidth, page=203]{Брусиловский.pdf}
\end{center}

Glaso это специалист, который, значит, вот работает в Скандинавии, работал, вот и публикации, им достаточно много лет, но они апробированы эти корреляции.
Значит, на примере исследований нефтей Северного моря.
И тоже есть вот.
То есть, у Стендинга, несмотря на то, что он работал только с Калифорнийскими нефтями, которые достаточно тяжелые, полученные им корреляции неплохо каким-то образом, но вполне неплохо описывают свойства и средних нефтей, и не только тяжелых нефтей.
А Glaso работал с нефтями средней плотности.
Я об этом почему говорю?
Потому что, чтобы понимать все-таки корреляции при каких условиях, на каком экспериментальном материале могли быть получены.
Вот видите, здесь в данном случае написано, что для нефтей с плотностью, здесь есть уточнение, чтобы одни параметры для нефтей с плотностью меньше, чем средней нефти.
Средней нефти помните?
От 0.85 до 0.87.
Значит, если меньше чем 0.85, то это для легких получается.
А для нефтей средних и тяжелых другие параметры.

\begin{center}
\includegraphics[width=\textwidth, page=204]{Брусиловский.pdf}
\end{center}

Рекомендации даются, это для инженеров, какие, когда, что использовать.
И если у нас в данном случае, тут говорится о необходимости корректировки, когда есть сероводород и диоксид углерода и азот.
Знаем, когда концентрация их в растворенном газе, мы можем делать поправки, это влияет на давление насыщения, на оценки.

\begin{center}
\includegraphics[width=\textwidth, page=205]{Брусиловский.pdf}
\end{center}

Теперь, Al-Marhoun, это Египетский специалист, то есть это нефть Ближнего Востока.
И, значит, соответственно, такая корреляция.

\begin{center}
\includegraphics[width=\textwidth, page=206]{Брусиловский.pdf}
\end{center}

Это самое известное. Вообще этих корреляций огромное количество.
И поэтому, для каких нефтей брать, в принципе, имеет значение.
Но тут нужно понимать, что это все для оценок, когда у вас нет экспериментальных данных.
Значит, теперь, вот это иллюстрация того, как, смотрите, вот у нас фиксировано, как плотность растворенного газа влияет на давление насыщения.
Вот, у нас есть газ, гамма газа 0.75, это у нас, значит, в основном из метана, этана состоящий газ.
Если только из метана, то это будет 0.6 относительная плотность.
Значит, тут есть этан и пропан, но не слишком много.
Значит, и мы видим, что давление насыщения для одного и того же газосодержания растёт с увеличением плотности сепарированной нефти.
Ну, вот это очевидно, очевидный факт.
А вот, значит, это хорошо здесь видно.
Значит, и справа уже зависимости приведены для случая, когда у нас растворённый газ более тяжёлый, был 0.75 относительная плотность, а стала 0.85.
И плотность для тех же плотностей нефти.
Мы видим просто визуально, что давление насыщения с увеличением жирности газа, увеличением его плотности уменьшается.
И я вам говорил уже, когда характеризовал составы, что если у нас битуминозная нефть.
Битуминозная нефть, это как раз первый, который здесь указан, 0.9 и выше.
И обычно в этих нефтях растворён сухой газ.
И поэтому давление насыщения обычно очень быстро возрастает с увеличением газосодержания.
Потому что представим себе бинарную систему, состоящую из очень тяжёлого компонента, там скажем, $C_{20}$ и метана.
Небольшое добавление метана будет приводить к очень значимому увеличению давления насыщения пластовой нефти.
Это вот иллюстрация.

\begin{center}
\includegraphics[width=\textwidth, page=207]{Брусиловский.pdf}
\end{center}

\begin{center}
\includegraphics[width=\textwidth, page=208]{Брусиловский.pdf}
\end{center}

Затем, вот по давлению насыщения мы закончили.
Теперь, как оценить газосодержание?
Вот формула, Стендингом полученная.
Вы можете сравнить с формулой для давления насыщения.
Для оценки давления насыщения требовалось знание газосодержания.
А мы перевернули эту формулу.
Стендинг перевернул.
И получаем формулу для оценки газосодержания при известном давлении насыщения.
То есть, вот текущее газосодержание при давлении $P$.
А давление $P$ соответствует давлению насыщения при этом газосодержании.
Это вот.
Значит, вот оценить можно по этой формуле.
Опять же, ну вот тут я внизу сделал примечание.
Инверсия выражения для давления насыщения.
И зависимость газосодержания от давления практически линейная.
Тут все написано.
Значит, все обозначения вам уже понятны.
Для оценки, так же как для оценки давления насыщения нужно знать какие характеристики?
Относительную плотность газа, относительную плотность нефти, рабочую температуру, рабочее давление.

\begin{center}
\includegraphics[width=\textwidth, page=209]{Брусиловский.pdf}
\end{center}

И значительно более трудоемкие для вычислений.
Ну понятно, для компьютеров уже.
Значит, это вот Velarde и McCain, 1999 года.
Для оценки текущего газосодержания при давлении меньше, чем давление насыщения.
Нам нужно знать газосодержание при давлении насыщения, текущее пластовое давление и вот эти коэффициенты $a_1$, $a_2$, $a_3$, которые зависят от относительной плотности растворенного газа и от давления насыщения нефти.
Ну то есть тут науки никакой нет.
Тут просто обработка большого числа, очень большого числа экспериментальных исследований.
И полезные для инженеров соотношения.
Соотношения, которые легко запрограммировать и в Excel.
И вот я должен сказать, в том файле Excel, по свойствам, которые Александр Иванович Адегов, он увлекался этой темой.
Занимался ей много лет, в Роснефти.
И потом когда к нам пришел тоже.
Вот там большая часть этих эмпирических зависимостей приведена.
И вам даны задачи.
На втором десятке.
Там всего 15.
Где требуется сравнить значения каких-то параметров пластовой нефти по различным корреляциям.
Значит вы с помощью той программы можете это сделать.
А можете и сами запрограммировать вот эти формулы, которые приведены в презентации.
И главное, что нужно, какие исходные данные еще раз?
Относительная плотность газа, относительная плотность нефти.
Понятно, рабочая температура, рабочее давление.
Но если давление насыщения нужно определить, то давление нам не нужно.
$\gamma_o$, $\gamma_g$ и температура, естественно, что и $R_{sb}$.
Ну то есть газосодержание при давлении насыщения.
Все очень просто.

\begin{center}
\includegraphics[width=\textwidth, page=210]{Брусиловский.pdf}
\end{center}

Это опять же тоже перевернутая зависимость для давления насыщения была.
А теперь вот она стала зависимостью для газосодержания.
Glaso -- это по исследованиям для месторождения Северного моря.

\begin{center}
\includegraphics[width=\textwidth, page=211]{Брусиловский.pdf}
\end{center}

По месторождениям Ближнего Востока те же авторы.
Только теперь уже не давление насыщения, а газосодержание.

\begin{center}
\includegraphics[width=\textwidth, page=212]{Брусиловский.pdf}
\end{center}

И здесь вот такой комментарий, что если по корреляциям Стендинга мы имеем линейную зависимость газосодержания от давления при снижении давления, то, например, по Velarde McCain у нас учитывается, что газ не сухой, а плотность его выше, чем у сухого газа.
И представленные рекомендуемые корреляции, они позволяют учесть отклонение вот этой вот прямой при низких давлениях для газосодержания.
Это имеет значение при решении задач, технологических задач, когда нам нужно оценить газосодержание, на какой глубине нужно, например, установить сепараторы на нефтяных месторождениях, скважины для отвода, для того, чтобы не сгорели глубинные насосы.
Такие вот технологические задачи.

\begin{center}
\includegraphics[width=\textwidth, page=213]{Брусиловский.pdf}
\end{center}

Теперь опять же из той же серии зависимостей газосодержания от плотности растворенного газа.

\begin{center}
\includegraphics[width=\textwidth, page=214]{Брусиловский.pdf}
\end{center}

\begin{center}
\includegraphics[width=\textwidth, page=215]{Брусиловский.pdf}
\end{center}

Теперь такое свойство, как сжимаемость нефти.
Здесь обозначение $C_{0}$; у нас принято в наших регламентах вообще $\beta$ обозначать сжимаемость нефти.
И когда я рассказывал о сжимаемости нефти там было $\beta=-\frac{1}{V}\frac{dV}{dP}$.
Ну, вот это вот параметры и есть.
Здесь приведена корреляция, и видно, что сжимаемость нефти зависит от газосодержания, плотности растворенного газа и плотности растворенной нефти, и давления.
Используется при давлении выше, чем давление насыщения.

\begin{center}
\includegraphics[width=\textwidth, page=216]{Брусиловский.pdf}
\end{center}

Вот график зависимости от давления коэффициента сжимаемости.
Изотермический, это не сжимаемость.
Точно что?
Точно это изотермический коэффициент сжимаемости.
Вот так называется.

\begin{center}
\includegraphics[width=\textwidth, page=217]{Брусиловский.pdf}
\end{center}

Ну вот еще раз, как ведет себя плотность пластовой нефти.
Значит, излом при давлении насыщения.
Уже объяснялись эти зависимости.

\begin{center}
\includegraphics[width=\textwidth, page=218]{Брусиловский.pdf}
\end{center}

Теперь плотность нефти по Стендингу.
Вот перед вами вот эту формулу вы видите, вот примечание написал.
Это не корреляция, а выводится из материального баланса.
И почему 1.21?
Просто плотность воздуха умножается на относительную плотность газа растворенного.
И вместо 1.205 точной плотности воздуха написано 1.21.
Вот откуда этот коэффициент.
Это не эмпирика.
Здесь примечания все есть.
И можно эту формулу даже самим вывести.
Совсем это делается несложно.
И в частности в моей книге 2002 года есть вывод вот этой формулы.
То есть это не корреляция.
Значит при давлении выше, чем давление насыщения, плотность нефти, когда мы знаем плотность при давлении насыщения, вот она таким образом определяется, зная сжимаемость и при нашем давлении $P$, вот эта формула применяется.

\begin{center}
\includegraphics[width=\textwidth, page=219]{Брусиловский.pdf}
\end{center}

Ну, дальше плотность нефти по корреляции McCain.
Я ее не использовал.
Тут много шагов и итераций еще.
В общем, я как-то очень скептически на всё это смотрю.
Потому что нужно совсем простые методы корреляции применять в инженерных расчетах.
Если они усложненные, то это уже очень проблематично.
Смотреть сколько шагов, всё это не вызывает доверия большого.

\begin{center}
\includegraphics[width=\textwidth, page=220]{Брусиловский.pdf}
\end{center}



\begin{center}
\includegraphics[width=\textwidth, page=221]{Брусиловский.pdf}
\end{center}



\begin{center}
\includegraphics[width=\textwidth, page=222]{Брусиловский.pdf}
\end{center}

Теперь объемный коэффициент.
При давлении меньше, чем давление насыщения.
Вот есть формула Стендинга.
Опять же, мы должны знать газосодержание текущее, при этом давление $P$, плотность растворенного газа и дегазированной нефти, и рабочую температуру.
Рабочую температуру, а при давлении выше, чем давление насыщения, опять же, используем объемную упругость для оценки объемного коэффициента.

\begin{center}
\includegraphics[width=\textwidth, page=223]{Брусиловский.pdf}
\end{center}

Объемный коэффициент, конечно, это не по McCain, но я уж оставил так, как предлагал мой коллега писать в свое время.
Это просто материальный баланс.
И приведен он в книге McCain.
Это не значит, что это по McCain.
Это то, что я говорил из материального баланса.

\begin{center}
\includegraphics[width=\textwidth, page=224]{Брусиловский.pdf}
\end{center}

Вот формула Glaso.

\begin{center}
\includegraphics[width=\textwidth, page=225]{Брусиловский.pdf}
\end{center}

Вы видите, что это не запоминать не надо ничего.
Это просто нужно использовать.
Использовать в Excel простые формулы для инженерных расчетов.

\begin{center}
\includegraphics[width=\textwidth, page=226]{Брусиловский.pdf}
\end{center}

Я тоже корреляции использовал.
Для какого случая?
Значит, вот это полезно.
Мы решили оценить в свое время, это было лет 15 назад, применимость корреляций Стендинга для месторождений Западной Сибири, которые эксплуатируют и разрабатывают наши компании.
Десятки месторождений многопластовых и так далее.
Мы разбили наши нефти по плотности, по типам.
Первый вы видите легкая нефть, 830-850, средняя нефть, тяжелых и битуминозных.
Вы видите, что тяжелых нефтей у нас в наших месторождениях гораздо меньше, чем средних и легких.
Легких достаточно много; а битуминозных еще меньше, чем тяжёлых.
Ну и слава Богу.
Потому что высоковязкие нефти сложно.
Битуминозные нефти, я уже говорил, например, в Татнефти специальная организация Битумнефти, там огромные запасы в Татарии битумов и битуминозной нефти, там термические методы воздействия и прочее, это не характерно для месторождений [не распознано] той компании, где я работаю.
Так вот, разбили мы и решили применить корреляции Стендинга.
Это вот объем, сейчас.
Объем, по-моему.
А, ну да, вспоминаю.
И вот отклонение.
Имели экспериментальные данные и решили проверить, насколько точно, какое отклонение, если оценивать по корреляциям Стендинга, и получили вот такие диаграммы.
Ну, просто это фактические данные нанесены.

\begin{center}
\includegraphics[width=\textwidth, page=227]{Брусиловский.pdf}
\end{center}

И какие же мы выводы сделали?
Что оказывается, подтвердили, что применение метода Стендинга позволяет с точностью до 5\%.
Это очень приличная точность при отсутствии лабораторных данных.
Оценивать объемный коэффициент битуминозных, тяжелых и средних нефтей во всем исследованном диапазоне газосодержания пластовой нефти, а легких нефтей до газосодержания 170 метр куб на метр куб.

И где используют корреляции?
Вот тут написано, являются альтернативой применению метода аналогий.
И вот сейчас будет пример.
Использовали раньше, иногда и сейчас используют.
Метод аналогий совершенно неуместно бывает.
Бывает совершенно неуместно.
То есть, для оценки пересчётного коэффициента при отсутствии результатов исследований представительных глубинных проб.

\begin{center}
\includegraphics[width=\textwidth, page=228]{Брусиловский.pdf}
\end{center}

Вот пример использования корреляций.
Кстати, можно с их помощью получать и зависимости для гидродинамического моделирования.
Показано сравнение полученных с применением корреляций Стендинга зависимостей для гидродинамического моделирования, для Black Oil, объемного коэффициента, и точное с помощью адекватной термодинамической модели.
Тут достаточно низкое газосодержание, поэтому и объемный коэффициент невысокий, поэтому результаты не так уж и далеки.
Но с повышением газосодержания, конечно же, точность корреляций падает существенно.

\begin{center}
\includegraphics[width=\textwidth, page=229]{Брусиловский.pdf}
\end{center}

Так вот, вот форма 6-ГР (ГР - Геологоразведка).
И справа данные, которые были в форме статистической отчетности.
Это каждая компания представляет в конце каждого года.
Для одного из месторождений такие данные были для Юрских пластов.
При различном газосодержании, вот $\text{Ю}_0$ и $\text{Ю}_0^1$, посмотрите, существенно различном, был один и тот же объемный коэффициент представлен.
Такого быть не может, мы с вами знаем, потому что объемный коэффициент очень существенно зависит от газосодержания.
И когда мы использовали эти данные, да еще и удивило то, что плотность сепарированной нефти сильно отличалась для близких очень пластов $\text{Ю}_0$ и $\text{Ю}_0^1$.
Вот когда уточнили результаты фактически, там поверхностных проб, уточнили плотность, но главное, что когда использовали фактическое газосодержание, вот 180, 100 и 110, мы его не подвергали сомнению и уточнили объемный коэффициент по корреляции Стендинга.
Мы получили совершенно физически нормальные значения, уточнили значение объемного коэффициента.
Потому что просто нелепо, не могут быть одинаковые значения объемного коэффициента при различном газосодержании.

\begin{center}
\includegraphics[width=\textwidth, page=230]{Брусиловский.pdf}
\end{center}



\begin{center}
\includegraphics[width=\textwidth, page=231]{Брусиловский.pdf}
\end{center}



\begin{center}
\includegraphics[width=\textwidth, page=232]{Брусиловский.pdf}
\end{center}



\begin{center}
\includegraphics[width=\textwidth, page=233]{Брусиловский.pdf}
\end{center}



\begin{center}
\includegraphics[width=\textwidth, page=234]{Брусиловский.pdf}
\end{center}

Значит, ну и теперь, понятно вам, зависимость вязкости от давления.
И здесь приведено, от чего зависит вязкость дегазированной нефти.
Ну, тут приведено от плотности, во-первых, дегазированной нефти, от температуры.
Ну, возможно, в некоторых случаях используют так называемый характеристический фактор Ватсона $K_W$, который отражает групповой углеродный состав нефти.

\begin{center}
\includegraphics[width=\textwidth, page=235]{Брусиловский.pdf}
\end{center}

Ну, вот и для этого фактора Ватсона, характеристический состав, видно корреляция.

\begin{center}
\includegraphics[width=\textwidth, page=236]{Брусиловский.pdf}
\end{center}

И можно оценить по Стендингу вязкость, вот логарифм отношения вязкости дегазированной нефти к плотности сепарированной, к плотности нефти, который в формулу включает характеристический фактор.
Вот.

\begin{center}
\includegraphics[width=\textwidth, page=237]{Брусиловский.pdf}
\end{center}

Ну, это всё эмпирика, запоминать ничего не надо.
Ничего не надо запоминать.
Вот просто по Стендингу есть вязкость нефти, по Beal-Standing.

\begin{center}
\includegraphics[width=\textwidth, page=238]{Брусиловский.pdf}
\end{center}

Есть вязкость нефти по Glaso.
Это всё те же авторы, которые для других свойств корреляции предлагают.

\begin{center}
\includegraphics[width=\textwidth, page=239]{Брусиловский.pdf}
\end{center}

И Al-Khafaji -- это Ближний Восток.
И здесь суть такая, что вязкость дегазированной нефти, она же очень сильно отличается от вязкости нефти газированной.
Ну, вот оценивают вязкость нефти дегазированной по корреляции, а затем её используют, уточняют, какая же будет вязкость для газированной нефти.

Сейчас я просто в хорошем темпе показываю, что же есть.
Что же есть по корреляциям.
И дальше, мы рассмотрим для газоконденсатных систем.
Минут 10-15 займёт.
И потом быстренько корреляции для газа и для воды, и всё.
И я надеюсь, что 30-40 минут нам хватит, чтобы завершить наш курс.
Итак, это вам просто полезно, полезно знать, какие корреляции, значит, заложены.
Вот можно, во-первых, запрограммировать в Excel, самим где-то использовать.
И чтобы не просто по программе лазить, по названию даже, вот, Al-Khafaji, Glaso, и тут соответствующие свойства.
И в программе посмотреть, какие же, сравнить, значит, какие значения, свойства разные авторы дают и так далее.
Ну, вот.
И выбрать для себя приемлемые корреляции для своих каких-то расчётов.
Если вам это будет нужно.

\begin{center}
\includegraphics[width=\textwidth, page=240]{Брусиловский.pdf}
\end{center}



\begin{center}
\includegraphics[width=\textwidth, page=241]{Брусиловский.pdf}
\end{center}



\begin{center}
\includegraphics[width=\textwidth, page=242]{Брусиловский.pdf}
\end{center}

Но сотни корреляций, на самом деле, предложены.
Не десятки даже, а сотни.
Ну, для отдельных свойств, значит, очень много.
И тут показано просто, что разные авторы использовали разные источники.
Значит, сколько проб, из каких регионов, какой диапазон данных и так далее.
По газосодержанию, по температурам, по давлениям.
Вот.
И вопрос в том, ну, что же использовать.

\begin{center}
\includegraphics[width=\textwidth, page=243]{Брусиловский.pdf}
\end{center}

И вот даются такие рекомендации, разные подходы к выбору.
Ну, понятно.
Либо, если вот программа, кстати говоря, вот какие корреляции.
Но, насколько я понимаю, в разных прикладных программах есть, используют корреляции, есть настройки по умолчанию.
Значит, кто-то использует, естественно, по умолчанию, не зная, что за корреляции.
Ну, надеясь на авторов программ.
Значит, вот.
Есть, возможно, другие правила.
Если это не стандартные программы, нет возможности выбора, то можно оценивать на основе плотности нефти, на основе региона.
По аналогии того, как на соседнем месторождении.
Ну, понятно.
В общем, ну, тут показано, еще раз.
Стендинг делал свои корреляции на основе нефтей Калифорнийских, а там были в основном тяжелые нефти.
Но, повторяю, его корреляции, тем не менее, они очень неплохо и для нефтей других типов.
Но не с повышенным газосодержанием работают.
Для Северного моря -- Glaso, там легкая нефть.
Для Ближнего Востока – средняя нефть.
И, значит, Bill McCain, он для... рекомендует универсальные корреляции.
Кстати, вот я хочу сказать.
Вышла книжка McCain со специалистами из Schlumberger и еще одним специалистом из фирмы [не распознано] по корреляциям.
State of the art -- состояние вопроса и там рекомендуется... есть рекомендации по использованию корреляций.
И большинство корреляций -- это те, которые в этой презентации есть.
Либо, значит, если они предлагают улучшенные корреляции, то буквально там на несколько процентов или десятые доли процентов лучше, чем известные корреляции.
Ну, есть рекомендации, в частности, те, которые даны в этой презентации, они вполне годятся.

\begin{center}
\includegraphics[width=\textwidth, page=244]{Брусиловский.pdf}
\end{center}

Дальше по поводу газоконденсатных систем.

Для газоконденсатных систем мы корреляции не используем, за исключением Z-фактора.

\begin{center}
\includegraphics[width=\textwidth, page=245]{Брусиловский.pdf}
\end{center}



\begin{center}
\includegraphics[width=\textwidth, page=246]{Брусиловский.pdf}
\end{center}



\begin{center}
\includegraphics[width=\textwidth, page=247]{Брусиловский.pdf}
\end{center}



\begin{center}
\includegraphics[width=\textwidth, page=248]{Брусиловский.pdf}
\end{center}



\begin{center}
\includegraphics[width=\textwidth, page=249]{Брусиловский.pdf}
\end{center}



\begin{center}
\includegraphics[width=\textwidth, page=250]{Брусиловский.pdf}
\end{center}



\begin{center}
\includegraphics[width=\textwidth, page=251]{Брусиловский.pdf}
\end{center}



\begin{center}
\includegraphics[width=\textwidth, page=252]{Брусиловский.pdf}
\end{center}



\begin{center}
\includegraphics[width=\textwidth, page=253]{Брусиловский.pdf}
\end{center}



\begin{center}
\includegraphics[width=\textwidth, page=254]{Брусиловский.pdf}
\end{center}



\begin{center}
\includegraphics[width=\textwidth, page=255]{Брусиловский.pdf}
\end{center}



\begin{center}
\includegraphics[width=\textwidth, page=256]{Брусиловский.pdf}
\end{center}



\begin{center}
\includegraphics[width=\textwidth, page=257]{Брусиловский.pdf}
\end{center}



\begin{center}
\includegraphics[width=\textwidth, page=258]{Брусиловский.pdf}
\end{center}



\begin{center}
\includegraphics[width=\textwidth, page=259]{Брусиловский.pdf}
\end{center}



\begin{center}
\includegraphics[width=\textwidth, page=260]{Брусиловский.pdf}
\end{center}



\begin{center}
\includegraphics[width=\textwidth, page=261]{Брусиловский.pdf}
\end{center}



\begin{center}
\includegraphics[width=\textwidth, page=262]{Брусиловский.pdf}
\end{center}



\begin{center}
\includegraphics[width=\textwidth, page=263]{Брусиловский.pdf}
\end{center}



\begin{center}
\includegraphics[width=\textwidth, page=264]{Брусиловский.pdf}
\end{center}



\begin{center}
\includegraphics[width=\textwidth, page=265]{Брусиловский.pdf}
\end{center}



\begin{center}
\includegraphics[width=\textwidth, page=266]{Брусиловский.pdf}
\end{center}



\begin{center}
\includegraphics[width=\textwidth, page=267]{Брусиловский.pdf}
\end{center}



\begin{center}
\includegraphics[width=\textwidth, page=268]{Брусиловский.pdf}
\end{center}



\begin{center}
\includegraphics[width=\textwidth, page=269]{Брусиловский.pdf}
\end{center}



\begin{center}
\includegraphics[width=\textwidth, page=270]{Брусиловский.pdf}
\end{center}



\begin{center}
\includegraphics[width=\textwidth, page=271]{Брусиловский.pdf}
\end{center}



\begin{center}
\includegraphics[width=\textwidth, page=272]{Брусиловский.pdf}
\end{center}



\begin{center}
\includegraphics[width=\textwidth, page=273]{Брусиловский.pdf}
\end{center}

И я вам хочу сказать на примере глубокопогруженной залежи, на примере результатов исследований моделирования, отдельные особенности, которые присущи для газоконденсатных систем с повышенным...
Уникальноконденсатных газоконденсатных систем, то есть тех, где высокая концентрация пентанов плюс вышекипящих в пластовом газе.
Вот какие особенности для глубокопогруженных залежей, характеризующихся повышенными давлениями и повышенным содержанием $C_{5+}$ и выше.
Да плюс еще содержащих и сероводород и $CO_2$.
Смотрите, значит у нас давление для... Это реальная залежь, это, по-моему, это Карачаганак, да?
Да, Карачаганак.
Значит, это начальное давление 57.3 мегапаскаля, там аномально высокое пластовое давление.
Глубина почти 5 километров и давление начала конденсации, как я говорил, оно близко к гидростатическому всегда, и 48.3 мегапаскаля.
То есть это необычно высокие пластовые давления, но они всегда будут присутствовать в глубокопогруженных залежах.
И Z-фактор, мы привыкли к тому, что Z-фактор меньше единицы.
Вот там будет диаграмма.
Мы дальше будем рассматривать корреляции для Z-фактора очень-очень-очень коротко, и там будет диаграмма, значит, Каца-Стендинга.
И по этой диаграмме обычно определяют Z-фактор, значит, инженеры, ну или по корреляциям, полученным по этим диаграммам.
И там Z-фактор, значит, он обычно меньше единицы.
Но для повышенных давлений Z-фактор существенно превышает единицу.
И вы видите, что при начальном давлении 57 мегапаскалей Z-фактор 1.4 для этой смеси существенно.
Почему нам важно?
Еще раз вспомним, давайте, формулу, применяемую для подсчета запасов газа, вот то, что мы рассматривали.
И нам нужно реальные-реальные оценки Z-фактора.
Так вот, значит, в случае уникальных составов, в случае повышенных давлений, нам нужно использовать, ну, конечно же, экспериментальные данные.
А кроме того, корреляции могут уже не дать таких вот достаточно точных результатов, а вот экспериментальные данные.
И плюс к этому еще результаты PVT-моделирования с применением уравнений состояния, но обязательно использование шифт-параметра.
А вообще говоря, я, значит, создал, ну, сделал, значит, кубическое уравнение состояния для повышенных давлений, которое очень хорошо в сравнении с экспериментальными данными при повышенных давлениях, вплоть до 1000 атмосфер, описывает Z-фактор.
Ну, это я вот, это все в книжке 2002 года описано.
И я хочу обратить просто ваше внимание на величину Z-фактора, что она существенно выше единицы.
Это не ошибка, это так и есть.

Теперь, динамическая вязкость.
Мы с вами говорили, что границей между динамической вязкости пластовой нефти и природными газами является величина 0.1.
И здесь при повышенных давлениях вы видите, уже свыше, ну, в общем, 0.07 динамическая вязкость, и даже выше.
А обычно величины динамической вязкости при давлениях 100, 200 атмосфер -- это 0.01, 0.02 мПа*с.
То есть динамическая вязкость существенно растет.
И поскольку динамическая вязкость влияет на характеристики фильтрации нашей смеси, поэтому нам при моделировании фильтрации при таких давлениях нужно обязательно учитывать, использовать хорошие данные, хорошие формулы для адекватного описания динамической вязкости.
Еще один момент.
У нас это уникальноконденсатная система, поэтому при исследовании в данном случае контактной конденсации, посмотрите, сколь высокая объемная доля ретроградной жидкой фазы у нас для данной системы.
Она максимальная насыщенность достигает почти 30\%.
Эта контактная конденсация; будет дифференциальная тоже показана.
Ну, а дальше я расскажу.
Про алгоритмы и результаты CVD, конечно, я сейчас не буду рассказывать, но мы можем оценивать не только газоотдачу и конденсатоотдачу, но и компонентоотдачу.

\begin{center}
\includegraphics[width=\textwidth, page=274]{Брусиловский.pdf}
\end{center}

\begin{center}
\includegraphics[width=\textwidth, page=275]{Брусиловский.pdf}
\end{center}

Дальше.
Какие еще особенности?
Особенность такая, что в результате ретроградной конденсации у нас очень сильно меняется компонентный состав пластового газа.
И это видно по относительной плотности газа по воздуху.
И наряду с тем, что у нас давление уменьшается, вот видите начальное здесь давление равно давлению начала конденсации.
Это то давление, с которым начинается моделирование CVD.
Немедленно после снижения давления ниже давления начала конденсации у нас самые тяжелые фракции группы $C_{5+}$, они интенсивно выпадают.
Поэтому плотность по воздуху газа резко уменьшается.
Газовая фаза.
И мы добываем уже пластовый газ, который сильно при незначительном снижении давления пластового, незначительном отборе, у нас характеристики добываемого газа, содержание в нём конденсата резко изменяется.

\begin{center}
\includegraphics[width=\textwidth, page=276]{Брусиловский.pdf}
\end{center}

\begin{center}
\includegraphics[width=\textwidth, page=277]{Брусиловский.pdf}
\end{center}

И вот посмотрите, а это результаты уже процесса CVD.
Дифференциальная конденсация написана, но на самом деле CVD моделировался, constant volume depletion.
Так вот, объемная доля ретроградной жидкой фазы, она тут превышает, кривая чуть-чуть выше, чем по контактной.
Так вот превышает 30\%.
Это означает, что у нас будет в пласте фильтрация и ретроградной жидкой фазы, мы предполагаем.
И поэтому оценку коэффициента извлечения конденсата по результатам процесса CVD делать некорректно для таких залежей.
Это будет заниженная оценка, поскольку у нас ретроградная углеводородная жидкая фаза при ее насыщенности максимальной, как вот здесь, почти треть порового пространства, она будет фильтроваться.

\begin{center}
\includegraphics[width=\textwidth, page=278]{Брусиловский.pdf}
\end{center}



\begin{center}
\includegraphics[width=\textwidth, page=279]{Брусиловский.pdf}
\end{center}

И это видно по относительной плотности газа по воздуху, а также будут данные, посмотрите, пожалуйста, содержание группы С5+, вот на сухой газ.
То есть, у нас начальное содержание, потенциальное содержание, вот формулы я вам рассказывал, 714 г/м$^3$ сухого газа.
И посмотрите, мы снизили на 30 бар, и у нас уже меньше 500.
Вопрос заключается, ну и продолжает резко уменьшаться.

\begin{center}
\includegraphics[width=\textwidth, page=280]{Брусиловский.pdf}
\end{center}

Вопрос заключается в том, значит, вот график, который всё показывает.
Верхняя кривая на сухой газ, нижняя на пластовый газ.
Так на какое же количество конденсата нам рассчитывать при добыче газа?
Это же уникальноконденсатная залежь.

Вот это вот, это проблема.
Это проблема, значит, вот одна из проблем при разработке месторождения уникальноконденсатного.
Сейчас у меня 2 минуты идёт загрузка.
Моя презентация вдруг пропала.
И мы ещё немножечко с вами послужим, до конца дойдём.
Видите, уже 280-ый, да, ведь слайд, а у нас всего 304.
Ну, чуть-чуть осталось.
Чуть-чуть терпите, если кто устал.
И мы с вами полностью посмотрим презентацию.
Вот, сейчас.
Презентация загружена, только я...
Название, оно...
Сейчас, секунду.
А, вот.
Всё, ушла, слава Богу.

\begin{center}
\includegraphics[width=\textwidth, page=281]{Брусиловский.pdf}
\end{center}

Вот теперь коэффициент излечения, как формируется.
В начале, когда у нас нет ретроградной конденсации, у нас динамика коэффициента излечения, что $C_{5+}$, что сухого газа одинаковый.
Но как только у нас давление снизилось ниже давления начала ретроградной конденсации, резкое различие пошло.
И в итоге из-за ретроградной конденсации у нас сухой газ, как обычному в месторождении природного газа, при одной атмосфере мы почти 100\% добываем, почти, потому что в ретроградном конденсате у нас газ чуть-чуть растворяется, и поэтому не 100\%, а 99,5\% здесь.
А, значит, конденсатом мы, если вот традиционно до одной физической атмосферы снижать давление, добудем всего 26\%.
Но реально давление забрасывания, оно, скажем, может быть и 5 мегапаскалей, может быть и выше, может быть и выше.
Но вот если ориентироваться на 5 мегапаскалей для глубокозагруженной залежи с низкопроницаемыми коллекторами, мы конденсата всего 22.3\% добудем при разработке на режиме истощения.
А на другом режиме мы и не разрабатываем.
И это существенно ниже, чем традиционная нефтеотдача, которая чуть меньше 40\%, 35\% обычно бывает.
Но на нефтяных месторождениях мы всегда поддерживаем давление, значит, выше давления насыщения.
А здесь вот на истощение, все у нас газоконденсатные залежи разрабатываются на истощение пластовой энергии в нашей стране.
И поэтому здесь если сравнивать с тем, что будет происходить на газоконденсатной, в газоконденсатной залежи мы получим величину значительно меньше, чем традиционная нефтеотдача.

\begin{center}
\includegraphics[width=\textwidth, page=282]{Брусиловский.pdf}
\end{center}

Вот эта зависимость.

\begin{center}
\includegraphics[width=\textwidth, page=283]{Брусиловский.pdf}
\end{center}



\begin{center}
\includegraphics[width=\textwidth, page=284]{Брусиловский.pdf}
\end{center}

И еще для одного, это Астраханское месторождение.
Результаты, это вот зависимость потенциального содержания $C_{5+}$ в пластовом газе при снижении давления.
Пока мы не снизили ниже давления начала конденсации, оно равно гидростатическому 40 мегапаскалей, у нас изменения не происходит в потенциальном содержании конденсата в газе.
А затем снижается из-за ретроградных потерь, но вы видите, в начале не очень интенсивно снижается, а потом уже интенсивно.
Но это зависит от компонентного состава газа.
Здесь есть и кислые компоненты, и другие компоненты.
То есть это все очень индивидуально.
И чем выше содержание конденсата, мы это уже видели, тем более интенсивны будут ретроградные потери.

\begin{center}
\includegraphics[width=\textwidth, page=285]{Брусиловский.pdf}
\end{center}



\begin{center}
\includegraphics[width=\textwidth, page=286]{Брусиловский.pdf}
\end{center}

Теперь по поводу корреляции для природного газа, очень коротко.

\begin{center}
\includegraphics[width=\textwidth, page=287]{Брусиловский.pdf}
\end{center}

Вот она, знаменитая диаграмма Стендинга-Каца, которая была опубликована в 40-х годах прошлого века.
Это получено по данным экспериментальным для метана фактически.
Вы видите, это Z-фактор и зависимость приведенного давления, приведенной температуры.
Как мы с вами когда вот рассматривали свойства компонентов, и вот мы с вами уже говорили, приведенное давление -- это давление, деленное на критическое для чистого компонента.
Теперь приведенная температура -- это абсолютная температура, деленная на температуру критическую.
Что касается смесей природных, то мы используем приведенные параметры, это диаграммы, и для смеси можно использовать, для фазы, для природного газа.
И при этом нам нужно оценить для нашей смеси газовой уже давление не критическое, а псевдокритическое.
И не критическую температуру, а псевдокритическую температуру.
Что это такое?
Мы представляем, что наша смесь, что есть некий эффективный условный компонент, свойства которого эквивалентно свойствам нашей природной системы, природного газа.
И мы оцениваем, какая же температура и какое давление были бы у этого условного компонента, свойства которого эквивалентны свойствам нашей смеси.
И вот мы для оценки псевдокритического давления, мы называем не критическое давление, не критическую температуру, а псевдо, потому что у нас псевдочистый компонент, заменяет нашу смесь.
Вот существуют разные формулы, самая простая формула используется правило Кэя, я сейчас не буду вам об этом говорить, это в учебниках есть по разработке газовых месторождений.
А вот для практики очень полезно для инженеров, для практической работы оценка, то есть вот формулы простые для оценки псевдокритического давления, псевдокритической температуры по относительной плотности газа.
То есть относительную плотность газа определяется, как ее определять вы знаете рассчитывать, отношение молекулярной массы газа к молекулярной массе воздуха, ну все это мы подробно уже смотрели, оценивается псевдокритическое давление, псевдокритическая температура, рассчитывается при заданных давлении и температуре приведенное давление и температура, и вот с этого графика раньше снимали, какое же значение Z-фактора.
И это использовали и для подсчета запасов, и для технологических расчетов, и вот эти диаграммы.

\begin{center}
\includegraphics[width=\textwidth, page=288]{Брусиловский.pdf}
\end{center}

А на основе этой диаграммы были разные корреляции получены, вот одна на первый взгляд сложная, есть другие рекомендации для Z-фактора, но это просто одна из возможных.

\begin{center}
\includegraphics[width=\textwidth, page=289]{Брусиловский.pdf}
\end{center}

А я вам привожу вот, это значит вот из справочника, который мы написали с Григорием Федоровичем Гуревичем еще в 1984 году, значит и он привел в этом справочнике приведена формула про...

Итак, вот эта формула, значит, она достаточно точная для инженерных расчетов и очень удобная.
Очень удобная для оценки Z-фактора углеводородных газов.
Если у нас всякий сероводород, $CO_2$ и так далее, давайте использовать уравнение состояния, нет нормальных корреляций, и не нужно тут сыр бор городить, есть кислые компоненты, используйте уравнение состояния, и вы получите нормальные оценки Z-фактора природного газа и все другие свойства, которые вам нужны, и не нужны упрощенные корреляции, если состав необычный.

\begin{center}
\includegraphics[width=\textwidth, page=290]{Брусиловский.pdf}
\end{center}

Ну, плотность газа с Z-фактором однозначно связана, поэтому тут, значит, тоже.
$P=\rho ZRT$, все, ничего больше в этой формуле нет.

\begin{center}
\includegraphics[width=\textwidth, page=291]{Брусиловский.pdf}
\end{center}

Теперь объемный коэффициент газа, вот его определение дано, и взаимосвязь между Z-фактором, рабочим давлением и температурой, и стандартными.
Теперь тут важно, что чтобы, значит, вот я уже говорил, что давление, чтобы стандартное и рабочее были в одинаковых единицах, чтобы температура была абсолютная, то есть никакие не градусы Цельсия, а либо градусы Кельвина, ну, да, мы же используем градусы Кельвина, тут нам другие не нужны, никакие Ранкины и прочие, ни в коем случае не Фаренгейты, не Цельсия, это абсолютная температура тут используется.
Вот, и таким образом мы объемный коэффициент газа оцениваем.
Эта формула, я проверял, она правильная, и, значит, можно оценивать объемный коэффициент газа, может нам в различных технологических вопросах по технологии добычи газа понадобится.

\begin{center}
\includegraphics[width=\textwidth, page=292]{Брусиловский.pdf}
\end{center}

Дальше, вязкость газа Lee-Gonzalez-Eakin, формула неоднократно проверенная, все тут просто.
И особенности по вязкости газа я вам сказал при повышенных давлениях, при повышенных давлениях, вот, газ может быть необычный.
Обычно его динамическая вязкость от одной до двух сотых процента, редко-редко выше, ну, фу, двух сотых, мПа*с, не процент.

\begin{center}
\includegraphics[width=\textwidth, page=293]{Брусиловский.pdf}
\end{center}



\begin{center}
\includegraphics[width=\textwidth, page=294]{Брусиловский.pdf}
\end{center}

Коэффициент межфазного натяжения на границе газ-нефть, значит, нам нужно относительную плотность нефти знать, и мы тогда можем оценить величину поверхностного натяжения.
И тут даже нет в этой корреляции, значит, относительной плотности газа, потому что все зависит от, ну, в общем, того, с какой нефтью газ контактирует, для оценки его поверхностного натяжения.

\begin{center}
\includegraphics[width=\textwidth, page=295]{Брусиловский.pdf}
\end{center}

И коэффициент межфазного натяжения на границе газ-вода тоже единственная формула, других и нет.
И здесь зависимость только от давления и температуры.

\begin{center}
\includegraphics[width=\textwidth, page=296]{Брусиловский.pdf}
\end{center}



\begin{center}
\includegraphics[width=\textwidth, page=297]{Брусиловский.pdf}
\end{center}

И последнее, это очень коротко, значит, по корреляциям для пластовых вод, вообще пластовыми водами гидрогеологи занимаются, и здесь очень коротко, значит, приводятся сведения о классификации пластовых вод по ионному составу, по степени минерализации.
Это просто для сведения.
Кто-то подзабыл, кто-то не знал.
Вот для этого.

\begin{center}
\includegraphics[width=\textwidth, page=298]{Брусиловский.pdf}
\end{center}

Значит, пластовые воды, они есть гидрокарбонатные, сульфатные, хлоридные, в зависимости от преобладания тех или иных анионов.

\begin{center}
\includegraphics[width=\textwidth, page=299]{Брусиловский.pdf}
\end{center}

И вот основное, что нам нужно, это понятие минерализации воды, то есть её количество солей в воде.
Есть дистиллированная вода, а у нас в пласте всегда она минерализована, соли растворены.
И, значит, вот есть понятие ppm, parts per million, значит, для массовых концентраций, взаимосвязь ppm с миллиграмм на килограмм и так далее.
Значит, вот.

\begin{center}
\includegraphics[width=\textwidth, page=300]{Брусиловский.pdf}
\end{center}

И вот формула для плотности минерализованной воды в случае, смотрите, вот формула, в случае, значит, если вода у нас в стандартных термобарических условиях, то с учётом доли солей в водном растворе, процент массовый, вот это вот важно, там не перепутать какие проценты, процент массовый, мы можем оценить плотность воды в стандартных условиях.
А для того, чтобы, значит, нам рассчитать плотность воды в пластовых условиях, применяется нижняя формула, то есть числитель – это плотность воды в стандартных условиях, а знаменатель у нас объёмный коэффициент, просто объёмный коэффициент пластовой воды.
Значит, и мы с вами узнаём, какова плотность воды в пластовых условиях.
Вот всё, значит, по плотности.

\begin{center}
\includegraphics[width=\textwidth, page=301]{Брусиловский.pdf}
\end{center}

По вязкости воды это написано Beggs и Brill, это Brill, это известный тоже американский специалист.
Я с ним знакомился, когда был в университете Талсы, имели беседу, это был 90-й год, когда я первый раз поехал на конференцию по приглашению SPE, 32 года назад.
И очень хороший специалист, который является автором книги, изданной, вот, Библиотека SPE, серии SPE по течению газо-жидкостных смесей в скважинах.
То есть, вот, Витсон, Тёртис по Phase Behaviour, Брилл по течению в скважинах.
Вот, поскольку эти корреляции используются при моделировании течения воды или нефтеводогазовых систем в скважинах, вот он с соавторами, с коллегами предложил такую корреляцию на основе обработки данных экспериментальных исследований.

\begin{center}
\includegraphics[width=\textwidth, page=302]{Брусиловский.pdf}
\end{center}

Теперь объёмный коэффициент пластовой воды таким образом вычисляется.
Это всё, значит, сведения из монографий SPE, из статей, и широко известные корреляции на Западе.
Они же применяются и нашими инженерами для оценки свойств воды.
Просто обработка имеющихся экспериментальных данных.

\begin{center}
\includegraphics[width=\textwidth, page=303]{Брусиловский.pdf}
\end{center}

Сжимаемость воды, изотермический коэффициент сжимаемости, зависит только от давления в корреляции и температуры.

\begin{center}
\includegraphics[width=\textwidth, page=304]{Брусиловский.pdf}
\end{center}

И вот это, значит, Брусиловский Александр Иосифович, который с вами полноценно провёл курс 4-дневный, 10-ого октября, даже 5-дневный, 10-ого октября и с 24 по 27 октября.
Спасибо вам за внимание. Я уверен, что материал будет вам полезен.
И желаю вам больших успехов. Всего вам хорошего.

\end{document}
