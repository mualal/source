\documentclass[main.tex]{subfiles}

\begin{document}

\section*{Лекция 26.10.2022 (Брусиловский А.И.)\markboth{ЛЕКЦИЯ 26.10.2022 (БРУСИЛОВСКИЙ А.И.)}{}}
\addcontentsline{toc}{section}{Лекция 26.10.2022 (Брусиловский А.И.)}

\subsubsection{Задача газоконденсатных исследований -- изучение газоконденсатной характеристики (ГКХ)}

\begin{center}
\includegraphics[width=\textwidth, page=115]{Брусиловский.pdf}
\end{center}

Итак, мы с вами прослушали часть курса, касающуюся исследований пластовых нефтей.
Переходим к исследованию не менее важной углеводородной системы, такой как газоконденсатная.
Как вы знаете, в нашей стране огромные запасы природного газа, 16\% мировых запасов, и значительная часть из них -- это газоконденсатные системы.
То есть природные газы, в состав которых входят пентаны плюс вышекипящие.
И исследование газоконденсатных систем -- это важная составляющая, важная часть деятельности инженеров в области изучения природных углеводородных систем, проектирования разработки и, конечно, подсчета запасов.
И содержание конденсата влияет на схемы обустройства промыслов.
Ну и важная часть.
Мы же сосредоточимся на физико-химических свойствах, прежде всего, не вникая в технологии отбора и так далее.
То есть, принципиальные решения для того, чтобы понять, какие сложности стоят перед теми, кто обосновывает компонентный состав природных газоконденсатных систем, прогнозирует его изменения в процессе разработки залежей.
И, значит, принципиальный вопрос.
Газоконденсатная характеристика, как вы уже прочитали на слайде, изучается с целью очень похожей на те цели, которые стоят перед теми, кто занимается изучением характеристик пластовых нефтей, то есть подсчет запасов (у нефтяников это нефть, растворенный газ и отдельные компоненты; у газовиков подсчет запасов газа, конденсата, имеется в виду пентаны плюс вышекипящие, стабильный конденсат), составление проекта разработки месторождения, составление проекта обустройства промысла и определение направлений использования конденсата.

\begin{center}
\includegraphics[width=\textwidth, page=116]{Брусиловский.pdf}
\end{center}

Что же необходимо для подсчета запасов газа и конденсата?
При исследовании скважины замеряют дебиты газа и конденсата, отбирают пробы газосепарации и насыщенного конденсата, уже в лаборатории отобранные пробы исследуют (определяют компонентный состав и так далее).
И исследования какие?
Это физико-химические исследования, и нас прежде всего интересуют термодинамические исследования, направленные на определение компонентного состава пластового газа, а для этого мы должны определить компонентный состав проб сырого конденсата и отсепарированного газа.
И затем уже определяется компонентный состав пластового газа, так же как и у нефтяников это делается путем расчёта материального баланса.
Формула будет показана, и подход очень похожий.
Теперь определяется потенциальное содержание конденсата, этана, пропана, бутана в пластовом газе.
Напомню, что когда мы рассматривали пластовые нефти, то мы говорили о том, что если в составе растворённого газа концентрация этана равна или превышает 3\% мольных, то подсчитывается запас этана, пропана, бутанов в растворённом газе.
При подсчёте запасов в газоконденсатных залежах тоже определяется содержание, помимо конденсата, этана, пропана, бутанов, если у нас концентрация этана превышает 3\%.
Теперь по результатам исследований в лаборатории рассчитывается компонентный состав пластового газа.
И я подготовил достаточно подробный пример того, как это делается.
То есть вы сможете понять, я вам расскажу коротко и ясно, как это делается.
И затем в результате исследований термодинамических в лаборатории в сосудах высокого давления определяют давление начала ретроградной конденсации, определяют давление максимальной конденсации, определяют Z-фактор, коэффициент сверхсжимаемости, а также определяют коэффициент извлечения конденсата при разработке залежи до остаточного давления, равного одной физической атмосфере.
Значит, написано, вот это исследование в лаборатории, оно называется дифференциальная конденсация, когда мы хотим спрогнозировать, какую же долю конденсата от его начальных запасов мы можем извлечь в результате разработки залежи без поддержания пластового давления.
Должен отметить, что это важный эксперимент, дифференциальная конденсация, а сейчас делается контактно-дифференциальная конденсация, опять же потому что все оборудование импортное практически в лабораториях, очень хорошее оборудование.
Методика проведения экспериментов тоже такая, которая осуществляется в ведущих западных лабораториях, и поэтому называется CVD, Constant Volume Depletion.
Более подробно я вам расскажу далее, что это такое.
Смысл, что мы прогнозируем, какую же долю конденсата от его начальных запасов мы можем извлечь при разработке залежи на режиме истощения пластовой энергии, ну или иначе без поддержания давления.
Я должен сказать, что в нашей стране в силу разных обстоятельств все газоконденсатные объекты разрабатываются без поддержания пластового давления.
Это вызвано как экономическими причинами недостаточно высокой стоимости конденсата, высокой стоимостью компрессоров в технологиях поддержания пластового давления, а пластовое давление поддерживается для того, чтобы не допустить ретроградной потери конденсата.
Значит, вот.
И просто в силу и экономических причин, и отсутствия необходимых компрессоров достаточной мощности и для достаточно высокого давления у нас разработка осуществляется на режиме истощения пластовой энергии.

\begin{center}
\includegraphics[width=\textwidth, page=117]{Брусиловский.pdf}
\end{center}

Для составления проекта разработки месторождения требуется определить балансовые запасы газа и конденсата.
А для того, чтобы определить балансовые запасы конденсата, нам нужно определить начальное потенциальное содержание конденсата в пластовом газе.
А для прогнозирования добычи конденсата, C5+ выше, когда я говорю конденсат, я имею в виду стабильный конденсат, это пентаны плюс вышекипящие (ценнейшее сырье для обрабатывающей промышленности; в самых разных областях применяется и очень ценно).
Это аналог легких фракций нефти и еще более легких фракций нефти.
И для химической, и для нефтехимической промышленности.
Вот, значит, конденсат это ценнейшее сырье.
Так вот, нам нужно прогнозировать потенциальное содержание конденсата в пластовом газе.
Тогда, если мы знаем эти данные, они получаются в результате термодинамических исследований дифференциальной конденсации или constant volume depletion в лабораториях.
Мы можем прогнозировать объемы добычи конденсата в соответствии с прогнозными объемами добычи газа.

\begin{center}
\includegraphics[width=\textwidth, page=118]{Брусиловский.pdf}
\end{center}

Теперь для использования конденсата так же как, теперь все очень похоже на то, что для нефти.
Для использования нефти или конденсата необходимо знать объемы добываемого конденсата.
В случае нефти это объемы добычи нефти.
Естественно, когда мы говорим о нефти, мы говорим о товарной нефти.
А здесь мы говорим о группе C5+ выше, пентаны плюс вышекипящие.
Это аналог товарной нефти.
Фракционный состав стабильного конденсата, то есть групповой углеводородный состав конденсата, то есть соотношение в нем, парафиновых или алканных, нафтена, ароматики, они (соотношения) определяют направление использования конденсата и товарную характеристику целевых фракций конденсата.
Если бы мы заменили на нефть, тоже было бы все правильно.
Товарная характеристика целевых фракций нефти, групповой углеводородный состав нефти, фракционный состав, все это нефтяники определяют, так же как и газовики.
И это делается по большей части по одинаковым отраслевым стандартам.
Для нефти и для конденсата, для природных углеводородных систем.

\begin{center}
\includegraphics[width=\textwidth, page=119]{Брусиловский.pdf}
\end{center}

Задача газоконденсатных исследований, в целом задача изучения её характеристики газоконденсатной.
И для изучения этой характеристики газоконденсатные исследования проводятся в 2 этапа.
Первый этап промысловый, а второй лабораторный.
В свою очередь, то есть промысловый этап, так же как и у нефтяников, проводится с целью получения проб, то есть информации о том, какова же наша природная углеводородная система, которую мы будем добывать.
И должен сказать, что исследования промысловые, газоконденсатные, они даже сложнее с точки зрения получения представительной информации, чем исследования пластовых нефтей.
В исследовании пластовых нефтей, как мы уже знаем, основным способом получения информации о пластовой нефти является отбор глубинных проб.
А в случае невозможности их отбора, отбираются пробы газосепарации и насыщенного конденсата, измеряется дебит, соответствующий газосепарации насыщенного конденсата.
И в соответствии с этими дебитами, если мы поделим дебит конденсата насыщенного на дебит газосепарации, мы получаем такую величину, как конденсатогазовый фактор.
Это величина, обратная газовому фактору, который используют нефтяники.
Просто у нас обычно объемы конденсата существенно меньше, чем объемы газа, а у нефтяников наоборот.
Поэтому у нефтяников используется газовый фактор, объем газа делится на объем нефти, а у газовиков оперируют с конденсатогазовым фактором.
Еще раз.
Измеряют дебит газосепарации, измеряют дебит насыщенного конденсата.
И поделив расход или дебит насыщенного конденсата на дебит газосепарации, получают такую величину, как конденсатогазовый фактор, соответствующий условиям его отбора на промысле при газоконденсатных исследованиях.
И, повторяю, для газовиков основным является отбор поверхностных проб с последующей рекомбинацией в соответствии с конденсатогазовым фактором в лаборатории.
И таким образом получают состав пластового газа.
А у нефтяников это делается в случае невозможности отбора глубинных проб.
Глубинные пробы в случае исследований на газоконденсатность только в последнее время стали рассматривать как альтернативу отбору поверхностных проб с последующей рекомбинацией.
Почему?
Потому что для глубокопогруженных залежей с низкопроницаемыми коллекторами практически очень затруднительно получить пробы поверхностные при соблюдении стационарности потока в стволе скважины.
И отсутствие пульсации, и значит, все это очень затруднительно.
А для низкопроницаемых коллекторов и дебиты сами, и сами расходы, они небольшие, и они не позволяют соблюсти необходимые условия для получения представительных проб насыщенного конденсата и газосепарации с последующей рекомбинацией.
И значит, исследованием в лаборатории, исследовать в лаборатории рекомбинированные пробы имеет смысл только в том случае, когда мы уверены, что рекомбинированная проба, это касается и нефти, и газоконденсатных систем, тем, что эта проба соответствует составу пластовой углеводородной смеси.
Тогда мы в результате лабораторных исследований сможем получить правильные характеристики и прогнозировать.
Правильные характеристики имеют в виду физико-химические свойства и содержание конденсата, и правильно прогнозировать его динамику от изменения пластового давления.
Ну, об исследованиях я ещё дальше расскажу, и, конечно же, акцент я сделаю прежде всего на той технологии, которая традиционна и отработана в газовой промышленности, а с тем, чтобы вы хорошо поняли, как определяется компонентный состав пластового газа, и также я расскажу об основных экспериментальных исследованиях, чтобы вы поняли физический смысл, почему они проводятся, без акцента на подробности технологий, потому что они в специальных изданиях приводятся, а я делаю обзор, а те, кто заинтересуется или по работе будущей будет сталкиваться уже с необходимостью более подробного изучения, уже нужно читать специальные издания, руководства, отраслевые стандарты и так далее.

\begin{center}
\includegraphics[width=\textwidth, page=120]{Брусиловский.pdf}
\end{center}

Итак, методы проведения первичных промысловых исследований скважин.
Основной метод в газовой промышленности при исследовании газоконденсатных систем -- это метод промышленных отборов.
Что это такое?
Это пусть у разведочной скважины осуществляется привязка большой сепарационной установки, специальной именно использованной для разведочных скважин.
Раньше использовали, например, в учебниках это есть французскую установку "<Порт-э-тест">, "<Порт-э-тест"> -- это название.
Сейчас могут быть другие названия, но смысл такой, что это емкость большого объема, в которую поступает полностью вся добываемая углеводородная смесь из скважины.
Это оборудование, сепаратор, имеет различные датчики, счетчики установленные и так далее, с тем, чтобы можно было измерять расход газа, расход жидкой фазы, измерять давление и температуру.
То есть, все необходимое для того, чтобы получить пробы газосепарации и насыщенного конденсата, на который в этом сепараторе при заданных термобарических параметрах разделяется поступающая из скважины смесь, фактически из пласта.
При этом существует требование, чтобы течение в стволе скважины было равномерное, фактически стационарное, без существенных колебаний.
С тем, чтобы мы получили в итоге в правильном соотношении расход газосепарации и насыщенного конденсата.
И могли замерить, определить конденсатогазовый фактор, и затем в лаборатории пробы газосепарации и насыщенного конденсата рекомбинировать в соответствии с конденсатогазовым фактором.
Если у нас будут значительные колебания в расходах, будет пульсирующий режим работы скважины, либо такая вещь, как недостаточная скорость, недостаточный дебит для полного выноса ретроградного конденсата, который выпадает в стволе скважины, мы не сможем получить представительную пробу пластовой газоконденсатной смеси.
Это основная проблема, которая существует при газоконденсатных исследованиях.
Особенно она обострилась с увеличением глубин бурения, которым соответствуют низкопроницаемые коллектора и повышенное содержание конденсата.
В общем, эта проблема очень сложная и до сих пор окончательно нерешённая.
Это также не менее сложно, а может быть и более сложно, чем получать представительные пробы нефти с повышенным газосодержанием или пробу околокритических нефтей.
А для нефтей традиционных, там, ещё раз повторяю, просто для закрепления, там проблем-то особых нет.
С невысоким газосодержанием, где чётко виден излом PV-зависимости, и мы знаем, что мы можем нашим пробоотборникам отобрать представительную пробу нефти в случае её однофазного течения в стволе скважины.
А для газоконденсатных систем всё гораздо сложнее.
Ещё и потому, что, во-первых, PV-зависимости с изломом не существует, даже для нефтей с повышенным газосодержанием, уже, как мы отмечали с вами, не существует излома PV-зависимости.
И для нефтей с повышенным газосодержанием применяются оптические анализаторы, фиксирующие начало выделения газовой фазы.
А для газоконденсатных систем тоже сейчас используются оптические анализаторы, но, в общем, сложность в том, что в отличие от нефтяных систем, где давление насыщения зачастую весьма существенно ниже пластового давления, и существует запас по давлению в пласте, когда наша нефть находится в однофазном состоянии.
Так вот, для газоконденсатных систем, если у нас в пласте гидростатическое давление, для подавляющего большинства газоконденсатных систем давление начала ретроградной конденсации равно начальному (или очень близко к нему) пластовому давлению.
Вот такая особенность.
Ну, в общем, я не буду опережать события, немножко подробнее я об этом сейчас расскажу.
Так вот, метод промышленных отборов считается приоритетным при газоконденсатных исследованиях.
Он проводится на разведочном этапе, на разведочных скважинах при одноступенчатой сепарации, а когда...
Значит, это для получения первичной информации, для подсчета запасов, для проведения термодинамических исследований, для проведения физико-химических исследований конденсата, определения компонентного состава нашего пластового газа.
Вот всё это при одноступенчатой сепарации проводятся эксперименты на разведочных скважинах.
И в самый начальный период эксплуатации.
Теперь, исследования при двухступенчатой сепарации проводятся для прогнозирования, для прогнозирования схемы, для определения наиболее эффективной схемы обустройства газового промысла и определяются изотермо-изобары конденсации с целью определить термобарические условия наиболее выгодные для выделения конденсата при эксплуатации.
Это вот по двухступенчатой... Двухступенчатая сепарация для этого проводится.
Но на начальном этапе на разведочных скважинах осуществляется исследование при одноступенчатой сепарации.
И получение проб газосепарации насыщенного конденсата с последующей рекомбинацией в лаборатории в соответствии с измеренными дебитами и величиной конденсатогазового фактора.
Я многократно уже повторил это для того, чтобы вы чётко совершенно усвоили это.
Теперь, помимо метода промышленных отборов используется также метод малых отборов.
Отбирается часть потока.
Это не только...
В случае двухступенчатой сепарации, когда используется метод промышленных отборов, из общего потока отбирается часть потока для исследования второй ступени.
Для исследования наиболее выгодных условий промысловой сепарации.
Я об этом уже сказал.
Тогда отбирается это в малую термостатируемую сепарационную установку, отбирается часть потока.
А метод малых отборов, это когда у нас часть потока в районе устья отбирается, довольно малая часть потока.
Затем отбираются пробы и считаются, и затем исследуются в лаборатории.
Это не для определения направлений, не для определения схемы обустройства промысла, а это единственный способ получения представления о составе пластовой газоконденсатной системы.
Это делается в случаях, когда мы не можем доставить большой промышленный сепаратор к устью разведочной скважины.
Тогда используются такие установки, в которые отбирается часть потока.
Но это очень под большим вопросом.
Представительность использования этого метода как основного.
Статистика показывает, что часто метод малых отборов даёт неверную, искажённую информацию о составе пластовой смеси.
Почему это происходит?
Дальше я уже не буду этого касаться.
Это в регламентах можно (в инструкциях) прочесть более подробно.
У нас газоконденсатная смесь в стволе скважины на устье практически всегда находится в гетерогенном (газожидкостном) состоянии.
При этом совершенно неравномерно распределён поток этот газожидкостный по сечению скважины.
Стараются гомогенизировать этот поток перед отбором.
Перед тем, как часть потока будет направлена в малый сепаратор на устье скважины.
Много разных конструктивных ухищрений делается для того, чтобы этот поток был равномерным у устья скважины.
То есть гомогенизацию делают.
Но далеко не всегда это удаётся.
Кроме того, какая существует проблема?
Дело в том, что газожидкостный поток не только неравномерный по сечению, но и по вертикали тоже.
У нас большие глобулы конденсата могут неравномерно по сравнению с течением газа двигаться в стволе скважины.
А чем ближе мы к устью, тем у нас больше.
При высоком содержании конденсата, тем больше будут эти глобулы.
Возможность гомогенизации зависит от структуры газожидкостного потока.
При различной доле жидкости, то есть конденсата, выделившегося из пришедшей в скважину смеси, по мере уменьшения давления, уменьшения температуры в стволе скважины, по мере движения к устью, у нас осуществляется ретроградная конденсация.
И если содержание конденсата большое, то меняется структура потока, если у нас доля жидкой фазы будет значительной.
Для уникальных конденсатных залежей это может быть структура потока такая же, как для нефтей с повышенным газосодержанием.
Очень сложная проблема.
И повторяю, с помощью метода малых отборов далеко не всегда удается, даже с помощью гомогенизации, отобрать представительные пробы.
Теперь вот.
Альтернативой отбору поверхностных проб, то что всегда было, повторяю, главным у тех, кто исследовал системы на газоконденсатность, является отбор глубинных проб с применением пластоиспытателей MDT и других.
В основном MDT.
И я уже показывал картинки, когда мы рассматривали вопросы получения проб для пластовых нефтей.
Я имею в виду картинки для MDT.
Смысл заключается в том, что MDT позволяет отбирать пластовые флюиды при минимальной депрессии.
При минимальной депрессии в зоне отбора, но если для нефтей у нас обычно существует запас по давлению между давлением пластовым и давлением насыщения, то для газоконденсатных систем, повторяю, ретроградная конденсация начинается практически для многих сразу после снижения давления ниже пластового.
То есть только мы создаем депрессию, а она для низкопроницаемых пластов даже в MDT может составлять несколько десятков бар.
Настолько сейчас сложные условия.
То мы уже и в MDT получаем пластовый газ, обеднённый содержанием пентан плюс высшикипящих С5+.
Часть уже выпала.
Вот такая вещь.
А мы также можем с вами вспомнить, когда мы рассматривали фазовые диаграммы типа пластовых флюидов и фазовые диаграммы в координатах давления-температура, то мы говорили, что чем выше содержание С5+ в пластовой смеси, тем более интенсивная ретроградная конденсация будет осуществляться.
Это так же как для нефтей.
Чем выше газосодержание, тем более бурное разгазирование осуществляется для нефтей при снижении давления перехода из области однофазной в область двухфазную, гетерогенную.
Ну вот похожие вещи для газоконденсатных систем, только здесь у нас осуществляется ретроградная конденсация, а не разгазирование.
И вот это большая проблема для получения представительных проб пластового газа.
И кроме того, когда мы получаем пробу в пластоиспытатель, в MDT, мы получили пробу, мы подняли давление с помощью современных пластоиспытателей, давление внутри MDT, оно больше чем пластовое, оно поднимается за счет поджимки азота или что-то другое, и мы доставляем ту смесь, которая в том фазовом состоянии, который мы получили и создали в MDT.
А вопрос в том, а что же за смесь у нас пришла в MDT на забой.
И повторяю, если депрессия составляет несколько десятков бар, то эта вот смесь, которая вошла в MDT, она может и не соответствовать пластовому газу.
В общем, это проблема.
Эта проблема изучается, и пути получения представительных газоконденсатной системы для низкопроницаемых пластов на данном этапе -- это актуальная научная проблема.
Научная, инженерная проблема, но имеется в виду наука по разработке, эксплуатации месторождений газоконденсатных.
Также, когда мы имеем дело с двухфазными залежами, тоже ведь там предельно насыщенная у газонефтяного контакта и нефть, и пластовый газ.
И тоже, если мы исследуем в районе газонефтяного контакта, мы традиционными способами...
Нам трудно, даже если пласт не низкопроницаемый, но для двухфазных залежей довольно трудно определить представительный состав газоконденсатного флюида.
Но это специальная проблема, мы сейчас не будем её касаться подробно, просто я говорю о том, какие вопросы существуют.
Так же, как и нефть пластовая, она предельно насыщена у газонефтяного контакта, и при малейшей депрессии начинается выделение газа, и мы глубинным пробоотборником уже не можем отобрать с вами представительную пробу нефти.
В случае, когда у нас отбор осуществляется вблизи газонефтяного контакта.
И вот подобные проблемы для газоконденсатных систем, где тоже немедленно ретроградная конденсация осуществляется при снижении давления.
Но для газоконденсатных систем эта проблема не только для двухфазных залежей, а вообще, повторяю, для газоконденсатных систем, пластовое давление в которых равно гидростатическому или очень близко к нему.
Значит, проблема ретроградной конденсации, проблема получения представительных проб пластовой газоконденсатной системы, значит, вот с помощью пластоиспытателей.
Тоже вопрос стоит большой.
Если депрессия в несколько атмосфер, мы можем получить представительную пробу.
А если она в несколько десятков атмосфер, то из-за ретроградной конденсации, пробы полученные в глубины пробоотборник не будет представительной.
Поэтому даже сейчас газовики совершенно разные мнения.
Например, в Новатэке не являются сторонниками использования пластоиспытателей для газоконденсатных систем.
Именно из-за того, что мы не можем контролировать, ну, не можем получать исходную гомогенную систему.
Просто я вам для примера.
Это я доподлинно знаю из мнений разных специалистов, причем специалистов высокого класса, из обсуждений на государственной комиссии по запасам соответствующих месторождений.
То есть кто-то является сторонником MDT, а другие против и даже категорически против такого подхода для газоконденсатных систем.
А статистика не говорит в пользу ни того, ни другого. То есть достаточно большое количество неудачных исследований.
В то же время, должен сказать, что в общем это является актуальной проблемой.
Как получать представительные пробы пластовых газоконденсатных систем в случае низкопроницаемых коллекторов и глубокопогруженных залежей.

\begin{center}
\includegraphics[width=\textwidth, page=121]{Брусиловский.pdf}
\end{center}

Значит, какие основные факторы влияют на достоверность параметров газоконденсатных характеристик при промысловых исследованиях скважин?
Вы видите на слайде и перечень, и голубым отражены те вопросы, которые мы более подробно рассмотрим.
Во-первых, время работы скважины на режиме.
Скважина должна работать на стационарном режиме для получения не случайной, а представительной пробы, это касается и использования промышленных сепараторов, и пластоиспытателей.
Теперь, кстати говоря, в случае пластоиспытателей, там еще существует большая проблема.
Их используют после бурения скважин на разведочных скважинах.
Я уже говорил тоже и про нефтяные.
При исследовании нефтей, значит, долгое время не могли решить проблему того, что часть углеводорода, содержащаяся в буровом растворе, они попадали в MDT тоже, и искажали состав пластовой смеси.
Это касалось и нефти, это касается и газоконденсатных систем.
И этому тоже те авторы пластоиспытателей постоянно совершенствуют конструкцию, оборудование, технологии с тем, чтобы не допустить попадания посторонних флюидов в камеру пробоотборника, чтобы не искажались пробы пластового газа.
И для нефти это характерно.
Теперь о депрессии на пласт.
Это чуть позже.
Пропускная способность сепаратора.
Для того, чтобы у нас было качественное отделение газовой фазы от жидкой, у нас сепаратор должен работать на режиме где-то в половину своей пропускной способности.
Иначе у нас будет недостаточно качественное отделение газа от жидкости, и капельки жидкости будут уходить с газовой фазы.
И затем при рекомбинации отобранных проб газосепарации насыщенного конденсата, у нас будет искаженный состав пластового газа.
В регламентах, в инструкциях об этом написано.
Пропускная способность сепаратора должна быть достаточной.
Именно поэтому же еще большие сепараторы применяют.
Но не только поэтому, а для того, чтобы не случайные пробы отбирать.
Чтобы достаточно большое время было наполнение сепаратором вот этой насыщенной жидкости и так далее.
В общем, большого объема сепараторы.
Они предпочтительны, явно предпочтительны при газоконденсатных исследованиях.
И позволяют, кроме того, сгладить, если есть пульсации.
Теперь скорость восходящего.
Значит, у нас проблема связана не только с тем, что может выпадать конденсат в призабойной зоне и искажаться состав пластовой смеси, который приходит на забой скважины.
Но и то, что тоже важная вещь.
Значит, у нас скорость восходящего потока в газоконденсатной смеси в насосно-компрессионных трубах должна быть достаточной для полного равномерного выноса выпавшего ретроградного конденсата в стволе скважины.
Если у нас дебит будет недостаточный, а дебитом определяется скорость потока в насосно-компрессорных трубах.
Если дебит будет недостаточным, то скорость будет недостаточной (низкой), и жидкая фаза в стволе скважины не сможет равномерно быть вынесенной из ствола скважины.
И это будет приводить и к пульсациям, и к накоплению даже конденсата у забоя скважины, потому что если скорость маленькая, то выпадающий в стволе скважины конденсат по стенкам может стекать на забой.
А затем, накопившись, будет выброс этого конденсата.
То есть проблем достаточно с точки зрения гидродинамики газожидкостного потока, не только термодинамики.
И не только того, что происходит в пласте, но важно при газоконденсатных исследованиях рассматривать комплексно проблему и течения в пористой среде, и в стволе скважины при газоконденсатных исследованиях.

И условия сепарации, по ним рекомендация такая, что сепарация при газоконденсатных исследованиях, вот в этом сепараторе типа "<Порт-э-тест">, в который с устья идет добываемый поток, там давление должно быть не выше, чем 60 бар.
Почему?
Потому что, если давление не выше 60 бар, на практике стараются где-то 40-45 бар, чтобы было давление.
Тогда в газе сепарации конденсата очень мало.
То есть, есть пентаны и гексаны, но в малом количестве, и при транспортировке, и в лаборатории, эти вот лёгкие углеводороды, $C_5$, $C_6$, они не выпадают из газа сепарации.
И состав его не искажается при лабораторном исследовании компонентного состава газа сепарации.
А если давление в контейнере с газом сепарации, в баллоне с газом сепарации отобранои, будет выше 60 бар, то будет больше растворено лёгких компонентов С5 плюс выше, и они при исследованиях в лаборатории выделятся, и даже при транспортировке они могут выделяться в баллоне, на стенке скважины, и будет искажён состав газа сепарации.
Это из практики.
Значит, все эти вопросы, касающиеся, когда может выделяться конденсат из газа сепарации, если кого-то будет интересовать, или какое-то подтверждение для себя, значит, все это можно рассчитывать.
Рассчитывать с применением уравнения состояния, задаётся компонентный состав газа сепарации, задаются термобарические условия, и вот можно посмотреть, будет ли при заданном диапазоне температур в заданном диапазоне давлений из газа сепарации выделяться жидкость.
Если она будет выделяться, значит, эти давления, эти термобаритические условия, они плохи для отбора газа сепарации при газоконденсатных промысловых исследованиях.

\subsubsection{Депрессия на пласт}

\begin{center}
\includegraphics[width=\textwidth, page=122]{Брусиловский.pdf}
\end{center}

Дальше. Так вот, два вопроса, которые мы рассмотрим более подробно. Депрессия на пласт.
Насыщенные залежи углеводородов, давление равно давлению начала ретроградной конденсации, необходимо исследовать с депрессией до 10\% от пластового давления.
Значит, казалось бы, а почему до 10\%?
Ведь конденсат уже будет выделяться, и выделяться он будет, если может прилично уже выделиться из газовой фазы.
Так вот, смысл в том, почему до 10\%?
Это из практики, тоже из специальных исследований, что в начале выделяющийся конденсат представляет собой мелкодисперсную фазу.
То есть, капельки выделяются, капельки жидкости очень мелкие, и этот конденсат, который выделяется в призабойной зоне, капельки эти не успевают консолидироваться, и они с газовой фазой попадают на забой скважины. 
То есть, иначе говоря, значит, если у нас депрессия до 10\% от пластового давления, то считается, что состав пластовой смеси, поступающей на забой в ствол скважины, он не отличается от состава пластового газа.
Ну, я сказал, почему.
Мелкодисперсная, ретроградная жидкая фаза, и капельки еще не успевают консолидироваться, и не успевают в пористой среде откладываться, то есть, оседать на поверхность пористой среды.
Ну, вот так современные такие представления.
Значит, если же у нас депрессия выше, то капельки будут консолидироваться, и будут уже оседать на поверхности пористой среды, и состав поступающей смеси на забой скважины будет обеднен.
Обеднен содержанием конденсата по сравнению с пластовым газом.
Теперь, если у нас легкий конденсат, легкий -- это плотность которого меньше либо равна 700 кг на метр кубический, это легкий конденсат, то из практики следует, что такие скважины могут исследоваться с депрессией, такие залежи могут исследоваться с депрессией до 15\%.
Это из практики.
Просто проводились исследования скважин с различной депрессией, и сделаны были выводы, что для легкого конденсата меньше 700 кг на метр куб плотность, можно создавать депрессию до 15\%.
Это из практики.
Это написано в инструкциях, в руководствах теми, кто занимался этим вопросом.
Это тоже, вообще говоря, вопрос исследований, и для конкретного объекта, например, не лишнее провести исследование на нескольких режимах с разной депрессией и сравнить конденсатогазовый фактор, который при этом получается.
Но из предыдущего опыта вот такие выводы следуют.
И недонасыщенные залежи можно исследовать с депрессией до 20\% и больше.
Это залежи обычно с аномально высоким пластовым давлением.
Это залежи, например, на Астраханском серогазоконденсатном месторождении проводились большие исследования сотрудниками Института АстраханьНИПИгаз, квалифицированные люди.
Я назову фамилию Лапшин Владимир Ильич, который уже давно и долгое время работает во ВНИИГАЗе.
Мы с ним в свое время контактировали, когда проводились исследования на газоконденсатность на Астраханском месторождении.
И практикой было показано, что недонасыщенные залежи можно исследовать с депрессией до 20\% и больше.
Что значит больше?
Ну, все зависит от различия между пластовым давлением и давлением начала ретроградной конденсации.
На Астраханском месторождении оно очень большое, там пластовое давление превышает 60 мегапаскалей.
Там 63 мегапаскаля, но там залежь достаточно большой толщиной газоносности, и давление меняется.
Но в целом, вот так вот, больше 60 мегапаскалей.
А глубина 4 километра, гидростатическое давление, значит, у нас 400 бар.
Исследования в бомбе PVT показали, что для рекомбинированных проб пластового газа, полученных с разной депрессией, давление начала конденсации равно гидростатическому, близко к нему очень, и компонентный состав пластового газа при различных депрессиях на разных режимах газоконденсатных исследований был, в общем, практически одинаковым.
То есть, можно было получить представительный состав пластового газа, когда у нас есть значительное...
Ну, то есть, для месторождения с аномально высоким пластовым давлением значительно проще получить состав пластового газа, представительный состав.
И даже вот для этих месторождений, даже если представляются там низкопроницаемые коллектора, ну на Астраханском, кстати, тоже.
Значит, и...
То вот там как раз достаточно применить пластоиспытатель, и вы точно получите представительный состав пластового газа, потому что при депрессии, которая формируется при использовании пластоиспытателя, она гораздо меньше, чем...
Значит, давление, при котором смесь приходит в пластоиспытатель, оно гораздо больше, чем давление начала конденсации в случае, если наша смесь пластовая из залежи с аномально высоким пластовым давлением.
А вот для большинства залежей с гидростатическим давлением даже использование пластоиспытателя, это в общем в низкопроницаемых пластах большая проблема.
Актуальная проблема, ее следует изучать и предлагать, что и делается сейчас, вот в НТЦ Газпромнефть специалисты сосредоточены на изучении этой проблемы тоже.
Те, кто занимается... значит, у нас есть центр исследования, центр по исследованию и разработке месторождений природных газов, вот там проводятся исследования по совершенствованию технологий газоконденсатных исследований для получения представительных проб.
То есть для решения проблемы получения представительных проб.

\subsubsection{Минимальный необходимый дебит (МНД)}

\begin{center}
\includegraphics[width=\textwidth, page=123]{Брусиловский.pdf}
\end{center}

Теперь, что касается минимально необходимого дебита (МНД) при газоконденсатных исследованиях.
Эта проблема связана с проблемой минимально необходимой скорости для полного выноса ретроградного конденсата из ствола скважины.
Это потому, что дебитом определяется... дебитом и конструкцией скважины определяется скорость течения нашей смеси.
И, значит, нужно рекомендовать для практики минимально необходимый дебит для полного выноса капельной жидкости.
И однозначно связано это со скоростью выноса.
Если мы связываем наш дебит с конструкцией, с диаметром НКТ (насосно-компрессорные трубы), учитываем давление, температуру в стволе скважины, реальные свойства газа, Z-фактор.
И вот вы видите формулу на взаимосвязи скорости течения, дебита, температуры, Z-фактора, давления, диаметра НКТ.
И из формулы... вот она первично записана для скорости.
Значит, поменяв местами переменные, можно получить выражение для оценки минимально необходимого дебита для выноса жидкой фазы.

\begin{center}
\includegraphics[width=\textwidth, page=124]{Брусиловский.pdf}
\end{center}

Значит, сейчас, на данном этапе, считается, что при скоростях порядка 2.5 м/с конденсат выносится непрерывно сразу после пуска скважины.
Это на основе и практики, и исследований экспериментальных.
Это современное положение.
Для насыщенных углеводородных систем расчёт минимально необходимого дебита производится у башмака насосно-компрессорных труб, то есть в самом низу.
А для недонасыщенных углеводородных систем, ну вот, недонасыщенные, скажем, с АВПД (с аномальным высоким пластовым давлением), расчёт минимально необходимого дебита производится в сечении НКТ, где давление равно давлению начала конденсации.

\begin{center}
\includegraphics[width=\textwidth, page=125]{Брусиловский.pdf}
\end{center}

И я подготовил таблицу, которая плохо видно, но я расскажу.
Вот к этой проблеме относятся.
По вопросу минимальной необходимой скорости газа, необходимой для установившегося выноса конденсата при газоконденсатных исследованиях скважин.
В инструкции 1980 года, это инструкция отраслевая, значит, ссылка тут есть, да?
И в руководстве по исследованию скважин более позднего издания, 1995 года, там говорится о том, что минимальная необходимая скорость для полного выноса капельной жидкости, для полного выноса конденсата при газоконденсатных исследованиях должна быть не меньше 4 метров в секунду.
Значит, это по результатам экспериментальных исследований было в свое время, в конце 70-х годов, сделан такой вывод, и последующие промысловые исследования подтвердили, что да, при скорости 4 и более метров в секунду мы получаем представительный состав смеси.
Но тогда пластовые давления, которые практически не было, значит, вот залежи с повышенным пластовым давлением, ну, где-то было 2-2.5 километра, с увеличением глубин бурения, значит, возрастают давления пластовые, а раз возрастает пластовое давление, возрастает и плотность газа, соответствующего этим давлением, и при одном и том же дебите, чем выше давление, то есть выше плотность газа, тем меньше скорость в стволе скважины, которая обеспечивает это количество газа.
Это понятно совершенно.
Поэтому с увеличением глубин бурения, увеличением давления, значит, уже появились рекомендации о том, что достаточно скорости 2.5 метра в секунду, да, для башмака фонтанных труб, для полного выноса выпадающего ретроградного конденсата из ствола скважины.
Это впервые появилась монография 1997 года, изданная в Ухте специалистами СеверНИПИгаза.
Надо сказать, что там очень хорошая школа по газоконденсатным исследованиям, и лаборатория, и, значит, и вот специалисты, это группа специалистов написала эту монографию, исходя из собственного опыта исследований скважин не только в Республике Коми, в том числе в Вуктыльском месторождении, знаменитом, о котором я рассказывал, когда приводил примеры газоконденсатных залежей.
Вот там впервые появилась.
Затем эта, значит, величина, она уже отражена была в инструкции по комплексным исследованиям газовых и газоконденсатных пластовых скважин.
И дело в том, что одним из авторов этой инструкции является специалист, который на пенсии уже, он работал активно в 80-е, 90-е годы, Корчашкин.
Вот один из авторов этой монографии СеверНИПИгаза 1997 года, и он же написал материал в руководство Газпром 2011 года.
Инструкция, но это руководящий рекомендации Газпром.
Не то что отраслевой стандарт для всех компаний, нет, внутри Газпрома, но Газпром у нас, так сказать, основная компания по добыче газа.
И поэтому инструкция, она включает в себя Р Газпром.
Мне понравилось, там глава 9, запомните, часть 1, глава 9 по газоконденсатным исследованиям.
И в части 2, там и примеры, там и примеры приводятся.
И там глава 5 посвящена газоконденсатным исследованиям.
Там отражено современное состояние инженерной практики при исследовании газоконденсатных систем.
И из этих разделов можно почерпнуть также и те проблемы, которые существуют.
То есть очень насыщенное достаточно издание.
Как в любом практическом руководстве, там нет изобилия формул и постановок математических, а там уже инженерные рекомендации для инженеров, и в том числе выводы из имеющихся исследований физико-математических, но не в виде формул, а в виде уже выводов, понимаете, для инженеров.
То есть, это отличное руководство.
Там тоже есть какие-то ляпы, как и в любом издании, ляпы или же неправильные терминологии, или же выводы, с которыми трудно согласиться, но в целом очень полезное руководство.
Р Газпром, главы 9, часть 1, и глава 5, часть 2.
Это я для тех, кто будет специализироваться по газоконденсатным исследованиям, и по крайней мере одного человека из слушателей, я знаю, который этим интересуется, и наверняка он уже знакомился с этим руководством.
А вот последние два источника –- это классика.
Первая -- Тёрнер с соавторами.
Это работа, которая опубликована в Journal of Petroleum Technology 1969 год.
Казалось бы, очень старая работа, но она настолько тщательно была сделана, а специалисты квалифицированы, она, пожалуй, остаётся единственной зарубежной классической работой по рекомендациям о выносе жидкости в случае газожидкостных потоков.
И там в отличие от формул, которые приведены в наших инструкциях, даже современных, там скорость, там правильно, там формула зависит от физических свойств газовой и жидкой фаз.
Какие физические свойства?
Плотности и поверхностного натяжения между фазами.
Так вот, дело в том, что в зависимости от давления, это определяющий фактор, значит, у нас с увеличением давления плотность газовой фазы растёт, плотность жидкой фазы из-за растворения в ней газа, с увеличением давления, в случае газо-жидкостного контакта газ растворяется в жидкой фазе, поэтому её плотность, она совершенно нелинейная.
Определяется массообменными процессами.
И поверхностное натяжение между фазами очень сильно уменьшается при увеличении давления.
Вот приведён график зависимости поверхностного натяжения между жидкой и газовой фазами от давления.
И это поверхностное натяжение, оно в формуле есть, которую опубликовали Тёрнер с соавторами, и при исследовании вопросов о том, какова же скорость для полного выноса жидкой фазы, очень полезно быть знакомым с этими исследованиями.
На них ссылаются и в учебнике по разработке газоконденсатных месторождений профессор Ширковский, он специализировался на газоконденсатных исследованиях, это известный специалист, который работал на кафедре разработки и эксплуатации газовых и газоконденсатных месторождений Губкинского института.
И как раз в то время, когда я в аспирантуре учился, и потом уже я защищал докторскую, в 1994 году он был моим оппонентом.
В общем, один из немногих специалистов, которые хорошо понимали физическую суть и были знакомыми с методами, и специально интересовался, при каких же условиях газоконденсатные исследования дадут нам возможность получения представительных проб пластовой газоконденсатной системы.
Естественно, что в то время давление пластовые были ниже, низкопроницаемых коллекторов тогда почти не было, и поэтому сейчас, отталкиваясь от этих исследований, выводов, нужно уже развивать всё это.
Нужно развивать, ну и это, конечно, на основе хорошей физико-математической подготовки, которую вам дают в вашей Высшей Школе, которая называется, как я недавно узнал, переименовали, теоретической механики и математической физики.
Вот математическая физика, понимание физики, владение математическими методами, ну, очень-очень необходимы для эффективного моделирования разработки и эксплуатации месторождений нефти газа.
И последняя, ниже формула, она похожа на то, что Тернер в 68-м году опубликовал, в 69-м году.
Так вот, чуть позже, чем это американцы делали, наши специалисты из разных институтов, вот фамилии авторов тут написаны, это очень известные профессора, известные, ну, то есть они стали профессорами в итоге, а в начале, значит, они просто проводили эффективные исследования по течению газожидкостных смесей и опубликовали формулу, в которой тоже скорость зависит от поверхностного натяжения, от плотности фаз.
Ну, вот, это вот те, значит, исследования, которые существуют в этой области.
Итак, вот, Р Газпром, книга 97-го года по исследованию природных газоконденсатных систем, вот исследования, которые проводились гидродинамиками, на которые я указал, это вот приведенные источники, на которые нужно опираться, нужно опираться и следовать дальше.
Я подробно очень остановился, потому что это важная проблема в получении представительных проб газоконденсатных систем, пластовых газов из низкопроницаемых коллекторов глубокопогруженных залежей.

\subsubsection{Определение компонентного состава пластового газа}

\begin{center}
\includegraphics[width=\textwidth, page=126]{Брусиловский.pdf}
\end{center}

Теперь об определении компонентного состава пластового газа.
Значит, вот мы получили пробы после промысловых газоконденсатных исследований.
Давайте с вами уже переходим к лабораторным исследованиям.
Мы получили с вами пробы газа сепарации в баллоне.
Этот газ сепарации получен.
Контейнер с насыщенным конденсатом.
И мы определяем хроматографией компонентный состав газа сепарации.
Затем дегазируя пробу, насыщенную из контейнера с насыщенным конденсатом, мы получаем, исследуем газ дегазации.
При дегазации, это одна атмосфера и 20 градусов Цельсия, или допускается температура окружающей среды, у нас получается газ дегазации и дегазированный конденсат.
То есть из насыщенного конденсата.
И у нас в стране, за рубежом нет, у нас в стране дополнительно, если у нас дегазированный конденсат имеет плотность больше чем 700 кг/м$^3$, то осуществляется его дебутанизация.
И из дегазированного конденсата в результате дебутанизации выходит газ дебутанизации и остается у нас дебутанизированный конденсат, практически целиком состоящий из пентанов плюс вышекипящих.
Это осуществляется в ректификационной колонке лабораторной.
И в результате мы получаем с вами исходные данные для расчёта компонентного состава пластового газа.

\subsubsection{Принцип расчёта компонентного сотава пластового газа}

\begin{center}
\includegraphics[width=\textwidth, page=127]{Брусиловский.pdf}
\end{center}

Мы с вами, вот пример, вспомните пожалуйста, был пример, как мы на основе однократного разгазирования рассчитывали состав пластовой нефти.
Там у нас две составляющие были: компонентный состав газа стандартной сепарации однократного разгазирования и компонентный состав дегазированной нефти.

Здесь же у нас, когда мы проводим дегазацию и дебутанизацию, то есть самый общий случай, мы должны компонентный состав пластовой смеси вот по этой формуле рассчитать.
Мы должны иметь число молей каждой составляющей, то есть газа сепарации, и компонентный состав, значит число молей каждого компонента в этом газе сепарации, должны знать число молей дегазированного конденсата.
Только что я вам рассказал, значит вот, пробу насыщенного конденсата.
И его компонентный состав, значит, даёт возможность определить число молей каждого компонента в дегазированном конденсате.
Если проводится дебутанизация, то мы, значит, вычисляем, мы знаем число молей газа дебутанизации, и зная компонентный состав этого газа, знаем число молей $i$-ого компонента в газе дебутанизации, и остался только у нас насыщенный, остался дебутанизированный конденсат, из которого удалены все компоненты газовые, значит, до бутана включительно, и в том числе азот, $CO_2$, то есть остались только пентаны плюс вышекипящие, поэтому называется дебутанизированный конденсат.
Значит, определяя число молей этого конденсата, зная его объём, зная его плотность, измеренную экспериментально, и молекулярную массу, и зная состав этого дебутанизированного конденсата определенным хроматографическим методом, так же, как и для газов, мы с вами по определению, суммируя число молей $i$-ого компонента во всех составляющих, и делим на число молей всех компонентов, суммируя число молей всех компонентов, газа сепарации, то есть, иначе определяем мольную долю $i$-ого компонента.
В сумме, значит, мольные доли по всем компонентам дадут единицу.
Всё достаточно просто.

\subsubsection{Расчёт состава пластового газа газоконденсатной залежи}

\begin{center}
\includegraphics[width=\textwidth, page=128]{Брусиловский.pdf}
\end{center}

Дальше.
Вот теперь мы перейдём к...
Тут мелко, но это просто в точности соответствует содержанию взятому в инструкции.
В инструкции по газоконденсатным исследованиям.
И обозначения соответствует этой инструкции.
Значит, единственное, я предварил, и вам гораздо легче будет воспринимать, что мольная доля определяется, напомню, отношение мольной доли компонента к сумме молей всех компонентов.
Значит, у нас исходный...
Вопрос остаётся, а как определить число молей разных составляющих?
Вот это...
У нас какие исходные данные?
Значит, вот это самое удельное количество выделяю, выход сырого конденсата.
Вот этот самый конденсатогазовый фактор.
Называется выход сырого конденсата.
Сантиметр кубический на метр кубический газа сепарации.
И здесь прям конкретный пример.
Объём контейнеров, в которые отобран сырой конденсат.
Дальше, когда этот сырой конденсат дегазировали и дебутанизировали, из него вот объёмы газодегазации и дебутанизации, они тут, которые вышли из контейнера тоже заданного объёма в), вот указанное количество газа, выделяемого а) и б).
Напоминаю вам, когда указывают количество газа, объём в литрах, это всё в стандартных термобарических условиях.
Не в условиях контейнера, а в стандартных термобарических условиях.
Значит, вот эти...
Теперь, объём дебутанизированного конденсата, который выделился в результате дебутанизации насыщенного конденсата.
Его плотность, дебутанизированного конденсата, С5+, при стандартных условиях, и молекулярная масса.
А также в дебутанизированном конденсате хроматографией определили мольные доли изо и нормального пентана.
Значит, сейчас с большим количеством компонентов имеют дело, разделяют группу С6+ выше, огромное количество фракций и так далее.
Ничего этого в этом примере нет, потому что принцип не изменен.
Нам нужно определять число молей компонентов, как было указано, компонентный состав газов сепарации, дегазации, дебутанизации.
Значит, компонентный состав дебутанизированного конденсата.
И с точки зрения принципа определения компонентного состава нам не важно.
У нас дебутанизированный конденсат, его состав определен изонормальный пентан группы С6+ или группы С6+ еще на 20 фракций.
Понимаете, отделена.
Сейчас вот в руководстве Газпрома, издание 2011 года, а вообще ссылка руководства Газпрома 2010 года, там, по-моему, до С10+, если я не ошибаюсь, состав.
Но принцип-то, там немножко...
Принцип совершенно тот же.
Совершенно тот же.
У нас состав газов сепарации определяется в процентах мольных.
Компонентный состав жидкости, конденсата определяется в процентах массовых.
А потом, зная молекулярные массы составляющих компонентов, пересчитывают в мольные проценты.
Это для дегазированного конденсата, для дебутанизированного конденсата.
Это я вам говорю, чтобы вы понимали.
Чтобы обратили внимание, что газы сразу в мольных процентах, которые равны объёмным.
В руководствах пишут даже объёмные проценты, но это...
Я уже объяснял, что лучше писать мольные.
Значит, а жидкость в начале в массовых процентах, а с помощью молекулярной массы и формул, которые я показывал в начале, пересчитывают в мольные доли.

\subsubsection{Последовательность расчёта состава пластового газа}

\begin{center}
\includegraphics[width=\textwidth, page=129]{Брусиловский.pdf}
\end{center}

И вот тут приведены формулы простейшие.
Как нам определить число молей?
И вот я проверял.
Я взял вот этот руководящий документ Газпрома и самым внимательным образом проверил формулы, которые там приводятся, которые соответствуют формулам в более ранних инструкциях 80-го, 95-го.
Ничего не изменилось.
Только количество компонентов группы С6+, оно больше стало.
А принцип один и тот же.
Значит, вот расчёт.

\begin{center}
\includegraphics[width=\textwidth, page=130]{Брусиловский.pdf}
\end{center}

Теперь я вам покажу табличку, где использованы эти формулы.
И в итоге вам будет всё совершенно понятно.
Вот у нас расчёт состава пластового газа.
При расчёте состава пластового газа мы исходим из тысячи молей газа сепарации.
Смотрите, колонка 3 внизу, газ сепарации, грамм-молей.
Ну, упрощённо говорят, молей, грамм молей.
Значит, вот у нас колонка 3.
У нас всего тысяча молей газа сепарации.
Зная компонентный состав газа сепарации, полученный хроматографией, мы определили число молей каждого компонента (метана, этана, пропана и так далее, С6+, тут ещё азот и диоксид углерода рассматривается).
Дальше.
Сколько же у нас газа дегазации молей каждого компонента?
По формуле, которая приведена на предыдущем слайде, вы захотите, можете...
Мы можем с вами определить количество молей газа, выделяющихся при дегазации, и соответствующие тысячи молей газа сепарации, то есть базовым отталкиваемся от того, что газ сепарации тысяча молей.
Оно равно А умноженное на Q и делённое на В.
То есть обозначения те же.
Я сейчас смотрю на руководство Газпрома, где я внимательно проверял формулу этого 2010 года, страница 136, и на слайд.
И мы видим, что число молей газа дегазации 17.94.
Давайте посмотрим на табличку.
Да, вот 17.94 всего молей газа дегазации.
Мы определили его компонентный состав хроматографией, и это дано в процентах.
И умножив общее число молей на мольную долю каждого компонента, получаем число молей соответствующих компонентов.
Например, 17.94 умноженное на 0.637 мольную долю метана, получаем 11.43.
То же самое для газа дебутанизации.
У нас 3.34 моля газа дебутанизации образовалось в результате дебутанизации дегазированного конденсата.
И это количество молей соответствует тысячам молям газа сепарации.
Значит, зная компонентный состав газа дебутанизации, мы определяем число молей газа дебутанизации, колонка 7.
Дальше у нас остается дебутанизированный конденсат.
Мы определили общее число молей дебутанизированного конденсата, приходящее на тысячу грамм-молей газа сепарации.
У нас получилось 16.12 числа молей дебутанизированного конденсата.
Мы определили его. Тонкость.
Сначала мы определили массовую долю каждого компонента в результате газожидкостной хроматографии.
А затем, зная молекулярную массу каждого компонента, мы пересчитали в мольную долю.
И вот у нас колонка 8.
Это в процентах мольная доля компонентов в дебутанизированном конденсате.
И вы видите, что у нас нет газовых так называемых компонентов, что этот дебутанизированный конденсат состоит только из нормального пентана и группы С6+ выше.
Если бы мы подробно, так как пишет Р Газпром, например, разбивали С6+ на фракции, то просто бы мы мельтешение цифр имели бы и мало, что увидели бы на экране, а суть-то абсолютно та же.
Вот в итоге мы имеем число молей всех составляющих, которые и в знаменателе, и в числителе для каждого компонента нам нужны.
В знаменателе общее число молей всех компонентов.
Теперь перейдем к колонке 12.
В колонке 12 просуммировано число молей.
Самая нижняя строчка –- это всего молей у нас 1037.39.
Из них 1000 – это число молей газа сепарации, а 37.39 – это соответствующие этим 1000 молям газа сепарации всех остальных составляющих (газ дегазации, дебутанизация, дебутанизированный конденсат).
И, зная дальше, мы определили для каждого компонента, просуммировав по всем составляющим, число молей.
Дальше мы пронормировали, должны пронормировать, то есть поделить число молей каждого компонента на сумму, то есть для метана, если мы это сделаем, 873.43, поделим на 1037.39, мы получим с вами 0.8419, или в процентах 84.19.
Это мольная доля метана в составе пластового газа.
Итак, мы сделали такую нормировку и умножим на 100, чтобы получить проценты.
Для каждого компонента, вот колонка 13, в итоге мы получили состав пластового газа.
Именно так определяется компонентный состав.
Теперь, вот мы получили состав пластового газа в процентах мольных.
Это основа для термодинамических расчетов в программных комплексах.
И сепарации, и течения в стволе скважины, и прочее, прочее.
Если вы знаете состав пластового газа, и вы его в качестве исходной информации в программу ввели, ну, только, конечно, С6+ нужно на фракции разбивать.
Ну, в технических отчетах, которые сейчас приводятся по данным исследований гаоконденсатных, там гораздо более подробно.
Не ограничивается С6+, а там, повторяю, достаточное количество компонентов для того, чтобы, введя их в программу, правильно описать массообменные процессы, которые будут происходить с нашим пластовым газом.

\subsubsection{Определение потенциального содержания конденсата (C5+) в пластовом газе ПС5+, г/м3}

\begin{center}
\includegraphics[width=\textwidth, page=131]{Брусиловский.pdf}
\end{center}

Значит, дальше.
А дальше, значит, главное, мы определили компонентный состав.
Теперь, что нам важно еще?
И это соответствует, повторяю, вот формулы, которые приводятся.
Нам нужно рассчитать потенциальное содержание конденсата в пластовом газе.
Мы первоначально всё делали на тысячу молей газа сепарации.
Вот когда получали число молей других составляющих, потом получаем число молей всех составляющих, просуммировали и так далее, и получили компонентный состав, мольный, пластового газа.
Значит, еще важная вещь –- это, так называемое, потенциальное содержание конденсата С5+, в пластовом газе.
И вот первая формула, и та, которая приводится в инструкциях, начиная с инструкции 1975 года, ну, там это руководство, это отраслевая инструкция, начиная с инструкции 1980 года, и руководство 1995, никаких изменений нет, всё правильно.
Значит, вот приводится с учетом тех обозначений, которые вы видите на слайде и на предыдущих слайдах, зная соотношение, зная конденсатогазовый фактор, полученный, измеренный на промысле, зная объём контейнера, зная, значит, объёмы выделившегося газа дегазации, дебутанизации, зная содержание С5+ в газе дегазации, ну, то есть, L1, L2, L3, вот они, просто эти обозначения стандартные, и, значит, малярную массу в отсепарированном газе и так далее, мы получаем потенциальное содержание конденсата в граммах на метр кубический газа сепарации.
Значит, я почему акцентирую, потому что в издании, значит, руководства 1980 года по всему Советскому Союзу, это официальное издание было, там было просто грамм на метр кубический.
Ну, мы же имеем, нас интересует и пластовый газ, и сухой газ.
И вот не было сказано, что на метр кубический газа сепарации, и я сам выводил эти формулы, значит, проверял, проверял.
Они правильные, и единственное, значит, вот нужно, вот тут вот правильно написано всё, грамм на метр кубический отсепарированного газа.
Вот это важно очень.
А нам при проектировании разработки и при подсчёте запасов, нам нужно знать, не на метр кубический отсепарированного газа, а на метр кубический пластового газа и на метр кубический сухого газа.
Пластовый газ включает все компоненты.
Значит, если мы знаем потенциальное содержание конденсата в граммах на метр кубический отсепарированного газа, то мы, вот пункт 2, умножив это потенциальное содержание на 1000, у нас же газа сепарации 1000 молей, и поделив на суммарное количество молей всех составляющих в пластовом газом, помните, там 1039 или что-то, мы получаем потенциальное содержание компонентов С5+, то есть стабильного конденсата, приходящегося на метр кубический пластового газа.
А затем мы также из таблички знаем, какова доля сухого газа в пластовом газе -- это уже из компонентного состава.
Я вам напомню.
Значит, вот правая самая 13-я колонка, это предыдущий слайд 130-й.
Значит, и вот у нас всего молей 1030, сейчас, секундочку, да, значит, у нас в сумме 100\% всех компонентов, а сухой газ, это газ, не включающий пентаны плюс вышекипящие.
Вот если мы из 100\% вычтем содержание пентанов и С6+, мы получим с вами долю процентного содержания компонентов, составляющих сухой газ.
И вот $y_{C5+}$, то есть мы выч ли из единицы, единица минус $y_{C5+}$, в процентах, делим на 100.
Вот $y_{C5+}$ -- это суммарное содержание C5+, в пластовом газе мы его уже знаем, определив состав пластового газа.
И получаем потенциальное содержание в граммах на метр кубический сухого газа.
Вот основная информация, которая нам нужна для подсчета запасов и проектирования разработки.
Когда мы проектируем разработку, мы прогнозируем объемы добычи пластового газа, и зная начальное содержание, ну, вернее так, значит, смотрите, мы сначала про подсчет запасов, это же начальное содержание.
Начальное содержание в нашей пластовой смеси по результатам исследований разведочных скважин мы получили, что у нас 78.2 грамма на метр кубический пластового газа.
Мы подсчет запасов осуществляем на пластовый газ и на сухой газ.
Рассчитав объем геологических запасов пластового газа, мы умножим на потенциальное содержание конденсата в граммах на метр кубический пластового газа, и получим с вами запасы, геологические запасы конденсата С5+.
А если, также мы рассчитываем геологические запасы сухого газа, и тогда мы, чтобы определить, каковы запасы конденсата, мы должны потенциальное содержание конденсата в грамм на метр кубический сухого газа умножить на геологические запасы сухого газа.
Это одно и то же, это одни и те же запасы будут, что когда мы используем грамм на метр кубический пластового газа, умножаем на запасы пластового газа, или же грамм на метр кубический сухого газа умножаем на запасы сухого газа.
Одна и та же величина для запасов С5+ будет.
В данном случае у нас среднеконденсатная система.

\subsubsection{Классификация газоконденсатных залежей по начальному содержанию конденсата (C5+)}

\begin{center}
\includegraphics[width=\textwidth, page=132]{Брусиловский.pdf}
\end{center}

И напоминаю, что газоконденсатные залежи по начальному содержанию конденсата, они классифицируются на низкоконденсатные, среднеконденсатные, высококонденсатные, уникальноконденсатные.
Вот то, что вы видите, это повторение слайда.
Но формулы, которые не были ранее приведены, это как нам определить потенциальное содержание конденсата в граммах на метр кубический сухого газа и пластового газа.
Это верхняя формула на сухой газ, а нижняя на пластовый.
Когда мы знаем мольную долю в процентах содержания пентанов плюс вышекипящих в пластовом газе, знаем молекулярную массу С5+ в пластовом газе, и тогда мы, используя формулу, которую мы можем рассчитать, потенциальное содержание конденсата в граммах на метр кубический сухого газа и пластового газа.
Вот эти формулы, они простые совсем.
Вы используете, когда будут соответствующие у вас практические вопросы, или же у вас кто-то попросит ответить на вопрос.
Вот вам дан состав пластового газа, дана молекулярная масса пентанов плюс вышекипящих, какого потенциальное содержание конденсата.
И вот вы видите формулы, считаете по ним, определяете.

Теперь, ну вот, это такой очень важный этап мы прошли.
У газа получили состав, теперь вы знаете, как получают компонентный состав пластового газа.
И как вы видите, немножечко, немножечко это сложнее, чем для нефти.
Там мы просто по результатам однократного разгазирования получали две составляющие, значит вот растворенный газ и дегазированная нефть.
В случае, когда мы по однократному разгазировали, определяем, а это основное.
Основное всегда делает лаборатория, рассчитывает по результатам стандартной сепарации нефти компонентный состав.
А тут мы определили компонентный состав, нужно было нам знать, здесь больше составляющих.
Газ сепарации, газ дегазации, газ дебутанизации и дебутанизированный конденсат.
Ну просто больше составляющих.
Если же, а за границей дебутанизацию не осуществляют.
Они только, они делали это работая у нас в стране.
А если брать их, смотреть отчеты для их месторождения, там никакой дебутанизации не производилось.
Так вот, значит, все просто, больше, больше просто составляющих.
Все вам теперь должно быть предельно понятно.

\subsubsection{Оценка коэффициента конденсатоотдачи}

\begin{center}
\includegraphics[width=\textwidth, page=133]{Брусиловский.pdf}
\end{center}

Теперь я все-таки, прежде чем перейти к экспериментальным исследованиям, термодинамическим.
Дальше. Теперь, я говорил, что важной задачей является оценка коэффициента конденсатоотдачи для газоконденсатной системы.
Если у нас потенциальное содержание конденсата меньше 25 граммов на метр кубический пластового газа, то экспериментальных исследований обычно не проводят, по крайней мере, требований таких нет.
Это уже, может быть, инициатива просто.
Но требований таких нет.
И определяют коэффициент извлечения конденсата по результатам разработки на режиме истощения пластовой энергии, Depletion, до одной физической атмосферы.
Вот таким простейшим образом по зависимости коэффициента извлечения конденсата от отношения суммы пентана, пропана, бутанов к концентрации пентанов плюс вышекипящих, зависимость была получена, тут никакой теории нет, просто на основе имеющихся практических результатов, опыта, разработки и исследований месторождений термодинамических лабораторий.
В 70-х годах этот график был построен, по которому можно, зная компонентный состав пластового газа, оценить, ну именно оценить, а какую же долю C5+ выше мы извлечем из пласта при достижении пластового давления одной физической атмосферы.
То есть мы просчитали, вот C2, это концентрация этана, пропана, бутанов, поделили на концентрацию в пластовом газе C5+, провели соответствующую вертикальную линию до пересечения с зависимостью, и на оси ординат у нас будет величина конденсатоотдачи.
Такой комментарий у меня по этому поводу, по поводу при одной физической атмосфере.
Значит, также оценивают и конденсатоизвлечение по результатам экспериментальных исследований, формулу я вам покажу, значит, при достижении давления одной физической атмосферы.
Но в пласте на практике, конечно же, никакой одной физической атмосферы не достигается.
Минимальное давление, иначе говоря, давление забрасывания, оно зависит от характеристик пласта и глубины, коллекторских свойств, и, значит, оно, это вообще говоря, определение давления забрасывания, это величина технико-экономических расчетов.
Но для оценки, причем это завышенная оценка, исходя из того состава компонентного, который мы имеем, значит, снижают давление до одной физической атмосферы, получают возможность с помощью простой формулы, которая будет приведена, вот сказать, какая доля конденсата при этом с газовой фазой будет добыта.
Соответственно, оставшаяся доля до 100\% безвозвратно потеряна в виде ретроградного конденсата.

\subsubsection{Важнейшие результаты лабораторных термодинамических исследований на установках PVT}

\begin{center}
\includegraphics[width=\textwidth, page=134]{Брусиловский.pdf}
\end{center}

Самые важные результаты лабораторных термодинамических исследований -- это давление начала конденсации, Z-фактор при начальном пластовом давлении и давлении начала конденсации и показатели дифференциальной конденсации.
Сейчас делают контактную дифференциальную конденсацию в соответствии с опытом зарубежным, поскольку и методики, и аппаратура сплошь зарубежные.
И не только поэтому, но это основная причина.

\subsubsection{Увеличение объёма при постоянной массе (составе) смеси (constant composition expansion). Контактная конденсация}

\begin{center}
\includegraphics[width=\textwidth, page=135]{Брусиловский.pdf}
\end{center}

Значит, проводят так...
Вот для нефтей мы с вами просматривали два вида экспериментальных исследований термодинамических.
Там был контактный процесс, контактное разгазирование и дифференциальное разгазирование.
Но разгазирование -- это для нефтей, для газированных жидкостей.
А для пластовых конденсатсодержащих газов, это не разгазирование -- это конденсация.
А так, значит, всё очень с точки зрения физической химии, термодинамики, терминологии похоже.
Значит, проводят исследования при постоянной массе смеси -- это контактный процесс, форма выделения конденсата из природного газа, при которой суммарный компонентный состав и масса системы не изменяются.
Для нефтей, значит, похожее определение было, только там разгазирование из пластовой нефти, при которой суммарные, главное, что суммарный компонентный состав и масса системы не изменяются.
Значит, выделившаяся фаза, в данном случае конденсат, для нефти был газ, находится в равновесии с газовой фазой.
Для нефти была в равновесии выделившаяся газовая фаза находится в равновесии с жидкой углеводородной фазой.
То есть, это контактная конденсация, удаление, значит, не происходит флюида, и PV-соотношения не строят для газоконденсатных систем, просто потому, что это бессмысленное занятие, мы не можем определить давление начала ретроградной конденсации по PV-зависимости.
Но, значит, схема проведения контактной конденсации, она очень похожа.
Значит, мы приводим нашу систему к давлению на, мы не знаем сначала давление начала конденсации, значит, мы вводим, ну, максимальное давление начала конденсации равно пластовому давлению.
Значит, мы приводим систему на 3-5 мегапаскалей выше, загружаем систему в сосуд и создаем такой объем, значит, у нас система на несколько мегапаскалей выше, чем пластовое давление.
И, значит, соответственно, если мы правильно загрузили систему в правильном соотношении, а загружают газ сепарации, насыщенный конденсат, и в соотношении с соответствующим измеренным конденсатогазовым фактором, ну, то есть создают модель пластового газа.
Значит, у нас система при давлении выше, чем пластовое, она должна быть в однофазном газовом состоянии.
Дальше, значит, если мы будем увеличивать объем системы, давление будет падать, и вот $p_s$ –- это saturation pressure, значит, $p_1$, вот мы выше, чем давление насыщения, много выше, на несколько единиц мегапаскалей, на несколько десятков бар.
Зачем мы увеличиваем объем? Снижаем давление.
Однофазное состояние у нас остается.
И, наконец, у нас, вот если визуально, то запотевает окошечко, запотевает стекло, значит, вот на окошечке нашего PVT-сосуда.
Это значит, мы достигли давления начала ретроградной конденсации.
И для современных установок есть оптические анализаторы, которые показывают, что у нас вот началось выделение, значит, жидкой фазы -- давления начала ретроградной конденсации.
И при дальнейшем увеличении объема выделяется ретроградная жидкая фаза, значит, ну и вот это показано, раз мы увеличиваем объем, давление уменьшается, и, значит, этот объем ретроградной жидкой фазы увеличивается до давления максимальной конденсации.

Давайте с вами вспомним фазовые диаграммы.
Фазовые диаграммы, значит, которые, кстати, ну да, они строятся для смеси начального состава, то, что, постоянного состава, то, что соответствует контактной конденсации.
И вот при каком-то давлении затем у нас мы достигли максимального объема выпавшего ретроградного конденсата, и при дальнейшем увеличении объема начнется процесс уже прямого испарения.
Это вот схема экспериментальных исследований контактной конденсации.
Основной смысл контактной конденсации для нас -- определение давления начала ретроградной конденсации, а также можно определить Z-фактор, экспериментальное значение Z-фактора при, значит, любом давлении в однофазной области, то есть начальном пластовом, и давление начала конденсации нас интересует.

\begin{center}
\includegraphics[width=\textwidth, page=136]{Брусиловский.pdf}
\end{center}

Теперь вот пример.
Пример результатов лабораторных исследований контактной конденсации, Constant Composition Expansion, то есть расширение при постоянном составе.
Вот график приведен по результатам приведенных исследований.
На что хочу обратить внимание?
Постоянная температура пластовая, значит, фиксируется объем ячейки и объем жидкости, которая в этой ячейке выпадает.
Ну и, естественно, соответствующее давление.
Теперь вы видите справа пластовые потери сантиметр куб на стандартный метр кубический.
Это измеряют объем.
Вот измерили же объем жидкости, выпавшей, значит, и относят к объему пластового газа, приведенному к стандартным термобарическим условиям.
Значит, пластовая смесь, которую мы в бомбе PVT загрузили, в соответствии, то есть рекомбинировали, рекомбинировали газ сепарации и насыщенный конденсат в соответствии с конденсатогазовым фактором на промысле измеренным.
Потом перемешали все это, привели давление к начальному.
В начальном у нас вот это начальное состояние должно быть газообразное.
И это количество смеси мы приводим к объему стандартных метров кубических.
Сейчас я вот расскажу, как это делается.
И относим объем конденсата к объему стандартных метров кубических начальной пластовой смеси.
Для меня всегда было очень как-то непонятно.
Ну, непонятно, это трудно оценить, вот это величину.
А в зарубежных отчетах гораздо более все четко, ясно и более понятно, например, с точки зрения поведения гидродинамических расчетов.
Просто объемная доля жидкости, которая выпадает в объеме сосуда, вот она фиксируется, она фиксируется, объемная доля жидкости.
И мы знаем текущий объем сосуда, мы знаем, какой объем жидкости в этом сосуде, вот делим одно на другое, получаем объемную долю.
Вот понятно.
И когда мы моделируем, например, процесс constant composition expansion, и имеем результаты в зарубежных отчетах контактной конденсации, и переводится там объемная доля жидкости, мы можем сравнить с результатами нашего математического моделирования очень легко.
Такая величина, как сантиметр куб на стандартный метр куб, я в своей программе, которую писал в конце 80-х и 90-х, которую внедрял во многих институтах, я, конечно, этот показатель, я и то, и другое приводил, приводил по результатам расчетов.
Но в зарубежных программах не всегда есть такой показатель, как сантиметр куб на стандартный метр куб.
И, кстати говоря, в наших отчетах, старых 70-х годов, 80-х годов, до того, как пришли зарубежные компании, там приводились данные сантиметр куб на стандартный метр куб, а объемная доля жидкости не приводилась.
И очень трудно было сравнивать поначалу, когда по PVTI, например, я считал, правильно, неправильно, насколько наша математическая модель близка или нет, а по своей программе я считал, у меня было и то, и другое, то есть удобнее было.
Но по зарубежным программам, пока отчеты не стали соответствовать, сейчас они соответствуют и зарубежным стандартам в отчеты наших известных лабораторий, там ГЕОДАТА очень хорошая лаборатория и другие, ну, не буду называть, это не то, что реклама.
За исключением одной лаборатории, которую я уже называл, которая специализировалась по исследованию пластовых нефтей, у меня нет отрицательных каких-то эмоций по поводу других лабораторий, и люди стремятся совершенствовать всё и приводить в порядок.
Сейчас можно всё.
Дальше пошли.
Можно использовать и хорошо, удобно использовать отчеты, сравнивать с математическим моделированием, то есть всё на гораздо более высоком уровне.

\subsubsection{Понятие Z-фактора (коэффициента сжимаемости)}

\begin{center}
\includegraphics[width=\textwidth, page=137]{Брусиловский.pdf}
\end{center}

Теперь, также мы говорим, что Z-фактор мы измеряем.
Z-фактор -- это технологии разработки и подсчёта запасов природных газов, это основной физический параметр, зная который мы можем очень многое рассчитывать, моделировать.
Я напоминаю, что такое Z-фактор, для тех, кто забыл или этой проблемой вообще не занимался.
Есть слева у нас уравнение состояния идеального газа.
Я, когда буду делать обзор по уравнениям состояния, которые применяются...
Я там немножко напомню.
Первый раз я напоминаю про уравнение состояния идеального газа.
$pV=NRT$, $N$ -- число молей смеси нашего газа, равное массе газа делённой на его молекулярную массу.
И для одного моля у нас $pV=RT$, $R$ -- это универсальная газовая постоянная, $T$ -- это абсолютная температура.
Универсальная газовая постоянная, она должна соответствовать величине той размерности давления, температуры абсолютной и объёма, который используется.
Это понятно из курса физики.
А уравнение состояния реального газа справа, оно описывает свойства газа, которые отличаются от идеальных для реальных газов, и вводится поправочный коэффициент Z-фактор.
Вот и всё.
Z-фактор позволяет учесть реальные свойства газа, газовую фазу.
Это я напоминаю просто сведения из учебника физики.
Дальше будет график.
Когда будем рассматривать корреляции для расчёта Z-фактора реального газа, там будет и диаграмма.
Диаграмма приведена как определять не по уравнению состояния, а как в инженерных расчётах, не имея программы с уравнением состояния, как определять Z-фактор.
Как это делали всегда при подсчёте запасов и технологических, при подсчёте запасов, например.

\subsubsection{Лабораторные исследования природных газов}

\begin{center}
\includegraphics[width=\textwidth, page=138]{Брусиловский.pdf}
\end{center}

Теперь по поводу Z-фактора.
Тут два случая его определения по экспериментальным данным.
Первый случай, когда у нас сухой газ.
Как мы можем определить Z-фактор газа?
По поводу Z-фактора, коэффициент сжимаемости.
Вообще часто в технологии определяют, не часто, а почти всегда в инженерных задачах, либо по корреляциям, либо с применением уравнений состояния Z-фактор для природного газа.
Экспериментально, если нужно определить на основе экспериментальных данных, то, как определяют Z-фактор, тут два варианта.
Первый -- это сухой газ, в котором нет пентанов плюс вышекипящих.
В этом случае в бомбе PVT при том давлении и температуре, для которых нам нужно знать Z-фактор, определяется объем, который занимает наша смесь.
Дальше, эта смесь выпускается из бомбы PVT и при стандартных условиях, то есть одна физическая атмосфера и температура стандартные, ну, окружающей среды, в формуле $T_0$, значит, это стандартная, по идее.
Не обязательно, главное, чтобы замерять температуру, все в абсолютных величинах, измеряется объем, который занимает эта смесь.
Откуда эта формула?
Вот на предыдущем слайде $pV=NZRT$.
В бомбе PVT у нас смесь N молей газа, и выпустили мы эту смесь, эти N молей газа.
Значит, $pV=NZRT$.
И мы приравниваем, значит, мы вот что делаем.
Мы относительно универсальной газовой постоянной, например, перепишем, вот это вот $pV=NZRT$, или даже относительно произведения, N числа молей на R.
Значит, и эта величина в условиях рабочих при $p$ и $T$, и в условиях стандартных $p_0$, $T_0$ одинаковые, значит, мы приравниваем соответствующие соотношения, делаем, и получаем, и затем определяем из этого соотношения $Z=pVT_0/(p_0V_0T)$.
Температура абсолютной величины.
Именно таким образом, как я понимаю, определяют, ну не то чтобы именно так, определяет экспериментально величины Z-фактора, когда строили диаграмму Стендинга-Каца, для метана она была построена, вот так для чистых веществ или для смесей строится сухих газов, определяется зависимость Z-фактора, давление, температура.

Теперь, значит, экспериментально.
Если у нас газоконденсатная смесь, тут несколько по-другому.
Почему?
Потому что у нас тоже измеряется объем, занимаемый смесью при рабочих давлении и температуре в сосуде PVT, затем то количество смеси, объем который был замерен в рабочих условиях, выпускается из бомбы PVT и при атмосферных условиях эта смесь, из нее выделяется, из этого газа выделяется жидкость, это конденсат.
Это не в чистом виде С5+, но это конденсат.
И получается так: у нас из двух составляющих, это вот второй пункт, из двух составляющих у нас есть газ, газовая фаза, и есть жидкая фаза.
И мы хотим воспользоваться формулой $Z=pV/(NRT)$, то есть формулой, соответствующей закону для реального газа.
И нам единственное, что давление мы знаем в рабочих условиях, температуру знаем, объем знаем, нам число молей неизвестно той смеси, которая занимала объем $V$ при рабочих условиях $p$ и $T$.
Значит мы измеряем, эта смесь, она на две фазы разделилась при атмосферных условиях.
Значит мы должны измерить объем жидкости, которая при атмосферных условиях образовалась, измерить ее плотность, молекулярную массу.
И таким образом мы можем определить число молей этой жидкости при атмосферных условиях.
Вот формула написана: первое слагаемое -- это число молей этой жидкости.
И определить число молей газа, который выделился из нашей пластовой смеси при приведении к атмосферным условиям.
Мы измерили его объем и разделим на объем, занимаемый одним молем газа при атмосферном давлении, текущей температуре, при стандартных условиях естественно, при стандартных условиях, 24.04 литра.
Мы считаем, что у нас температура 20 градусов Цельсия, вот при этих 293 K, вот при этих условиях у нас объем моля 1.2404.
Значит мы определили число молей газовой фазы, сложили, а в сумме это и есть число молей той смеси, которая была выпущена из сосуда, и которая в сосуде при давлении P и T, занимала объем равный V.
Вот так определяют Z-фактор газоконденсатной смеси.

\subsubsection{Подсчёт запасов свободного газа объёмным методом (иллюстрация использования Z-фактора пластового газа)}

\begin{center}
\includegraphics[width=\textwidth, page=139]{Брусиловский.pdf}
\end{center}

Дальше пошли.
Вот говорится много о Z-факторе, но на самом деле в технологии для природных газов это основное физическое свойство, как бы везде присутствующий во всех технологических расчетах, и в частности я хотел вам показать, как используется Z-фактор при расчете начальных геологических запасов свободного газа, это не все знают.
Значит вот написана формула, первые сомножители, это объем газонасыщенной пористой среды определяют.
Затем у нас есть сомножитель, где у нас начальное давление $p_0$, величина обратная Z-фактору, некое остаточное давление и умноженное на соответствующее этому давление, величина обратная Z-фактору.
Делится на стандартное давление, стандартное давление это у нас одна физическая атмосфера.
Но если мы используем мегапаскали, просто у нас и в числителе, и в знаменателе должна быть единая размерность.
Если в числителе мегапаскали, значит в числителе, в знаменателе у нас стандартное давление 0.101325.
Если у нас бары, соответственно бары и в числителе, и в знаменателе.
И умножаем на отношение температуры, абсолютно стандартной, это 293.15 K, на пластовую, тоже в Кельвинах, ни в коем случае не Цельсия, потому что это просто неправильные результаты.
Когда абсолютная температура должна, а по давлению просто требования одинаковой размерности в числителе и в знаменателе.
Так вот, это просто формула выводится, вы можете сами ее вывести, это из материального баланса.
И она позволяет нам оценить запасы газа при данных характеристиках газонасыщенной пористой среды.
Также эта формула позволяет оценить тот объем газа, который будет добыт из залежи при снижении давления в ней от начального до некого текущего.
То есть, это полезная очень формула.
И особенность, которую не все знают, когда подсчитывают геологические запасы газа, то считается, что давление в пласте, окончательно, то есть, давление, это не одна физическая атмосфера.
Одна физическая атмосфера на устье скважины.
Значит, по барометрии, значит, наше давление, вот в формуле остаточная, это не одна физическая атмосфера.
Это давление, которое будет в пласте, когда на устье у нас будет одна физическая атмосфера.
Это такая особенность, которую далеко не все знают в учебниках или кто-то в подсчете из запасов.
Значит, при этом переход от устьевого давления к остаточному давлению в пласте происходит по барометрической формуле.
В этой барометрической формуле, значит, считается, пренебрегают потерями давления в течении газа в стволе скважины.
То есть, считается только столб, столб гравитационный, гравитационный столб по барометрической формуле.
И, значит, мы должны знать глубину $h$, и должны знать с вами относительную плотность газа.
Относительную плотность, мы с вами о ней очень много уже говорили.
И относительная плотность, это поэтому, поскольку у нас относительная плотность, у нас есть некий сомножитель 1.205.
Но это просто мы от относительной плотности газа хотим перейти к абсолютной плотности газа для того, чтобы гравитационный столб посчитать.
А вспомним, что плотность воздуха равняется 1.205 кг на метр кубический.
Плотность воздуха.
Значит, умножаем плотность воздуха на относительную плотность нашего газа, получаем абсолютную плотность нашего газа при стандартных условиях.
Ну и дальше идёт вот это стандартные данные, используемые в подсчёте запас в барометрической формуле.
Все размерности данные.
Теперь по поводу второго.
То есть, мы должны что знать?
Мы должны знать Z-фактор при начальном пластовом давлении.
И должны знать Z-фактор при давлении остаточном.
Есть у нас остаточное давление.
Теперь чему равно остаточное давление?
Зависит оно, естественно, от плотности газа.
Чем он жирнее, тем будет больше столб.
Но и сильно зависит от глубины скважины.
Чем больше глубина скважины, тем столб газа будет больше.
И давление в пласте, соответствующее одной атмосфере на устье, оно будет больше.
Но всё равно это всего-навсего несколько атмосфер, несколько бар.
Я оценивал для глубокопогруженных залежей.
У меня получалось 5 бар для глубины в 4 или сколько-то километров.
Это можно самим просто оценить.
Взять относительную плотность газа, предположим, 0.8 или 0.9, если более жирный газ.
Пластовый газ любой.
И оценить при разных глубинах какое же давление в пласте будет остаточное.
Остаточное для использования вот этой формулы.
Вот такая вещь.
Часто, наверное, пренебрегают тем, что остаточное давление не равно одной физической атмосфере.
Потому что кто будет пересчитывать, если только эксперты такие дотошные очень.
А так и большой погрешности не будет.
Ну вот.

А вот ещё интересно для тех, которые разработкой занимаются.
Это можно по этой формуле оценить динамику добычи газа при снижении давления от начального до заданного.
Потому что мы можем задать...
$P_{\text{ост}}$ может быть любым давлением.
Это же уравнение материального баланса для сухого газа, для залежи природного газа.
Без учёта выпадения конденсата.
И мы можем таким образом, задав интересующее нас давление, оценить, какой объём газа будет при этом добыт при снижении давления от начального до заданного.
И для этого нам нужно тоже знать Z-фактор.
Как при начальном давлении, так и при заданном.
Это для месторождений с уникальным составом, где много сероводорода, диоксида углерода.
Ну вот Прикаспии определялся Z-фактор экспериментально.
А вообще, если вы уверены, это можно вполне делать с применением уравнения состояния.
Либо корреляцией.
Но если в уравнение состояния вы задаете компонентной состав и хорошее уравнение состояния, которые в программных вычислительных комплексах, там они применимы для сложных составов, когда есть сероводород, диоксид углерода и так далее.
То корреляции, которые используются, они в основном были получены для сухих газов.
И их точность, она требует...
Ну, если есть сомнения, можно сравнить с тем, что дает уравнение состояния, что дает корреляции, ну и так далее.
Вот. И понятно. Очень понятно.

\subsubsection{Расчёт балансовых запасов конденсата газового стабильного (КГС), этана, пропана, бутанов и неуглеводородных компонентов}

\begin{center}
\includegraphics[width=\textwidth, page=140]{Брусиловский.pdf}
\end{center}

Дальше поехали.
Теперь по поводу потенциального...
Для чего используется потенциальное содержание, я же говорил.
Конденсата.
Мы зная запасы газа, в данном случае пластовый газ, $Q_{\text{Г}}$, значит, пластовый газ, мы умножив на потенциальное...
А, ну, собственно говоря, не важно.
Умножив на потенциальное содержание конденсата, граммов на метр кубический пластового газа, посчитать запасы конденсата.
Если у нас потенциальное содержание граммов на метр кубический пластового газа, то в этой формуле нужно запасы пластового газа использовать.
А если потенциальное содержание граммов на метр кубический сухого газа, значит, у нас в сомножителе, в числителе должны быть запасы сухого газа для подсчета конденсата.
Ну, аналогичные формулы, они используются для подсчета запасов компонентов (этанов, пропанов, бутанов, значит, и...)
Здесь $\Pi_i$, формула такая же, как для конденсата, но $\Pi_i$ относится к этану, пропану, бутанам, и в руководствах вот сказано, что, давайте вы, потенциальное содержание этана, пропана, бутанов определяется умножением молярного процента содержания каждого из них в пластовом газе на коэффициенты, соответственно, 12.5, 18.3, 24.2.
И вопрос в том, что это за коэффициенты, откуда они взяты?
Ну, вот, значит, это соответствует, соответствует, вот, значит, молярное...
Давайте $\Pi_i$ равняется молярное процентное содержание, да, умножается на молекулярную массу компонента, делится на 2.404.
Так вот, если мы молярную массу этана, 30, я пренебрегаю всякими десятыми, сотыми, разделим на 2.404, мы получим 12.5.
Для пропана, молекулярная масса 30, $C_2H_6$, мы получим, разделим на 2.404, получим 18.3.
И то же самое для бутанов.
Вот, откуда эти коэффициенты в руководствах, и там не сказано, значит, как они получены, эти коэффициенты.
Ну, а мы с вами вот знаем, что из формулы, из понятия, значит, ну, вот таким образом, да.
Все это соответствует расчету потенциального содержания компонентов в газовой смеси.
Если мы потенциальное содержание компонентов вот так рассчитали и умножили на запасы газа, значит, мы получили запасы не только конденсанта, а и компонентов.

\subsubsection{Дифференциальная и контактно-дифференциальная конденсация (CVD)}

\begin{center}
\includegraphics[width=\textwidth, page=141]{Брусиловский.pdf}
\end{center}

Значит, итак, про контактную конденсацию.
Первый тип экспериментальных исследований мы с вами знаем.
Теперь дифференциальная и контактно-дифференциальная конденсация.
Значит, в чём смысл?
Смысл в том, что мы выпускаем газовую фазу, и за счёт выпуска газовой фазы из сосуда у нас изменяется давление.
И мы говорили о дифференциальном разгазировании пластовой нефти.
То есть, когда у нас за счёт изменения объёма осуществилось соответствующее уменьшение давления, осуществлялось разгазирование пластовой нефти, и дальше мы ступенчато осуществляли дифференциальное разгазирование.
А вот по поводу дифференциальной конденсации тут смысл совершенно другой.
Там мы получали зависимости свойств нефти и газа от давления для использования при гидродинамическом моделировании разработки месторождений.
При этом непосредственно вопросы прогнозирования показателей разработки мы не затрагивали.
Просто динамику свойств нефти и растворённого газа, которые мы используем дальше при моделировании разработки.
А вот здесь процесс контактной дифференциальной конденсации, эксперимент, он непосредственно связан с моделированием.
Просто мы моделируем процесс разработки залежи на режиме истощения пластовой энергии.
Но понятно, что предполагая, что у нас объём сосуда, сосуд PVT, он моделирует нам объём газонасыщенного пространства.
Понятно, что мы не учитываем всякие капиллярные, сорбционные силы, и неоднородный коллектор, но это всё понятно.
Это просто аналог материального баланса, очень хорошо нам позволяющий моделировать динамику газоотдачи, динамику состава добываемого газа, динамику конденсатоотдачи, то есть при разработке газоконденсатной залежи.
В точной постановке процесс дифференциальной конденсации, именно дифференциальной конденсации, N-компонентной смеси, он описывается, приведённой системой обыкновенных дифференциальных уравнений, и я просто заодно вам вставлю в курс дела.
И используя уравнение состояния, мы можем в точной постановке, решив эту систему обыкновенных дифференциальных уравнений, описать дифференциальную конденсацию или истощение газоконденсатной системы.
Теперь экспериментально.
Раньше дифференциальную конденсацию моделировали до поступления иностранной аппаратуры, и при этом непрерывно выпускали газ с определённым темпом, достигали давления заданного, фиксировали объём выпускаемого газа и его характеристики, и после нескольких загрузок, до 10 загрузок пластовой смеси, которые необходимы были для получения соответствующих характеристик процесса, именно процесс дифференциальный моделировался.
Проблема возникала, во-первых, в том, что нужно несколько загрузок делать, но главная проблема, это то, что у нас темп выпуска газа, он всё равно не позволял, реальный темп выпуска газа при экспериментальных исследованиях, он приводил к неравновесности системы, и вместе с газовой фазой из бомбы PVT удалялись капельки конденсата, поскольку система при непрерывном удалении газа, она не могла прийти в равновесие, и в общем она была неравновесна.
Оценку этой неравновесности очень хорошо описал Новопашин Владимир Фёдорович, он сделал оценки соответствующие, это из института ТюменьНИПИгаз, очень грамотный человек, он сейчас уже ушёл на пенсию, но в общем он руководил лабораторией ТюменьНИПИгаз газоконденсатных исследований, поэтому перешли эту проблему неравновесности, её удалось решить при переходе к технологии, которая за рубежом используется, то есть собственно вот Constant Volume Depletion, то есть когда у нас дифференциальный процесс, он заменяется ступенчатым процессом, ступенчатым процессом, это тоже неточное описание дифференциального процесса, но это лучше, чем недостижение равновесия в дифференциальном процессе, когда есть неопределённость в темпе удаления газа и так далее.
Что же делают в процессе Constant Volume Depletion?
Приводят нашу рекомбинированную систему, или глубинную пробу, если MDT отобрали, приводят нашу систему к давлению, да, ну вот по контактной конденсации мы же определили давление начала ретроградной конденсации, поэтому мы уже знаем при каком давлении, то есть начнётся ретроградная конденсация, давление начала, и мы к этому давлению приводим нашу смесь.
Дальше мы контактно увеличиваем, не удаляя ничего из сосуда, увеличиваем объём, и поскольку начали с давления начала ретроградной конденсации, у нас выделился ретроградный конденсат.
Мы зафиксировали давление $p_2$, мы зафиксировали объём, и дальше мы открываем вентиль и при этом постоянном давлении $p_2$, поджимая, удаляем газ.
При этом у нас давление постоянное в этом процессе, значит у нас удаляется газовая фаза, равновесная с жидкой фазой, а мы, кстати говоря, достигли давления $p_2$, нашу смесь мы мешаем, мешаем, мешаем, и добиваемся того, что у нас именно равновесное состояние при давлении $p_2$, выпавшего ретроградного конденсата и пластового газа.
Вот здесь уже чётко совершенно равновесие.
Затем мы выпускаем газовую фазу при постоянном давлении и в отличие от дифференциального разгазирования пластовой нефти, мы объём, мы не всю газовую фазу удаляем, а мы вот этот этап заканчиваем, когда наш объём рабочий становится равным первоначальному, который был при давлении начала конденсации.
При этом у нас уже есть выпавший конденсат, мы замерили объём газа, который вышел, компонентный состав его, мы можем определить Z-фактор, который при текущем давлении $p_2$ у нашего газа, и потенциальное содержание, то есть всё, что нам нужно.
Дальше повторяется процесс, уже опять увеличиваем объём рабочий контактно, то есть происходит контактная конденсация, давление снижается ниже, чем $p_2$, фиксируем давление $p_3$, ретроградный конденсат ещё его объём увеличился, дальше мы опять перемешиваем, чтобы у нас всё было в равновесии, и дальше опять выпускаем газ до достижения объёма, равного первоначальному, то есть как бы объём сосуда PVT.
Смысл этого процесса в том, у нас же газонасыщенная пористая среда, если мы пренебрегаем деформацией и поступлением воды, она неизменна.
Вот мы и возвращаемся к этому неизменному начальному объёму.
Мы таким образом ступенчато моделируем процесс контактно-дифференциальной конденсации, и это нам позволяет оценить, что же происходит при разработке на режиме истощения пластовой энергии в газоконденсатном пласте.
И ещё раз возвращаясь к физике, дифференциальная конденсация осуществлялась с таким темпом, другого просто не дано было на практике, что у нас газ и ретроградный конденсат не были в равновесии, и темп удаления газа гораздо выше, чем на практике, чем при разработке месторождения.
И это раз.
Несколько загрузок нужно делать, это два.
А здесь у нас недостаток в том, что дифференциальный процесс, он заменяется контактно-дифференциальным.
Но сейчас, когда вы видите отчёты зарубежных компаний или наших лабораторий, вот это изменение давления, оно буквально через каждые 10 бар, это ступени, потому что современные аппаратуры позволяют быстро измерения проводить.
Поэтому этот процесс близок к дифференциальному, не моделируя в точной постановке этот процесс.
Вот такая вещь.
Это общепринятая международная практика.
Но главное, что мы моделируем процесс истощения.

\subsubsection{Пример результатов лабораторных исследований контактно-дифференциальной конденсации (CVD)}

\begin{center}
\includegraphics[width=\textwidth, page=142]{Брусиловский.pdf}
\end{center}

\begin{center}
\includegraphics[width=\textwidth, page=143]{Брусиловский.pdf}
\end{center}

Теперь в каком виде отчёты публикуются?
Сейчас вы видите и в табличной форме, и в графической.
У нас есть давление, при котором осуществлялись шаги контактно-дифференциальной конденсации.
И вы видите, какая объёмная доля жидкости выпала в процессе истощения при постоянном объёме.
И формула приведена, которую можно легко вывести самим тоже.
Это оценка коэффициента излечения конденсата при достижении давления, равного одной физической атмосфере в пласте и в бомбе PVT.
У нас так, в числителе.
По результатам контактной дифференциальной конденсации, можно оценить коэффициент извлечения конденсата.
Где у нас обозначение показано, $Q$ -- это объём конденсата, оставшийся в ячейке при одной физической атмосфере, соответственно, $\rho_{\text{ск}}$ стабильного конденсата, плотность конденсата, оставшегося в ячейке, дегазирована.
Дегазированный конденсат, оставшийся в ячейке, и его плотность приведена к 20 градусам Цельсия, как требуется.
И вот, что ещё.
Это потенциальное содержание С5+, в граммах на метр кубический газа.
Так вот, о потенциальном содержании уже мы много говорили.
Это начальное, начальное имеется в виду.
Значит, если вы используете грамм на метр кубический газа сепарации, то должно быть и в знаменателе, и в числителе, и, значит, это... и объём конденсата, оставшегося в ячейке, $Q$, да, это отнесённый к...
Ну, в общем, потенциальное содержание должно быть и в числителе, и в знаменателе, либо грамм на метр кубический сухого газа, либо пластового, либо газа сепарации.
И это влияет на объём конденсата, оставшегося в ячейке.
Это...
Формулу лучше самим вывести.
Это объём конденсата, это не просто объём конденсата, это объём конденсата, приходящийся на метр кубический начального газа, либо газа сепарации, либо сухого, либо пластового.
Вот что это такое.
Тут не очень...
Я обращу внимание, просто только сейчас обратил внимание, но просто здесь нужно уточнить.
Здесь нужно уточнить.
Это слайд номер 143, значит, $Q$.
Вот размерность.
Но это точно, потому что я эту формулу выводил, а на слайде не проверил.
Скопировал с другой презентации, не проверил.
В общем, и потенциальное содержание, и вот это Q должны быть на метр кубический одного и того же газа, либо газа сепарации, либо пластовый газ, либо сухой.
Тогда эта формула даст нам правильную оценку коэффициента излечения конденсата при одной физической атмосфере.

\subsubsection{Цели исследования сухого и жирного газа}

\begin{center}
\includegraphics[width=\textwidth, page=144]{Брусиловский.pdf}
\end{center}

Теперь осталось два слайда.
То есть мы с вами рассмотрели газоконденсатные из пяти типов, которые мы рассматривали.
Black oil, летучая нефть, газоконденсатная система.
И вот остались сухой и жирный газ.
Но ретроградных потерь ни в жирном, ни в сухом газе, ретроградных потерь в пласте нет, поэтому тут никаких экспериментов не проводится, и по CVD, скажем, и так далее.
А вот и даже для сухого газа и контактная конденсация не проводится, поскольку выпадения жидкой фазы, углеводородной нет.
А для жирного газа можно определить экспериментально при условиях промысловой сепарации, сколько выделится конденсата.
Это также можно делать и математическим моделированием.
И главное тут определение свойств газа, для подсчета запасов, для проектирования разработки и проектообустройства направления использования газа.

\subsubsection{Основные параметры, определяемые при исследовании свойств сухого и жирного газов}

\begin{center}
\includegraphics[width=\textwidth, page=145]{Брусиловский.pdf}
\end{center}

Теперь вот основные параметры, которые определяются при исследовании сухого и жирного газа.
Тут конденсата нет.
Важно тут что?
Z-фактор важен.
Ну, раз мы Z-фактор знаем, значит мы и объемный коэффициент знаем, потому что в формуле Z-фактор.
Значит, нам нужно знать вязкость газа, но она экспериментально не определяется, она определяется по корреляциям.
То же касается жирного газа, значит, для жирного газа нам нужно знать информацию об условиях термбарических, выпадении конденсата и его количестве и свойствах.
Вот это вот экспериментально в бомбе PVT может быть сделано.
И, естественно, математическое моделирование тоже может быть сделано, но экспериментальные исследования, конечно, они первичны, потому что математической моделью при недостаточном опыте термодинамического моделирования в фазовых равновесиях можно неправильно смоделировать соотношение, при каких условиях будет конденсат выделяться.

\end{document}
