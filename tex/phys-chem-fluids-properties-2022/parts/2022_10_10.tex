\documentclass[main.tex]{subfiles}

\begin{document}

\section{Лекция 10.10.2022 (Брусиловский А.И.)}

\begin{center}
\includegraphics[width=\textwidth, page=1]{Брусиловский.pdf}
\end{center}

Видите на слайде "<PVT-свойства нефти, газа и пластовой воды">, а у вас курс называется "<Физико-химические свойства пластовых флюидов">.
Это по сути одно и то же.
Чтобы вы понимали: у нефтяников и газовиков принят термин PVT-свойства (давление, объём, температура) для физико-химических свойств флюидов.

\begin{center}
\includegraphics[width=\textwidth, page=2]{Брусиловский.pdf}
\end{center}

Я очень коротко представлюсь.
Я сотрудник Научно-Технического Центра Газпром-Нефти.
В 1974 году я закончил Московский институт нефти и газа (сейчас он называется университет) по специальности "<Прикладная математика">.
С 1977 по 1980 учился в аспирантуре.
В 1980 защитил кандидатскую диссертацию.
Работал на кафедре разработки и эксплуатации газовых и газоконденсатных месторождений института имени Губкина.
Это была ведущая кафедра, которая не уступала по своей компетенции и кругу вопросов научно-исследовательскому институту.
Занимался вопросами, связанными с моделированием фазовых превращений с применением уравнений состояния для крупнейших газоконденсатных месторождений Советского Союза (совершенно разных месторождений по месту нахождения, в том числе и тех, которые сейчас относятся к независимым государствам -- Узбекистан, Казахстан и так далее, но основное конечно это Россия).

Далее в 1987 году был образован институт проблем нефти и газа Российской Академии Наук.
Я перешёл туда в лабораторию газонефтеконденсатоотдачи.
Этот институт был образован на базе Губкинского университета.
Тогда полные сил маститые учёные руководили лабораториями, я работал в составе одной из лабораторий и в 1994 году защитил докторскую диссертацию.
Далее работал в должности главного научного сотрудника этой лаборатории.

В 2002 году я перешёл в Научно-Технический Центр СибНефти, которая в 2006 стала Газпром-Нефтью.
С тех пор я работаю в этой компании.
По сути 20 лет я занимаюсь вопросами обоснования свойств нефти и газа месторождений этой крупной компании.
То есть я имею большой теоретический багаж на основе того, что было наработано в Академии Наук и до этого в Губкинском Институте, и большой практический опыт.
Поэтому я постараюсь в лекциях рассказать вам квинтэссенцию (как бы основные моменты), которые вам будут полезны.
И это даст вам возможность лучше ориентироваться в терминологии и в том, что обычно используется на практике.
Всё это концентрируется вокруг физико-химических или PVT-свойств пластовых флюидов.

\begin{center}
\includegraphics[width=\textwidth, page=3]{Брусиловский.pdf}
\end{center}

Причём есть такая особенность: если до сравнительно недавних пор Газпром-Нефть занималась разработкой нефтяных месторождений, то в последние годы это и месторождения газоконденсатные, с нефтяной оторочкой или же нефтяные с газовой шапкой.
И даже есть месторождения чисто газоконденсатные.
Про конкретные месторождения я тоже слайд подготовил.

Итак, цель курса: понять связь физико-химических свойств пластовых флюидов месторождений нефти и газа с их компонентным составом и термобарическими условиями.

Первый тезис: любые свойства природных углеводородных систем определяются компонентным составом и термобарическими условиями.
Основой являются фундаментальные положения физической химии и, как её части, термодинамики многокомпонентных систем.
Серьёзная наука физическая химия развивается, начиная практически с 19 века, и её теоретические положения нашли практическое применение при обосновании свойств нефтяных и газовых месторождений.
Но фундаментальные основы отличаются общностью

\begin{center}
\includegraphics[width=\textwidth, page=4]{Брусиловский.pdf}
\end{center}

\begin{center}
\includegraphics[width=\textwidth, page=5]{Брусиловский.pdf}
\end{center}

\begin{center}
\includegraphics[width=\textwidth, page=6]{Брусиловский.pdf}
\end{center}

\begin{center}
\includegraphics[width=\textwidth, page=7]{Брусиловский.pdf}
\end{center}

\begin{center}
\includegraphics[width=\textwidth, page=8]{Брусиловский.pdf}
\end{center}

\begin{center}
\includegraphics[width=\textwidth, page=9]{Брусиловский.pdf}
\end{center}

\begin{center}
\includegraphics[width=\textwidth, page=10]{Брусиловский.pdf}
\end{center}

\begin{center}
\includegraphics[width=\textwidth, page=11]{Брусиловский.pdf}
\end{center}

\begin{center}
\includegraphics[width=\textwidth, page=12]{Брусиловский.pdf}
\end{center}

\begin{center}
\includegraphics[width=\textwidth, page=13]{Брусиловский.pdf}
\end{center}

\begin{center}
\includegraphics[width=\textwidth, page=14]{Брусиловский.pdf}
\end{center}

\begin{center}
\includegraphics[width=\textwidth, page=15]{Брусиловский.pdf}
\end{center}

\begin{center}
\includegraphics[width=\textwidth, page=16]{Брусиловский.pdf}
\end{center}

\begin{center}
\includegraphics[width=\textwidth, page=17]{Брусиловский.pdf}
\end{center}

\begin{center}
\includegraphics[width=\textwidth, page=18]{Брусиловский.pdf}
\end{center}

\begin{center}
\includegraphics[width=\textwidth, page=19]{Брусиловский.pdf}
\end{center}

\begin{center}
\includegraphics[width=\textwidth, page=20]{Брусиловский.pdf}
\end{center}

\begin{center}
\includegraphics[width=\textwidth, page=21]{Брусиловский.pdf}
\end{center}

\begin{center}
\includegraphics[width=\textwidth, page=22]{Брусиловский.pdf}
\end{center}

\begin{center}
\includegraphics[width=\textwidth, page=23]{Брусиловский.pdf}
\end{center}

\begin{center}
\includegraphics[width=\textwidth, page=24]{Брусиловский.pdf}
\end{center}

\begin{center}
\includegraphics[width=\textwidth, page=25]{Брусиловский.pdf}
\end{center}

\begin{center}
\includegraphics[width=\textwidth, page=26]{Брусиловский.pdf}
\end{center}

\begin{center}
\includegraphics[width=\textwidth, page=27]{Брусиловский.pdf}
\end{center}

\begin{center}
\includegraphics[width=\textwidth, page=28]{Брусиловский.pdf}
\end{center}

\begin{center}
\includegraphics[width=\textwidth, page=29]{Брусиловский.pdf}
\end{center}

\begin{center}
\includegraphics[width=\textwidth, page=30]{Брусиловский.pdf}
\end{center}

\begin{center}
\includegraphics[width=\textwidth, page=31]{Брусиловский.pdf}
\end{center}

\begin{center}
\includegraphics[width=\textwidth, page=32]{Брусиловский.pdf}
\end{center}

\begin{center}
\includegraphics[width=\textwidth, page=33]{Брусиловский.pdf}
\end{center}

\begin{center}
\includegraphics[width=\textwidth, page=34]{Брусиловский.pdf}
\end{center}

\begin{center}
\includegraphics[width=\textwidth, page=35]{Брусиловский.pdf}
\end{center}

\begin{center}
\includegraphics[width=\textwidth, page=36]{Брусиловский.pdf}
\end{center}

\begin{center}
\includegraphics[width=\textwidth, page=37]{Брусиловский.pdf}
\end{center}

\begin{center}
\includegraphics[width=\textwidth, page=38]{Брусиловский.pdf}
\end{center}

\begin{center}
\includegraphics[width=\textwidth, page=39]{Брусиловский.pdf}
\end{center}

\begin{center}
\includegraphics[width=\textwidth, page=40]{Брусиловский.pdf}
\end{center}

\begin{center}
\includegraphics[width=\textwidth, page=41]{Брусиловский.pdf}
\end{center}

\begin{center}
\includegraphics[width=\textwidth, page=42]{Брусиловский.pdf}
\end{center}

\begin{center}
\includegraphics[width=\textwidth, page=43]{Брусиловский.pdf}
\end{center}

\begin{center}
\includegraphics[width=\textwidth, page=44]{Брусиловский.pdf}
\end{center}

\begin{center}
\includegraphics[width=\textwidth, page=45]{Брусиловский.pdf}
\end{center}

\begin{center}
\includegraphics[width=\textwidth, page=46]{Брусиловский.pdf}
\end{center}

\begin{center}
\includegraphics[width=\textwidth, page=47]{Брусиловский.pdf}
\end{center}

\begin{center}
\includegraphics[width=\textwidth, page=48]{Брусиловский.pdf}
\end{center}

\begin{center}
\includegraphics[width=\textwidth, page=49]{Брусиловский.pdf}
\end{center}

\begin{center}
\includegraphics[width=\textwidth, page=50]{Брусиловский.pdf}
\end{center}

\end{document}