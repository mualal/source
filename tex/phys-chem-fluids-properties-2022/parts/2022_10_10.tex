\documentclass[main.tex]{subfiles}

\begin{document}

\section{Лекция 10.10.2022 (Брусиловский А.И.)}

\begin{center}
\includegraphics[width=\textwidth, page=1]{Брусиловский.pdf}
\end{center}

Видите на слайде "<PVT-свойства нефти, газа и пластовой воды">, а у вас курс называется "<Физико-химические свойства пластовых флюидов">.
Это по сути одно и то же.
Чтобы вы понимали: у нефтяников и газовиков принят термин PVT-свойства (давление, объём, температура) для физико-химических свойств флюидов.

\begin{center}
\includegraphics[width=\textwidth, page=2]{Брусиловский.pdf}
\end{center}

Я очень коротко представлюсь.
Я сотрудник Научно-Технического Центра Газпром-Нефти.
В 1974 году я закончил Московский институт нефти и газа (сейчас он называется университет) по специальности "<Прикладная математика">.
С 1977 по 1980 учился в аспирантуре.
В 1980 защитил кандидатскую диссертацию.
Работал на кафедре разработки и эксплуатации газовых и газоконденсатных месторождений института имени Губкина.
Это была ведущая кафедра, которая не уступала по своей компетенции и кругу вопросов научно-исследовательскому институту.
Занимался вопросами, связанными с моделированием фазовых превращений с применением уравнений состояния для крупнейших газоконденсатных месторождений Советского Союза (совершенно разных месторождений по месту нахождения, в том числе и тех, которые сейчас относятся к независимым государствам -- Узбекистан, Казахстан и так далее, но основное конечно это Россия).

Далее в 1987 году был образован институт проблем нефти и газа Российской Академии Наук.
Я перешёл туда в лабораторию газонефтеконденсатоотдачи.
Этот институт был образован на базе Губкинского университета.
Тогда полные сил маститые учёные руководили лабораториями, я работал в составе одной из лабораторий и в 1994 году защитил докторскую диссертацию.
Далее работал в должности главного научного сотрудника этой лаборатории.

В 2002 году я перешёл в Научно-Технический Центр СибНефти, которая в 2006 стала Газпром-Нефтью.
С тех пор я работаю в этой компании.
По сути 20 лет я занимаюсь вопросами обоснования свойств нефти и газа месторождений этой крупной компании.
То есть я имею большой теоретический багаж на основе того, что было наработано в Академии Наук и до этого в Губкинском Институте, и большой практический опыт.
Поэтому я постараюсь в лекциях рассказать вам квинтэссенцию (как бы основные моменты), которые вам будут полезны.
И это даст вам возможность лучше ориентироваться в терминологии и в том, что обычно используется на практике.
Всё это концентрируется вокруг физико-химических или PVT-свойств пластовых флюидов.

\begin{center}
\includegraphics[width=\textwidth, page=3]{Брусиловский.pdf}
\end{center}

Причём есть такая особенность: если до сравнительно недавних пор Газпром-Нефть занималась разработкой нефтяных месторождений, то в последние годы это и месторождения газоконденсатные, с нефтяной оторочкой или же нефтяные с газовой шапкой.
И даже есть месторождения чисто газоконденсатные.
Про конкретные месторождения я тоже слайд подготовил.

Итак, цель курса: понять связь физико-химических свойств пластовых флюидов месторождений нефти и газа с их компонентным составом и термобарическими условиями.

Первый тезис: любые свойства природных углеводородных систем определяются компонентным составом и термобарическими условиями.
Основой являются фундаментальные положения физической химии и, как её части, термодинамики многокомпонентных систем.
Серьёзная наука физическая химия развивается, начиная практически с 19 века, и её теоретические положения нашли практическое применение при обосновании свойств нефтяных и газовых месторождений.
Но фундаментальные основы (и вообще фундаментальная наука) отличаются общностью приложений.
И для физической химии много очень приложений в промышленности.
Только одно из них –- это обоснование свойств природных углеводородных систем, каковыми являются нефть и природные газы.

Теперь очень важно понять, а как же получать информацию о пластовых флюидах, которые залегают на километровых глубинах.
Километр –- это очень небольшая глубина.
Для глубокопогруженных залежей (пласты, залегающие на глубине 4 или 5 километров) давление составляет сотни бар и температура может значительно превышать 100 градусов по Цельсию.
И чем глубже, тем выше температура, тем выше давление.
Поэтому нам важно получать пробы пластовых нефтей и пластовых газов, которые бы правильно отражали их компонентный состав, а следовательно и их свойства при термобарических условиях залегания в пластах.
Это нужно для проектирования разработки.

Для промысловой подготовки нам нужно уметь прогнозировать свойства при термобарических условиях, соответствующих промысловой сепарации и транспортировке флюидов.
Причём если в пласте у нас один компонентный состав флюида (практически всегда в нефти растворён газ), то при промысловой обработке газ от нефти отделяется (часто в природных газах растворены компоненты пентаны плюс вышекипящие, которые называются стабильный конденсат, и эти пентан плюс вышекипящие выделяются из газоконденсатной смеси).
Всё это происходит при достаточно высоких давлениях и широком диапазоне температур, которые в пласте всегда положительные, при промысловой подготовке у нефтяников тоже положительные температуры, а у газовиков температура при промысловой подготовке может быть отрицательной и составлять несколько десятков градусов со знаком минус (потому что технологии выделения конденсата из газа и выделения растворённого газа из пластовой нефти совершенно отличаются именно технологией).

Я вам должен сказать, что в практике обучения нужно обязательно знать и особенности свойств пластовых нефтей, и особенности свойств природных газоконденсатных смесей, потому что современные нефтяные компании, такие как Роснефть, Газпромнефть и другие, в том числе и Лукойл и так далее (нас прежде всего интересуют наши отечественные компании) разрабатывают месторождения самые разнообразные: и нефтяные, и газоконденсатные, и нефтегазоконденсатные и так далее.
Я вам хочу сказать, что у нас долгое время (и сейчас тоже) в крупнейших нефтяных вузах есть отдельно кафедры разработки нефтяных месторождений и отдельно кафедры разработки газоконденсатных месторождений, и зачастую те, кто преподаёт и обучается на кафедре нефтяных месторождений, имеют очень слабое представление о технологии и свойствах газоконденсатных систем.
И наоборот: те, кто изучает технологии разработки газовых месторождений, имеют слабое представление о нефтях.
Это из моего опыта.
А нужно иметь обязательно представление и о том, и о другом, потому что сейчас, кстати говоря, условия залегания сильно отличаются и очень часто встаёт вопрос о рациональной разработке двухфазных залежей.
То есть, это нефтяные залежи с газовой шапкой, и нужно понимать и физику, и как готовить добываемую продукцию, и как добиться повышения нефтеотдачи, газоотдачи, конденсатоотдачи.
В общем, это очень большие проблемы.

Проектирование осуществляется на базе комплексного использования результатов промысловых исследований, лабораторных исследований и математического моделирования технологических процессов.
Сейчас для всего этого есть стандартные программные комплексы.
И эти стандартные комплексы (кем бы и где бы они ни создавались -- за границей или у нас) при моделировании фазового состояния используют уравнение состояния и основы термодинамики многокомпонентных систем.
Поэтому тот очень краткий обзор, который я вам прочитаю по уравнениям состояния, включает в себя постановку основных задач (математическую постановку для моделирования фазового состояния).
Это во всех программных комплексах используется.
И это, так называемые, теоретические основы, их важно понимать.
Повторю ещё раз, что при моделировании фазового состояния используется основа термодинамики многокомпонентных систем в совокупности с уравнениями состояния, а также результаты лабораторных исследований.

А второе направление -- это применение так называемых корреляций для оценки PVT-свойств нефти, газа и пластовой воды при проведении технологических расчетов.
Когда мы зачастую не знаем компонентный состав, когда мы имеем дело с теплофизическими свойствами (это прежде всего динамическая вязкость), то для оценки PVT-свойств используются корреляции.

То есть в инженерной практике комплексно используется и моделирование с применением серьезной теории, алгоритмов и уравнений состояния, а также корреляций, которые основаны на обработке большого количества экспериментальных данных, полученных на промысле и в лабораториях.

\begin{center}
\includegraphics[width=\textwidth, page=4]{Брусиловский.pdf}
\end{center}

Где же мы используем PVT-свойства?

Здесь нужно знать тонкости, о которых я расскажу.

PVT-свойства используются:

1) для подсчета запасов нефти, газа и конденсата в месторождениях;

2) для инженерных расчетов при проектировании (начальная стадия, когда создаётся проект разработки) и мониторинге (мониторим выполнение проекта) разработки месторождений.
При мониторинге мониторим, насколько проектные данные (прогнозные) совпадают с фактическими.
Это еще зависит и от того, насколько надежными были исходные данные при проектировании разработки, при подсчете запасов.
Потому что зачастую данные о PVT-свойствах пластовых нефтей, газов, конденсатов, полученные в период разведки, затем корректируются, так как далеко не всегда сразу получают так называемые промысловые представительные пробы пластовых флюидов в период разведки.
И потом уже фактические данные разработки говорят о том, что исходные данные, принятые на основе первичных исследований, не вполне правильные.
Это происходит из-за сложности получения так называемых представительных проб флюидов и пластовых нефтей, и газоконденсатных систем.
И я расскажу, какие же возникают трудности для того, чтобы вы ориентировались. 

3) для расчёта потоков флюидов в скважинах и трубопроводах.
Требуются знания PVT-свойств, так как даже если поток углеводородного флюида вблизи забоя однофазный, то вследствие падения давления в скважине по мере движения флюида к поверхности происходит выделение газа, который растворён в пластовых условиях в пластовой нефти.
Соответственно, поток из однофазного (в данном случае не говорю о том, что всегда есть вода, так как считаю её инертным флюидом) превращается в двухфазный.
То есть углеводородная система из однофазной превращается в двухфазную (когда есть и газовая фаза, и жидкая углеводородная фаза), а здесь для проведения расчётов требуется уже серьёзная гидродинамика.
Здесь возможны разные режимы течения и в скважинах, и в трубопроводах.
И этому уделялось тоже очень большое внимание.
Это серьёзная гидродинамика и серьёзные эксперименты, которые на протяжении десятков лет проводятся, и за рубежом, и в нашей стране.

4) для расчёта наземного оборудования.
Тоже важно знать PVT-свойства, так как они касаются технологических схем обустройства месторождений, то есть это и сепарация, и различные аппараты для выделения газа из нефти, конденсата из природного газа и так далее.
Отдельные компоненты выделяются обычно на заводах, на газоперерабатывающих заводах, газохимических комплексах.
А вот первичное выделение на поверхности тоже требует понимания фазовых превращений и умения их прогнозировать.
Повторяю, что для этого существуют современные универсальные программные комплексы, но нужно понимать, что в их основе, какая теория, на чём основаны расчёты, и понимать физическую сущность получаемых результатов.
Обо всём этом я постараюсь вам рассказать, и это является основной моей задачей.
Такой широкий обзор обусловлен моей многолетней практикой как в теоретическом плане, так и при решении практических задач, особенно в процессе последних 20 лет работы в Научно-Техническом Центре "<Газпромнефть">.

\subsection{Классификация, поведение и исследование пластовых углеводородных систем}

\begin{center}
\includegraphics[width=\textwidth, page=5]{Брусиловский.pdf}
\end{center}

\begin{center}
\includegraphics[width=\textwidth, page=6]{Брусиловский.pdf}
\end{center}

Я буду пользоваться мной написанной монографией, учебники разные, вот я просто вам частью зачитаю какие-то вещи, которые позволят вам лучше ориентироваться и тем более, что формулировки были продуманы в процессе написания.
Те, кто заинтересуется первоисточниками, я потом их дам.

Нужно чётко понимать, что нефть и природные газы -- это смеси углеводородных и неуглеводородных веществ, залегающих в недрах Земли.
Входящие в смесь индивидуальные вещества называются её компонентами, а компонентный состав пластового флюида, ещё раз вам повторяю, и температура залежи определяют тип смеси (нефть или природный газ).

Что такое нефть в общем случае?
Это природная многокомпонентная система, находящаяся в пластовых условиях в жидком состоянии (т.е. если вы открыли залежь и пластовый углеводородный флюид находится в жидком состоянии, то вы открыли нефтяную залежь).
Эта углеводородная смесь (пластовая нефть) состоит из соединений парафиновой, нафтеновой и ароматической групп.
Но в то же время в неё входят и неуглеводородные соединения, которые представлены на слайде.
В пластовых нефтях есть азот, диоксид углерода, сероводород, гелий.
Каждый компонент привносит особенности в поведение пластовых флюидов и в направление переработки и использования компонентов пластового флюида.

Про алканы вы знаете из химии, но я кратко повторю.
Алканы имеют химическую формулу $C_nH_{2n+2}$ (состоят из углерода и водорода).
Входят в состав также циклоалканы, их называют также нафтенами ($C_nH_{2n}$), арены (ароматика) $C_nH_{2n-6}$.
Кроме углеводородов в состав нефтей входят высокомолекулярные вещества асфальтены.
Их молекулярная масса может достигать 10000.
В этих соединениях не только углеводороды, но и кислород, и другие компоненты.
Другими словами, это очень высокомолекулярные соединения, которые могут быть в небольшом количестве, но они, выделяясь в твердую фазу, могут создавать большие проблемы с точки зрения разработки и эксплуатации месторождений, особенно в призабойной зоне и при течении в стволе скважины.
Есть такие термобарические условия, при которых могут выпадать в твердую фазу асфальтены и парафины.
Парафины -- это углеводороды с числом атомов углерода от 17 до 64.
И они тоже в твердую фазу могут выпадать и при пластовых условиях, и в стволе скважины, и требуются специальные меры и реагенты для удаления этой твердой фазы и уменьшения её отрицательного влияния на фильтрацию флюидов в пористой среде и течение в стволе скважины.
Неуглеводородные компоненты, которые я назвал уже, могут тоже очень сильно влиять на товарные свойства флюидов, которые мы добываем.

Я должен вам сказать, что существуют специальные издания и физико-химические лаборатории и так далее, которые специализируются на том, как нам эксплуатировать нефть, и как нам эксплуатировать залежи с высоким содержанием сероводорода, диоксида углерода и других нежелательных компонентов.
Значит, что касается терминологии...
Я сейчас буду говорить о нефти.
В инженерной практике используют термины:

1) "<пластовая нефть">, которую я уже называл.
Это залегающая в недрах Земли смесь природных углеводородных систем и с другими компонентами, которые находятся в жидком агрегатном состоянии;

2) "<промысловая нефть"> -- это безводная жидкая составляющая продукции добывающих скважин в трубопроводах и других объектах промыслового обустройства нефтяных месторождений;

3) "<сырая нефть"> -- это добываемая из скважин смесь промысловой нефти и попутной воды.
То есть, "<сырая нефть">, когда вы услышите такой термин, это смесь промысловой нефти и попутной воды;

4) "<товарная нефть"> -- это нефть, подготовленная к поставке потребителям и соответствующая требованию действующих нормативных технических документов.
Именно "<товарная нефть"> является продукцией нефтяного промысла, поступающая в систему Транснефти.
Транснефть -- это объединение, которое занимается транспортировкой подготовленной нефти до потребителей.

\begin{center}
\includegraphics[width=\textwidth, page=7]{Брусиловский.pdf}
\end{center}

Я рассказывал вам про алканы.
На этом слайде представлены примеры алканов.
Видите, метан -- $CH_4$, этан -- $C_2H_6$ и так далее.
Общая формула -- $C_nH_{2n+2}$.
И тут есть такая особенность.
В программных комплексах очень часто, да и в литературе бывает, $CH_4$ пишут просто метан -- $C_1$, этан -- $C_2$ и так далее.
То есть ограничиваются числом атомов углерода в молекуле углеводорода.
Такая аббревиатура часто встречается в программных комплексах, чтобы вы понимали, что такое $C_1$, $C_2$, $C_3$, $C_4$ и так далее.

Теперь, в стандартных условиях. Каковы стандартные условия?
Это одна физическая атмосфера и 20 градусов Цельсия, ну или 293,15 градусов Кельвина.
Значит, 20 градусов Цельсия -- это в нефтяной и газовой промышленности у нас в стране такое стандартное условие.
Метан, этан, пропан, бутаны в стандартных условиях как отдельные компоненты находятся в газовом агрегатном состоянии.
Начиная с пентанов (могут быть изопентан или нормальный пентан; изо -- это изомер) в природных смесях при стандартных условиях являются жидкостью (есть еще неопентан, но в природных смесях он практически отсутствует).
А начиная с углеводородов, в которых число атомов углерода 17 и выше, они находятся при стандартных условиях в твердом состоянии (как отдельные компоненты).
Но когда эти компоненты находятся в смеси, то в зависимости от компонентного состава в пластовых условиях они либо в газовом, либо в жидком агрегатном состоянии.
Вот эти смеси.
То есть одно дело индивидуальные вещества, а другое дело их смеси.
Это и есть то замечание, которое я хотел отметить.

\begin{center}
\includegraphics[width=\textwidth, page=8]{Брусиловский.pdf}
\end{center}

Я очень тщательно готовил презентацию для вас, чтобы основные понятия отразить обязательно, и чтобы вы могли, те кто не знает, чтобы вы могли их как-то запомнить.

В физической химии в расчётах фазового равновесия используют мольный компонентный состав пластовых флюидов.
В учебниках вы можете увидеть, что наряду с мольным есть ещё объёмный состав, массовый состав.
Имейте в виду, что в науке, в физической химии, в термодинамике многокомпонентных систем и в программных комплексах для расчёта фазовых превращений нам нужно знать мольный компонентный состав смеси.
Не объёмный, не массовый, а именно мольный.
Так вот, по определению отношение числа молей $i$-ого компонента к общему числу молей смеси, это есть мольная доля $i$-ого компонента.
И в примечании я написал для вас, потому что не все знают, не всех знали, но теперь будете знать, что такое моль.
Моль это сокращённо, а вообще-то полный термин это грамм-тире-моль вещества.
Количество вещества в граммах численно равное относительной молекулярной массе этого вещества.
И вот это молекулярная масса показывает, во сколько раз масса молекул данного вещества больше 1/12 массы атома углерода.
И, например, вот у нас есть метан, $CH_4$.
Молекулярная масса углерода 12, молекулярная масса водорода, атомная масса, значит это $H$, это единичка.
И значит молекулярная масса метана 16.
Молекулярная масса 16, тогда моль метана это 16 грамм.
Для этана $C_2H_6$ у нас получается молекулярная масса 12 на 2 плюс 6 это 30.
И значит моль равен 30 грамм.
Вот теперь, какова связь между мольными и массовыми долями $i$-ого компонента смеси?
Значит у нас получается такая связь, что если мы знаем массовую долю компонента смеси, то поделив массовую долю $i$-ого компонента на его молекулярную массу (это представлено в числителе $\omega_i/M_i$, и поделив на сумму таких частных для всех компонентов смеси, то мы с вами получаем мольную долю компонента.
Почему?
Потому что это из определения мольной доли.
Это если вам известна массовая доля компонента в смеси, то нужно знать молекулярную массу всех компонентов и мы можем рассчитать мольную долю компонента в смеси.
Или наоборот, если мы знаем мольную долю компонента в смеси, то мы можем рассчитать по формуле справа внизу массовую долю компонента в смеси.
В любом случае нам нужно знать молекулярную массу любого компонента смеси.
Это для справки.

\begin{center}
\includegraphics[width=\textwidth, page=9]{Брусиловский.pdf}
\end{center}

Теперь, вот ещё раз, какая используется терминология, и вы лучше будете понимать.
Пластовые нефти -- природная многокомпонентная система, находящаяся в пластовых условиях в жидком состоянии.
Это основа.
То, что в жидком состоянии в пластовых условиях изначально -- это значит смесь наша нефть.
И состоят они преимущественно из углеводородных соединений парафиновой и нафтеновой ароматической группы.
А природные газы -- это смеси, которые в пластовых условиях находятся в газовом агрегатном состоянии.
Они могут находиться в виде отдельных скоплений -- это газовые залежи, или в виде газовых шапок нефтегазовых залежей -- это свободный газ, газовая шапка -- свободный газ над нефтяной оторочкой.
Либо это свободные газы.
А также природные газы могут находиться в растворённом состоянии -- в нефти или в пластовой воде.
Практически всегда в пластовых нефтях есть растворённый газ в том или ином количестве.
И также отдельные компоненты, такие как метан, в значительной меньшей степени этан, пропан,
и диоксид углерода, и сероводород -- они хорошо растворяются в пластовой воде.

\begin{center}
\includegraphics[width=\textwidth, page=11]{Брусиловский.pdf}
\end{center}

Теперь я хотел бы привести примеры компонентного состава нефтяных месторождений и газовых месторождений.
На этом слайде у нас примеры составов для нефтяных месторождений.
Для чего?
Для того, чтобы вы понимали, что очень разнообразные составы есть по концентрации (по содержанию отдельных компонентов).
И именно компонентный состав, по нему можно сказать, какой же растворён газ в нефтяном месторождении, прямо по компонентному составу пластовой нефти.
Ну, нужен определённый опыт, но это всё нарабатывается.
И каково же газосодержание, то есть количество газа растворённого можно оценить, условно говоря, небольшое, средненькое или очень высокое.
По компонентному составу пластового газа даже если вы ещё не знаете результаты лабораторных исследований, а просто перед вами компонентный состав.
Так вот, по нефтяным месторождениям.
Здесь только крупные месторождения.

Ромашкинское месторождение -- это уникальное месторождение в Татарии, которое до сих пор разрабатывается, а начали разрабатывать в 1945 году сразу после окончания Великой Отечественной войны.
И это месторождение уникальное по запасам (там миллиарды тонн нефти) и было основой по добыче нефти в Советском Союзе после исчерпания запасов нефти в Закавказе.
Во время войны очень большое значение имела добыча нефти в Баку, в Чечне, в Ингушетии и так далее.
Ну вот в Закавказском регионе колоссально.
А после войны Ромашкинское месторождение стало основным источником добычи нефти.
Значит, мы видим с вами, что по приведенному составу у нас метана совсем немного, а очень много в растворенном газе этана, пропана и бутана.

Вы видите, что просто большущая концентрация в растворенном газе этана, пропана, значит, этот газ –- это очень ценное сырье для переработки.
А метана, то есть это нельзя назвать, что метан там основной компонент в растворенном газе.
Нет, это многокомпонентная система, растворенный газ, поэтому в Татарии используют, ну, то есть там и газопереработка, специальный комплекс есть не только для нефтепереработки, но и для газопереработки.
Ну, вот это теперь.
Вот растворенный газ у нас включает до бутана включительно, азот и диоксид углерода.
В Ромашкинском месторождении сероводорода практически нет (по тому составу, который приведен).
Но, значит, и немного диоксида углерода.
Но вообще надо сказать, что в месторождениях Татарии и Башкирии есть и сероводород, и сера.
И вот что я вам чуть подальше, чуть попозже вам могу сказать.
Но, так сказать, вот очень большой, большой бич для нефтяников, если в составе нефти есть сера, вот чуть подальше я вам расскажу, что существует классификация по содержанию серы и других компонентов, вот чуть подальше.
А это просто примеры, примеры, живые примеры компонентного состава пластовых нефтей.

Самотлорское месторождение (Западно-Сибирское) -- это гигант, который стал основным для добычи нефти после того, как были открыто.
Оно было открыто в начале 60-х годов.
Вот эти гигантские месторождения Западной Сибири, они подхватили эстафету от месторождений Татарии.
Это уже многопластовые месторождения и здесь в качестве примера показано, что для различных пластов могут сильно отличаться свойства пластовой нефти.
Что мы видим по пласту A$_4$ Самотлорского месторождения?
Значит, в основном в растворенном газе у нас метан.
Мало этана, мало пропана, бутанов чуть больше и небольшое количество азота и диоксида углерода.
Азот всегда присутствует в составе растворенных газов, практически всегда.
И есть практически всегда диоксид углерода.
Но слава богу, что нет сероводорода.
Сероводород очень неприятный компонент, который нужно извлекать, он влияет на коррозию и так далее.
Я чуть дальше об этом скажу.
Значит, а в пласте B$_8$ Самотлорского месторождения мы видим, что у нас более жирный газ, в составе растворенного газа уже меньше метана, больше этана, значительно больше пропана
и существенно больше бутанов, чем в пласте А$_4$.

Талинское месторождение -- это месторождение, которое разрабатывает Роснефть в Западной Сибири.
И, значит, видно, что в растворенном газе достаточно много этана, пропана.
И видно, что еще?
Газосодержание высокое на этом месторождении.
То есть количество растворенного газа более высокое, чем, скажем, в пласте В$_8$ Самотлорского месторождения.
И оно больше, чем в Самотлорском месторождении в исходной пластовой нефти.
Это можно как вообще оценить?
Это просто суммируется так называемый газовый компонент.
То есть, те, что в стандартных условиях, в газовом агрегатном состоянии.
Метан, этан, пропан, бутаны.
Ну вот в Талинском месторождении азота и диоксида углерода нет, и сероводорода нет.
Но если бы были, они бы тоже суммировались с этими углеводородами, от метана до бутанов, чтобы понять, какова доля так называемых газовых компонентов.

И вот почему у нас в этой таблице $С_{5+}$, это пентаны плюс вышекипящие?
Потому что это группа компонентов, смесь которых в стандартных пластовых условиях является жидкостью.
Обычно у нефтяников принято состав представлять до $С_{7+}$.
Но у газовиков важное значение имеет $С_{5+}$ для того, чтобы вы видели для сравнения вот этой таблицы по составу пластовых нефтей и по компонентному составу пластовых природных газов как в следующей табличке будет.
Вот поэтому здесь представлены газовые компоненты плюс компонент обобщенный $С_{5+}$ (в сумме единица получается состав).
Ну и Приобское месторождение гигантское, которое основное по запасам, которые сейчас в Западной Сибири разрабатываются.
Оно разрабатывается и Роснефтью, и Газпромнефтью, лицензионные участки.
И вот мы видим, что газосодержание в пласте AC$_{10}$ не очень высокое, это сразу можно определить по концентрации метана и невысокой концентрации этана в составе пластовой нефти.

Дальше месторождение, которое разрабатывает Газпромнефть.
Сугмутское месторождение.
Это все многопластовые месторождения, в таблице указаны конкретные пласты.
Ну вот 30\% мольных метана в составе пластовой нефти и достаточно жирный газ, в котором есть и этаны, и пропаны, бутаны.
Ну такой вот состав, невысокое, но и немаленькое газосодержание.
Газосодержание там, как я помню, я всеми этими месторождениями занимался, но газосодержание на Сугмуте в пласте БС9-2 больше, чем в пласте АС10 Приобского месторождения.
Вот если вы сравните, просуммируете концентрации компонентов от метана до бутанов, добавите к ним азот, $CO_2$ и $H_2S$ и просто даже если вы посмотрите на содержание $С_{5+}$, так тоже можно не суммировать, потому что в сумме 100\%, значит, просто посмотрите, что на Приобке в пласте АС$_{10}$ у нас $С_{5+}$ существенно мольная доля выше, чем в пласте БС9-2 Сугмута, и значит газосодержание больше.

И теперь многопластовое Ярайнерское месторождение.
Значит, там по самый верхний пласт, пласт ПК.
И вот что интересно.
Посмотрите на растворённый газ.
В пластах ПК у нас растворён очень сухой газ, состоящий преимущественно из метана.
То есть буквально небольшие очень концентрации этана, пропана, бутанов, неуглеводородных компонентов и подавляющая концентрация –- это метан.
Это так называемый сухой газ.
Это характерно для пластов ПК.
Они неглубокие -- полтора километра или даже чуть меньше.
Более глубокопогруженные залежи характеризуются меньшим содержанием метана, большим содержанием его гомологов.
И вот этот пласт БВ2 характеризуется меньшим вообще газосодержанием.
Опять же, давайте посмотрим $С_{5+}$, 57,6 в пласте БВ2 и 46,8 в пласте ПК-20 Ярайнерского месторождения.
Но всё остальное – это газовые компоненты.
Значит, понятно, что растворённого газа в пласте БВ2 больше, чем в ПК-20.
Не в абсолютных величинах, не в запасах, потому что запасы зависят от запасов нефти, от объёма нефтенасыщенного порового пространства.
А именно, если мы сравниваем компонентный состав, мы видим, что у нас в ПК сухой газ, почти не содержащий гомологов метана, в пласте БВ2 уже такой типичный нефтяной газ, в котором содержится не только метан, но и гомологи его, а Юрские залежи характеризуются ещё более глубоко погруженные, ЕВ1 на Ярайнере характеризуются жирным составом растворённого газа.
Посмотрите, сколько у нас этана, пропана, бутанов.
И значительно меньшей долей С5+ выше, а значит, значительно большей долей газообразных компонентов, и, следовательно, значительно большим газосодержанием, или ещё аналог - это газонасыщенность, этой пластовой нефти, Юрской залежи, и Ярайнерского месторождения.
Вообще, для Юрских залежей характерно, но не всегда, высокое газосодержание, и, что практически всегда, то, что в растворённом газе у нас много гомологов метана.
И вот последнее, тоже Юрская залежь, ещё одного месторождения Газпромнефти, и вот здесь мы видим, сразу же, что газосодержание в этом пласте невысокое, потому что С5+ больше 70%.
А в газе растворённом у нас, значит, наряду с метаном, есть и гомологи, и этан, и пропан, и это газ, так называемый, такой полужирный.
То есть этот газ, кстати, вот если сравнить с газом Ромашкинского месторождения, вы видите там метана в газе Крапивинского Юрская залежи, и Ромашкинского, первая строчка, практически равное количество.
Но в Ромашкинском месторождении значительно больше в растворённом газе этана, пропана, значит, для газопереработки, для переработки этого газа Ромашкинское месторождение более привлекательное.
Теперь, ещё на что хотел обратить внимание, значит, я пропустил, Тенгизское месторождение, Тенгизское месторождение открыли советские геологи, это уникальное нефтяное месторождение, которое находится в Казахстане, но открыли его в советское время, и там, обратите внимание, 16\% сероводорода в растворённом газе, то есть его много, и это очень большая проблема, всё оборудование в антикоррозионном исполнении, абсолютно всё, значит, и когда там загорелась скважина, то есть на разведочной скважине там жуткий, совершенно жуткий был пожар, который не могли потушить около года, и даже подступиться не могли именно из-за того, что это очень агрессивный газ растворённый, содержащий большое количество сероводорода.
То есть в итоге, конечно, потушили и так далее, но огромное количество нефти и газа природного было сожжено, и было очень сложно.
Это основное месторождение было в Казахстане, сейчас, может быть, там другие есть, и оно уникальное не только составом, оно характерно тем, что там аномально высокое пластовое давление свыше 800 бар при глубине 3 километра, насколько я помню, а давление свыше 800 бар, очень высокая температура, значит…
Ну, так вот, с одной стороны, высокое давление, это большая разница между начальным пластовым давлением и давлением насыщения, которое порядка 30 мегапаскалей, то есть 50 мегапаскалей разница между начальным пластовым давлением и давлением насыщения.

Это значит, что можно долгое время разрабатывать месторождение на упругом режиме, не поддерживая давление, казалось бы, и нефть будет в пласте в однофазном состоянии и можно достаточно просто рассчитать, какая нефтеотдача может быть достигнута на упругом режиме.
Это, кстати, делается без помощи сложных гидродинамических расчетов, а зная изотермический коэффициент сжимаемости или объемную упругость пластовой нефти, которая получается на основе лабораторных исследований, также может быть оценена с применением уравнений состояния или каких-то корреляций.
Ну, конечно, самое лучшее -- это лабораторные исследования.
Ну и уравнение состояния даёт нам возможность оценить изотермический коэффициент сжимаемости.
Я это делал, когда занимался месторождениями при Каспии, еще работая в университете имени Губкина.
Это было давно -- 40 лет назад.
Было очень интересно.
Всё было необычно абсолютно.
И, кстати говоря, тогда были проблемы с исследованиями пластовой нефти Тенгизского месторождения из-за компонентного состава пластовой нефти.
Если сейчас всё оборудование в антикоррозионном исполнении, то для высоких давлений до 1000 атмосфер и для высоких температур тогда такого оборудования не было.
И только потом уже поступило импортное оборудование и так далее, и так далее.
Были большие сложности.
Было важно научиться оценивать с применением уравнений состояния, потому что не всегда имелись экспериментальные данные из-за сложного компонентного состава пластовой нефти и наличия в ней такого количества сероводорода.
Вот это что касается нефтяных месторождений.
Каждое из них уникально по компонентному составу.
Каждое требует тщательного подхода для прогнозирования разработки и промысловой подготовки.

\begin{center}
\includegraphics[width=\textwidth, page=10]{Брусиловский.pdf}
\end{center}

Мы с вами про нефть послушали.
Теперь природные газы.

Для природных газов, для Сеноманских залежей характерно, что это очень сухой газ в основе которого метан.
Вот вы видите состав Медвежьего месторождения.
Это все уникальные месторождения с колоссальными запасами.
Медвежье было открыто в 1974 году.
Я как раз закончил институт и первая моя командировка была в Надым.
Только-только начали освоить месторождение Медвежье.
Я туда прилетел в феврале месяце, температура воздуха -50 градусов по Цельсию.
И, в общем, самая главная мысль была -- из автобуса сразу в какой-нибудь подъезд.
Невозможно, холодно было.
Никаких ни гостиниц, ни домов толковых тогда ещё не было (в 1975 году).
И, в общем, это на всю жизнь у меня осталось.
И при этом люди там работали, разрабатывали месторождения, геологи, эксплуатационщики.
Просто в совершенно неимоверных климатических условиях.
И там, что интересно, что я еще запомнил.
То, что никогда не заглушали двигатели у грузовиков, потому что потом можно было не завести.
Круглые сутки они работали.
Вот это вот Медвежье месторождение, Сеноманская залежь.
На Уренгое, который недалеко, в том же регионе, тоже Сеноманская залежь, практически состоящая из метана.
Небольшое количество гомологов и отсутствие конденсата (пентанов плюс вышекипящих).
Там на Уренгое написано 0,01, но это ничто.
То есть сухой газ.
А уже в более глубоких Валанжинских залежах появляется конденсат.
Причем в приличном очень количестве.
Это на том же Уренгойском многопластовом месторождении, которое до сих пор разрабатывается и которое долгие десятилетия являлось основой добычи газа в Советском Союзе и в 90-х годах в России.
Вот там, помимо газа, в Валанжинских залежах добывают еще и конденсат.
И вы видите, что достаточно этот газ, газоконденсат на этот залежи.
И там приличное количество этана и немало пропана.
Но главное, что конденсат.
Конденсат, вот его БУ-14, 5,62\% мольных.
$C_5$ плюс выше, то есть стабильного конденсата.
Это высокое значение.
Дальше я вам скажу.
Это вот по содержанию $C_5$ плюс выше.
Я могу и сейчас сказать вам.
Просто вы потом вспомните, вам будет легче.
Газоконденсатные залежи по содержанию конденсата в пластовом газе классифицируются так: если меньше 25 грамм на метр кубический газа, то это низкоконденсатные залежи и для них часто даже не проводится экспериментальных исследований по свойствам пластового газа, по выделению конденсата.
Если у нас больше 25, но меньше 100 граммов на метр кубический газа, то это среднеконденсатные залежи по содержанию конденсата.
От 100 до 500 граммов на метр кубический газа -- это высококонденсатные залежи.
А свыше 500 грамм на метр кубический -- это уникальноконденсатные залежи.
И видите, вот это терминология грамм на метр кубический.
Это я вам покажу формулу, когда рассчитывают состав пластового газа и определяют потенциальное содержание в нём $C_5$ плюс выше.
И я более подробно об этом вам расскажу дальше.
Есть материал (слайды), чтобы вы понимали, как получается состав пластового газа.
Его рассчитывают на основе проведенных лабораторных исследований.
Так вот, Ямбургское месторождение тоже, значит, оно высококонденсатное по содержанию.
Вуктыльское месторождение, Коми-Республика, оно практически уже доразработано.
Разрабатывают на уже остаточные запасы.
Оно чем вообще уникально?
По запасам.
Там, вот видите, в начальном составе 6,4\% $С_5$ плюс выше.
Начальные запасы конденсата, если переходить уже к запасам, 300 миллионов тонн пентанов плюс вышекипящих.
То есть это очень-очень богатый конденсатом.
И по концентрации это высококонденсатное месторождение.
Не уникальноконденсатное, но высококонденсатное.
Там что-то 360, по-моему, граммов на метр кубический конденсата $C_5$ плюс.
Теперь следующее Оренбургское, Астраханское и Карачаганакское -- это месторождения Прикаспия.
Во всех месторождениях Прикаспия есть сероводород, диоксид углерода.
И всё оборудование там в антикоррозионном исполнении.
И что важно, значит, ещё Оренбургское месторождение, которым я очень плотно занимался, и внедрение моей кандидатской диссертации было именно для расчёта для Оренбургского месторождения, для прогнозирования состава добываемого газа, ну, в общем, много чего я там моделировал.
Это был первый опыт применения уравнений состояния в нашей стране для месторождений.
Не для транспорта, а для месторождений.
Также и Астраханское месторождение тоже.
Значит, вот ещё раз хочу сказать, всё оборудование в Прикаспийских месторождениях в антикоррозионном исполнении –- это очень дорогое удовольствие, очень.
Но учитывая, что все эти месторождения на Оренбургском свыше 1 триллиона кубометров запаса были, оно сейчас практически разработано.
Начальное пластовое давление 200 бар.
Значит, и сейчас уже, так сказать, на стадии, ну, уже окончательного, ну, забрасывания практически разработки.
То есть, это… Чем они, эти ценны месторождения, кроме запасов и так далее, они все в европейской части.
Если Медвежье, Уренгой – это север западной Сибири, это Ямбург, ещё севернее.
Это очень тяжёлые условия.
То Оренбургское, Астраханское, Карачаганакское –- это, значит, европейская часть,
это совершенно другой климат.
Сейчас Карачаганак относится к Казахстану, но в своё время было открыто советскими геологами,
и мы, значит, много занимались этим месторождением, и по изучениям, изучением, изучением этого месторождения.
Так вот, Оренбургское там есть, и конденсат там есть.
Первый завод по производству этана; там был построен первый завод по производству гелия.
Здесь гелий не указан, но, значит, именно на Оренбургском месторождении был построен гелиевый завод.
Значит, если у нас в составе пластового газа концентрация, вот именно вот свободного газа, нерастворённого, свободного газа больше, чем 0,05, то рассчитываются запасы, рассчитываются запасы этого газа, и был построен гелиевый завод.
Гелий является основой для...
Ну, в общем, это стратегический компонент.
Вот сейчас, кстати, гелий есть в составе Ковыккинского месторождения, газ которого в Китай направляется.
И вот я когда смотрел состав Ковыккинского месторождения, там были промышленные концентрации гелия.
Вот любопытно, извлекают его, прежде чем отправлять за границу или нет.
Ну, то есть раньше с этим строго всё было.
Теперь Астраханское месторождение уникальное по содержанию диоксида углерода и сероводорода.
То есть это просто уникальные запасы серы, и оно было там, помимо того, что там аномально высокое пластовое давление, высокая пластовая температура, то есть глубина 4 километра, пластовое давление свыше 60 мегапаскалей, то есть аномально высокое пластовое давление.
И оно предполагалось, его ввели в разработку во второй половине 80-х годов, и предполагалось, что оно будет основой для добычи серы в странах экономической взаимопомощи, ну, СЭВы.
А потом оказалось, когда всё было разрушено, все связи, в чём проблема, в чём большая проблема?
В том, что добывается газ, в этом газе огромная доля сероводорода, и, значит, это очень...
Это тяжёлый газ, он тяжелее воздуха, кстати говоря, также, как и газ Оренбургского месторождения, он тоже тяжелее воздуха.
Обычно газы легче воздуха, и, например, можно проводить исследования в атмосферу.
В данном случае, в Прикаспии, исключено абсолютно исследование в атмосферу, и поскольку газ тяжелее воздуха, нужно абсолютно герметичное оборудование.
Это очень-очень большая проблема.
И скважины, и трубопроводы, и просто были, ну, так сказать...
Были случаи в первое время, из-за, опять же, неопытности, когда выпускали газ, он скапливался в низинах, и люди погибали.
Потому что, ещё раз хочу сказать, газ, который тяжелее воздуха, это большая проблема.
Но, слава богу, что углеводородные газы, они, так сказать, легче воздуха.

Теперь, я дальше покажу вам формулу, как определить газ по компонентному составу.
Газ тяжелее воздуха или легче.
Но, вот, если у нас нет измерений, вот просто компонентный состав.
Мы можем, я вам сразу скажу, мы можем рассчитать молекулярную массу этого газа, сумму молекулярных масс и компонентов, на мольную долю компонента, да?
Поделим на молекулярную массу воздуха, это 28,96.
И это отношение молекулярных масс газа и воздуха, оно равно отношению их плотностей.
Значит, вот, и вообще, плотность, есть понятие, относительная плотность газа по воздуху.
Это вот отношение молекулярной массы газа к молекулярной массе воздуха.
Если она меньше единицы, значит, газ не будет скапливаться.
И он, ну, легче, легче воздуха.
А если больше единицы, значит, он будет скапливаться, и нельзя проводить исследования, нельзя проводить исследования в атмосферу, потому что газ никуда улетучиваться не будет.
Значит, вот такая вещь.
И это на Астраханском месторождении, пожалуй, единственное месторождение в своём роде, когда добыча газа, она ограничивается добываемым количеством серы, сероводорода.
Из сероводорода выделяется сера, и, значит, а куда её, куда её девать?
Значит, вот, поэтому добыча газа, значит, она определяется объёмом реализуемой серы.
Вот такая интересная вещь.
Ну, и последнее, Карачаганагское месторождение.
Значит, это уникальное, но оно сейчас в Казахстане, неважно.
Я просто хочу, чтобы вы понимали, какое же многообразие природа приготовила нам.
Значит толщина в этом месторождении больше тысячи, ну, почти километр газоносной части.
Там есть внизу, ну, местами, нефтяная оторочка.
И когда мы изучали это месторождение, то столкнулись с тем, это было впервые, столкнулись, это, ну, вот, да, мы изучали вот, начиная с середины 80-х годов.
Это период очень интенсивных исследований в вновь открытых месторождениях, когда и при Каспии начали разрабатывать.
Так вот, посмотрите, пожалуйста, значит, наверху, в интервал перфорации 41-50 метров, это у нас концентрация метана выше, чем у подошвы.
Значит, но давайте посмотрим на концентрацию конденсата, $C_5$ плюс выше.
Если вот наверху у нас, это уникальная конденсатная залежь, то есть там свыше или около 500 граммов на метр кубический у кровли, у кровли, вот как раз кровли, 40-50, там где-то 41-50, и значительно более высокой концентрации $C_5$ плюс километром ниже у подошвы залежи.
Значит, запасы у нас далеко не однородны, и подсчёт запасов делается с учётом изменения компонентного состава по глубине залежи.
И это всё необходимо учитывать при разработке залежей, при подсчёте запасов, то есть если мы сталкиваемся с залежами с большой толщиной продуктивного пласта.
Так, это вот обзор по месторождениям я завершил.

\begin{center}
\includegraphics[width=\textwidth, page=12]{Брусиловский.pdf}
\end{center}

Ещё раз я напоминаю, какие типы месторождений (залежей) существуют.
Это информация из всяких регламентов по подсчёту запасов, то есть это официальная информация.
Ну вы видите, что нефтяное только нефть, насыщенная в различной степени газом; газонефтяное - это двухфазная залежь уже, нефть и газ.
Так вот, нефтяная часть залежи превышает по объёму газовую часть залежи.
Вот интересно, что то, запасы чего меньше, оно стоит в начале, то есть газонефтяное месторождение или залежь, если газ впереди, значит его запасы по объёму меньше, чем запасы нефти.
А нефтегазовая, соответственно, наоборот, поскольку нефть в этом слове впереди, то её запасы меньше, чем запасы газа.
Газовые залежи с нефтяной оторочкой, в которых газовая шапка превышает по объёму нефтяную часть залежи.
Ну это всё вот и газонефтяное, и нефтегазовое - это двухфазные залежи.
Есть чисто, вот как есть чисто нефтяное месторождение, есть чисто газовое.

Это вот месторождение Медвежье, о котором я говорил.
Чисто газовое месторождение -- только свободный газ.
Газ этот в основном состоит из метана, что касается Медвежьего месторождения.
Ну понятно, что могут быть газы и с более высоким содержанием гомологов метана.

Теперь газоконденсатные месторождения.
Это месторождения, содержащее в своём составе пентаны плюс вышекипящие, то есть $C_5$ плюс выше.
И примеры их я тоже вам приводил многочисленные.
Нефтегазоконденсатная, когда у нас в составе есть нефть, газ и растворённый в газе конденсат.
Вот на самом деле эта терминология, значит, когда я с ней знакомился, меня вызвало смущение, что и газонефтяные, и нефтегазовые на самом деле.
Вот нефтегазоконденсатная, она чем отличается?
Это двухфазные залежи, так же как газонефтяная и нефтегазовая, но в составе пластового газа, газовой шапки растворён конденсат.
Так вот, когда у нас конденсата в газовой шапке нет, это тоже всё можно даже математически смоделировать.
И только в том случае у нас нет растворённого конденсата, то есть пентанов плюс вышекипящих в газовой шапке, когда у нас нефть тяжёлая.
Дальше я расскажу о том, какие нефти, как их различают.
Но вот если нефть тяжёлая, тогда в составе этой нефти, вот в растворённом газе, он почти целиком состоит из метана типа Медвежьего, да, вот газ, только там-то речь идёт о свободном газе, а я говорю про растворённый.

Так вот, если у нас тяжёлая нефть, то в составе пластовой нефти, то есть растворён сухой газ всегда, и если у нас двухфазный залежь и газовая шапка над нефтяной оторочкой, а в нефтяной оторочке нефть тяжёлая, и там растворён метан, газовая шапка не будет содержать конденсат, она будет практически представлять собой сухой газ.
Это всё абсолютно корреспондирует с термодинамикой многокомпонентных систем, это можно моделировать, и это можно изучать экспериментально, а можно и математическое моделирование осуществлять.
То есть для тяжёлых нефтей, в них всегда, если есть растворённый газ, он всегда сухой, и газовая шапка всегда в ней отсутствует конденсат, и она будет в основном состоять из метана.

\begin{center}
\includegraphics[width=\textwidth, page=13]{Брусиловский.pdf}
\end{center}

Теперь дальше, про терминологию применяемую.
Что такое промысловая нефть, сырая нефть и товарная нефть?
Вы видите, это устоявшаяся терминология, она используется в стандартах, в регламентах.

Промысловая нефть –- это безводная жидкость, составляющая продукции добывающих скважин, а вот сырая нефть –- это смесь промысловой нефти и попутной воды.
И уже в сырой нефти содержится в той или иной степени растворённый газ.
Но имеется в виду, сырая нефть -– это вот рассматривает, добываемая нефть, в которой есть растворённый газ, в ней есть вода, минеральные соли, мехпримеси.
И сырую нефть готовят на промысле для того, чтобы получить товарную нефть, которая будет сдаваться потребителю.
Значит, вот в систему Транснефти для дальнейшей транспортировки.
И эту нефть сырую нужно готовить.
Так вот, для товарной нефти существуют чёткие критерии по тому, как она должна быть подготовлена, и для того, чтобы её сдать в систему Транснефти.
Вот об этом мы сейчас и поговорим.

\begin{center}
\includegraphics[width=\textwidth, page=14]{Брусиловский.pdf}
\end{center}

Поскольку у нас… сейчас вот мы говорим про физико-химические свойства,
и вообще в названии у вас курса "<Физико-химические свойства">.
Ну вот, так вот, по физико-химическим свойствам, степени подготовки, содержания сероводорода и лёгких меркаптанов нефть подразделяют на классы, типы, группы и виды.
Меркаптаны –- это радикалы, содержащие серу.

\begin{center}
\includegraphics[width=\textwidth, page=15]{Брусиловский.pdf}
\end{center}

Классы нефти по её сернистости, то есть наличию серы.
Малосернистая –- меньше или равно шесть десятых процента массовых.
Сернистая нефть -– до одной целой восьми десятых процента массовых серы.
Высокосернистая -– до трёх с половиной, и особо высокосернистая – свыше трёх с половиной процента массового.
К сожалению… да, так вот, значит, вот чем меньше серы, тем выше стоимость нефти.
Вот наличие серы, оно сильно вообще удешевляет стоимость товарной нефти.
Значит, и вот это такая проблема.
Значит, ну вот по сернистости –- это класс нефти.

\begin{center}
\includegraphics[width=\textwidth, page=16]{Брусиловский.pdf}
\end{center}

Теперь по плотности товарной нефти.
Нефть различается на следующий типы.
Особо лёгкая –- это меньше 830 килограмм на метр кубический.
Лёгкая -– до 850, средняя -– до 870, тяжёлая –- 870-895, и бетуминозная –- свыше 895.
Теперь вот какие комментарии.

У нас особая лёгкая нефть характерна для глубокопогруженных пластов.
Это Юрские залежи и это Ачимовские залежи.
Вот для них там характерна особо лёгкая нефть.
Значит, наиболее часто встречающаяся нефть –- это средняя.
От 850 до 870 килограмм на метр кубический.
Имеется в виду, уже когда мы дегазировали нефть, подготовили её к сдаче потребителю.
Вот замеряем плотность и вот такая у нас нефть.
Ну и можно оценивать ещё по устьевым пробам нефти.
Какая же у нас нефть в пласте?
Вот на такие типы разделяется.
Теперь Битуминозная нефть, тяжёлая нефть, как я говорил, там вот пластовая нефть, если содержит растворённый газ, ну всегда практически есть растворённый газ, да, то он очень сухой.
Для лёгкой нефти, и особо лёгкой нефти, у нас растворённый газ практически всегда содержит приличное количество этана, пропана, бутанов, то есть этот газ очень привлекателен для переработки.


\begin{center}
\includegraphics[width=\textwidth, page=17]{Брусиловский.pdf}
\end{center}

А вот разделяются вот так вот, да, вот насчёт типов нефти.
До этого мы про плотность только говорили, потому что этот показатель, он важен, когда мы, ну это важнейший показатель, тип нефти определяться плотностью, если мы нефть поставляем на экспорт, то там тоже 5 типов, но обратите внимание, плотность измеряется не при 20 градусах Цельсия, это дегазированная нефть, а при 15 градусах Цельсия.

Почему? Потому что за рубежом стандартные условия отличаются от тех, которые используются у нас в стране.
У нас в стране по давлению они не отличается, это одна физическая атмосфера, вообще говоря, одна физическая атмосфера это 1,01325 бар.
Это из отраслевого стандарта, там написано 1 бар.
А там стандартные температуры 60 градусов Фаренгейта.

60 градусов Фаренгейта это 15,56 градусов Цельсия.
Ну, сократили, и вот в стандарте написали 15 градусов Цельсия.
Когда за рубеж предполагается экспортировать нефть, то измеряют ее плотность при 15 градусах Цельсия, одной физической атмосфере, и уже показаны нормы для типа особо лёгкой, легкой, средней, тяжёлой и битуминозной нефтей.
Понятно, что раз температура меньше, чем 20 градусов Цельсия, значит значение плотности будет несколько выше.
То есть при 15 градусах, на 5 градусов ниже, чем 20, значит, вот, соответственно, плотность меняется.
Помимо плотности для нефти, поставляемой на экспорт, приводится выход фракций процента объемных.
Причем я видел ссылки, ну, тех, кто не занимается этим, ну, так сказать, невнимательно отнесся, там написано процент массовый.
Нет, в регламенте, в регламенте, в отраслевом стандарте, вернее, в государственном стандарте ГОСТе, написано процент объемный.
До температуры 200 градусов Цельсия сколько выкипает?
Это бензины, кстати.
При температуре до 300 градусов Цельсия сколько выкипает?
Это керосин.
И сколько выкипает при температуре, значит, свыше 350 градусов Цельсия?
Это уже масла, дистиллят и когда интересует, и так далее.
То есть это характеризует, вот этот выход фракций процента объемных характеризует уже направление переработки, возможное направление переработки нефти, сколько там разных фракций с точки зрения использования в промышленности.
И также есть такой показатель, как массовые доли серы.
Вот обратите внимание, выход фракций процента объемных, а уже сера или массовые доли парафина -- это процент именно массовый.
И вот мы с вами про доли серы уже говорили.
Малосернистость, сернистость, особо высоко сернистость.
Для нефти, поставляемой на экспорт, вот такие градации.
Они несколько отличаются от нефти, которая...
Если вы сравните с табличкой, которая до этого была.
Давайте.
Вот классы нефти по сернистости.
Давайте.
Сейчас, секунду.
Ну вот нефть сернистая здесь.
0,6-1,8.
Это то же самое, то же самое.
А вот...
Сейчас, секунду.
Высокосернистость...
Сейчас, секундочку, извините, пожалуйста.
А, ну вот тут нет.
Класс 0.
Вот на экспорте, когда вот экспорт, да?
Тип, тип, тип, тип.
Вот тип 0 - это меньше, чем 0,6.
А когда не на экспорт, то у нас...
Ну, в общем, тут есть различия, неважно.
В общем, факт тот, что тоже малосернистость, сернистость, высокосернистость, вот разные нормы.
И вот наличие серы, наличие серы - это...
Чем её больше, тем хуже.
Опять это коррозия, это...
И это сильно удешевляет стоимость этой нефти.
И по массовой доле парафина также отличаются нефти, поставляемые на экспорт, что массовая доля парафина, она не должна превышать 6\%.
Не может быть нефти, поставляемой на экспорт, содержание парафина больше, чем 6\%.
Парафин -- это твёрдые углеводороды, парафины, вообще говоря.
От $C_{17}$ до $C_{64}$.
И существуют специальные методики для исследования парафинности нефтей.
И, значит, вот наличие парафина, который может выпадать в стволе скважины, в оборудовании, да и в призабойной зоне, если у нас температура в пласте невысокая, вот наличие парафина -- это, значит, вот создаёт большие проблемы.
И для избавления от парафина используют и механические способы, и химию, и так далее.
То есть это специальные лаборатории, специальные исследования и занимаются тем, как нам эксплуатировать месторождение с парафинностями нефтями.
А их немало, этих месторождений, в том числе и в Татарии, и в Башкирии, ну и так далее.
Это уже по-разному.

\begin{center}
\includegraphics[width=\textwidth, page=18]{Брусиловский.pdf}
\end{center}

Теперь существует градация нефтей, группы нефти уже по степени подготовки.
Это очень важный показатель.
Значит, массовая доля воды не должна превышать полпроцента.
Это вот при сдаче в систему Транснефти.
То есть товарная нефть.
И обратите, пожалуйста, внимание, что давление насыщенных паров нефти, вот оно не более 500 мм ртутного столба при температуре 37,8 градуса Цельсия.
Это температура [не распознано] называется.
Почему я на этом останавливаюсь?
Так вот, для обычных нефтей давление насыщенных паров после подготовки этих нефтей на промысле и перед сдачей для средних нефтей, тяжелых, битуминозных, оно обычно ниже чем 66,7 килоПаскалей или 500 мм ртутного столба.
И никаких проблем в этом плане нет.
Потому что там нет в этой нефти легких фракций.
А вот в Юрских залежах, короче, для особо легких нефтей, особо легкие по плотности, я вам говорил, там нефти, их плотность меньше чем 830 кг/м3 при стандартных условиях, там есть легкие фракции.
И давление насыщенных паров этой нефти, вот если мы эту нефть готовим,
а потом приводим к одной атмосфере и 20 градусов Цельсия, вот эта нефть, да?
И оказывается, что давление насыщенных паров такой нефти, особо легкой, особо легкой это касается, оно при 37,8 градусах Цельсия оказывается выше чем 500 мм ртутного столба.
Это явилось полной неожиданностью в начале 2000-х годов, когда стали эксплуатировать Новогоднее месторождение, ну вообще, Юрские пласты, ну в частности в нашей компании Новогоднее месторождения, ну не важно, не будем называть название месторождения, при переходе к Юрским пластам, то есть с увеличением глубин бурения, когда нефти легкие, особо легкие оказались в пластах нужно дополнительные усилия принимать для того, чтобы перед сдачей в систему Транснефти избавиться от легких фракций.
Вот это вот была большая проблема для производственников, потому что нужно было перестраивать систему подготовки нефти и кроме того это уменьшало объемы добычи нефти.
Если у нас чем больше особо легких фракций, то есть с низкой температурой начала кипения, и тем больше их испарится при подготовке нефти к транспорту для того, чтобы давление насыщенных паров не превышало 500 мм ртутного столба.
И оценки, вот я только пришёл на работу в СибНефть, и сразу нужно было оценить, каковы же потери будут компании от введения этого ГОСТа.
Если бы температура нефти была не превышением при температуре, вот нефть сдаваемая в Транснефти, её упругость паров не должна была превышать 500 мм ртутного столба при температуре сдачи потребителя.
То есть не нормировалась, ну то есть при какой температуре вы сдаёте, при такой сдаёте.
Эта температура обычно была 20 градусов Цельсия считалось.
А с введением нового ГОСТа нужно было оценивать упругость паров, чтобы не превышала 500 мм ртутного столба при 37,8 градусах Цельсия.
И когда мы грели эту нефть, ну так сказать моделировали нагрев этой нефти особо лёгкой до 37,8 градусов Цельсия, то до 6\% массовых этой нефти испарялось.
То есть это колоссальное, если переводить на тонны, учитывая добычу, учитывая запасы, понимаете, это вот большая проблема.
Она в итоге была решена смешением этой особо лёгкой нефти с нефтью более тяжёлой, тогда упругость паров уменьшалась и так далее.
Но в общем всегда возникают какие-то инженерные проблемы на практике и для их решения требуется теоретические знания, понимание сути.
И этого понимания у многих нефтяников не было, которые не сталкивались с особо лёгкой нефтью.
Для обычных нефтей начало кипения, оно может быть значительно выше, чем для лёгких нефтей, оно может превышать и 100 градусов Цельсия начало кипения.
И может быть, ну может быть и меньше, конечно, но когда у нас особо лёгкие нефти, и у нас начало кипения порядка 50 градусов Цельсия, то это совсем другая ситуация.

\begin{center}
\includegraphics[width=\textwidth, page=19]{Брусиловский.pdf}
\end{center}

Вот теперь виды нефти по массовой доле сероводорода и лёгких меркаптанов.
Значит, вот тоже регламентируется, что массовая доля сероводорода может быть не более, чем написано для разных групп нефти.
Метил и этил меркаптанов -- это радикалы $RSH$, содержащие серу.
Так что у вас примут в систему Транснефти, то есть предприятие должно не то что стремиться, оно обязано уже в систему Транснефти продавать нефть с содержанием сероводорода не более, чем регламентируемое, и по упругости паров, и по количеству воды не более половины процента, то есть масса всего.
Поэтому системы промысловой подготовки, они такие очень серьёзные, серьёзные вот эти установки подготовки нефти и так далее.

\begin{center}
\includegraphics[width=\textwidth, page=20]{Брусиловский.pdf}
\end{center}

Теперь вот пример.
Значит, когда у нас изучили свойства дегазированной нефти, то ей присваивается некое условное обозначение.
И это вот данные из государственного стандарта, прямо примеры оттуда.
И это обозначение в итоге, вот когда вы в проекте разработки или в каком-то проектном документе вы описываете свойства дегазированной нефти, характерные для какого-то пласта, по пластам в подсчёте запасов, в проектных документах на разработку.
Там есть главы по физико-химическим свойствам нефтей.
Так вот, там вы указываете, и скажем, для какого-то пласта сколько скважин исследовано, сколько проб у вас.
И проб имеется в виду устьевые пробы или пробы, взятые после промысловой подготовки, ну, дегазированной нефти.
Их исследовали в лаборатории, определяли там все показатели, и плотность, и молекулярную массу  по температурам кипения и так далее.
И в итоге присваивают некое условное обозначение.
И вот здесь вот написано, что в соответствии с тем, о чём я вам рассказал по массовой доле серы, по плотности при стандартных условиях.
При 20 градусах Цельсия имеется в виду одна атмосфера физическая, конечно же.
Концентрации хлористых солей, массовой доли воды и так далее, количество сероводорода.
Вот обозначают.
Четыре цифры.
И добавляют в конце ГОСТ, то есть государственный стандарт, R51858-2002.
То есть это в этом 2002 году этот стандарт был выпущен, потом там произошли небольшие изменения, небольшие, значит, и вот так вот.
То есть, вот если...
А когда мы говорим о нефти, поставляемой на экспорт, то в соответствии тоже с тем, о чём я вам рассказал по массовой доле серы, по плотности объёмной...
Но в дополнение ещё по объёмной доле фракции до 200, как я говорил, это бензин, до 300 – это керосин, плюс массовой доли парафина.
Концентрации хлористых солей.
То есть уже обозначение такое, какое здесь показано, и оно несколько отличается от того, что пишут для нефти, которое не поставляется на экспорт.

\begin{center}
\includegraphics[width=\textwidth, page=21]{Брусиловский.pdf}
\end{center}

Вот это теперь сведения, которые весьма полезны для будущих нефтяников и которые вообще, так сказать, тех, кто интересуется ситуацией современной
и вообще понимания, какие же марки нефти существуют, где они добываются, где эта нефть добывается.
И когда говорят про марку нефти, про стоимость, всё время в средствах массовой информации говорят, стоимость марки нефти WTI, есть стоимость марки нефти Brent.
Что это такое? Почему?
Мало кто это знает.
Так вот, марки нефти.
Различные их, их довольно много.
И они характеризуются свойствами, прежде всего плотностью и содержанием серы.
Мы видим, что марка Brent, она дорогая.
Это нефть, добываемая в Северном море.
То есть это англичане, это норвежцы.
Это особо лёгкая нефть, потому что, вспомним давайте, нефть с плотностью меньше 830 кг/м3, это особо лёгкая.
То есть это особо лёгкая нефть и малосернистая, меньше чем 0,6\% массовых, 0,37\%.
То есть она малосернистая, особо лёгкая, и она очень дорогая поэтому.
Практически те же характеристики у нефти WTI.
Это нефть, добываемая в Техасе.
Значит, наша нефть, Urals, марка Urals, добывается в Западной Сибири и в Поволжье.
Плотность соответствует средней плотности.
Я напоминаю, средняя плотность -- это 850-870 кг/м3.
И это уже сернистая нефть.
Сернистая нефть, и её стоимость ниже именно из-за наличия серы.
Ну, значит, вот это основной недостаток.
Значит, теперь есть сорт, добываемый в Западной Сибири.
Называется Siberian Light.
И мы видим, по плотности это какая нефть?
Лёгкая.
Вспомним, что лёгкая нефть -- 830-850 кг/м3.
И содержание серы, оно на грани.
Малосернистая и сернистая.
То есть, это Siberian Light.
Лёгкая нефть, да?
Лёгкая, ну, по плотности она лёгкая.
И в ней серы.
Серы, в общем, немного.
Вот она поэтому достаточно дорогая.
А в Восточной Сибири есть ESPO, характеристики которой практически такие же, как и у Siberian Light.
Но всё-таки граница в 0,6 содержания серы пройдена.
Больше, чем 0,6.
Поэтому это сернистая нефть.
И, соответственно, немножко...
Ну, подешевле она.
Теперь вот компания "<Газпромнефть"> на шельфе добывает на уникальном Приразломном месторождении.
Это шельф Арктика.
Это Арктика, поэтому Arctic Oil.
Вот. Какая это нефть?
Это битуминозная нефть.
То есть очень тяжёлая.
Битуминозная нефть.
И серы там от 2\% до 3\%.
То есть стоимость её, из-за наличия содержания серы, конечно же, меньше, чем у Brent, у WTI.
Вот так вот.
Ну, и пример на слайде для Иранской нефти.
Он для нас не такой актуальный.
Вот Приразломное месторождение для нашей компании действительно актуально.
Так вот, в зависимости от того, из каких пластов добывается нефть, из Юрских залежей нефть значительно более лёгкая, чем из залежей ПК или других.
И вот от этого стоимость нефти определяется.
Ну и, конечно, ещё раз про содержание серы.
Практически во всех месторождениях России содержится очень мало малосернистой нефти.
Так природа распорядилась.

\begin{center}
\includegraphics[width=\textwidth, page=22]{Брусиловский.pdf}
\end{center}

Теперь, вот тоже для справки.
Это мой коллега в своё время где-то высмотрел.
Тоже по плотности содержания серы.
Какова доля нефти в разных странах по сортам.
Ну вот по содержанию серы и плотности имеется в виду.
Вот это любопытные данные.
По России тут нет, это имеется в виду зарубежное.
Вот в Норвегии, например, очень в сильной степени зависит от добычи нефти.
То есть как там открыли месторождения в 70-х годах.
Это страна, которая, грубо говоря, влачила в существование.
Там ничего особенного-то не было.
И они просто обогатились именно на добыче нефти.
Это сейчас одна из богатейших стран по уровню жизни и так далее.
И вот там лёгкая нефть составляет 73\%.
Вы знаете что?
73\% в Норвегии - это по содержанию серы.
И нефть при этом лёгкая, поэтому её стоимость высокая.
Ну и вот анализируя эту таблицу, сейчас мы уж не будем на это время тратить, но я хочу отметить.
Вот Венесуэла.
В Венесуэле, во-первых, высокое содержание серы.
И во-вторых, тяжёлая нефть.
Там очень высокая доля не просто тяжёлой нефти, а битуминозной нефти.
В общем, одно время наша компания имела контакты с Венесуэлой.
Поэтому я запомнил, что там много битуминозной нефти, тяжёлой нефти.
Вот это я закончил.
Ту часть общую, которая касается характеристики физико-химических свойств поверхностных нефтей.
Примеры с пластовыми нефтями и газами.

\subsection{Фазовые диаграммы пар-жидкость}

\begin{center}
\includegraphics[width=\textwidth, page=23]{Брусиловский.pdf}
\end{center}

И следующая важная тема.
Она очень важная.
Это уже мы переходим к термодинамике.
Переходим к теории.
Это фазовые диаграммы пар-жидкость.

Итак, тема у нас очень важная для понимания тех физических процессов, которые происходят в углеводородных системах.
Фазовые диаграммы пар-жидкость.

\begin{center}
\includegraphics[width=\textwidth, page=24]{Брусиловский.pdf}
\end{center}

Для введения я вам прямо зачитаю из своего учебника.
Ну и кому-то я напомню.
Может быть, изучали в зависимости от той специальности, которую
вы получаете.
Диаграммы, в которых по осям координат откладывается значение термодинамических параметров или функций состояния, называются термодинамическими диаграммами.
В случае многокомпонентной системы термодинамическая диаграмма является также диаграммой, по осям которой откладывается термодинамическое свойство и состав.
А вот конкретно про фазовые диаграммы.
Термодинамические диаграммы, в которых по осям координат откладывается давление, температура, мольный объём и наносятся кривые фазового равновесия, называются фазовыми диаграммами.
Для многокомпонентных систем по осям координат может откладываться и состав.
Кривой фазового равновесия называется линия на фазовой диаграмме, соответствующая состояниям равновесно сосуществующих фаз.
Мы с вами рассматриваем случаи, когда находится в равновесии пар и жидкость.
Вообще пар -- это аналог газа.
Просто когда в равновесии с жидкостью газ находится, то часто говорят именно пар и парожидкостное равновесие.
Но пишут и газожидкостное равновесие.
То есть если пар, то это обязательно имеется в виду насыщенный жидкостью.
Так вот фазовые диаграммы полезны для лучшего понимания закономерности парожидкостного равновесия в системах природных углеводородов, поэтому я этому уделяю особое внимание и в своих книжках я отдельные главы посвящаю фазовым диаграммам пар-жидкость.
Потому что совершенно бывает недостаточно того, что пишут в учебниках традиционных.
Ну не всегда, да.
Для понимания тонкости, в каком случае у нас, какой.
И этим фазовыми диаграммами, кроме всего прочего, определяется тип месторождения.
Какие у нас месторождения, это я вам покажу на картинках.

Ну а сначала мы о месторождениях не говорим.
Мы просто говорим о теории, о фазовом равновесии.
И существует отличие между особенностями фазового равновесия чистых веществ и систем, состоящих из двух и более компонентов.
Важнейшее отличие заключается в том, что при фазовом равновесии однокомпонентной системы, то есть чистого вещества, метан, этан, пропан и так далее, чистое вещество, она расслаивается на фазы, состоящие только из данного чистого вещества.
А если система состоит из двух и более компонентов, то в общем случае доли компонентов в равновесных фазах не совпадают.
То есть, если у нас расслаивает система на пары жидкость,
то в общем случае компонентный состав, ну а следовательно и свойства при заданных термобарических условиях будут различными.
И это обусловливает уже для двухкомпонентных систем такие физические явления, которые не наблюдаются для чистых веществ.
Для систем с числом компонентов больше двух характерны
многие физические явления, которые наблюдаются в бинарных
системах.
Поэтому на основе фазовых диаграмм бинарных смесей иллюстрируется закономерности, присущие парожидкостному равновесию многокомпонентных систем.
А также в учебниках поясняется используемая терминология.
Ну вот теперь напомним правила фаз Гиббса.
Если $N$-компонентная система, находится в состоянии термодинамического равновесия под воздействием двух интенсивных параметров -- давления и температуры (интенсивные параметры -- это те параметры, которые не зависят от размера) и число фаз равно $m$, то число независимых переменных, характеризующих эту систему $F$ равно $n$ минус $m$ плюс 2.
$F$ это число термодинамических степеней свободы.
Буква $F$ от английского слова freedom, свобода.
$n$ -- это число компонентов в системе и $m$ -- это число фаз.
А параметр $F$, вот freedom, называется числом термодинамических
степеней свободы.
В зависимости от его значения, система делится на инвариантные,
когда у нас $F$ равно 0.
1-вариантные $F$ равняется 1.
2-вариантный $F$ равно 2.
Ну и так далее.
Минимальное число фаз $m$ равняется 1, поэтому максимальное число степеней свободы ($F$ максимальное) равно $n$ плюс 1.

Вот что на слайде об этом не говорится, но я вам скажу.
Вот это равенство, $F$ максимум равняется $n$ плюс 1, дает размерность пространству необходимую для представления полной фазовой диаграммы $n$-компонентной системы.
Для 1-компонентной системы, это максимальное число степеней свободы равно 2, да?
И фазовое состояние такой системы может быть представлено
в двумерной плоскости.
Для чистого вещества, для 1-компонентной системы, фазовое состояние может быть представлено на плоскости давление температура.
Что мы увидим эту картинку.
Для числа компонентов равное 2, $F$ максимум равно 3, и для описания фазового состояния бинарной смеси, необходимо трехмерное пространство в координатах давления температура и мольная доля одного из компонентов (например, второго).
Для тройной системы необходимо четырехмерное пространство и так далее.
Поскольку максимальный, теперь дальше.
Правило фаз дает ответ на два основных вопроса.
Первый: каково максимальное количество фаз в системе с числом компонентов $n$?
Максимальному числу фаз соответствует число степеней свободы равное 0.
И получается, что максимальное количество фаз в системе с числом компонентов $n$ равно $n+2$.
Так вот, для 1-компонентной системы, чистое вещество, $F=1$, это означает, что существует взаимно однозначное соответствие между температурой и давлением, при котором вещество находится в состоянии парожидкостного равновесия.
Такой давление называется давлением насыщенного пара вещества и на картинке я вам покажу.
Зависимость между давлением насыщенного пара и температурой изображается линией на фазовой диаграмме в координатах давления температуры.
Для 2-компонентной системы, уже для заданного значения температуры, можно изобразить фазовую диаграмму в координатах давления, состава равновесных фаз и так далее.
Для 3-компонентной системы $F=3$ и нам нужны уже объёмные диаграммы, которыми на практике практически не используются.
А для систем с числом компонентов более 3, в инженерной практике
используется в основном только диаграммы давления и температуры, которые на плоскости можно изобразить для смеси с любым числом компонентов.
Поэтому не случайно в учебниках вы можете увидеть именно
диаграммы в координатах давления температуры.
В специальной литературе по физхимии, по фазовому равновесию можно увидеть и другие, в том числе и в моей книжке можно увидеть самые разнообразные фазовые диаграммы, но в литературе по разработке месторождений, по свойствам пластовых флюидов, именно применительно к разработке месторождений для инженеров, используют диаграммы в координатах давления и температуры.
Так вот, уже заканчивая этот слайд, который вы прочитали, конечно же, вот это правило фаз, оно даёт ответ на два основных вопроса.
Каково максимальное количество фаз в системе с числом компонентов $n$, то есть $n+2$, да?
И какое число независимых переменных в систему нужно задать для однозначного определения значений остальных переменных.
Вот это тоже очень важно.
Второе.
Какое число независимых переменных в систему нужно задать для однозначного определения значения остальных переменных.
Это когда я вам буду показывать остановки задачи расчёта фазового равновесия, то как раз можно будет вспомнить правило фаз Гиббса, что мы и сделаем.

\begin{center}
\includegraphics[width=\textwidth, page=25]{Брусиловский.pdf}
\end{center}

Вот теперь перейдём к фазовым диаграммам.
Сначала чистые вещества.
Вот зависимость давления насыщенного пара чистого вещества от температуры начинается в тройной точке $H$ и оканчивается в критической точке $C$.
Точка $H$ называется тройной точкой, так как она соответствует единственным для данного вещества значением давления и температуры, при которых оно одновременно может находиться в трёхфазном равновесии пар-жидкость-твёрдое тело.
Ещё раз.
Точка $H$ -- это равновесие пар-жидкость-твёрдое тело.
И это единственное сочетание термобарических параметров.
Вот тройная точка.
И это следует из правила фаз ипса.
При $N=1$ и $m=3$.
Число термодинамических степеней свободы $F=0$.
Критическая точка для однокомпонентной системы (точка $C$) характеризуется наивысшими значениями давления и температуры, при которых могут сосуществовать паровая и жидкая фазы.
В области, ограниченной слева линией равновесия жидкость-твёрдое тело, кривой плавления, а снизу кривой $HC$, равновесие жидкость-пар.
Вот это вот, ещё раз хочу сказать, кривая $HC$ -- это кривая насыщенного пара.
При любых давлениях вещество находится в однофазном жидком состоянии, то есть внутри области, ограниченной кривой плавления и ограниченной снизу кривой упругости паров или насыщенного пара $HC$, вот внутри написано жидкость.
Вещество находится в жидком агрегатном состоянии.
Слева от тройной точки $H$ проходит линия равновесия газ-твёрдое тело, которую называют кривой сублимации.
Теперь вернёмся к пар-жидкость.
При любых давлениях ниже давления насыщенного пара или сублимации и заданной температуре, вещество находится в однофазном газовом состоянии.
Теперь, я уже отметил, но повторю, что термины пар и газ являются синонимами.
Слово пар наиболее часто употребляется для обозначения газовой фазы, находящейся в равновесии с жидкой фазой.
Теперь, при любой температуре выше критической, то есть вот эта вот точка $C$ характеризуется температурой, которая называется критической и при любых давлениях вещество находится в однофазном состоянии.
Принимая условие, что во всей этой области состояние вещества является газовым, ну то есть при температурах выше критической температуры мы будем называть состояние газовым.
Получаем, что при любых давлениях выше критического, давление выше чем $P_c$, ну то есть критического, критическая изотерма является условной границей, различающей жидкое и газовое состояние вещества.
Почему условной?
Ну потому что при переходе через критическую температуру и давление выше критического мы не увидим никаких менисков раздела между фазами, никаких видимых изменений не будет.
Будет плавный переход из жидкого агрегатного состояния в газовое.
То есть вот если мы переходим, например, переход из точки $e$ в точку $d$, видимых изменений не будет, но в точке $e$ мы
в жидком агрегатном состоянии, в точке $d$ мы в газовом агрегатном
состоянии, которое, я сразу отмечу, могут называть и закритическим.
Ну так сказать, это все терминология.
То есть, при температурах выше критической у нас газовое агрегатное состояние.
Теперь, поскольку при этих термобарических условиях не существует поверхности раздела между фазами, то переход происходит при непрерывном изменении свойств.
Вот я вам об этом сказал, но и тут же я смотрю в тексте вот это написано и так далее.
Например, газ, характеризующийся, вот давайте посмотрим вот это вот, то что пунктирный путь.
Например, газ, характеризующийся параметрами в точке $a$ можно в процессе изобарического нагревания привести в состояние соответствующей точке $b$, затем повысив давление выше критического при постоянной температуре, перевести вещество в закритическое состояние.
Ну вот, да, закритическое.
По сути это газовое состояние все равно.
В точке $d$ и изобарически уменьшив температуру ниже критической и идти в жидкое состояние, в точке $e$ уже будет жидкое состояние.
В точке при температурах меньше критической у нас будет жидкое агрегатное состояние и соответствующим давлением.
Ну вот дальше я вам уже, сейчас одну секундочку, уже понятно.
Теперь может быть было бы полезно, может быть было бы полезно, я вам еще скажу, может быть было бы полезно привести диаграмму в координатах давления-удельный объем чистого вещества.
Но я вам скажу так на словах.
Смотрите, для чистого вещества, если мы пойдем, если мы находимся при температуре больше, чем температура тройной точки и меньше чем критическая температура и давление у нас низкое, ниже давление у нас упругости паров, вот мы находимся в газовом агрегатном состоянии
при какой-то температуре фиксированной.
И мы начинаем увеличивать давление изотермически.
У нас с вами температура, повторяю, меньше чем критическая.
Мы изотермически увеличиваем давление, например, в сосуде сжимая, уменьшая объем.
Значит, у нас с вами давление будет расти и когда мы при данной температуре достигнем давления насыщенного пара, образуется первая капля равновесной жидкости.
И вот мы будем дальше сжимать и пока всё вещество не перейдет в жидкое агрегатное состояние у нас давление не изменится.
Это характерно для чистого вещества.
Ну а когда у нас вещество перейдет на вот это давление упругости паров в жидкое состояние полностью, дальнейшее сжатие приведет к резкому увеличению давления.
Итак, во всем диапазоне от тройной точки до критической температуры это всё будет происходить именно таким образом.
Это характеризуется в координатах давления-удельный объем.
В учебниках физики и так далее.
Но вот это для чистого вещества.
А вот почему я об этом говорю.
Мы сейчас к смесям перейдем уже.
И там и там не так.
Там всё не так.
Всё не так.

\begin{center}
\includegraphics[width=\textwidth, page=26]{Брусиловский.pdf}
\end{center}

Теперь еще очень полезная информация.
Вот я и вставил именно для лекции для вас.
Это информация о свойствах углеводородных и неуглеводородных компонентов, которые входят в состав природных углеводородных смесей и нефтей, и газов.
И что мы здесь видим.
Здесь это сведения из справочников просто.
Значит, о малярной массе, о критическом давлении.
Вспомним, сразу после упругости паров критическое давление.
То есть максимальное давление, при котором, вернее, сейчас
критическое давление.
Это давление, соответствующее критической точке.
Критической точке, то есть крайние точки на кривой упругости паров.

Критическая температура -- это максимальная температура, при которой возможно для чистого вещества, для данного каждого вещества, жидкое агрегатное состояние.
Выше этой температуры вещество находится в газовом агрегатном состоянии.
Критический объём -- это объём, который соответствует объёму в критической точке.
И ацентрический фактор -- омега.
Я об ацентрическом факторе сейчас вам скажу.
Но прежде я хочу, чтобы вы обратили внимание на очень большой диапазон изменения критической температуры для веществ, входящих в природные углеводородные системы.
Молярная масса определяется химической формулой.
Мы уже с вами считали метан приблизительно 16, здесь более точно.
Этан 30 определяется.
Пропан напомню углерод 3 на 12 -- 36, плюс 8 водород -- это даёт 44.
Более точно здесь в таблице.
Каждый компонент характеризуется своим критическим давлением.
И обратим внимание, что нормальные алканы по мере увеличения числа атомов углерода, у них такая тенденция есть к уменьшению критического давления.
Это не значит, что всегда так, но это тенденция чёткая.
С увеличением числа атомов углерода в алканах критическое давление падает.
Ну вот, за исключением того, что у метана и у этана близкие критические давления, например, у пропана уже меньше, у нормального бутана ещё меньше, у нормального пентана ещё меньше и так далее.
Но вот эта тенденция уменьшения давления проявляется до...
Ну то есть темп уменьшения давления не является постоянным, потому что иначе бы у нас для углеводородов с большим числом атомов углерода, для алканов, мы бы очень быстро достигли критического давления близкого к нулю.
Такого не бывает.
Такого не бывает, это я вам сразу могу сказать.
В справочниках вот самое известное Рид и Шервот.
Рид и Шервот-Прауснец, там приводится свойство где-то до $C_{20}$, а дальше нет сведений в этих справочниках, может быть каких-то специальных, но вот эти данные о критическом давлении, критическом давлении компонентов, они потом были аппроксимированы.
В общем нам нужно знать критическое давление, критическую температуру для того, чтобы характеризовать компоненты при моделировании с использованием уравнений состояния.
Это я вам просто на будущее говорю, потом мы вернёмся.
И вот были предложены разные способы, разные корреляции для критической температуры, критического давления.
В том числе и я в своё время предложил очень эффективный метод, который позволил сразу улучшить моделирование природных углеводородных систем с применением уравнений состояния, но потом он совершенствовался.
Так вот, дальше, такой показатель как ацентрический фактор.
И вот внизу прямо определение этого ацентрического фактора.
Дело в том, что он характеризует, нам неявно отклонение формы молекул и вещества от сферической.
И с увеличением числа атомов углерода у наc величина ацентрического фактора растёт.
Дело в том, что например для метана, его форма была показана, $CH_4$ -- это практически сферическая молекула.
Так вот, ацентрический фактор, он близок, вот 1/100.
А вот для инертного газа, этот ацентрический фактор равен, ну нулю, нулю равен.
Значит с увеличением числа атомов углерода, у нас длина цепочки, длина цепочки молекулярно, она увеличивается.
И значит, а для изомеров, ещё и от этой цепочки, ответвления всякие возможны.
И вот эта форма, которая отличается от сферической, так далее, она приводит к тому, что у нас увеличивается значение ацентрического фактора.
И вот, исторически было так, что Ван-дер-Ваальс, знаменитый голландский физик, который предложил уравнение, которое стало основой для моделирования в том числе углеводородных систем.
Я об этом тоже, я подробно пишу в книжках и кратко скажу в обзоре.
И которое явилось первым уравнением среди так называемых кубических уравнений состояния, которые сейчас очень широко применяются в инженерной практике.
Так вот, он показал, что если ввести такие параметры, как приведенное давление, это давление абсолютное, деленное на критическое давление.
Приведенная температура, это абсолютная температура, делённая на критическую температуру.
Критическая как раз для чистого вещества.
То в этих параметрах, приведенных, $P$ приведенное, $T$ приведенное, мы можем получить зависимость давления насыщенного пара универсальную как бы, но дело в том, что это теоретическая зависимость.
А на практике у нас для веществ давление насыщенного пара отличается от этой универсальной зависимости.
Именно потому, что у нас разные физические эффекты для веществ присутствующих.
И в том числе симметрия молекул, и квантовые эффекты и прочее, и прочее.
Много всего.
И вот для того, чтобы учесть особенности вещества, был предложен такой параметр, как ацентрический фактор.
Его впервые предложил Вильсон в 1955 году.
Это никакие не нефтяники, не газовики, это все ученые физики, физической химии занимающиеся.
И вот для, значит, вот он, откуда эта формула возникла, откуда эти величины, что для вещества характеризующегося универсальной формой зависимости приведенного давления насыщенного пара от приведенной температуры при значении приведённой температуры, равной 0,7, приведенное давление насыщенного пара равняется 1/10.
И вот он предложил для того, чтобы учитывать отклонение свойств вещества от молекул, имеющих симметрию, подчиняющейся универсальной зависимости приведенного давления насыщенного пара от приведенной температуры, значит, учесть отклонение для данного вещества приведенного давления при температуре насыщенного пара равной 0,7 от универсальной зависимости.
Вот для универсальной зависимости, раз у нас давление насыщенного пара при приведенной температуре 0,7 равняется 1/10, значит, десятичный логарифм от этой величины равен -1 от
1/10.
И вот мы из десятичного логарифма величины давления насыщенного пара при температуре, приведенной 0,7 вычитаем величину при такой же приведённой температуре приведенного давления насыщенного пара конкретного вещества.
И вот мы получаем величину ацентрического фактора.
То есть, если ацентрический фактор равен нулю, а он равен нулю для, по-моему, криптона или что-то такое, или крайне близок к нулю, ну, в общем, для инертных газов.
Это, значит, их зависимость давления насыщенных паров очень близка к универсальной, которую Ван-дер-Ваальс показал.
Так?
А уже для углеводородов самая маленькая величина ацентрического фактора, то есть наиболее он близок к универсальной зависимости насыщенного пара для метана.
И потом, с увеличением числа атомов углерода, у нас отклонение насыщенного пара от универсальной зависимости увеличивается, увеличивается, увеличивается.
И поэтому ацентрический фактор, вот его величина, раз это отклонение увеличивается значит и ацентрический фактор у нас растёт.
Ну вот я немножко, я хотел просто вам объяснить.
Это можно в учебниках прочесть.
Но я хотел вам объяснить, что за ацентрический фактор.
Потому что из графика зависимости вы для чистого вещества увидели, что такое критическая температура, что такое критическое давление.
Вы знаете, что такое молярная масса.
Критический объём -- это объём в критической точке.
А вот я вам объяснил тот параметр, происхождение которого, оно, которое используется при моделировании с применением уравнений состояния, почему вот собственно это важно.
И откуда берётся его значение.
Интересно, что в разных справочниках, причём всемирных таких и всемирно известных, значения ацентрического фактора немножечко отличаются.
Ну в сотых долях.
Ну значит потому что, потому что кривые упругости паров, они тоже в разных справочниках, ну в общем, немножечко отличаются.
Немножечко отличаются данные.
Это всё экспериментальные данные для углеводородов, значит, которые были получены для углеводородов, по-моему, до $C_{18}$ или что-то такое.
И это прецизионные исследования и немножечко, и в зависимости от источника этих данных, значит, вот ацентрический фактор он немножечко разный.
Но это неизбежность.
Значит, дальше поехали.

\begin{center}
\includegraphics[width=\textwidth, page=27]{Брусиловский.pdf}
\end{center}

Вот теперь мы от чистых веществ перейдём к бинарным смесям.
И фазовые диаграммы бинарных смесей, они принципиально иные.
Вот фазовые диаграммы смесей.
Уже двухкомпонентные системы, при парожидкостном равновесии имеют свои особенности, существенно отличающие эти системы от чистых веществ, от однокомпонентных.
Для систем, содержащих более одного компонента, используется такая характеристика фазы, как и её состав.
Для чистого вещества состав мы не...
Он, я уже говорил, значит, он для различных фаз одинаков.
Это просто чистые вещества.
А вот начиная с бинарных систем составы фаз разные.
И в частном случае только в критической точке они могут совпадать.
Теперь состав фаз определяется долями каждого компонента в смеси, составляющей фазы.
В термодинамике многокомпонентных систем применяется понятие мольной доли, о которой мы с вами уже знаем.
Мольная доля компонентов в смеси, представляющей...
Ну, то есть в фазе.
В данном случае фаза -- это смесь.
Может быть газовая фаза, может быть жидкая, а может быть...
Ну да.
Отношение числа молей компонента к общему числу молей всех компонентов.
Вот вспоминаем один из первых слайдов.
Теперь вот мы с вами рассмотрим фазовую диаграмму в координатах давления и температуры.
Я вам должен сказать, поскольку я уже говорил, что именно в координатах давления и температуры мы можем рассматривать фазовые диаграммы для любого числа компонентов смеси.
Поэтому для единообразия, мы посмотрели для чистого вещества, сейчас мы посмотрим для бинарных смесей, а потом мы уже рассмотрим для многокомпонентных систем, то есть с числом компонентов $N$.
Именно это можно рассматривать на плоскости.
Именно в координатах давления и температуры мы можем на плоскости рассматривать фазовую диаграмму с любым числом компонентов.
Приступаем.
Термобарические условия существования двухфазного парожидкостного состояния смеси фиксированного состава принято представлять на диаграммах давления и температуры.
То, что я говорил.
Диаграмма давления и температуры отражает значительные отличия многокомпонентных систем от чистых веществ.
Если для чистого вещества данная диаграмма представляет собой кривую упругости паров, которую мы рассмотрели подробно, которая одновременно является кривой точек кипения и конденсации.
Вот на кривой упругости паров у нас в равновесии пар и жидкость, то есть кипения и конденсации.
То для бинарных систем область, эта область, лежащая между кривыми упругости паров обоих компонентов смеси и ограниченная кривыми точек росы и точек кипения, смыкающимися в критической точке.
Вот у нас, давайте рассмотрим вот это рисуночек.
Вот у нас есть с вами компонент 1, ну вот $C_1$, ну 1 компонент и компонент 2.
И вот у нас с вами $AC_1$ и $BC_2$ -- это кривые упругости паров, соответственно более летучего, более летучий это вот там где $AC_1$ и менее летучего компонента смеси, оканчивающегося критическими точками $C_1$ и $C_2$.
Кстати, обращаю ваше внимание, вот только сейчас мы говорили о том, что для более тяжелых углеводородов у нас критическое давление меньше, и вот вы видите, что на этой диаграмме у нас для более легкого углеводорода критическое давление выше, чем для второго углеводорода, более тяжелого.
Ну и соответственно температура критическая для более легкого углеводорода меньше, чем критическая температура для более тяжелого углеводорода.
То есть всё в соответствии тут.
Вот на диаграмме изображены три области парожидкостного равновесия, соответствующие трём разным составам бинарной смеси.
Вот здесь уже мы фигурируем составом.
Если в смеси преобладает компонент 1, то есть более летучий, то зона двухфазного состояния находится внутри области ограниченной линии $A' C'B'$.
То есть мы, смотрите, мы прибавили к первому компоненту второй компонент в небольшом количестве, и у нас уже образовалась область двухфазного существования, ограниченная вот $A'C'B'$.
Двухфазная система с преобладающим количеством первого компонента.
И вот эта область двухфазного существования, то есть внутри этой области у нас парожидкость.
Значит, при увеличении в составе смеси доли компонента 2 область двухфазного существования смещается вправо и расширяется.
То есть если мы добавим еще компонента 2 к нашей смеси, получим смесь с двумя штрихами, то у нас область двухфазного существования будет $A''C''B''$ она больше, чем область $A'C'B'$, так как мы добавили второго компонента.
Значит, ее максимальные размеры этой области обычно достигаются при примерно равном, называемом эквимолярном, содержании компонентов 1 и 2.
То есть максимальная область сосуществования вот этой бинарной смеси при приблизительно эквимолярном, равном мольном содержании первого и второго компонента.
Этому случаю соответствует изображенная на рисунке область $A''C''B''$. Причем $A''С''$ -- это кривая точек кипения, а $B''C''$ -- это кривая точек росы.
$C''$ - это критическая точка, в которой составы паровой и жидкой фаз становятся одинаковыми.
Это характерно для критической точки смеси.
Преобладание в смеси компонента 2 приводит к тому, что область парожидкостного равновесия смещается ближе к кривой упругости паров этого компонента и сужается.
И этому случаю соответствует область $A'''B'''C'''$, где кривые точек кипения, это линия $A'''C'''$ и кривая точек росы -- это $B'''C'''$ , а смыкаются они тоже в критической точке $C'''$.
Для любых составов бинарной смеси справа от критической изотермы и вне области двухфазного равновесия смесь находится в газовом состоянии.
То есть если наша температура выше, чем критическая температура смеси и мы находимся вне области двухфазного существования, то смесь находится в газовом агрегатном состоянии.
А если же температура ниже критической, то вне двухфазной области смесь находится в жидком состоянии.
В предельных случаях, когда доля одного из компонентов становится равна единице, область двухфазного существования переходит в кривую упругости паров этого компонента.
Это понятно.
То есть мы начинали, мы отмечали первый компонент $AC'$ -- кривая упругости его паров, добавили компонента 2, второго компонента, то есть смесь сделали уже, и у нас уже образовалась двухфазная область, не кривая упругости, а область.
И так далее.
То, что мы с вами уже слышали.
Теперь пунктирная линия, огибающая критических точек, пунктирная линия бинарных смесей, начинающаяся в критической точке компонента 1 и оканчивающаяся в критической точке компонента 2.
Это вот огибающая критических точек.
Вот она нарисована.
С увеличением содержания менее летучего компонента 2 критическая температура смеси непрерывно растёт.
Иначе ведёт себя критическое давление.
Оно вначале увеличивается, достигает максимума и затем снижается.
Вот я опередил, сказал вам об этом.
Но ещё раз повторение мать учения.
Критическая температура смеси, вот это очень важно,
критическая температура смеси всегда больше критической температуры первого компонента и меньше второго.
То есть всегда больше критической температуры для бинарной смеси более летучего компонента и меньше, чем критическая температура менее летучего компонента.
А критическое давление смеси может значительно превышать критическое давление как первого, так и второго компонента.
Я не случайно акцентирую ваше внимание на этом, потому что я вам напомню об этом, что когда мы будем вычислять псевдокритическую температуру и давление, который применяется для расчёта, скажем, $Z$-фактора газов, никакого отношения к реальной критическому давлению псевдокритическое давление не имеет.
Я вам физически объясню происхождение этого, кстати, нигде не написано и очень зря.
Ни в одном справочнике, ни в регламентах, что такое псевдокритическое давление, что такое псевдокритическая температура, особенно про давлению.
Теперь вот на что очень важно.
Следует обратить внимание, почему так много времени я уделил бинарным смесям.
Потому что понятия и явления, рассмотренные при описании фазовых диаграмм давления и температуры бинарных смесей остаются справедливыми и для многокомпонентных систем.
Тут еще раз, я вот сейчас смотрю в текст своего учебника, если вас заинтересует, я вам потом ссылку дам на 100-страничное учебное пособие "<Теоретические основы фазового превращения углеводородных смесей">.
Я это смотрю для того, чтобы более четко, чтобы вы поняли, о чем идет речь и какие явления характерны.
Это не потому, что я не могу прочитать так.
Просто, когда я писал это учебное пособие, я выверял каждое предложение.
Поэтому тут важно.

\begin{center}
\includegraphics[width=\textwidth, page=28]{Брусиловский.pdf}
\end{center}

Но вот этот рисунок, это уже мы переходим к природным смесям, многокомпонентным.
Вот и еще раз, вот я здесь себе даже восклицательный знак поставил.
Важное отличие двух и многокомпонентных систем от чистых веществ, я повторяю это, заключается в том, что критическое давление и температура смеси не являются одновременно максимальными значениями давления и температуры, при которых возможно сосуществование равновесных паровой и жидкой фаз для смеси, начиная от двух, от бинарных.
Это принципиально отличает смеси от чистых веществ.
И вот на примере предыдущей, вот как раз это все и, вот вы видите, критическая температура, ну то о чем я вам говорил, просто очень важно это понимать, что если для чистого вещества критическая температура
характеризует максимальную температуру двухфазного существования, возможно двухфазного существования чистого вещества, то для смеси это не так.
Возьмем любую смесь и возьмем ее критическую температуру, то есть температуру критической точки, мы видим, что при температуре выше, чем температура критическая, у нас в системе может существовать двухфазное состояние.
И тоже касается и давления.
Поэтому тут это важно понимать.
Теперь в частном случае критическое давление может быть равно, а ну вот все, теперь вот мы перешли.
Есть вот светлая область, это область двухфазного существования.
Вот красненькая точка, критическая точка, это вот точка, в которой сходятся линии разгазирования и конденсации.
Если мы проведём изотерму, равную критической температуре, то, что мы увидим, у нас при температуре ниже критической и вне области двухфазного существования, у нас жидкая фаза.
Для углеводородной системы это значит, что это нефть.
То есть для нефтяного месторождения, для нефтяного месторождения пластовая температура должна быть ниже, чем критическая температура смеси.
Продолжаем.
В области вне двухфазной, как мы уже с вами обсуждали, и температуре выше критической, у нас находится система в газовом агрегатном состоянии.
Поэтому область газовых месторождений, месторождений природных газов, для них характерно то, что пластовая температура превышает критическую температуру смеси.
Теперь ещё некие характерные величины.
Вот мы с вами проведём касательную, вертикальную касательную к области двухфазного существования.
И мы с вами, вот температура, соответствующая этой касательной называется крикондентерма.
При температуре, если у нас пластовая температура превышает критическую для смеси и меньше, чем крикондентерма, это область существования газоконденсатных месторождений, то есть месторождений газа с ретроградной конденсацией.
Если наша температура, пластовая температура превышает крикондентерму, то это месторождение природного газа, в котором нет ретроградной конденсации в пласте.
Мы об этом ещё поговорим, о разных типах флюидов.
Но вот уже на примере схемы фазового диаграммы, поскольку нужно несколько раз повторить, то вы обязательно лучше усвоите.
Опыт показывает, что не все усваивают, но нужно стараться.
Итак, если температура выше крикондентермы, у нас при изотермическом снижении давления у нас не будет процесса ретроградной конденсации.
Жидкая фаза не образуется, сколько бы мы не снижали
давление.
Поэтому это характерно именно для газовых месторождений.
Вот типа того, что я вам говорил, Сеноманская залежь Медвежьего месторождения, Уренгойского.
Сухие, когда у нас в основном метан.
Это месторождение сухого газа.
А вот гораздо более сложные процессы для моделирования, для понимания это месторождение газоконденсатные.
Это то, что здесь на слайде обозначено как месторождение газа с ретроградной конденсацией.
И мы более подробно рассмотрим, что же происходит.
А месторождения нефтяные, значит для них характерно, что наша пластовая температура меньше, чем критическая температура нашей системы.
И тогда мы имеем дело с нефтяным месторождением.
Вот такая вещь.
И существование, вот еще я замечу.
В частном случае критическое давление может быть равно криконденбаре.
Но это именно частный случай.
Но при этом критическая температура будет меньше крикондентермы.
Это относится...
Ну это вот...
Да.
Это...
Ну не важно, я это отметил.
Теперь я уже буду переходить к следующему слайду.

\begin{center}
\includegraphics[width=\textwidth, page=29]{Брусиловский.pdf}
\end{center}

Вот сейчас.
Вот.
Вот следующий слайд.
Значит мы с вами...
Зоны ретроградных явлений -- это очень важная вещь для газоконденсатных месторождений будут отличаться в зависимости от того, находится ли критическая точка слева или справа от точки на фазовой диаграмме соответствующей криконденбаре.
Еще раз смотрите.
Криконденбара -- это максимальное давление, при котором возможно двухфазное существование нашей смеси.
Криконденбара определяется касательно к двухфазной области.
Так вот.
Вот здесь у нас критическая точка обозначена $C$, а $G$ -- точка соответствует криконденбаре.
Это и на левом, и на правом рисунках.
Вот теперь отдельно.
Давайте рассмотрим на отдельных слайдах, чтобы более крупно
было.
Левый и правый рисунки.

\begin{center}
\includegraphics[width=\textwidth, page=30]{Брусиловский.pdf}
\end{center}

Вот это вот левый рисунок.
Вот этот рисунок относится к случаю, когда критическая точка находится слева от точки $G$, от криконденбары.
Я еще раз вам хочу сказать, это ни в одном учебнике не описано, но я очень хочу, чтобы вы понимали суть.
В традиционных учебниках приводятся фазовые диаграммы, которые вот на следующем слайде.
Вот такая.
Во всех учебниках по разработке месторождения приводится
такая диаграмма (см следующий слайд).
На самом деле, то есть, когда критическая точка у нас правее, чем криконденбара.
На самом деле, часто бывает так, это зависит от компонентного состава, что критическая точка у нас левее от криконденбары.
И просто мы должны посмотреть, а что же происходит в одном
и во втором случае.
И вот мы рассмотрим последние два процесса.
Первый процесс -- это изобарическое изменение температуры при давлении больше, чем критическое, но меньше, чем давление в точке $G$.
То есть, $P_G$.
Вот это у нас будет давление $P_1$.
И второй случай -- это изотермическое изменение давления при температуре больше, чем критическая температура и меньше, чем температура криконденбары.
Конечно, крикондентерма.
Я оговорился.
Вот точка $M$ соответствует крикондентерме.
То есть, еще раз.
Первый - изобарическое.
Мы рассмотрим случай.
Изобарическое изменение температуры при давлении выше, чем критическом и меньше, чем криконденбары.
И изотермическое значение давления при температуре выше критическим и меньше, чем крикондентерма.
Особенно нам будет с практической точки зрения важен второй
случай.
Но первый тоже мы рассмотрим для полноты.
Итак, начнем.
Изобарическое изменение температуры.
Чтобы вы просто понимали, что происходит с многокомпонентными углеводородными системами при изменении термобарических параметров.
Первое.
Изобарическое изменение температуры при давлении выше, чем критическое и меньше, чем криконденбара.
Пусть у нас давление равно $P_1$.
И температура равняется $T_A$.
В точке $A$.
Смесь находится в жидком состоянии.
Повышаем температуру.
При достижении критической температуры $T_C$ смесь плавно без образования поверхности раздела, переходит из жидкого состояния в газовое.
Мы не увидим ничего визуально.
Но теоретически мы перешли через критическую изотерму.
И значит мы перешли из жидкого агрегатного состояния в газовое агрегатное состояние.
Затем продолжаем повышать температуру изобарически.
В точке $D$ газовая фаза становится насыщенной.
То есть из неё выделяется первая капля жидкости.
При дальнейшем повышении температуры протекает
ретроградный процесс.
Вот мы впервые о нём говорим.
Ретроградный, то есть обратный, который противоречит нашим представлениям.
Дело в том, что наши представления показывают, что при повышении
температуры у нас жидкость не образуется.
У нас газ нагревается.
При повышении температуры, если у нас жидкость, то она будет испаряться.
А тут что происходит?
Вот у нас было газовое агрегатное состояние.
И в точке $D$ выделилась первая капля жидкости.
И при дальнейшем повышении температуры протекает ретроградный
процесс.
Конденсируется жидкая фаза, количество которой достигает
максимум в точке $E$.
При увеличении температуры процесс становится прямым.
Вот при повышении температуры, при температуре свыше, чем та, которая в точке $E$.
Жидкая фаза испаряется и в точке $F$ исчезает последняя капля.
То есть у нас система становится полностью в газовом агрегатном
состоянии.
Ну и понятно, что при последующем повышении температуры просто будет газ нагреваться.
Это вот изобарически.
Итак, на отрезке $DE$ при повышении температуры происходит процесс ретроградной конденсации.
При понижении температуры на этом отрезке происходит процесс ретроградного испарения жидкой фазы.
Чем ближе давление к критическому или к криконденбаре, тем короче интервал температур, на котором происходит ретроградное явление.
Область ретроградных явлений при изобарическом изменении температуры находится внутри замкнутой кривой $CEGDC$.
Вот области ретроградных явлений при изобарическом изменении температуры.
Теперь второй процесс.
Изотермическое изменение, и он нам с практической точки зрения даже более важен.
Изотермические процессы, они происходят при промысловом, в колоннах, в колоннах, в области, в промысловом, в колоннах, в колоннах, химической технологии, вот это вот.
В пласте изобарическое изменение температуры, это такой редчайший вариант, не такой актуальный.
Всё-таки нас больше интересует изотермическое изменение давления, связанное с, когда мы разрабатываем месторождение, например, на режиме истощения, у нас происходит изотермическая, у нас в пласте обычно постоянная температура, значит равная пластовой, и у нас происходит
изменение давления.
Вот теперь давайте рассмотрим изотермическое изменение давления при температуре больше, чем критическое, но меньше, чем кондентерма.
То есть это диапазон температур от $T_C$ до $T_M$.
Что же там будет происходить?
В случае, когда у нас критическая точка, вот она левее, чем
при конденбара, пусть $T=T_1$, и смесь находится в однофазном газовом состоянии, в точке $H$.
Вот эта точка $H$ -- однофазное газовое состояние.
Предположим, что это начальные пластовые условия термобарические.
Снижаем давление.
В точке $L$ смесь становится насыщенной, из неё выделяются
первые капли жидкости.
При дальнейшем снижении давления происходит ретроградный
процесс.
Мы снижаем давление, и вдруг конденсируется жидкая фаза, количество которой достигает максимума в точке $S$.
Почему опять же ретроградный процесс необычный, непривычный?
Когда мы снижаем давление, для нас привычно, для нас непривычно, что может образовываться жидкая фаза.
Вот когда мы снижаем давление в жидкости, в нефтяном пласте, в нефтяном газированной жидкости, снижаем давление, и образуется первый пузырёк газа.
Это нормально.
При дальнейшем снижении давления у нас разгазирование происходит.
Это прямые процессы.
То есть в нефтяных пластах у нас происходят прямые процессы.
А вот если у нас газоконденсатная система, то есть когда у нас температура, пластовая температура в диапазоне между критической и крикондентермой, там происходят обратные процессы.
И вот это вот, это очень-очень важно.
Это определяет вообще прогнозирование при разработке газоконденсатных месторождений.
И умение этого моделировать, это вот как раз применением уравнения состояния, математическое моделирование.
Вот это вот, что оно позволяет прогнозировать, моделировать.
Это процесс не простой, но это поддаётся моделированию.
Итак, при снижении давления, в точке $L$ у нас смесь становится насыщенной, и при дальнейшем снижении давления происходит ретроградный процесс, конденсируется жидкая фаза, количество которой достигает максимум в точке $S$.
При уменьшении давления в дальнейшем, процесс ниже давления, соответствующего точке $S$, процесс становится прямым, жидкая фаза испаряется и в точке $J$, внизу, исчезает.
Итак, на отрезке $LS$ при снижении давления происходит процесс ретроградной конденсации.
При повышении давления на этом отрезке происходит ретроградное испарение жидкой фазы.
Чем ближе температура к критической или к крикондентерме, тем короче интервал давлений, на котором происходит ретроградное явление.
Теперь, область ретроградных явлений наблюдается.
Область ретроградных явлений, наблюдающихся при изотермическом изменении давления, ограничена замкнутой кривой $CDGLFMSC$.
Вот это область ретроградных явлений.
И линия $MSC$ называется кривой максимальной конденсации, так как для любой температуры в диапазоне от критической до крикондентерме, максимальное количество жидкой фазы достигается при давлении, равном ординате соответствующей точке на линии $MSC$.
Это давление называется давлением максимальной конденсации.
При любой температуре в диапазоне от критической до крикондентермы на пунктирной кривой $CSM$ соответствующая точка соответствует давлению максимальной конденсации.
Линия $CDGLFM$ называется линией ретроградных точек росы.
Граница двухфазной области, начинающаяся в точке $M$ и уходящая вниз через точку $G$, называется линией прямых точек росы.
Таким образом, если критическая точка находится слева от крикондентермы, от точки $G$, то область ретроградных явлений, наблюдающихся при изобарическом изменении температуры является частью более крупной области ретроградных явлений, которая происходит при изотермическом изменении
давления.
То есть, область ретроградных явлений, наблюдающихся при изобарическом изменении температуры, при давлении $P_1$, она входит в более крупную область ретроградных явлений, которая происходит при изотермическом изменении давления.
Вот эта область $CDGLFMSC$.

Это первый случай, когда у нас критическая температура
была меньше, чем температура криконденбара.

\begin{center}
\includegraphics[width=\textwidth, page=31]{Брусиловский.pdf}
\end{center}

Вот теперь следующий рисунок.
Традиционно во всех учебниках пишут, переписывают просто из одного в другой, когда у нас критическая точка она выше, чем, правее, чем крикондентерма.
То есть, критическая температура, когда у нас больше, чем
температура, соответствующая крикондентерме.
Как и в рассмотренном сейчас только случае, процессы изобарического изменения температуры, изотермического изменения давления при определенных термобарических условиях сопровождаются ретроградными явлениями.
Они происходят соответственно в областях, ограниченных
замкнутыми линиями.
Вот теперь области ретроградных явлений, они не одна в другой, а они разделены, разделены критической точкой.
Вот одна область, она ограничена $CFGEC$.
И вторая область это $CLMSC$.
Однако, в отличие от случая, как я вот сейчас вам только
что сказал, рассмотренного ранее, в данном случае области
ретроградных явлений имеет только одну общую точку,
критическую точку $C$.
И кроме того, при изобарическом изменении температуры в области $CFGEC$, происходят иные процессы, чем в области, рассмотренной до этого, чем в случае рассмотренном для этого.
Теперь давайте нас для практической, с практической точки зрения интересует прежде всего процесс изотермического изменения давления.
Вот.
В области, вот давайте, изотермическое изменение давления, в области $CLMSC$ при изотермическом изменении давления, происходят ретроградные процессы, аналогично описанным выше процессам в области, вот на предыдущем, в области $CDGLFMSC$.
То есть, они похожи.
Изотермическое уменьшение давления сопровождается процессом ретроградной конденсации, а увеличение давления ретроградным испарением.
Линия $MSC$, линия, значит, это кривая максимальной конденсации, а линия $CLM$ -- линии ретроградных точек росы.
$MJN$ и ее продолжение вниз -- линия прямых точек росы.
То есть, у нас, еще раз, ретроградное явление происходит в области $CLMSC$ при изотермическом снижении давления.
Ну вот, на этом, ну то есть, те же самые процессы, но
я вам могу сказать, например, ну вот, предположим, начальные термобарические условия у нас в точке $H$, однофазное газовое состояние.
Мы снижаем давление, достигаем точки $L$.
Здесь у нас, значит, образуются первые капельки ретроградной жидкости, ну то есть, конденсата ретроградного.
При дальнейшем снижении давления будет увеличиваться объемная доля ретроградного конденсата, которая достигнет максимума в точке $S$.
Это все изотермический процесс при температуре $T_1$.
И затем, при снижении давления ниже точки $S$, происходит процесс прямого испарения, в который последняя капелька ретроградного конденсата, она исчезает в точке $J$.
И при дальнейшем снижении давления у нас, значит, чисто газовая фаза.
Что я хотел бы отметить, что, значит, вот это вот давление соответствует точке $J$, оно очень низкое обычно.
На практике оно никогда не достигается.
Поэтому, если у нас выпадает ретроградный конденсат, то это, ну выпадает, то есть, вот становится меньше конденсата в добываемом газе.
Значит, и кроме того, я хочу сказать, ну это вот просто я еще повторю.
Значит, это вот теоретически, это термодинамика многокомпонентных систем.
На практике, значит, у нас разные, значит, физические явления происходят, которые мешают прямому испарению выпавшего ретроградного конденсата.
И это не позволяет увеличивать конденсатоотдачу.
Но тем не менее, когда мы моделируем с применением уравнения состояния, с помощью модели многокомпонентной фильтрации, это для некоторых из вас я акцентирую, мы там используем классическую термодинамику.
И у нас именно, если мы будем рассматривать фазовые диаграммы, вот такие результаты при моделировании получатся.
Так что это вот все очень полезные сведения.
А как это расходится с практикой, это уже другое дело.
Совсем другое дело.

\begin{center}
\includegraphics[width=\textwidth, page=32]{Брусиловский.pdf}
\end{center}

Тут есть упражнения.
В прошлом году эти все упражнения, они посылались на вашу
кафедру.
Значит, вы, пожалуйста, знаете, скопируйте себе эти задания.
И они простые.
И вот это упражнение, оно соответствует просто тому, о чем я рассказывал.
И там нужно на, по-моему, сейчас.
Я, конечно, могу с вами сейчас рассмотреть это упражнение.
Но лучше, если это вы сами сделаете

\subsection{Типы пластовых углеводородных смесей}

\begin{center}
\includegraphics[width=\textwidth, page=33]{Брусиловский.pdf}
\end{center}

\begin{center}
\includegraphics[width=\textwidth, page=34]{Брусиловский.pdf}
\end{center}

\begin{center}
\includegraphics[width=\textwidth, page=35]{Брусиловский.pdf}
\end{center}

\begin{center}
\includegraphics[width=\textwidth, page=36]{Брусиловский.pdf}
\end{center}

\begin{center}
\includegraphics[width=\textwidth, page=37]{Брусиловский.pdf}
\end{center}

\begin{center}
\includegraphics[width=\textwidth, page=38]{Брусиловский.pdf}
\end{center}

\begin{center}
\includegraphics[width=\textwidth, page=39]{Брусиловский.pdf}
\end{center}

\begin{center}
\includegraphics[width=\textwidth, page=40]{Брусиловский.pdf}
\end{center}

\begin{center}
\includegraphics[width=\textwidth, page=41]{Брусиловский.pdf}
\end{center}

\begin{center}
\includegraphics[width=\textwidth, page=42]{Брусиловский.pdf}
\end{center}

\begin{center}
\includegraphics[width=\textwidth, page=43]{Брусиловский.pdf}
\end{center}

\begin{center}
\includegraphics[width=\textwidth, page=44]{Брусиловский.pdf}
\end{center}

\begin{center}
\includegraphics[width=\textwidth, page=45]{Брусиловский.pdf}
\end{center}

\begin{center}
\includegraphics[width=\textwidth, page=46]{Брусиловский.pdf}
\end{center}

\begin{center}
\includegraphics[width=\textwidth, page=47]{Брусиловский.pdf}
\end{center}

\begin{center}
\includegraphics[width=\textwidth, page=48]{Брусиловский.pdf}
\end{center}

\begin{center}
\includegraphics[width=\textwidth, page=49]{Брусиловский.pdf}
\end{center}

\begin{center}
\includegraphics[width=\textwidth, page=50]{Брусиловский.pdf}
\end{center}

\end{document}