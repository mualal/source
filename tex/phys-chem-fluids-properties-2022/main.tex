% !TeX spellcheck = en_US
% !TeX program = xelatex

\documentclass[a4paper,12pt]{article}
\renewcommand{\baselinestretch}{1.2}
\usepackage[utf8]{inputenc}
\usepackage[T2A, T1]{fontenc}
\usepackage[english, russian]{babel}

\usepackage{fontspec}
\setmainfont{Times New Roman}
\usepackage{setspace, amsmath}
\usepackage{amssymb}
\usepackage{dsfont}
\usepackage{epsfig}

\makeatletter
\let\@fnsymbol\@arabic
\makeatother

\usepackage{geometry}
\geometry{
a4paper,
total={170mm, 257mm},
left=20mm,
top=20mm,
}

\usepackage{systeme}
\usepackage{skak}
\usepackage{mathtools}
\usepackage{unicode-math}
\usepackage{array}
\usepackage{makecell}
\usepackage{subfiles}
\usepackage{hyperref}
\hypersetup{pdfstartview=FitH, linkcolor=black, urlcolor=blue, colorlinks=true}
\usepackage{framed}
\usepackage{graphicx}
\usepackage{caption}
\usepackage{subcaption}
\usepackage{color}
\usepackage{chngcntr}
\usepackage{tikz}
\usepackage{csquotes}
\usepackage{fancyhdr}

\pagestyle{fancy}
\fancyhf{}
\fancyhead[L]{\leftmark}
\fancyfoot[C]{\thepage}

\usepackage{float}
\floatstyle{plaintop}
\usepackage{enumitem}
\setlength{\parindent}{0pt}

\graphicspath{{./img/}}
\newcommand{\myPictWidth}{.95\textwidth}
\newcommand{\phm}{\phantom{-}}
\newcommand{\beq}{\begin{equation}}
\newcommand{\eeq}{\end{equation}}


\begin{document}
	\tableofcontents
	\title{Физико-химические свойства пластовых флюидов\\Конспект лекций}
	\author{Муравцев А.А.\thanks{конспектирует; email: almuravcev@yandex.ru}
	\and
	Брусиловский А.И.\thanks{лектор, Высшая школа теоретической механики, Санкт-Петербургский Политехнический университет. Дополнительные материалы к лекциям \href{https://csspbstu-my.sharepoint.com/:f:/g/personal/muravtsev_aa_edu_spbstu_ru/Epiacj6WFMBHqIF6E3YQgCMB7yi5NAA1ycqFLqrTZMhJ4w?e=i2agP0}{доступны по ссылке}.}}
	\maketitle
	\subfile{parts/2022_10_10}
	\newpage
	\subfile{parts/2022_10_24}
	\newpage
	\subfile{parts/2022_10_25}
	\newpage
	\subfile{parts/2022_10_26}
	\newpage
	\subfile{parts/2022_10_27}
\end{document}
