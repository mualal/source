\documentclass[main.tex]{subfiles}

\begin{document}

% \textcolor{red}{Вводная лекция}

\section{Вводная лекция (Базыров И.Ш.)}

\subsection{Озеро Пёнье (Peigneur, Пенёр)}

Пресноводное озеро площадью около $5\text{ км}^2$ со средней глубиной $3.3$ метра за один день (21 ноября 1980 года) стало солёным с максимальной глубиной около $396$ метров.

Ошиблись при расчётах траектории бурения скважины и пробурили её в соляную шахту. 
Вода из озера, двигаясь по скважине под действием силы тяжести, начала размывать соляную шахту.
На поверхности озера образовалась огромная воронка. И за считанные минуты глубина озера возросла в несколько раз.

Вывод. Необходимо очень серьёзно относится к своим инженерным решениям!

\subsection{История развития гидродинамических моделей}

\enquote{...В сущности все модели неправильные, но некоторые полезны...} Дж. Бокс

Создание 3D ГДМ не позволяет решить все задачи на всех месторождениях, необходима иерархия моделей.

Упрощённые модели могут решить только часть задач: некоторые процессы с помощью упрощённых моделей описать нельзя (а именно нелинейные процессы, при протекании которых происходит много сопутствующих процессов).

1) Аналитические модели (примерно с 1920 г.):
\begin{itemize}
	\item Buckley Leverett
	\item Muscat
	\item Dykstra Parson
	\item Arps
\end{itemize}
2) Численные модели / развитие симуляторов:
\begin{itemize}
	\item Nelson Pope
	\item CMG Suite
	\item BOAST
	\item ECLIPSE
	\item RMS
	\item Nexus
	\item tNavigator
	\item Intersect
	\item OPM
\end{itemize}
3) Необходимость развития упрощённых моделей и необходимость иерархии моделей:
\begin{itemize}
	\item CRM
\end{itemize}

\subsection{Что такое гидродинамическое моделирование?}
1) Набор уравнений:
\begin{itemize}
	\item неразрывность потока
	\beq\label{Continuity}
	\frac{\partial\left(\rho_f\varphi\right)}{\partial t}+\pmb{\nabla}\cdot\left(\rho_f\varphi \pmb{v_f}\right)=q_f(\pmb{x})
	\eeq
	\item закон Дарси
	\beq\label{Darcy}
	\pmb{W}=-\frac{k}{\mu_f}\cdot\pmb{\nabla} p
	\eeq
	\item сжимаемость флюида
	\beq\label{Compressibility}
	p-p_0=K_f\frac{\rho_f-\rho_f^0}{\rho_f^0}
	\eeq
\end{itemize}

На этих уравнениях строится основное уравнение гидродинамики пласта -- уравнение пьезопроводности.

2) Насыщенности и относительные фазовые проницаемости (для нескольких флюидов)

3) Геометрия (сложное строение пласта)

\subsubsection{Уравнение пьезопроводности (без упругости пласта)}

В предположении неподвижности скелета ($\pmb{v_s}\approx \pmb{0}$ и $\varphi(t)=\textrm{const}$) верно равенство $\pmb{W}\approx\varphi \pmb{v_f}$. Подставляя в закон Дарси \eqref{Darcy}, получаем:
\beq\label{DarcyWithSkeletNotMoving}
\varphi \pmb{v_f}=-\frac{k}{\mu_f}\cdot\pmb{\nabla} p
\eeq

Условие сжимаемости флюида \eqref{Compressibility} перепишем в дифференциальной форме:
\beq\label{CompressibilityDiff}
\frac{\partial p}{\partial t}=\frac{K_f}{\rho_f^0}\frac{\partial\rho_f}{\partial t}
\eeq

Учитывая предположение о неподвижности скелета, перепишем уравнение неразрывности потока:
\beq\label{ContinuityWithSkeletNotMoving}
\varphi\frac{\partial\rho_f}{\partial t}+\pmb{\nabla}\cdot\left(\rho_f\varphi\pmb{v_f}\right)=q_f(\pmb{x})
\eeq

Подставляя \eqref{DarcyWithSkeletNotMoving} и \eqref{CompressibilityDiff} в \eqref{ContinuityWithSkeletNotMoving}, при отсутствии источникового слагаемого ($q_f(\pmb{x})=0$) получаем:
\beq
\varphi\frac{\rho_f^0}{K_f}\frac{\partial p}{\partial t}-\pmb{\nabla}\cdot\left(\rho_f\frac{k}{\mu_f}\pmb{\nabla} p\right)=0
\eeq

При дополнительном условии слабосжимаемости флюида ($\rho_f\approx\rho_f^0=\textrm{const}$) получаем:
\beq
\frac{\partial p}{\partial t}=\frac{kK_f}{\mu_f\varphi}\pmb{\nabla}^2p
\eeq

Это уравнение пьезопроводности (без упругости пласта), полученное в приближении слабосжимаемого флюида, неподвижного и недеформируемого пласта.

\subsubsection{Уравнение пьезопроводности (в случае упругого пласта)}

Задача со звёздочкой.

\subsection{Пороупругость}

Использует методы механики сплошных сред к пористым средам.

В нефтяной индустрии описывает взаимозависимость величин пластового давления и деформаций пористой среды с изменением напряжённо-деформированного состояния во время разработки месторождений.

Матрица - материал, из которого сделан пористый скелет.

В твёрдой части выполняются уравнения упругости.

В уравнениях пороупругости величины, относящиеся к матрице, обозначаются индексом s, к жидкости - индексом f, к пористому скелету - без индексов.
Например, уравнение для плотности:
\beq
\rho=\rho_s\left(1-\varphi\right)+\rho_f\varphi
\eeq

Общая схема решения упругих (пороупругих) задач:
\begin{itemize}
\item определиться с неизвестными задачи (напряжения, деформации, перемещение, давление и т.п.)
\item сформулировать законы сохранения (закон сохранения массы - ЗСМ; закон сохранения количества движения - ЗСИ; закон сохранения момента количества движения - ЗСМИ; закон сохранения энергии - ЗСЭ; закон неубывания энтропии)
\item сформулировать кинематические соотношения и условия сплошности (связь между перемещениями и деформациями)
\item сформулировать определяющие уравнения (закон Гука - связь между напряжениями и деформациями; связь между плотностью среды и давлением)
\item сформулировать граничные и начальные условия
\end{itemize}

\subsection{Гидрогеомеханическое моделирование: компьютерное моделирование, виды совмещения}

\subsubsection{Виды гидро-геомеханического совмещения}

\begin{enumerate}
\item \underline{Постоянная сжимаемость}: объём пор является единственной функцией порового давления $V_p=f(P_p)$
\item \underline{Псевдосовмещение}: уплотнение и изменение горизонтального напряжения вычисляются с помощью простых соотношений между пористостью, проницаемостью и напряжением соответственно
\item \underline{Односторонняя связь}: информация передаётся только одним способом с модуля симулятора на геомеханику
\item \underline{Итеративный способ совмещения}: уникальная взаимосвязь между объёмом пор и изменением порового давления используется для оценки изменения объёма пор в моделировании пласта
\item \underline{Полное совмещение}: этот метод является глобальным симуляционным решением одновременно и неявно уравнения потока и уравнения структурного анализа в тех же линейных системах
\end{enumerate}

От первого к последнему возрастает качество, но падает скорость.


%\begin{figure}
%	\includegraphics[width=\textwidth]{1}
%\end{figure}

\end{document}
