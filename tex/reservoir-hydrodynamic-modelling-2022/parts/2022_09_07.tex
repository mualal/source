\documentclass[main.tex]{subfiles}

\begin{document}

% \textcolor{red}{Вводная лекция}

\section{Вводная лекция 07.09.2022 (Базыров И.Ш.)}

\subsection{Озеро Пёнье (Peigneur, Пенёр)}

Пресноводное озеро площадью около $5\text{ км}^2$ со средней глубиной $3.3$ метра за один день (21 ноября 1980 года) стало солёным с максимальной глубиной около $396$ метров.

Ошиблись при расчётах траектории бурения скважины и пробурили её в соляную шахту. 
Вода из озера, двигаясь по скважине под действием силы тяжести, начала размывать соляную шахту.
На поверхности озера образовалась огромная воронка. И за считанные минуты глубина озера возросла в несколько раз.

\textbf{Вывод.} Необходимо очень серьёзно относиться к своим инженерным решениям, ведь они очень сильно влияют на нашу с вами жизнь!

\subsection{История развития гидродинамических моделей}

\enquote{...В сущности все модели неправильные, но некоторые полезны...} Дж. Бокс

Создание 3D ГДМ не позволяет решить все задачи на всех месторождениях, необходима иерархия моделей.

Упрощённые модели могут решить только часть задач: некоторые процессы с помощью упрощённых моделей описать нельзя (а именно нелинейные процессы, при протекании которых происходит много сопутствующих процессов).

1) Аналитические модели (примерно с 1920 г.):
\begin{itemize}
	\item Buckley Leverett
	\item Muscat
	\item Dykstra Parson
	\item Arps
\end{itemize}
2) Численные модели / развитие симуляторов:
\begin{itemize}
	\item Nelson Pope
	\item CMG Suite
	\item BOAST
	\item ECLIPSE
	\item RMS
	\item Nexus
	\item tNavigator
	\item Intersect
	\item OPM \\
	Многие начали думать, что симулятор может смоделировать всё, но это неверно. И если не разбираться в физике, а просто, не понимая происходящего, строить модели в симуляторе, то вряд получится смоделировать что-нибудь разумное.
\end{itemize}

3) Необходимость развития упрощённых моделей и необходимость иерархии моделей:
\begin{itemize}
	\item CRM \\
	Чем нормальный инженер отличается от новичка: новичок просто считает в т-Навигаторе, выдаёт решение и говорит, что вот моё решение. Далее идёт к эксперту, который говорит ему, что это решение точно неверное. Новичок просто не понимает, откуда взялась эта экспертная оценка, а эксперты просто хорошо разбираются в упрощённых моделях, прикидывают порядок значений, которые должны получиться, и легко разбивают новичков.
\end{itemize}

\subsection{Что такое гидродинамическое моделирование?}
1) Набор уравнений:
\begin{itemize}
	\item неразрывность потока (уравнение переноса массы, записанное в дифференциальной форме)
	\beq\label{Continuity}
	\frac{\partial\left(\rho_f\varphi\right)}{\partial t}+\pmb{\nabla}\cdot\left(\rho_f\varphi \pmb{v_f}\right)=q_f(\pmb{x})
	\eeq
	\item закон Дарси
	\beq\label{Darcy}
	\pmb{W}=-\frac{k}{\mu_f}\cdot\pmb{\nabla} p
	\eeq
	\item сжимаемость флюида
	\beq\label{Compressibility}
	p-p_0=K_f\frac{\rho_f-\rho_f^0}{\rho_f^0}
	\eeq
\end{itemize}

На этих уравнениях строится основное уравнение гидродинамики пласта -- уравнение пьезопроводности.

2) Насыщенности и относительные фазовые проницаемости (для нескольких флюидов)

3) Геометрия (сложное строение пласта)

\subsubsection{Уравнение пьезопроводности (без упругости пласта)}

В предположении неподвижности скелета ($\pmb{v_s}\approx \pmb{0}$ и $\varphi(t)=\textrm{const}$) верно равенство $\pmb{W}\approx\varphi \pmb{v_f}$. Подставляя в закон Дарси \eqref{Darcy}, получаем:
\beq\label{DarcyWithSkeletNotMoving}
\varphi \pmb{v_f}=-\frac{k}{\mu_f}\cdot\pmb{\nabla} p
\eeq

Условие сжимаемости флюида \eqref{Compressibility} перепишем в дифференциальной форме:
\beq\label{CompressibilityDiff}
\frac{\partial p}{\partial t}=\frac{K_f}{\rho_f^0}\frac{\partial\rho_f}{\partial t}
\eeq

Учитывая предположение о неподвижности скелета, перепишем уравнение неразрывности потока:
\beq\label{ContinuityWithSkeletNotMoving}
\varphi\frac{\partial\rho_f}{\partial t}+\pmb{\nabla}\cdot\left(\rho_f\varphi\pmb{v_f}\right)=q_f(\pmb{x})
\eeq

Подставляя \eqref{DarcyWithSkeletNotMoving} и \eqref{CompressibilityDiff} в \eqref{ContinuityWithSkeletNotMoving}, при отсутствии источникового слагаемого ($q_f(\pmb{x})=0$) получаем:
\beq
\varphi\frac{\rho_f^0}{K_f}\frac{\partial p}{\partial t}-\pmb{\nabla}\cdot\left(\rho_f\frac{k}{\mu_f}\pmb{\nabla} p\right)=0
\eeq

При дополнительном условии слабосжимаемости флюида ($\rho_f\approx\rho_f^0=\textrm{const}$) получаем:
\beq
\frac{\partial p}{\partial t}=\frac{kK_f}{\mu_f\varphi}\pmb{\nabla}^2p
\eeq

Это уравнение пьезопроводности (без упругости пласта), полученное в приближении слабосжимаемого флюида, неподвижного и недеформируемого пласта.

\subsubsection{Уравнение пьезопроводности (в случае упругого пласта)}

Для упругого изотропного пласта можем записать известные соотношения пороупругости:
\begin{itemize}[parsep=-5pt]
\item на тензор полных напряжений \\
\beq
\pmb{T}=\sigma^0\pmb{I}+\left(\lambda I_1(\pmb{\varepsilon})-b\Delta p\right)\pmb{I}+2\mu\pmb{\varepsilon},
\eeq
где $\pmb{T}$ -- тензор полных напряжений; $\pmb{I}$ -- единичный тензор; $\pmb{\varepsilon}$ -- тензор полных деформаций; $\lambda=K-2G/3$ и $\mu=G$ -- константы (параметры) Ляме; $K$ -- модуль всестороннего сжатия; $G$ -- модуль сдвига; $I_1(\pmb{\varepsilon})$ -- след тензора полных деформаций; $b$ -- константа Био; $\Delta p$ -- изменение давления; $\sigma^0$ -- начальное напряжение
\item на пористость \\
\beq
\varphi = \varphi_0+bI_1(\pmb{\varepsilon})+\dfrac{1}{N}\Delta p,
\eeq
где $\varphi_0$ -- начальная пористость; $b$ -- константа Био; $I_1(\pmb{\varepsilon})$ -- след тензора полных деформаций; $N$ -- модуль Био; $\Delta p$ -- изменение давления.
\item условие равновесия \\
\beq
\pmb{\nabla}\cdot\pmb{T}=\pmb{0}
\eeq
\end{itemize}

Для флюида запишем:
\begin{itemize}[parsep=-5pt]
	\item закон Дарси \\
	\beq
	\pmb{W}=-\frac{k}{\mu_f}\cdot\pmb{\nabla} p,
	\eeq
	где $\pmb{W}=\pmb{v_f}-\pmb{v_s}$; $k$ -- проницаемость пласта; $\mu_f$ -- вязкость флюида; $\pmb{\nabla} p$ -- градиент давления
	\item условие на сжимаемость флюида
	\beq
	p-p_0=K_f\frac{\rho_f-\rho_f^0}{\rho_f^0},
	\eeq
	где $K_f$ -- сжимаемость флюида
	\item уравнение неразрывности потока при отсутствии источникового слагаемого (уравнение переноса массы, записанное в дифференциальной форме):
	\beq
	\frac{\partial\left(\rho_f\varphi\right)}{\partial t}+\pmb{\nabla}\cdot\left(\rho_f\varphi\,\pmb{v_f}\right)=0
	\eeq
\end{itemize}

Из уравнения неразрывности получаем:
\beq
\varphi_0\frac{\partial\rho_f}{\partial t}+\rho_0\frac{\partial\varphi}{\partial t}+\rho_0\pmb{\nabla}\cdot\pmb{W}+\rho_0\varphi_0\frac{\partial I_1(\pmb{\varepsilon})}{\partial t}=0,
\eeq
где
\beq
\frac{\partial I_1(\pmb{\varepsilon})}{\partial t}\equiv\pmb{\nabla}\cdot\pmb{v_s}
\eeq

А дальше через ряд свёрток и всяких операций получаем:
\beq
b\dot{I}_1(\pmb{\varepsilon})+\left(\frac{1}{N}+\frac{\varphi_0}{K_f}\right)\dot{p}=\frac{k}{\mu_f}\pmb{\nabla}^2p
\eeq

В осесимметричном случае при условии отсутствия деформации на бесконечности получаем:
\beq
\dot{p}=a\pmb{\nabla}^2p,
\eeq
где
\beq
a=\frac{kM}{\mu_f}\text{ и } M=\frac{b\left(b+\varphi\right)}{\lambda+2\mu}+\frac{1}{N}+\frac{\varphi}{K_f}
\eeq

\subsection{Пороупругость}

Пороупругость использует методы механики сплошных сред к пористым средам.

В нефтяной индустрии пороупругость описывает взаимозависимость величин пластового давления и деформаций пористой среды с изменением напряжённо-деформированного состояния во время разработки месторождений.

Матрица -- материал, из которого сделан пористый скелет.

В твёрдой части выполняются уравнения упругости.

В уравнениях пороупругости величины, относящиеся к матрице, обозначаются индексом s, к жидкости - индексом f, к пористому скелету - без индексов.
Например, уравнение для плотности:
\beq
\rho=\rho_s\left(1-\varphi\right)+\rho_f\varphi
\eeq

Общая схема решения упругих (пороупругих) задач:
\begin{itemize}
\item определиться с неизвестными задачи (напряжения, деформации, перемещение, давление и т.п.)
\item сформулировать законы сохранения (закон сохранения массы - ЗСМ; закон сохранения количества движения - ЗСИ; закон сохранения момента количества движения - ЗСМИ; закон сохранения энергии - ЗСЭ; закон неубывания энтропии)
\item сформулировать кинематические соотношения и условия сплошности (связь между перемещениями и деформациями)
\item сформулировать определяющие уравнения (закон Гука - связь между напряжениями и деформациями; связь между плотностью среды и давлением)
\item сформулировать граничные и начальные условия
\end{itemize}

\subsection{Гидрогеомеханическое моделирование: компьютерное моделирование, виды совмещения}

\subsubsection{Виды гидро-геомеханического совмещения}

\begin{enumerate}
\item \underline{Постоянная сжимаемость}: объём пор является единственной функцией порового давления $V_p=f(P_p)$
\item \underline{Псевдосовмещение}: уплотнение и изменение горизонтального напряжения вычисляются с помощью простых соотношений между пористостью, проницаемостью и напряжением соответственно
\item \underline{Односторонняя связь}: информация передаётся только одним способом с модуля симулятора на геомеханику
\item \underline{Итеративный способ совмещения}: уникальная взаимосвязь между объёмом пор и изменением порового давления используется для оценки изменения объёма пор в моделировании пласта
\item \underline{Полное совмещение}: этот метод является глобальным симуляционным решением одновременно и неявно уравнения потока и уравнения структурного анализа в тех же линейных системах
\end{enumerate}

От первого к последнему возрастает качество, но падает скорость.


%\begin{figure}
%	\includegraphics[width=\textwidth]{1}
%\end{figure}

\end{document}
