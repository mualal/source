\documentclass[main.tex]{subfiles}

\begin{document}


\section{Лекция 12.09.2022}

\subsection{Цели курса}

\subsection{Что такое модель?}

\subsection{Требования к моделям}

\subsection{Точность моделей}

\subsection{Виды моделей}

\subsection{Гидродинамическая модель}

\subsection{Цели гидродинамического моделирования (ГДМ)}

\subsection{Математическая основа ГДМ}

\subsection{Типы сеток ГДМ}

\subsection{Типы сеток ГДМ. LGR}

\subsection{Порядок нумерации ячеек сетки}

\subsection{Структура файла исходных данных для симулятора ECLIPSE}

\subsection{Справочники для симулятора ECLIPSE}

\subsection{Задание свойств в ячейках}

\subsection{Поток через ячейку}

\subsection{Несоседние соединения NNC}

\subsection{Проблемы пространственной дискретизации}

\subsection{Построение грида}

\subsection{Гидродинамические модели (схема)}

\subsection{Типы расчётных моделей}

\subsubsection{Модель нелетучей нефти}

\subsubsection{Композиционная модель}

\subsubsection{Термические модели}

\subsubsection{Модель двойной или мульти-среды}

\subsubsection{Модели линий тока}

\subsubsection{Proxy-модели}

\subsubsection{Суррогатные (мета) модели}

\subsection{Иерархия гидродинамических моделей}

\subsection{Местоположение моделирования в цикле нефтедобычи}

\subsection{Источники геологической информации в масштабе месторождения}

\subsection{Охват исследованием и погрешность}

\subsection{Исходные данные для гидродинамического моделирования}

\subsection{Подходы к построению ПДГГДМ}

\subsection{Ремасштабирование геомодели}

\subsection{Ремасштабирование структуры (upgridding)}

\subsection{Ремасштабирование свойств}

\subsection{Ремасштабирование проницаемости}

\subsection{Ремасштабирование геомодели. Контроль качества}

\subsection{Поверхностное натяжение}

\subsection{Смачиваемость}

\subsection{Капиллярное давление}

\subsection{Капиллярное давление для разных типов породы}

\subsection{J-функция Леверетта}

\subsection{Капиллярное давление. Лабораторные исследования}

\subsection{ОФП}

\subsection{Смачиваемость. Критерий Craig (1971)}

\subsection{Гистерезис ОФП}

\subsection{ОФП. Лабораторные исследования}

\subsection{Концевые точки ОФП в системе нефть-вода}

\subsection{Масштабирование ОФП}

\subsubsection{По горизонтали}

\subsubsection{По вертикали}

\subsection{Согласованность массивов в модели}

\subsection{Ремасштабирование (2-х фазный апскелинг ОФП)}

\subsection{Типы флюидов}

\subsection{Определение типа залежи по составу УВ}

\subsection{PVT-свойства}

\subsection{PVT-свойства нефти}

\subsection{PVT-свойства нефти. Корреляции}

\subsection{PVT-свойства "<живой нефти">}

\subsection{Варианты описания PVT в моделях Black Oil}

\subsection{Сжимаемость порового пространства}

\subsection{Сжимаемость порового пространства. Корреляции}

\end{document}
