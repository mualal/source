\documentclass[main.tex]{subfiles}

\begin{document}

\section{Лекция 12.09.2022 (Кайгородов С.В.)}

\subsection{Цели курса}

\includegraphics[width=\textwidth, page=7]{Kurs_OsnovyGDM_Kai_774_gorodovSV_v6_0.pdf}

\includegraphics[width=\textwidth, page=10]{Kurs_OsnovyGDM_Kai_774_gorodovSV_v6_0.pdf}

\subsection{Каналы распространения знаний по ГДМ}

\includegraphics[width=\textwidth, page=12]{Kurs_OsnovyGDM_Kai_774_gorodovSV_v6_0.pdf}

\subsection{Что такое модель?}

\includegraphics[width=\textwidth, page=15]{Kurs_OsnovyGDM_Kai_774_gorodovSV_v6_0.pdf}

\subsection{Требования к моделям}

\includegraphics[width=\textwidth, page=16]{Kurs_OsnovyGDM_Kai_774_gorodovSV_v6_0.pdf}

\subsection{Точность моделей}

\includegraphics[width=\textwidth, page=17]{Kurs_OsnovyGDM_Kai_774_gorodovSV_v6_0.pdf}

\subsection{Виды моделей}

\includegraphics[width=\textwidth, page=18]{Kurs_OsnovyGDM_Kai_774_gorodovSV_v6_0.pdf}

\subsection{Гидродинамическая модель}

\includegraphics[width=\textwidth, page=19]{Kurs_OsnovyGDM_Kai_774_gorodovSV_v6_0.pdf}

\subsection{Цели гидродинамического моделирования (ГДМ)}

\includegraphics[width=\textwidth, page=20]{Kurs_OsnovyGDM_Kai_774_gorodovSV_v6_0.pdf}

\subsection{Математическая основа ГДМ}

\includegraphics[width=\textwidth, page=21]{Kurs_OsnovyGDM_Kai_774_gorodovSV_v6_0.pdf}

\includegraphics[width=\textwidth, page=22]{Kurs_OsnovyGDM_Kai_774_gorodovSV_v6_0.pdf}

\subsection{Типы сеток ГДМ}

\includegraphics[width=\textwidth, page=23]{Kurs_OsnovyGDM_Kai_774_gorodovSV_v6_0.pdf}

\includegraphics[width=\textwidth, page=24]{Kurs_OsnovyGDM_Kai_774_gorodovSV_v6_0.pdf}

\subsection{Типы сеток ГДМ. LGR}

\includegraphics[width=\textwidth, page=25]{Kurs_OsnovyGDM_Kai_774_gorodovSV_v6_0.pdf}

\subsection{Порядок нумерации ячеек сетки}

\includegraphics[width=\textwidth, page=26]{Kurs_OsnovyGDM_Kai_774_gorodovSV_v6_0.pdf}

\subsection{Структура файла исходных данных для симулятора ECLIPSE}

\includegraphics[width=\textwidth, page=27]{Kurs_OsnovyGDM_Kai_774_gorodovSV_v6_0.pdf}

\subsection{Справочники для симулятора ECLIPSE}

\includegraphics[width=\textwidth, page=28]{Kurs_OsnovyGDM_Kai_774_gorodovSV_v6_0.pdf}

\subsection{Задание свойств в ячейках}

\includegraphics[width=\textwidth, page=29]{Kurs_OsnovyGDM_Kai_774_gorodovSV_v6_0.pdf}

\includegraphics[width=\textwidth, page=30]{Kurs_OsnovyGDM_Kai_774_gorodovSV_v6_0.pdf}

\includegraphics[width=\textwidth, page=31]{Kurs_OsnovyGDM_Kai_774_gorodovSV_v6_0.pdf}

\subsection{Поток через ячейку}

\includegraphics[width=\textwidth, page=32]{Kurs_OsnovyGDM_Kai_774_gorodovSV_v6_0.pdf}

\includegraphics[width=\textwidth, page=33]{Kurs_OsnovyGDM_Kai_774_gorodovSV_v6_0.pdf}

\includegraphics[width=\textwidth, page=34]{Kurs_OsnovyGDM_Kai_774_gorodovSV_v6_0.pdf}

\subsection{Несоседние соединения NNC}

\includegraphics[width=\textwidth, page=35]{Kurs_OsnovyGDM_Kai_774_gorodovSV_v6_0.pdf}

\subsection{Проблемы пространственной дискретизации}

\includegraphics[width=\textwidth, page=36]{Kurs_OsnovyGDM_Kai_774_gorodovSV_v6_0.pdf}

\includegraphics[width=\textwidth, page=37]{Kurs_OsnovyGDM_Kai_774_gorodovSV_v6_0.pdf}

\includegraphics[width=\textwidth, page=38]{Kurs_OsnovyGDM_Kai_774_gorodovSV_v6_0.pdf}

\subsection{Построение грида}

\includegraphics[width=\textwidth, page=39]{Kurs_OsnovyGDM_Kai_774_gorodovSV_v6_0.pdf}

\subsection{Гидродинамические модели (схема)}

\includegraphics[width=\textwidth, page=40]{Kurs_OsnovyGDM_Kai_774_gorodovSV_v6_0.pdf}

\subsection{Типы расчётных моделей}

\subsubsection{Модель нелетучей нефти}

\includegraphics[width=\textwidth, page=41]{Kurs_OsnovyGDM_Kai_774_gorodovSV_v6_0.pdf}

\subsubsection{Композиционная модель}

\includegraphics[width=\textwidth, page=42]{Kurs_OsnovyGDM_Kai_774_gorodovSV_v6_0.pdf}

\subsubsection{Термические модели}

\includegraphics[width=\textwidth, page=43]{Kurs_OsnovyGDM_Kai_774_gorodovSV_v6_0.pdf}

\subsubsection{Модель двойной или мульти-среды}

\includegraphics[width=\textwidth, page=44]{Kurs_OsnovyGDM_Kai_774_gorodovSV_v6_0.pdf}

\subsubsection{Модели линий тока}

\includegraphics[width=\textwidth, page=45]{Kurs_OsnovyGDM_Kai_774_gorodovSV_v6_0.pdf}

\subsubsection{Proxy-модели}

\includegraphics[width=\textwidth, page=46]{Kurs_OsnovyGDM_Kai_774_gorodovSV_v6_0.pdf}

\includegraphics[width=\textwidth, page=47]{Kurs_OsnovyGDM_Kai_774_gorodovSV_v6_0.pdf}

\subsubsection{Суррогатные (мета) модели}

\includegraphics[width=\textwidth, page=48]{Kurs_OsnovyGDM_Kai_774_gorodovSV_v6_0.pdf}

\includegraphics[width=\textwidth, page=49]{Kurs_OsnovyGDM_Kai_774_gorodovSV_v6_0.pdf}

\includegraphics[width=\textwidth, page=50]{Kurs_OsnovyGDM_Kai_774_gorodovSV_v6_0.pdf}

\subsection{Иерархия гидродинамических моделей}

\includegraphics[width=\textwidth, page=51]{Kurs_OsnovyGDM_Kai_774_gorodovSV_v6_0.pdf}

\subsection{Местоположение моделирования в цикле нефтедобычи}

\includegraphics[width=\textwidth, page=52]{Kurs_OsnovyGDM_Kai_774_gorodovSV_v6_0.pdf}

\includegraphics[width=\textwidth, page=53]{Kurs_OsnovyGDM_Kai_774_gorodovSV_v6_0.pdf}

\subsection{Источники геологической информации в масштабах месторождения}

\includegraphics[width=\textwidth, page=54]{Kurs_OsnovyGDM_Kai_774_gorodovSV_v6_0.pdf}

\subsection{Охват исследованием и погрешность}

\includegraphics[width=\textwidth, page=55]{Kurs_OsnovyGDM_Kai_774_gorodovSV_v6_0.pdf}

\subsection{Исходные данные для гидродинамического моделирования}

\includegraphics[width=\textwidth, page=56]{Kurs_OsnovyGDM_Kai_774_gorodovSV_v6_0.pdf}

\includegraphics[width=\textwidth, page=58]{Kurs_OsnovyGDM_Kai_774_gorodovSV_v6_0.pdf}

\subsection{Подходы к построению ПДГГДМ}

\includegraphics[width=\textwidth, page=57]{Kurs_OsnovyGDM_Kai_774_gorodovSV_v6_0.pdf}

\subsection{Ремасштабирование геомодели}

\includegraphics[width=\textwidth, page=59]{Kurs_OsnovyGDM_Kai_774_gorodovSV_v6_0.pdf}

\subsection{Ремасштабирование структуры (upgridding)}

\includegraphics[width=\textwidth, page=60]{Kurs_OsnovyGDM_Kai_774_gorodovSV_v6_0.pdf}

\includegraphics[width=\textwidth, page=61]{Kurs_OsnovyGDM_Kai_774_gorodovSV_v6_0.pdf}

\subsection{Ремасштабирование свойств}

\includegraphics[width=\textwidth, page=62]{Kurs_OsnovyGDM_Kai_774_gorodovSV_v6_0.pdf}

\subsection{Ремасштабирование проницаемости}

\includegraphics[width=\textwidth, page=63]{Kurs_OsnovyGDM_Kai_774_gorodovSV_v6_0.pdf}

\includegraphics[width=\textwidth, page=64]{Kurs_OsnovyGDM_Kai_774_gorodovSV_v6_0.pdf}

\subsection{Ремасштабирование геомодели. Контроль качества}

\includegraphics[width=\textwidth, page=65]{Kurs_OsnovyGDM_Kai_774_gorodovSV_v6_0.pdf}

\subsection{Поверхностное натяжение}

\includegraphics[width=\textwidth, page=66]{Kurs_OsnovyGDM_Kai_774_gorodovSV_v6_0.pdf}

Далее необходимо рассмотреть, каким образом флюиды взаимодействуют друг с другом и с пластом в ходе фильтрации. Это взаимодействие в основном связано с поверхностным натяжением. Считаем, что химического взаимодействия в пласте не происходит.

\subsection{Смачиваемость}

\includegraphics[width=\textwidth, page=67]{Kurs_OsnovyGDM_Kai_774_gorodovSV_v6_0.pdf}

\subsection{Капиллярное давление}

\includegraphics[width=\textwidth, page=68]{Kurs_OsnovyGDM_Kai_774_gorodovSV_v6_0.pdf}

Чем больше радиус капилляра, тем меньше капиллярное давление и на меньшую высоту поднимется вода от уровня равновесия (зеркала свободной воды).

Для одного и того же капиллярного давления: чем больше разница плотностей, тем на меньшую высоту флюид поднимется в капилляре.
Поэтому переходная зона между нефтью и водой значительно больше, чем переходная зона между нефтью и газом.
На самом деле, переходную зону между нефтью и газом моделируют очень редко: обычно просто задают газонефтяной контакт (ГНК) в пределах одной ячейки.

Нелинейная фильтрация связана с вязкостью жидкости и капиллярными эффектами (запирающий градиент / давление сдвига). Подумать об этом и почитать подробнее про нелинейную фильтрацию!

\includegraphics[width=\textwidth, page=69]{Kurs_OsnovyGDM_Kai_774_gorodovSV_v6_0.pdf}

Зеркало свободной воды = капиллярное давление равно нулю.
От этого уровня считается высота подъёма воды по капиллярам.

Определение ВНК (водонефтяного контакта) не так однозначно (есть несколько разных определений).

\includegraphics[width=\textwidth, page=70]{Kurs_OsnovyGDM_Kai_774_gorodovSV_v6_0.pdf}

\includegraphics[width=\textwidth, page=71]{Kurs_OsnovyGDM_Kai_774_gorodovSV_v6_0.pdf}

По виду капиллярной кривой можно судить об однородности коллектора и о размере пор.

Для узких пор полка (практически постоянное значение на графике) по капиллярному давлению находится выше, чем для широких пор.

Для неоднородного коллектора нет полки по капиллярному давлению (плавный переход).

\subsection{Капиллярное давление для разных типов породы}

\includegraphics[width=\textwidth, page=72]{Kurs_OsnovyGDM_Kai_774_gorodovSV_v6_0.pdf}

Во втором песчанике самый худший коллектор (самые узкие поры).

\subsection{J-функция Леверетта}

\includegraphics[width=\textwidth, page=73]{Kurs_OsnovyGDM_Kai_774_gorodovSV_v6_0.pdf}

$\sqrt{\dfrac{k}{\varphi}}$ характеризует извилистость поровых каналов.

Рассчитываем значения J-функции, и далее строим график, подобный представленному справа: отмечаем подсчитанные точки и аппроксимируем их некой зависимостью (которую в дальнейшем будем использовать в расчётах ГДМ).

На графике могут получиться не одно облако точек, а два или три (если есть несколько пластов с разными характеристиками или разные блоки на месторождении, в каждом из которых получился свой тип коллектора вследствие разных геологических процессов).
Тогда будет несколько аппроксимирующих кривых, которые можно использовать отдельно для каждого рассматриваемого блока или пласта соответственно.

\subsection{Капиллярное давление. Лабораторные исследования}

\includegraphics[width=\textwidth, page=74]{Kurs_OsnovyGDM_Kai_774_gorodovSV_v6_0.pdf}

Иногда проводятся керновые лабораторные исследования не с пластовыми флюидами.
Например, с ртутью и воздухом.
И полученные данные пытаются применить для пласта.
Но в наше время так делают только самые отсталые лаборатории.
Сейчас стараются извлекать флюид, имеющийся на месторождении, и использовать его в экспериментах с керном.
Если же исследование уже проведено в системе ртуть-воздух, то придётся их пересчитать в систему нефть-вода по формуле, представленной на слайде.
При этом понадобятся значения, представленные в таблице.

\subsection{ОФП}

\includegraphics[width=\textwidth, page=75]{Kurs_OsnovyGDM_Kai_774_gorodovSV_v6_0.pdf}

Поверхностное натяжение кроме капиллярного давления приводит ещё к взаимному сопротивлению фильтрации нескольких флюидов.

Относительная фазовая проницаемость (ОФП) флюида 1 в присутствии флюида 2 -- это некий множитель (зависящий от насыщенности флюида 1) перед абсолютной проницаемостью, который позволяет найти эффективную проницаемость флюида 1 в присутствии флюида 2.

В рассматриваемой на слайде ситуации (50\% воды и 50 \% нефти) из графиков зависимости ОФП от водонасыщенности видим, что эффективная проницаемость воды будет составлять 5\% от абсолютной проницаемости, а эффективная проницаемость нефти будет составлять 15\% от абсолютной проницаемости.

\subsection{Смачиваемость. Критерий Craig (1971)}

\includegraphics[width=\textwidth, page=76]{Kurs_OsnovyGDM_Kai_774_gorodovSV_v6_0.pdf}

По виду кривых ОФП можем сделать вывод о гидрофобности или гидрофильности рассматриваемой породы.

Для гидрофильной породы вода прилипает к стенкам поры. Следовательно, связанная водонасыщенность будет достаточно большой (как правило, больше 20\%) и максимальная ОФП по воде будет иметь небольшое значение (как правило меньше 0.3).
Кривая ОФП по воде прижата к оси абсцисс: точка пересечения кривых ОФП будет правее 50\% по насыщенности.

Для гидрофобной породы наоборот: нефть прилипает к порам, а вода нет. Следовательно, кривая ОФП по нефти более прижата, а по воде более поднята.
Связанная водонасыщенность меньше 15\%, максимальная ОФП по воде больше 50\%. Точка пересечения кривых ОФП будет левее 50\%.

\subsection{Гистерезис ОФП}

\includegraphics[width=\textwidth, page=77]{Kurs_OsnovyGDM_Kai_774_gorodovSV_v6_0.pdf}

В школьном курсе физики изучали гистерезис для упругих свойств (сжатие-растяжение) при преодолении определённого значения напряжения.

В рассматриваемом случае гистерезис наблюдается вследствие зависимости ОФП от направления фильтрации (вода вытесняет нефть или нефть воду).

\subsection{ОФП. Лабораторные исследования}

\includegraphics[width=\textwidth, page=78]{Kurs_OsnovyGDM_Kai_774_gorodovSV_v6_0.pdf}

Теория Баклея-Леверетта.
На основе ОФП можем рассчитать, каким образом будет происходить заводнение в пласте  (другими словами, как будет продвигаться фронт вытеснения).

ОФП совместно с соотношением вязкостей нефти и воды влияют на скорость распространения фронта заводнения и на величину скачка насыщенности.

$f_w(S_w)$ -- функция фракционного потока.

Графический анализ (по Уэлджу): зная угол наклона касательной к кривой фракционного потока (графику зависимости $f_w(S_w)$), можем найти скорость продвижения фронта заводнения.

Насыщенность в точке касания -- это насыщенность на фронте вытеснения.

Насыщенность в точке пересечения касательной и горизонтальной прямой $f_w=1$ -- это средняя насыщенность от нагнетательной скважины до края заводнения.

Таким образом, даже без построения модели, имея только ОФП и вязкости, можем многое рассказать о том, каким образом будет происходить вытеснение.

\includegraphics[width=\textwidth, page=79]{Kurs_OsnovyGDM_Kai_774_gorodovSV_v6_0.pdf}

Есть 2 режима лабораторных исследований: установившийся и неустановившийся.

\includegraphics[width=\textwidth, page=80]{Kurs_OsnovyGDM_Kai_774_gorodovSV_v6_0.pdf}

По стандартам все исследования должны проводиться на установившемся режиме.
Минус такого подхода: для низкопроницаемых образцов время установления может занимать месяц или даже несколько месяцев.
Это дорого.
Поэтому иногда проводят быстрые исследования на неустановившемся режиме, но это менее точно и не соответствует стандартам.

\includegraphics[width=\textwidth, page=81]{Kurs_OsnovyGDM_Kai_774_gorodovSV_v6_0.pdf}

\subsection{ОФП. Корреляции Corey и LET}

\includegraphics[width=\textwidth, page=82]{Kurs_OsnovyGDM_Kai_774_gorodovSV_v6_0.pdf}

Аппроксимация проводится с целью удобства: необходимо, чтобы ОФП были гладкими функциями.
Это позволяет легче находить решение при использовании численных схем.

Корреляция LET (появилась 15-20 лет) позволяет лучше описать лабораторные исследования: есть участки с разной выпуклостью/вогнутостью.

\subsection{Как задать ОФП в ГДМ, если есть несколько исследований?}

\includegraphics[width=\textwidth, page=83]{Kurs_OsnovyGDM_Kai_774_gorodovSV_v6_0.pdf}



\includegraphics[width=\textwidth, page=84]{Kurs_OsnovyGDM_Kai_774_gorodovSV_v6_0.pdf}

\subsection{Концевые точки ОФП в системе нефть-вода}

\includegraphics[width=\textwidth, page=85]{Kurs_OsnovyGDM_Kai_774_gorodovSV_v6_0.pdf}

\includegraphics[width=\textwidth, page=86]{Kurs_OsnovyGDM_Kai_774_gorodovSV_v6_0.pdf}

\subsection{Масштабирование ОФП}

\includegraphics[width=\textwidth, page=87]{Kurs_OsnovyGDM_Kai_774_gorodovSV_v6_0.pdf}

\includegraphics[width=\textwidth, page=88]{Kurs_OsnovyGDM_Kai_774_gorodovSV_v6_0.pdf}

\includegraphics[width=\textwidth, page=89]{Kurs_OsnovyGDM_Kai_774_gorodovSV_v6_0.pdf}

\includegraphics[width=\textwidth, page=90]{Kurs_OsnovyGDM_Kai_774_gorodovSV_v6_0.pdf}

\subsubsection{По горизонтали (по насыщенности)}

\includegraphics[width=\textwidth, page=91]{Kurs_OsnovyGDM_Kai_774_gorodovSV_v6_0.pdf}

\subsubsection{По вертикали}

\includegraphics[width=\textwidth, page=92]{Kurs_OsnovyGDM_Kai_774_gorodovSV_v6_0.pdf}

\subsection{Согласованность массивов в модели}

\includegraphics[width=\textwidth, page=93]{Kurs_OsnovyGDM_Kai_774_gorodovSV_v6_0.pdf}

\includegraphics[width=\textwidth, page=94]{Kurs_OsnovyGDM_Kai_774_gorodovSV_v6_0.pdf}

\includegraphics[width=\textwidth, page=95]{Kurs_OsnovyGDM_Kai_774_gorodovSV_v6_0.pdf}

\subsection{Ремасштабирование (2-х фазный апскелинг ОФП)}

\includegraphics[width=\textwidth, page=96]{Kurs_OsnovyGDM_Kai_774_gorodovSV_v6_0.pdf}

\includegraphics[width=\textwidth, page=97]{Kurs_OsnovyGDM_Kai_774_gorodovSV_v6_0.pdf}

\subsection{Типы флюидов}

\includegraphics[width=\textwidth, page=98]{Kurs_OsnovyGDM_Kai_774_gorodovSV_v6_0.pdf}

\includegraphics[width=\textwidth, page=99]{Kurs_OsnovyGDM_Kai_774_gorodovSV_v6_0.pdf}

\subsection{Определение типа залежи по составу УВ}

\includegraphics[width=\textwidth, page=100]{Kurs_OsnovyGDM_Kai_774_gorodovSV_v6_0.pdf}

\subsection{PVT-свойства}

\includegraphics[width=\textwidth, page=101]{Kurs_OsnovyGDM_Kai_774_gorodovSV_v6_0.pdf}

\subsection{PVT-свойства нефти}

\includegraphics[width=\textwidth, page=102]{Kurs_OsnovyGDM_Kai_774_gorodovSV_v6_0.pdf}

\subsection{PVT-свойства нефти. Корреляции}

\includegraphics[width=\textwidth, page=103]{Kurs_OsnovyGDM_Kai_774_gorodovSV_v6_0.pdf}

\subsection{PVT-свойства "<живой нефти">}

\includegraphics[width=\textwidth, page=104]{Kurs_OsnovyGDM_Kai_774_gorodovSV_v6_0.pdf}

\subsection{Варианты описания PVT в моделях Black Oil}

\includegraphics[width=\textwidth, page=105]{Kurs_OsnovyGDM_Kai_774_gorodovSV_v6_0.pdf}

\includegraphics[width=\textwidth, page=106]{Kurs_OsnovyGDM_Kai_774_gorodovSV_v6_0.pdf}

\subsection{Сжимаемость порового пространства}

\includegraphics[width=\textwidth, page=107]{Kurs_OsnovyGDM_Kai_774_gorodovSV_v6_0.pdf}

\includegraphics[width=\textwidth, page=108]{Kurs_OsnovyGDM_Kai_774_gorodovSV_v6_0.pdf}

\subsection{Сжимаемость порового пространства. Корреляции}

\includegraphics[width=\textwidth, page=109]{Kurs_OsnovyGDM_Kai_774_gorodovSV_v6_0.pdf}

\subsection{Упражнение 1. Упражнения на обработку и подготовку исходных данных}

\includegraphics[width=\textwidth, page=110]{Kurs_OsnovyGDM_Kai_774_gorodovSV_v6_0.pdf}

\includegraphics[width=\textwidth, page=111]{Kurs_OsnovyGDM_Kai_774_gorodovSV_v6_0.pdf}

\includegraphics[width=\textwidth, page=112]{Kurs_OsnovyGDM_Kai_774_gorodovSV_v6_0.pdf}

\includegraphics[width=\textwidth, page=113]{Kurs_OsnovyGDM_Kai_774_gorodovSV_v6_0.pdf}

\end{document}
