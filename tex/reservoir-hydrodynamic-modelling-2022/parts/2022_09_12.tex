\documentclass[main.tex]{subfiles}

\begin{document}

\section{Лекция 12.09.2022 (Кайгородов С.В.)}

\subsection{Цели курса}

\includegraphics[width=\textwidth, page=7]{Kurs_OsnovyGDM_Kai_774_gorodovSV_v6_0.pdf}

Курс построен таким образом, что сначала мы посмотрим, что такое модели, какие требования есть к моделям в общем виде, какие ограничения, типы моделей, типы симуляторов, вообще зачем нужно моделирование, в том числе в нефтедобыче.
И дальше перейдём непосредственно к созданию гидродинамических моделей, пошаговый переход от геологической модели, анализ исходных данных, PVT, ОФП, капиллярные давления.
В общем всё, что нужно для создания модели, как их (эти данные) преобразовывать и загружать в модель, как модель инициализировать, затем настраивать на историю эксплуатации, какие методы желательно не использовать, чтобы не испортить модель.
И как затем с помощью этой модели считать разные варианты прогнозов и оптимизации разработки, подбирать ГТМ и так далее.
И в конце поговорим про регламенты по моделированию.

Есть ещё отдельная презентация по самому софту, который используется сейчас у нас в Газпром-Нефти.
Это т-Навигатор, как корпоративный симулятор.
У нас безлимитная лицензия на него.
Это наш отечественный симулятор, который распространился по миру и уже теснит продукты Schlumberger и других вендоров, производителей программного обеспечения.
Такая история успеха, в которую и мы тоже приложили доля своего, так скажем, воздействия, да, когда тестировали этот симулятор, давали рекомендации по его доработке.

\includegraphics[width=\textwidth, page=10]{Kurs_OsnovyGDM_Kai_774_gorodovSV_v6_0.pdf}

Цели курса у нас: расширить знания в области инструментов управления разработкой месторождения; сформировать убеждение, что процесс создания модели достаточно простой, если в нём разобраться; сформировать понятия об основных требованиях к моделям, этапах создания моделей, целях и основных проблемах моделирования, а также развить навыки создания и адаптации моделей, расчёта прогнозных вариантов.

Презентация выстроена таким образом, что сначала немножко теория идёт, а потом какое-то практическое упражнение.
Но поскольку у нас сейчас нет доступа к симулятору (у вас непосредственно на месте, насколько я понимаю, нет симулятора), то мы сейчас практические упражнения сделать не сможем, но вместе с Ильдаром вы, наверное, это всё проделаете.

Так что эти 2 дня у нас будет теория, а дальше уже практика отдельно.
Надеюсь, что в голове ничего не перепутается.

Говорит Ильдар Шамилевич: я поэтому на курсе сейчас и присутствую, чтобы какую-то теорию повторить ещё раз перед тем, как давать упражнения.

Далее продолжает Сергей Владимирович.

Да, на практике, в общем, вы всё это закрепите, что я рассказываю, поэтому где-то себе там записывайте; если какие-то вопросы есть, тоже сразу можно задавать.
Не стесняться!
Задача разобраться в материале!
Сегодня до 12:30 у нас по плану, и завтра тоже также.
За 2 дня должны мы, в принципе, основные понятия разобрать.

\subsection{Каналы распространения знаний по ГДМ}

\includegraphics[width=\textwidth, page=12]{Kurs_OsnovyGDM_Kai_774_gorodovSV_v6_0.pdf}

\subsection{Что такое модель?}

\includegraphics[width=\textwidth, page=15]{Kurs_OsnovyGDM_Kai_774_gorodovSV_v6_0.pdf}

Давайте сразу. На группы делиться не будем. Онлайн это сложно сделать.
В общем виде, что такое модель? Безотносительно нефтянки, а вообще. 
Модель -- это по определению система, исследование которой служит средством для получения информации о другой системе, то есть это некое упрощённое представление реального устройства и/или протекающих в нём процессов и явлений.
Поскольку окружающий нас мир бесконечен (вследствие неисчерпаемости материи и форм её взаимодействия, как внутри себя так и с внешней средой) и сложен, то моделирование -- это на самом деле обязательная часть исследований и разработок, и неотъемлемая часть нашей жизни.
То есть мы, может быть, и не задумываемся, но на самом деле каждый день занимаемся моделированием.
Моделируем какие-то ситуации, представляем себе отклик какой-либо системы на наше воздействие, то есть, например, что будет, если мы потянем за ручку двери (она откроется).
Такое простейшее представление.

\subsection{Требования к моделям}

\includegraphics[width=\textwidth, page=16]{Kurs_OsnovyGDM_Kai_774_gorodovSV_v6_0.pdf}

Здесь представлены основные требования к моделям.

Например, желательно, чтобы модель можно было использовать как можно для более широкого круга задач (универсальность), но при этом не терять в точности модели.
Ведь если мы будем делать всеобъемлющую модель всего, то понятно, что это будет либо суперсложная модель, которую мы не сможем рассчитать, либо эта модель будет возпроизводить какие-то основные характеристики системы, а какие-то более тонкие эффекты не будет показывать.
Поэтому нужен баланс (целесообразная экономичность): с одной стороны у нас есть точность результатов, а с другой стороны у нас есть затраты на моделирование (время, данные, квалификация).
В итоге, необходимо соотносить затраты с требуемой детальностью модели.
Для простых задач можем строить простые модели, которые требуют меньше времени, меньше данных.

Когда появятся в общем доступе квантовые компьютеры, то, наверное, можно будет считать суперточные модели мгновенно и иметь полную информацию обо всём, что происходит.

\subsection{Точность моделей}

\includegraphics[width=\textwidth, page=17]{Kurs_OsnovyGDM_Kai_774_gorodovSV_v6_0.pdf}

При упрощении реальных систем мы пренебрегаем какими-либо связями или характеристиками системы, и соответственно это сразу же приводит к погрешностям.

Погрешность исходных данных может быть связана как с погрешностью приборов, так и с погрешностью проведения самих экспериментов (замеров значений этих параметров), так и с погрешностью интерпретации, связанной с погрешностью методики интерпретации и непосредственно применения этой методики.
То есть несколько таких направлений, которые могут приводить к погрешностям в исходных данных.

Далее, что посеешь, то и пожнёшь: качество результатов расчётов не может превысить точности исходных данных.

\subsection{Виды моделей}

\includegraphics[width=\textwidth, page=18]{Kurs_OsnovyGDM_Kai_774_gorodovSV_v6_0.pdf}

\subsection{Гидродинамическая модель}

\includegraphics[width=\textwidth, page=19]{Kurs_OsnovyGDM_Kai_774_gorodovSV_v6_0.pdf}

\subsection{Цели гидродинамического моделирования (ГДМ)}

\includegraphics[width=\textwidth, page=20]{Kurs_OsnovyGDM_Kai_774_gorodovSV_v6_0.pdf}

По сути гидродинамическая модель служит своего рода такой базой данных, в которой собираются все результаты исследований, интерпретации исследований.
И информация об одних и тех же свойствах может идти из разных исследований.
Когда мы всю эту информацию собираем воедино, то можем увидеть какие-то нестыковки, несоответствия и выявить погрешности в исходных данных, чтобы затем уточнить: действительно ли корректны рассматриваемые значения исходных параметров, полученных при интерпретации выбранных исследований.
Может быть, стоит их пересмотреть и разобраться в причинах несоответствия.

Лучше один раз увидеть, чем сто раз услышать -- визуализация объекта разработки важна (тогда легче представить, что с этим объектом можно делать и как этот объект будет реагировать на определённые воздействия; другими словами, легче понять отклик моделируемой системы на разные воздействия и обдумать, что необходимо делать для оптимизации процессов, имеющихся на месторождении, чтобы улучшить результаты в эксплуатации).
Визуализация позволяет представить, что происходит, а именно как в динамике меняются свойства пласта, провести анализ разработки, подобрать варианты оптимизации, варианты геолого-технических мероприятий и посчитать различные прогнозы.

На месторождении мы можем сделать мероприятие только 1 раз (и свойства пласта необратимо изменятся), а в модели можем сделать сколько угодно разных мероприятий (не оказывая при этом воздействие на реальный пласт).
Следовательно, модель является инструментом принятия решений и экономит нам средства (деньги и время) на то, чтобы подобрать оптимальный способ разработки (ведь можем перебрать много разных способов, выбрать лучший и его уже реализовывать на месторождении).

\subsection{Математическая основа ГДМ}

\includegraphics[width=\textwidth, page=21]{Kurs_OsnovyGDM_Kai_774_gorodovSV_v6_0.pdf}

Здесь приведены уравнения, на которых основана гидродинамическая модель. 
Это система уравнений, которая включает в себя уравнение неразрывности сплошной среды (по сути это закон сохранения массы).

Если у нас ещё есть какие-то тепловые методы (которые применяются на месторождении), то добавляются ещё уравнения сохранения энергии.
Но как правило, это делают достаточно редко, поэтому можно сказать, что в большинстве моделей это не учитывается, а именно мы считаем процессы изотермическими, никакого теплового воздействия в пласте не происходит (всё зависит только от давления).

Также в систему уравнений входит уравнение состояния сплошной среды, которое описывает, как изменяются свойства пласта, свойства флюидов при изменении давления и температуры (если всё-таки есть тепловое воздействие).

Ещё в систему уравнений входит закон движения (фильтрации), то есть по сути различные модификации закона Дарси.

Плюс начальные и граничные условия. На слайде всё приведено в дифференциальном виде. Можно поразбираться.
Более детально не стал рассказывать, это уже для тех, кому особо интересно есть отдельный курс (несколько часов рассказывается, как получаются эти уравнения, как их затем решать).
Но что можно отсюда заметить? Это нелинейное дифференциальное уравнение в частных производных.

\includegraphics[width=\textwidth, page=22]{Kurs_OsnovyGDM_Kai_774_gorodovSV_v6_0.pdf}

Как решать такие уравнения аналитически в общем виде, науке пока неизвестно.
Такие уравнения встречаются не только в гидродинамике, но и во всей физике.
Например, в теории относительности и гравитации есть уравнение Эйнштейна (тоже нелинейное дифференциальное уравнение в частных производных).

Но есть численные методы, которые позволяют нам решать такие уравнение приближённо.
Мы разбиваем пространство и время на отрезки конечных размеров. То есть в пространстве это будут такие ячейки (кусочки пространства), а по времени -- временные шаги. И мы говорим, что в одной этой ячейке на каждый определённый шаг по времени свойства имеют одно значение (т.е пористость, проницаемость, насыщенность фиксированы).
Но эти свойства могут меняться с каждым шагом по времени.
Наступил следующий временной шаг и свойства могут измениться в зависимости от потоков через грани ячеек.
Тогда мы можем сказать, что в соответсвии с этим упрощением мы можем уравнение аппроксимировать: производные по времени заменить конечными разностями, а интеграл по объёму ячейки заменить на интеграл по поверхности. Тогда у нас уравнения упрощаются, и получается система уже более простых уравнений, которую мы можем дальше решать.
Тоже есть определённая последовательность действий: линеаризация этих уравнений, решение СЛАУ и так далее.
Но сейчас детально рассматривать не будем.
Отсюда нужно только понять, что мы разбиваем пространство и время на элементарные отрезки, за счёт этого уравнения у нас упрощаются, и мы можем их решать на компьютере численными методами.
И получать за счёт этого приближённое решение.

\subsection{Типы сеток ГДМ}

\includegraphics[width=\textwidth, page=23]{Kurs_OsnovyGDM_Kai_774_gorodovSV_v6_0.pdf}

Как можно пространство разбить на ячейки?

Самое простое: нарезать параллелепипеды, тогда получится блочно-центрированная сетка ячеек.

Но пласт у нас неровный. Осадконакопление происходит неравномерно, либо происходят какие-то тектонические процессы после осадконакопления и формирования пласта.
Соответственно пласт какой-то изогнутый и с помощью блочно-центрированных ячеек эту изогнутость воспроизвести сложно, поэтому нужно придумать более гибкие ячейки, чтобы описать изгибы пласта под землёй.

Придумали сетки ячеек, которые называются геометрией угловой точки.
Для их построения задаются направляющие линии, и на этих направляющих линиях задаются глубины точек, которые являются вершинами для ячейки и таким образом плоскости граней ячейки могут быть повёрнуты куда угодно, т.е. ячейки становятся более гибкими.
На сегодняшний момент 3D геометрия угловой точки является самым популярным способом построения сетки для геологической/гидродинамической модели, чтобы описать строение пласта.

\includegraphics[width=\textwidth, page=24]{Kurs_OsnovyGDM_Kai_774_gorodovSV_v6_0.pdf}

Также есть так называемая сетка Вороного (или перпендикулярный бисектор).
Это локально ортогональная сетка, в которой грани соседних ячеек равноудалены от центров этих ячеек.
То есть если мы расставим точки центров ячеек и нарисуем грани этих ячеек так, чтобы они были равноудалены от этих точек центров, то получатся как раз шестиугольники (подобные пчелиным сотам).
Такая сетка позволяет более точно описать приток к скважине (дальше тоже это посмотрим).

\subsection{Типы сеток ГДМ. LGR}

\includegraphics[width=\textwidth, page=25]{Kurs_OsnovyGDM_Kai_774_gorodovSV_v6_0.pdf}

Сетку можно измельчать или укрупнять. Понятно, что если будем сетку измельчать, то их количество будет расти, для каждой из этих ячеек нам придётся решать уравнения фильтрации (как из одной ячейки в другую перетекает флюид), и это будет замедлять расчёт. Но с другой стороны можем более точно в какой-то области замоделировать течение флюидов.

Здесь (как всегда) приходится искать компромисс между точностью и скоростью.
Если нужно какие-то эффекты точно воспроизвести в заданной области, то можем сетку локально измельчить.
Но также могут быть ячейки, потоки в которых нам особо неинтересны (например, в тех ячейках, где течёт в основном вода) -- такие ячейки укрупняем (тем самым уменьшаем количество ячеек и сокращаем время расчёта).

Можем строить радиальную сетку, но на практике, честно говоря, ни разу не видел, чтобы кто-то пользовался.
На радиальной сетке проводят в основном теоретические расчёты, но на практике она не используется.

\subsection{Порядок нумерации ячеек сетки}

\includegraphics[width=\textwidth, page=26]{Kurs_OsnovyGDM_Kai_774_gorodovSV_v6_0.pdf}

Как происходит нумерация ячеек сетки?

Сначала изменяется координата по $x$, потом по $y$, потом по $z$.
Начинаем с левого верхнего угла (ячейка с координатами $\left(1,1,1\right)$), следующие ячейки $\left(2,1,1\right)$, $\left(3,1,1\right)$, $\left(4,1,1\right)$ и так далее. Здесь 8 ячеек по $x$.
Далее переходим ко второму ряду по $y$, начиная с ячейки $\left(1,2,1\right)$ переходим к $\left(2,2,1\right)$ и так далее.
После всех рядов по $y$ переходим на следующий слой по $z$.

Я это рассказываю, чтобы было понимание, в каком порядке номера ячеек меняются, чтобы можно было при визуализации найти какую-то ячейку, которая вам интересна. Например, если вы знаете, какую ячейку вскрывает скважина.

\subsection{Структура файла исходных данных для симулятора ECLIPSE}

\includegraphics[width=\textwidth, page=27]{Kurs_OsnovyGDM_Kai_774_gorodovSV_v6_0.pdf}

В т-Навигаторе тоже поддерживается формат Eclipse.
И в т-Навигаторе тоже считываются DATA-файлы, которые состоят из секций, в которые сгруппированы определённые ключевые слова, описывающие модель.
По сути это чем-то похоже на программирование: есть некая команда, которая воспринимается программой симулятором, и дальше идут некие параметры выполнения этой команды.

RUNSPEC = спецификация запуска. Eclipse создавали ещё в 80-е годы на Фортране и в это время ещё не было достаточного количества оперативной памяти, следовательно, нужно было заранее определять, сколько памяти потребуется модели для расчёта.
Поэтому в этой секции указывались основные характеристики: сколько в модели будет скважин, сколько моделируемых фаз, сколько разных PVT-таблиц. В общем, такие характеристики, чтобы под них забронировать оперативную память.
Сейчас таких проблем с оперативной памятью уже нет, но исторически такая секция RUNSPEC осталась.

В секции PROPS задаются PVT-свойства флюидов и SCAL свойства (special core analysis in laboratory) взаимодействия этих флюидов с пластом.
Для получения этих свойств проводится специальный анализ флюидов и керна в лаборатории.

Секция REGIONS используется, если нам нужно задать отдельные регионы, в каждом из которых свои свойства (например, свои свойства флюида).
Когда это нужно? Например, у нас есть несколько пластов на месторождении, и в каждом из этих пластов свойства отличаются, соответственно, можем записать их как разные регионы и для каждого региона задавать свои свойства.

Секция SOLUTION описывает инициализацию модели, т.е. начальные условия (до того, как начался расчёт): какое начальное состояние по насыщенности и так далее.

В секцию SUMMARY записываются те графики, которые хотим посмотреть по результатам расчёта.
Эта секция тоже относится к симулятору Eclipse, в т-Навигаторе эта секция необязательна (в нём настройка отображаемых графиков производится в самом интерфейсе программы -- галочками отмечаются графики, которые требуется отобразить).

\subsection{Справочники для симулятора ECLIPSE}

\includegraphics[width=\textwidth, page=28]{Kurs_OsnovyGDM_Kai_774_gorodovSV_v6_0.pdf}

Ключевые слова запоминать необязательно.
И для Eclipse, и для т-Навигатора, и для других симуляторов есть справочники, которые поставляются вместе с программой.
В этих справочниках есть технический мануал, в котором описаны уравнения, заложенные в расчёт, и есть мануал, который описывает сами ключевые слова (обычно сгруппированы по первым буквам).
Следовательно, можем найти необходимое ключевое слово и посмотреть, какие параметры нужны для этого ключевого слова.

Также есть примеры файлов-моделей с различными опциями.
Если хотим смоделировать какой-либо процесс (например, закачку полимера или водогазовое воздействие), то можем просто открыть папку с готовыми примерами (как правило, эта папка совпадает с корневой папкой, в которой лежит сам симулятор) и посмотреть, какие ключевые слова используются для моделирования этого процесса.
Затем вернуться в мануал и просмотреть эти ключевые слова, чтобы понять, что необходимо задавать для моделирования этих опций и воздействий.

\subsection{Задание свойств в ячейках}

\includegraphics[width=\textwidth, page=29]{Kurs_OsnovyGDM_Kai_774_gorodovSV_v6_0.pdf}

Свойства должны быть заданы для каждой ячейки, чтобы симулятор знал, как производить расчёт.
Как правило, эти значения присваиваются центру каждой ячейки; свойства можно задать явным перечислением и, если есть повторяющиеся значения, то их можно сгруппировать (т.е. записать, что свойство в $n$ ячейках имеет значение $a$).

Значения свойств ещё могут быть заданы в виде функции (в Eclipse ключевое слово OPERATE, в т-Навигаторе ключевое слово ARITHMETIC).

Schlumberger раньше поставлял FloViz и FloGrid. Сейчас они устарели, и Schlumberger их не продаёт.

Для того, чтобы сэкономить ресурсы, расчёт производится только в активных ячейках.
Активными считаются ячейки, в которых фактически происходит поток флюида.
То есть в ячейках с глинами (неколлекторами), где нет никаких потоков флюида, нет необходимости проводить какие-либо расчёты.
Соответственно, можем просто их исключить из расчёта (по-умолчанию неактивны ячейки с нулевыми пористостью (PORO) или песчанистостью (NTG, отношение количества эффективных толщин к общим толщинам)).

Также есть ключевое слово ACTNUM, которое непосредственно задаёт активные и неактивные ячейки. Т.е. мы или геолог с помощью этого ключевого слова можем самостоятельно отметить ячейки с коллектором (песчаником) или неколлектором (глинами).
 
\includegraphics[width=\textwidth, page=30]{Kurs_OsnovyGDM_Kai_774_gorodovSV_v6_0.pdf}

На этом слайде показаны примеры задания свойств в ячейках непосредственно по ячейкам, с группировкой ячеек, с помощью ключевого слова EQUALS, с помощью копирования COPY, а также с помощью арифметических операций (в Eclipse ключевое слово MULTIPLY, в т-Навигаторе можем использовать ключевое слово ARITHMETIC).

\includegraphics[width=\textwidth, page=31]{Kurs_OsnovyGDM_Kai_774_gorodovSV_v6_0.pdf}

На этом слайде показаны примеры задания свойств в ячейках с помощью ключевого слова BOX.

На слайде приведены примеры использования ключевого слова INCLUDE.
Файлы с большими массивами данных (кубами свойств) хранятся отдельно и подключаются к основному файлу с помощью ключевого слова INCLUDE.

\subsection{Поток через ячейку}

\includegraphics[width=\textwidth, page=32]{Kurs_OsnovyGDM_Kai_774_gorodovSV_v6_0.pdf}

Как рассчитывается поток через ячейку?
Поток определяется градиентом давления между ячейками и проводимостью (т.е. насколько легко будет течь флюид через границу).
Направление градиента и направление потока противоположны.
Градиент показывает направление возрастания какой-либо величины.

\includegraphics[width=\textwidth, page=33]{Kurs_OsnovyGDM_Kai_774_gorodovSV_v6_0.pdf}

Как считается проводимость?
На этом слайде показана формула расчёта проводимости для блочно-центрированной ячейки.
В эту формулу включаются песчанистость, размеры ячеек, проницаемость, разница глубин.
Дополнительно есть множитель MULTX (множитель проводимости).
Для чего нужен этот множитель? Посмотрим дальше.

\includegraphics[width=\textwidth, page=34]{Kurs_OsnovyGDM_Kai_774_gorodovSV_v6_0.pdf}

Для геометрии угловой точки используется чуть более сложная формула.
Запоминать эти формулы не нужно; они даны для информации: какие параметры влияют на поток через грани ячейки (а именно свойства самих ячеек и площадь граней, через которые происходит переток).

\subsection{Несоседние соединения NNC}

\includegraphics[width=\textwidth, page=35]{Kurs_OsnovyGDM_Kai_774_gorodovSV_v6_0.pdf}

Как правило переток происходит между ячейками, у которых индексы отличаются на единицу, но есть ряд случаев, когда необходимо, чтобы переток был между ячейками, у которых индексы отличаются больше чем на единицу.
Например, разлом со смещением, т.е. часть пласта у нас в результате тектонической активности сместилась относительно другой части пласта.
Для таких случаев симулятор создаёт так называемые несоседние соединения NNC (грубо говоря, прописывает взаимосвязи ячеек).
На самом деле об этом можно и не задумываться, так как такие соединения создаются в автоматическом режиме, но просто полезно для информации, что такое бывает.

В ячейках с радиальной геометрией идёт нумерация по часовой стрелке (вторая координата меняется по часовой стрелке), поэтому получается, что первая и последняя ячейки граничат друг с другом, но при этом их индексы отличаются больше чем на единицу (соответственно симулятор будет себе отмечать, что переток между этими ячейками должен быть).

То же самое для водоносных горизонтов.
Если они подключаются к каким-то неактивным ячейкам, то можно сделать так, чтобы были несоседние соединения, чтобы переток с водоносных горизонтов осуществлялся в модель.

Выклинивание: если какие-либо ячейки исключаются (из-за ключевых слов PINCH или MINPV), то, чтобы не создавать искусственный барьер, возникает (симулятор автоматически прописывает) несоседнее соединение между ячейками, примыкающими к исключённой ячейке.

\subsection{Проблемы пространственной дискретизации}

\includegraphics[width=\textwidth, page=36]{Kurs_OsnovyGDM_Kai_774_gorodovSV_v6_0.pdf}

За то, что мы прибегли к упрощению (а именно, воспользовались дискретным представлением пространства и времени), нам приходится платить точностью расчёта.
Из-за дискретизации пространства и времени возникает численная ошибка, которая называется численная дисперсия.
Она говорит о том, что, чем более грубые ячейки (чем более грубо мы разрезали месторождение на ячейки), тем менее точно будет описан процесс фильтрации.

Представим себе аналогию с разрешением картинки: если у нас есть картинка с разрешением 100 на 100 пикселей, то она чёткая; если же мы делаем меньше пикселей, то картинка становиться размазанной/размытой; и при определённом загрублении мы уже не можем понять, что изображено на картинке.
То же самое и в модели.
В каждой ячейке задаётся набор свойств.
Если мы сделаем слишком грубую сетку, то представления о распределении свойств под землёй будут искажены, и соответственно мы будем получать искажённое решение.

Как и везде в итоге необходимо искать баланс: и достаточно быстро, и достаточно точно.
Но бывает и не быстро, и не точно.

Помимо измельчения сетки есть ещё способ уменьшить численную дисперсию, а именно включить эту численную дисперсию в ОФП (получить при этом так называемую псевдо-ОФП), т.е. учесть что поток идёт более плавно по этим грубым ячейкам.
Про это расскажу более подробно чуть позже, когда будем рассматривать ОФП.

\includegraphics[width=\textwidth, page=37]{Kurs_OsnovyGDM_Kai_774_gorodovSV_v6_0.pdf}
Ещё один численный эффект, возникающий при дискретизации, это эффект ориентации сетки.
Он заключается в том, что время прихода флюида из одной точки в другую зависит от того, сколько ему нужно пройти ячеек.

Видим, что в случае, когда добывающие скважины расположены по диагонали ячеек сетки, вода к ним приходит позже.
Это такой чисто численный эффект, который нужно как-то исключить. 

\includegraphics[width=\textwidth, page=38]{Kurs_OsnovyGDM_Kai_774_gorodovSV_v6_0.pdf}

Для уменьшения/исключения эффекта ориентации сетки можно измельчить сетку, использовать альтернативные численные схемы (которые учитывают взаимодействие ячеек по диагонали; естественно эти вычислительные схемы усложняют расчёт и требуют дополнительных вычислительных ресурсов), можно использовать сетку Вороного (позволяет более точно смоделировать приток к скважине, т.е. уменьшить эффект ориентации сетки) или линии тока (но линии тока являются неким упрощением, когда мы решаем для насыщенности одномерную задачу; про линии тока поговорим ещё дальше по курсу).

Для ячеек Вороного (PEBI) тоже есть сложности с решением систем уравнений, ведь у PEBI самих граней, через которые течёт поток, больше (у прямоугольной ячейки 6 граней, у ячейки Вороного 8 граней), соответственно, и сами матрицы систем уравнений становятся сложнее для решения.
Углубляться не будем.

Вообще рекомендация такая: желательно ориентировать сетку ячеек по направлению основных потоков, которые в пласте происходят.
Эти потоки могут быть связаны как с сеткой скважин, так и с региональными стрессами (какими-либо разломами, трещиноватостями). Всё равно есть преимущественные направления фильтрации, и сетку желательно ориентировать так, чтобы она была в направлении этих потоков (в направлении фильтрации).

Сетку Вороного не нужно ориентировать; для сетки Вороного пропадает численный эффект ориентации сетки.
Но для сетки Вороного есть другая проблема: поток не всегда направлен перпендикулярно грани ячейки в случаях локального измельчения сетки.

\subsection{Построение грида}

\includegraphics[width=\textwidth, page=39]{Kurs_OsnovyGDM_Kai_774_gorodovSV_v6_0.pdf}

Здесь показаны разбиения сетки по вертикали.
Мы говорили о том, что есть эффект ориентации сетки по горизонтали, но на самом деле по вертикали он тоже может наблюдаться. В зависимости от того, как геолог сделает разбивку слоёв, может изменяться моделируемый процесс фильтрации.

Поэтому здесь важно тоже следить, чтобы вертикальная нарезка слоёв соответствовала тому, как фактически происходило формирование залежи.
Т.е. если это было равномерное осадконакопление без особенностей, то должна быть пропорциональная нарезка (здесь конечно возможны разные уплотнения слоёв под воздействием разных обстоятельств, но нарезка всё равно пропорциональная); если шло осадконакопление на фундамент (например, на какие-нибудь выветренные прогибы), то используется нарезка с налеганием; если шло осадконакопление равномерно, но затем часть этих осадков срезалась (эродировала) под воздействием каких-либо факторов (грубо говоря, сдуло часть накопленного песка или срезало ледником, который съехал откуда-то с горы), то используется нарезка с эрозией.

\subsection{Гидродинамические модели (схема)}

\includegraphics[width=\textwidth, page=40]{Kurs_OsnovyGDM_Kai_774_gorodovSV_v6_0.pdf}

На данном слайде представлено условное разделение моделей на типы по разным характеристикам (например, учитываем ли тепломассоперенос).

По размерности: может быть 3D полномасштабная модель, может быть 3D сектор, если мы считаем какую-то локальную задачу (например, рассчитываем оптимальное для бурения положение ствола скважины); с 2D моделью понятно -- это карты или разрезы; есть ещё 2.5D модели -- в одном слое модели задаётся распределение свойств, а по всем остальным слоям это распределение просто наследуется, т.е. визуально это выглядит как трёхмерная модель, но по вертикали не заложена реальная неоднородность (карту каждого слоя можем получить, зная карту в одном слое) -- вроде это не 2D, но и не 3D, поэтому иногда называют 2.5D.

Про остальные типы сейчас подробнее поговорим.

\subsection{Типы расчётных моделей}

\subsubsection{Модель нелетучей нефти}

\includegraphics[width=\textwidth, page=41]{Kurs_OsnovyGDM_Kai_774_gorodovSV_v6_0.pdf}

По фазам/флюидам, которые задаются в модели, выделяют модель нелетучей нефти (ещё называют её Black Oil) -- здесь моделируются три фазы (вода, нефть и газ), свойства фаз зависят только от давления (процессы изотермические), модели нелетучей нефти применимы для большинства способов разработки, которые используются у нас на месторождениях (это извлечение на режиме истощения, заводнение и т.д.)

На самом деле в рамках модели Black Oil есть ещё упрощённый метод для моделирования смешивающегося заводнения, но он не очень точный.
Поэтому при необходимости моделирования смешивающегося заводнения используем композиционную модель.

\subsubsection{Композиционная модель}

\includegraphics[width=\textwidth, page=42]{Kurs_OsnovyGDM_Kai_774_gorodovSV_v6_0.pdf}

В композиционной модели нефть и газ задаются покомпонентно, и уравнение фильтрации рассчитывается для каждой компоненты отдельно, т.е. представляется, что каждая компонента течёт сама по себе, и дальше в ячейке происходит смешивание этих компонент, и образуется какой-то флюид.
Т.е. помимо уравнения фильтрации в такой модели на каждый шаг расчёта ещё дополнительно происходит расчёт равновесия флюидов, т.е. на основе того, какие в этой ячейке (в данный шаг по времени) оказались компоненты, рассчитывается, в каком состоянии флюид находится (т.е. сколько газа, сколько нефти и состав).
Но понятно, что это кратно увеличивает время расчёта, поэтому такие модели рассчитываются только в крайних случаях, когда это действительно необходимо (т.е. для газоконденсатных месторождений, для месторождений с давлением, близким к давлению насыщения, когда происходит переход из одного фазового состояния в другое и обратно -- на каждом шаге это может происходить, либо закачка газа, который растворяется, либо нефть испаряется; в общем, когда флюиды находятся в таком пограничном состоянии и состав флюидов в значительной степени влияет на их свойства, то используется композиционная модель).

Если говорить про Газпром-нефть, то сейчас у нас таких моделей становится всё больше, поскольку мы берём у Газпрома месторождения с нефтяными оторочками, т.е. мы совместно их разрабатываем (Газпром добывает газ, а мы добываем нефть в этих оторочках).
Соответственно поведение нефти в оторочках иногда требует того, чтобы использовать композиционную модели, а именно, если мы систему поддержания пластового давления (ППД) пытаемся организовать, либо там содержится газоконденсат, который может выпадать испаряться, что-то ещё, то соответственно необходимо строить композиционную модель.

\subsubsection{Термические модели}

\includegraphics[width=\textwidth, page=43]{Kurs_OsnovyGDM_Kai_774_gorodovSV_v6_0.pdf}

Термические модели содержат 4 фазы (ещё добавляется твёрдая фаза -- кокс, например, когда мы используем внутрипластовое горение, то у нас часть нефти в пласте окисляется, сгорает и образуется твёрдая фаза кокс, которая тоже имеет определённые свойства и влияет на фильтрацию, т.е. она там закупоривает поры и так далее).

В этих моделях свойства фаз зависят не только от давления, но и от температуры: как вначале мы сегодня говорили, что в этом случае ещё добавляется уравнение сохранения энергии, т.е. тоже приходится гораздо больше уравнений решать, и такие модели тоже требуют гораздо больше вычислительных ресурсов.
И термические модели применяются только для тех случаев, когда у нас используются термические методы увеличения нефтеотдачи (как правило, они используются для высоковязких нефтей, чтобы снизить вязкость, чтобы нефть потекла; методы закачки пара; методы паротепловых обработок скважин Huff and Puff; циклическая закачка пара; закачка горячей воды или газа; внутрискважинные нагреватели; внутрипластовое горение).

\subsubsection{Модель двойной или мульти-среды}

\includegraphics[width=\textwidth, page=44]{Kurs_OsnovyGDM_Kai_774_gorodovSV_v6_0.pdf}

По типу пласта (типу материала, из которого создан пласт) модели тоже можно разделить: обычно это модель одной среды, но когда у нас есть кроме поровой составляющей ещё и какие-то трещины, которые вносят существенный вклад в фильтрацию, то может использоваться модель двойной (или мульти-) среды.
Т.е. если у нас трещиноватые, кавернозные коллектора (карбонаты, доломиты), то модель двойной среды может использоваться.

В этой модели как будто бы две среды: одна это поровая, а другая трещинная.
Для каждой из этих сред задаются свойства отдельно и рассчитывается переток из одной среды в другую.

Если у нас запасы в основном содержатся в матрице (т.е. в поровой части пласта коллектора), а фильтрация происходит только по трещинам, то это называется моделью двойной пористости.
Если же и матрица, и трещины достаточно проницаемы (фильтрация происходит и в матрице, и по трещинам), то это модель двойной проницаемости.

Как определить, какую модель лучше выбрать? Обычно смотрят по динамике работы скважин или по гидродинамическим исследованиям определяют эти характеристики, когда наступает отклик от матрицы, т.е. от поровой среды.

Кроме трещин у нас могут быть ещё и каверны, которые могут быть связаны или не связаны с трещинами, поэтому их тоже можно моделировать отдельной средой, когда это действительно необходимо (например, в случае трещинно-каверно-поровых доломитных коллекторов).
Тогда у нас возникает модель тройной пористости, четверной и так далее (мультипористости).
Т.е. можно сложные коллектора моделировать множественными средами, но опять таки это всё ведёт к усложнению модели и к увеличению времени расчёта, поэтому здесь нужно всегда искать баланс между скоростью и точностью.

Сейчас дойдём до логического завершения и сделаем перерыв.

\subsubsection{Модели линий тока}

\includegraphics[width=\textwidth, page=45]{Kurs_OsnovyGDM_Kai_774_gorodovSV_v6_0.pdf}

Про конечно-разностные модели мы проговорили (что мы разбиваем пространство и время на элементарные участки - на ячейки - и в каждой ячейке считаем уравнение).

Есть альтернативный вид моделей. Называется он моделями линий тока.
Здесь вместо решения уравнений на трёхмерной сетке решается одномерная задача для насыщенности.
Для этого строится линия тока таким образом, что касательная к каждой точке этой линии совпадает с вектором скорости фильтрации флюида, т.е. если мы построим к этим линиям касательные в каждой точке, то они будут показывать направление, в котором происходит фильтрация флюида.

Для давления мы продолжаем решать трёхмерную задачу, а для насыщенности задача упрощается.
Кроме того, что модели линий тока немного ускоряют расчёт, они предоставляют нам дополнительный способ визуализации процесса фильтрации.
Можно раскрасить линии от каждой скважины, и тогда мы видим, с какими скважинами данная скважина связана.
Также можно численно определить, какой процент флюида идёт от данной скважины к другой скважине.
 И это даёт нам дополнительную информацию, на основе которой мы можем проводить оптимизацию системы заводнения (т.е. понять какие нагнетательные скважины эффективно закачивают воду (эффективно вытесняют нефть), а какие неэффективно (т.е. гоняют воду по кругу); и соответственно перераспределить закачку из менее эффективных скважин в более эффективные и таким образом либо сократить количество добываемой воды, либо увеличить количество добываемой нефти, либо одновременно и то, и другое).

\subsubsection{Proxy-модели}

\includegraphics[width=\textwidth, page=46]{Kurs_OsnovyGDM_Kai_774_gorodovSV_v6_0.pdf}

Есть ещё Proxy-модели

\includegraphics[width=\textwidth, page=47]{Kurs_OsnovyGDM_Kai_774_gorodovSV_v6_0.pdf}

Proxy-модель является простой моделью, которая может быть применима для простых залежей, т.е. если у нас есть какие-то трёхмерные эффекты, которые необходимо воспроизвести, то такая модель не подойдёт.

На слайде представлены случаи, в которых 2D-Proxy-модель не подойдёт.

Пример: трёхмерные фигуры могут иметь в 2D одинаковую проекцию, но в 3D они значительно отличаются, т.е. не для всех случаев 2D модель достаточна.

\subsubsection{Суррогатные (мета) модели}

\includegraphics[width=\textwidth, page=48]{Kurs_OsnovyGDM_Kai_774_gorodovSV_v6_0.pdf}

Есть ещё (в последнее время появляются) так называемые суррогатные модели (или метамодели), которые стали особенно популярны в связи с появлением нейронных сетей, которые аппроксимируют поведение каких-либо систем.
Т.е. по сути это модель модели: сама модель это некое упрощение реальной системы, но всё равно она достаточно сложная, содержит в себе много различных связей, характеристик, долго считается, поэтому возникла следующая идея: давайте мы опишем модель с помощью каких-то более простых функций или нейронных сетей.
Для этого производится ряд вычислений на модели в симуляторе и результаты аппроксимируются какой-либо функцией (или нейронка обучается на эти расчёты). 
С помощью такой упрощённой модели можно какие-то характеристики считать.
Например, мы натренировали нейронку считать накопленную добычу нефти в зависимости от каких-то параметров модели, и дальше с помощью этой нейронки можем быстрее производить расчёты накопленной добычи нефти.

Здесь тоже есть определённые минусы. Всеобъемлющую суррогатную модель не построишь, для каждого параметра её придётся заново тренировать (или для каждого месторождения, или для определённых условий).
В общем, пока не получается сделать такую модель, чтобы она была натренирована на одних данных, а мы изменили систему разработки, например, и модель всё равно продолжала работать.
Другими словами, метамодель придётся переобучать и тратить на это время.

Но всё же для какого-то класса задач можно использовать метамодели, однако пока невозможно такими моделями заменить всё.

\includegraphics[width=\textwidth, page=49]{Kurs_OsnovyGDM_Kai_774_gorodovSV_v6_0.pdf}

Здесь представлен алгоритм построения суррогатной модели.

\includegraphics[width=\textwidth, page=50]{Kurs_OsnovyGDM_Kai_774_gorodovSV_v6_0.pdf}

Здесь показаны области применения суррогатных моделей.
Метамодели применяются для анализа чувствительности, для оптимизации (когда нам нужно сделать много расчётов, мы можем много расчётов произвести на этих простых (быстрых) метамоделях и потом несколько расчётов на исходных более сложных (медленных) моделях, чтобы подтвердить расчёты простых метамоделей).

\subsection{Иерархия гидродинамических моделей}

\includegraphics[width=\textwidth, page=51]{Kurs_OsnovyGDM_Kai_774_gorodovSV_v6_0.pdf}

Здесь ещё пара слайдов про иерархию моделей: чем более детальная и сложная модель, тем она требует больше данных на входе и больше выдаёт на выходе, но при этом требует больше трудозатрат, больше информации и больше времени на расчёт.
Соответственно, здесь нужно искать, опять повторюсь, компромисс (т.е. для каких-то простых задач достаточно и простых моделей, т.е. нет необходимости использовать сложные трёхмерные полномасштабные модели для каких-то простых задач; и наоборот для сложных задач простых моделей может быть недостаточно).

\subsection{Местоположение моделирования в цикле нефтедобычи}

\includegraphics[width=\textwidth, page=52]{Kurs_OsnovyGDM_Kai_774_gorodovSV_v6_0.pdf}

Дальше поговорим о данных, которые используются для построения гидродинамических моделей.
Давайте сделаем перерыв, сейчас у нас 11:05, перерыв до 11:15.
Проветримся, чтобы не уснуть.
Дальше поговорим о том, какие данные используются, как они обрабатываются перед тем, как попасть в модель.
Если есть какие-то вопросы, то тоже можете пока их готовить и через 10 минут задать.

\includegraphics[width=\textwidth, page=53]{Kurs_OsnovyGDM_Kai_774_gorodovSV_v6_0.pdf}

Здесь в виде облака тегов показано, с какими дисциплинами связано гидродинамическое моделирование, какого типа данные входят в ГДМ модель.

\subsection{Источники геологической информации в масштабах месторождения}

\includegraphics[width=\textwidth, page=54]{Kurs_OsnovyGDM_Kai_774_gorodovSV_v6_0.pdf}

На этом слайде показано, что есть разные источники геологической информации о месторождении и при этом они имеют разный охват пласта и масштаб исследований.

У нас есть с одной стороны лабораторные исследования керна, которые очень точные, т.е. мы вплоть до каждой поры можем рассмотреть образец, но при этом они имеют очень маленький масштаб исследования и только точечные данные (т.е. мы не можем по всему месторождению отобрать весь пласт и каждую пору рассмотреть; это просто какие-то элементы, из которых мы делаем выводы о строении пласта, о строении коллектора или неколлектора).

Дальше по размерности идут геофизические исследования скважин (ГИС), но у них по-прежнему информация только вблизи скважин.
ГИС менее точные, чем исследования на керне, но при этом у ГИС чуть больше охват исследования.

Гидродинамические исследования скважин позволяют нам охватить зону дренирования скважин, это уже десятки и сотни метров, но при этом имеют более высокую погрешность (более низкую точность), чем ГИС и керн.

Сейсморазведка позволяет охватить исследованиями всё месторождение или даже несколько месторождений, но при этом имеет самую высокую погрешность, т.е. с одной стороны мы имеем много данных, но при этом погрешность по определению структуры пласта (на какой глубине пласт находится) может составлять десятки метров.

\subsection{Охват исследованием и погрешность}

\includegraphics[width=\textwidth, page=55]{Kurs_OsnovyGDM_Kai_774_gorodovSV_v6_0.pdf}

Получается такой вот принцип неопределённости. По аналогии с принципом неопределённости Гейзенберга я попробовал записать: с одной стороны у нас увеличивается охват исследованием, а с другой стороны увеличивается погрешность, т.е. чем больше охват, тем больше погрешность.

В физике есть принцип неопределённости Гейзенберга, когда мы для квантовой частицы знаем либо её местоположение, либо её импульс (т.е. куда она движется).

Здесь мы тоже либо всё охватываем исследованием, но тогда имеем большую погрешность, либо имеем низкую погрешность, но тогда имеем очень маленький охват.
Поэтому данные у нас обладают такой неопределённостью: нам приходится их между собой комплексировать, искать какие-то хитрые методы, как друг с другом их увязать, ведь они разного масштаба, разного типа, записываются и хранятся по-разному.
Другими словами, есть много разных методик, как эти данные разных исследований между собой связывать и приводить в гидродинамическую модель.

\subsection{Исходные данные для гидродинамического моделирования}

\includegraphics[width=\textwidth, page=56]{Kurs_OsnovyGDM_Kai_774_gorodovSV_v6_0.pdf}

Здесь такая таблица, в которой я попытался свести разные типы данных, которые используются для гидродинамической модели, откуда они берутся и какие ключевые слова в основном используются.
Естественно это не всеобъемлющая таблица, но она показывает основную информацию, которая в гидродинамической модели используется.

Координаты и траектории скважин можем взять непосредственно из баз данных, но обычно берём из геомодели (так как они туда уже подгружены).

\includegraphics[width=\textwidth, page=58]{Kurs_OsnovyGDM_Kai_774_gorodovSV_v6_0.pdf}

На данном слайде выдержка из регламентного документа, которая говорит о том, что исходные данные могут содержать недостоверную информацию, и, если мы эту информацию занесём в модель, то соответственно такого же качества результат мы и получим (что посеешь, то и пожнёшь).
Поэтому перед тем, как модель запускать на расчёт, нужно проверить достоверность промысловой информации, непротиворечивость этой информации, поговорить непосредственно с коллегами на промысле, которые эту информацию записывали, всё ли действительно так, как оно занесено в базы данных.
И затем уже использовать эту модель.

\subsection{Подходы к построению ПДГГДМ}

\includegraphics[width=\textwidth, page=57]{Kurs_OsnovyGDM_Kai_774_gorodovSV_v6_0.pdf}

Здесь схематичный слайд о постепенном переходе со стандартного подхода поэтапного построения ГДМ модели к гибкому Agile подходу.

При стандартном подходе модель создаётся поэтапно. Т.е. сейсмик делает интерпретацию сейсморазведки, строит какие-то структурные поверхности, оценивает какие-то протяжённые тела, передаёт это всё петрофизику.
Петрофизик тоже интерпретирует свои ГИСы, передаёт эту одну интерпретацию геологу.
Геолог это всё сводит к геологической модели, передаёт гидродинамику одну эту геологическую модель.
Разработчик даёт гидродинамику какие-то результаты эксплуатации и исследований скважин, а у гидродинамика это всё не сходится.
Почему так происходит? Потому что все эти данные обладают неопределённостями и, если мы используем только какое-то одно значение из всего этого диапазона, то естественно оно может не соответствовать интерпретации по другому направлению.
Соответственно здесь должен быть гибкий подход, чтобы данные из разных направлений были в едином рабочем процессе (настраивались друг на друга), а с другой стороны сохранить диапазоны неопределённостей по интерпретации и протаскивать их на протяжении всего этого рабочего процесса (workflow) создания модели так, чтобы искать подходящее решение в пределах всех возможных вариаций параметров (при этом сохраняя взаимосвязи между параметрами).
Тогда все будут счастливы и модель будет более точная.

Мы такой подход реализовали, сейчас начинаем его тиражировать на разных месторождениях.
Сложно, конечно, пока это всё идёт; пока люди осознают, что это необходимо делать, приходится рисовать вот такие смайлики и показывать это схематично в виде пиктограмм. 

\subsection{Ремасштабирование геомодели}

\includegraphics[width=\textwidth, page=59]{Kurs_OsnovyGDM_Kai_774_gorodovSV_v6_0.pdf}

Теперь переходим непосредственно к созданию модели.
После того, как геолог создал свою статичную геологическую модель и передал его гидродинамику, бывают случаи, когда эту модель необходимо сделать более грубой, когда эта модель слишком детальная и эта детальность излишняя (только забирает ресурсы и никакой информации особо не несёт для гидродинамика).
Возникает необходимость сделать процедуру ремасштабирования, т.е. укрупнение модели (точнее укрупнение ячеек модели).
Эта процедура состоит из двух этапов: первое это upgridding (ремасштабирование сетки; изменение размеров и количества ячеек) и второе это upscaling (ремасштабирование свойств; т.е. после того, как мы получили большие ячейки, нам в эти большие ячейки нужно записать свойства, а именно осреднить значения свойств из маленьких ячеек и перенести эти осреднённые значения в большую ячейку).

Возникает такой вопрос: до какой степени нам модель можно укрупнять и когда следует остановиться, т.е. когда мы начнём терять в качестве?

Можно построить такой график (представлен на слайде): по оси $x$ откладываем количество слоёв по вертикали (здесь мы говорим про укрупнение по вертикали), а по оси $y$ откладываем погрешность в расчёте накопленной добычи нефти (FOPT), накопленной добычи воды (FLPT).
Т.е. мы сравниваем, насколько результаты вот этих накопленных показателей отличаются от модели с исходной геологической сеткой.
При уменьшении количества слоёв ошибка постепенно растёт, и в какой-то момент на графике возникает перегиб (ошибка начинает возрастать более резко), т.е. это является неким косвенным признаком того, что мы начинаем в этот момент терять какую-то информацию о геологическом строении, о неоднородности.
И соответственно можем сказать, что в этой точке перегиба у нас оптимум, дальше которого модель укрупнять не следует (стоит остановиться).
Итак, первым способом выбора степени укрупнения является нахождение оптимума (точки перегиба на графике).

Второй способ -- это просто ограничиться каким-то значением ошибки. Почему-то обычно привязываются к каким-то круглым значениям (5, 10 или 20 \%).

Но лучше всё-таки использовать способ, который показывает, в какой момент мы начинаем терять информацию о строении месторождения.

\subsection{Ремасштабирование структуры (upgridding)}

\includegraphics[width=\textwidth, page=60]{Kurs_OsnovyGDM_Kai_774_gorodovSV_v6_0.pdf}

По горизонтали рекомендация следующая: рекомендуется, чтобы между скважинами было не менее 3-5 ячеек, чтобы описать фильтрацию между скважинами.
Но с другой стороны, если у нас количество данных по месторождению ограничено, то стремиться к излишней детализации тоже не стоит, потому что точности не добавится (ведь новых данных нет), а время расчёта увеличится.
Но как минимум 3-5 ячеек всё таки желательно оставлять.
Справа на слайде приведён пример модели, которую передавали мне на экспертизу: и даже сложно различить, где какая скважина находится (здесь представлены горизонтальные скважины; крестиками помечены перфорации), настолько близко они расположены (в соседних ячейках), что, честно, даже непонятно, где какой ствол идёт; что таким образом пытались смоделировать тоже непонятно, естественно экспертизу такая модель не прошла и было рекомендовано сделать более детальную модель, чтобы между скважинами корректно воспроизводить процесс фильтрации.

\includegraphics[width=\textwidth, page=61]{Kurs_OsnovyGDM_Kai_774_gorodovSV_v6_0.pdf}

По вертикали желательно следить за тем, чтобы сохранялась расчленённость, нарезка слоёв и глинистые перемычки не пропадали.

Здесь на слайде тоже показан пример одной из моделей, которая проходила на экспертизу (левый рисунок на слайде).
Как делать не надо.
Видим, что в геологической модели и верхний, и нижний пласты достаточно расчленённые (на ГИС есть много белых глинистых перемычек).
А после укрупнения видим, что верхний пласт вообще склеился в один однородный массив, и в нижнем пласте тоже расчленённость пропала.

На правом рисунке на слайде есть пример, в котором глинистые перемычки пропали не внутри одного пласта, а даже между пластами, т.е. пласты, которые вообще не сообщаются гидродинамически, вдруг стали гидродинамически связанными.
Такое ужасное нарушение; модель стала совсем непригодной для расчётов, поскольку появилась вертикальная связь.

\subsection{Ремасштабирование свойств}

\includegraphics[width=\textwidth, page=62]{Kurs_OsnovyGDM_Kai_774_gorodovSV_v6_0.pdf}

После того, как мы укрупнили ячейки, нужно в эти ячейки перенести свойства.

Вот у нас пример здесь: были такие маленькие ячейки, теперь эти маленькие ячейки стали одной большой ячейкой. У маленьких ячеек были разные значения какого-то свойства. Какое значение теперь занести в большую ячейку?

Есть рекомендованные методы расчёта средних свойств и очерёдности расчёта: сначала мы рассчитываем среднее значение песчанистости (рассчитывается как среднее арифметическое, но взвешенное по объёму ячеек; формула представлена на слайде -- ячейки имеющие больший объём вносят больший вклад), следующей по очереди осредняется пористость (здесь среднее арифметическое, взвешенное на эффективный объём; эффективный объём -- это песчанистость, умноженная на геометрический объём), далее осредняется насыщенность (среднее арифметическое, взвешенное на поровый объём; поровый объём -- это пористость, умноженная на эффективный объём).
Это всё делается для того, чтобы воспроизвести запасы для модели с укрупнённой сеткой.

Если же делать всё-таки гидродинамически уравновешенную модель, то насыщенность можно не осреднять, а рассчитать по гидростатическому равновесию.
Про это (про расчёт насыщенностей) дальше мы тоже будем говорить, когда будем обсуждать инициализацию модели.

\subsection{Ремасштабирование проницаемости}

\includegraphics[width=\textwidth, page=63]{Kurs_OsnovyGDM_Kai_774_gorodovSV_v6_0.pdf}

Для проницаемости методы осреднения более разнообразны, поскольку проницаемость не является объёмной характеристикой, а является характеристикой, зависящей от направления фильтрации.

И получается, что если поток идёт параллельно напластованию, то мы можем использовать среднее арифметическое для осреднения.

Если поток идёт перпендикулярно напластованию, то рекомендуется использовать среднее гармоническое.

Если же пласт сильно неоднородный (сложно выделить направление напластования), то можно использовать среднее геометрическое или хитроумную комбинацию арифметических и гармонических.

Но самый лучший способ -- это осреднение на основе решения уравнений однофазной или многофазной фильтрации.
Т.е. что делается?
Фактически производится расчёт потоков (по формуле Дарси, грубо говоря) на мелких ячейках и затем на крупных ячейках обратным пересчётом рассчитывается проницаемость так, чтобы потоки через грани крупных ячеек были такими же, как и сумма потоков через грани мелких ячеек, которые составляют эту крупную ячейку. Т.е. основная задача -- это сохранить потоки, и таким образом подбирается проницаемость, чтобы эти потоки сохранились.

\includegraphics[width=\textwidth, page=64]{Kurs_OsnovyGDM_Kai_774_gorodovSV_v6_0.pdf}

Здесь говорится о том, какие условия задавать на границах при таком способе осреднения.

Если поток идёт по горизонтали, то мы говорим, что вертикального перетока нет.

Если идёт косая слоистость, то вычисляется полный тензор проницаемости.

Если идёт и горизонтальный поток, и вертикальный, то можно задать изменение давления на верхних границах, чтобы был переток.

\subsection{Ремасштабирование геомодели. Контроль качества}

\includegraphics[width=\textwidth, page=65]{Kurs_OsnovyGDM_Kai_774_gorodovSV_v6_0.pdf}

Как контролировать ремасштабирование?
Есть несколько способов.

Первое -- это геолого-статистический разрез (например, по песчанистости).
Здесь чисто визуально оценивается, сохранились ли глинистые перемычки, оцениваются доли коллектора с высоким и низким содержанием глины, т.е. такое визуальное сравнение графиков.

На слайде (на крайнем левом рисунке) красным показано среднее значение песчанистости в слоях по укрупнённой модели, а синей -- в исходной геологической модели.
Геолого-статистический разрез получается следующим образом: в каждом слое считается среднее арифметическое значение и наносится на график (по оси ординат -- слои, по оси абсцисс -- значения песчанистости).

Получается такая вот "<кардиограмма"> (крайний левый рисунок), на которой мы сопоставляем визуально, насколько хорошо сохранились глинистые перемычки.

Также можно сопоставить начальные запасы углеводородов (сохранились или не сохранились после ремасштабирования), эффективные толщины и ещё можно посмотреть гистограммы.

Гистограммы, конечно, один в один не совпадут, потому что количество крайних ячеек (точнее ячеек, которые имеют крайние значения, т.е. либо максимальные, либо минимальные) сократится, поскольку такие ячейки в ходе осреднения будут объединяться с ячейками с другими значениями.
Соответственно количество крайних значений уменьшится, и гистограмма как бы прижмётся к своему среднему значению, но при этом вид самой гистограммы должен быть одинаковый как до, так и после укрупнения ячеек.
Т.е. если мы видим какое-то смещение среднего значения или другой вид гистограммы, то это может говорить о том, что мы потеряли какую-то информацию о строении пласта в ходе этого укрупнения и нужно вернуться и проверить, всё ли правильно мы сделали, правильные ли методы осреднения использовали и не слишком ли грубо мы всё это сделали.

Это рассмотрели то, что касается статических свойств.

\subsection{Поверхностное натяжение}

\includegraphics[width=\textwidth, page=66]{Kurs_OsnovyGDM_Kai_774_gorodovSV_v6_0.pdf}

Далее необходимо рассмотреть, каким образом флюиды взаимодействуют друг с другом и с пластом в ходе фильтрации. Это взаимодействие в основном связано с поверхностным натяжением. Считаем, что химического взаимодействия в пласте не происходит (в обычной модели, без учёта химии).

\subsection{Смачиваемость}

\includegraphics[width=\textwidth, page=67]{Kurs_OsnovyGDM_Kai_774_gorodovSV_v6_0.pdf}

\subsection{Капиллярное давление}

\includegraphics[width=\textwidth, page=68]{Kurs_OsnovyGDM_Kai_774_gorodovSV_v6_0.pdf}

Чем больше радиус капилляра, тем меньше капиллярное давление и на меньшую высоту поднимется вода от уровня равновесия (зеркала свободной воды).

Для одного и того же капиллярного давления: чем больше разница плотностей, тем на меньшую высоту флюид поднимется в капилляре.
Поэтому переходная зона между нефтью и водой значительно больше, чем переходная зона между нефтью и газом.
На самом деле, переходную зону между нефтью и газом моделируют очень редко: обычно просто задают газонефтяной контакт (ГНК) в пределах одной ячейки.

Нелинейная фильтрация связана с вязкостью жидкости и капиллярными эффектами (запирающий градиент / давление сдвига). Подумать об этом и почитать подробнее про нелинейную фильтрацию!

\includegraphics[width=\textwidth, page=69]{Kurs_OsnovyGDM_Kai_774_gorodovSV_v6_0.pdf}

Зеркало свободной воды = капиллярное давление равно нулю.
От этого уровня считается высота подъёма воды по капиллярам.

Определение ВНК (водонефтяного контакта) не так однозначно (есть несколько разных определений).

\includegraphics[width=\textwidth, page=70]{Kurs_OsnovyGDM_Kai_774_gorodovSV_v6_0.pdf}

\includegraphics[width=\textwidth, page=71]{Kurs_OsnovyGDM_Kai_774_gorodovSV_v6_0.pdf}

По виду капиллярной кривой можно судить об однородности коллектора и о размере пор.

Для узких пор полка (практически постоянное значение на графике) по капиллярному давлению находится выше, чем для широких пор.

Для неоднородного коллектора нет полки по капиллярному давлению (плавный переход).

\subsection{Капиллярное давление для разных типов породы}

\includegraphics[width=\textwidth, page=72]{Kurs_OsnovyGDM_Kai_774_gorodovSV_v6_0.pdf}

Во втором песчанике самый худший коллектор (самые узкие поры).

\subsection{J-функция Леверетта}

\includegraphics[width=\textwidth, page=73]{Kurs_OsnovyGDM_Kai_774_gorodovSV_v6_0.pdf}

$\sqrt{\dfrac{k}{\varphi}}$ характеризует извилистость поровых каналов.

Рассчитываем значения J-функции, и далее строим график, подобный представленному справа: отмечаем подсчитанные точки и аппроксимируем их некой зависимостью (которую в дальнейшем будем использовать в расчётах ГДМ).

На графике могут получиться не одно облако точек, а два или три (если есть несколько пластов с разными характеристиками или разные блоки на месторождении, в каждом из которых получился свой тип коллектора вследствие разных геологических процессов).
Тогда будет несколько аппроксимирующих кривых, которые можно использовать отдельно для каждого рассматриваемого блока или пласта соответственно.

\subsection{Капиллярное давление. Лабораторные исследования}

\includegraphics[width=\textwidth, page=74]{Kurs_OsnovyGDM_Kai_774_gorodovSV_v6_0.pdf}

Иногда проводятся керновые лабораторные исследования не с пластовыми флюидами.
Например, с ртутью и воздухом.
И полученные данные пытаются применить для пласта.
Но в наше время так делают только самые отсталые лаборатории.
Сейчас стараются извлекать флюид, имеющийся на месторождении, и использовать его в экспериментах с керном.
Если же исследование уже проведено в системе ртуть-воздух, то придётся их пересчитать в систему нефть-вода по формуле, представленной на слайде.
При этом понадобятся значения, представленные в таблице.

\subsection{ОФП}

\includegraphics[width=\textwidth, page=75]{Kurs_OsnovyGDM_Kai_774_gorodovSV_v6_0.pdf}

Поверхностное натяжение кроме капиллярного давления приводит ещё к взаимному сопротивлению фильтрации нескольких флюидов.

Относительная фазовая проницаемость (ОФП) флюида 1 в присутствии флюида 2 -- это некий множитель (зависящий от насыщенности флюида 1) перед абсолютной проницаемостью, который позволяет найти эффективную проницаемость флюида 1 в присутствии флюида 2.

В рассматриваемой на слайде ситуации (50\% воды и 50 \% нефти) из графиков зависимости ОФП от водонасыщенности видим, что эффективная проницаемость воды будет составлять 5\% от абсолютной проницаемости, а эффективная проницаемость нефти будет составлять 15\% от абсолютной проницаемости.

\subsection{Смачиваемость. Критерий Craig (1971)}

\includegraphics[width=\textwidth, page=76]{Kurs_OsnovyGDM_Kai_774_gorodovSV_v6_0.pdf}

По виду кривых ОФП можем сделать вывод о гидрофобности или гидрофильности рассматриваемой породы.

Для гидрофильной породы вода прилипает к стенкам поры. Следовательно, связанная водонасыщенность будет достаточно большой (как правило, больше 20\%) и максимальная ОФП по воде будет иметь небольшое значение (как правило меньше 0.3).
Кривая ОФП по воде прижата к оси абсцисс: точка пересечения кривых ОФП будет правее 50\% по насыщенности.

Для гидрофобной породы наоборот: нефть прилипает к порам, а вода нет. Следовательно, кривая ОФП по нефти более прижата, а по воде более поднята.
Связанная водонасыщенность меньше 15\%, максимальная ОФП по воде больше 50\%. Точка пересечения кривых ОФП будет левее 50\%.

\subsection{Гистерезис ОФП}

\includegraphics[width=\textwidth, page=77]{Kurs_OsnovyGDM_Kai_774_gorodovSV_v6_0.pdf}

В школьном курсе физики изучали гистерезис для упругих свойств (сжатие-растяжение) при преодолении определённого значения напряжения.

В рассматриваемом случае гистерезис наблюдается вследствие зависимости ОФП от направления фильтрации (вода вытесняет нефть или нефть воду).

\subsection{ОФП. Лабораторные исследования}

\includegraphics[width=\textwidth, page=78]{Kurs_OsnovyGDM_Kai_774_gorodovSV_v6_0.pdf}

Теория Баклея-Леверетта.
На основе ОФП можем рассчитать, каким образом будет происходить заводнение в пласте  (другими словами, как будет продвигаться фронт вытеснения).

ОФП совместно с соотношением вязкостей нефти и воды влияют на скорость распространения фронта заводнения и на величину скачка насыщенности.

$f_w(S_w)$ -- функция фракционного потока.

Графический анализ (по Уэлджу): зная угол наклона касательной к кривой фракционного потока (графику зависимости $f_w(S_w)$), можем найти скорость продвижения фронта заводнения.

Насыщенность в точке касания -- это насыщенность на фронте вытеснения.

Насыщенность в точке пересечения касательной и горизонтальной прямой $f_w=1$ -- это средняя насыщенность от нагнетательной скважины до края заводнения.

Таким образом, даже без построения модели, имея только ОФП и вязкости, можем многое рассказать о том, каким образом будет происходить вытеснение.

\includegraphics[width=\textwidth, page=79]{Kurs_OsnovyGDM_Kai_774_gorodovSV_v6_0.pdf}

Есть 2 режима лабораторных исследований: установившийся и неустановившийся.

\includegraphics[width=\textwidth, page=80]{Kurs_OsnovyGDM_Kai_774_gorodovSV_v6_0.pdf}

По стандартам все исследования должны проводиться на установившемся режиме.
Минус такого подхода: для низкопроницаемых образцов время установления может занимать месяц или даже несколько месяцев.
Это дорого.
Поэтому иногда проводят быстрые исследования на неустановившемся режиме, но это менее точно и не соответствует стандартам.

\includegraphics[width=\textwidth, page=81]{Kurs_OsnovyGDM_Kai_774_gorodovSV_v6_0.pdf}

\subsection{ОФП. Корреляции Corey и LET}

\includegraphics[width=\textwidth, page=82]{Kurs_OsnovyGDM_Kai_774_gorodovSV_v6_0.pdf}

Аппроксимация проводится с целью удобства: необходимо, чтобы ОФП были гладкими функциями.
Это позволяет легче находить решение при использовании численных схем.

Корреляция LET (появилась 15-20 лет) позволяет лучше описать лабораторные исследования: есть участки с разной выпуклостью/вогнутостью.

\subsection{Как задать ОФП в ГДМ, если есть несколько исследований?}

\includegraphics[width=\textwidth, page=83]{Kurs_OsnovyGDM_Kai_774_gorodovSV_v6_0.pdf}



\includegraphics[width=\textwidth, page=84]{Kurs_OsnovyGDM_Kai_774_gorodovSV_v6_0.pdf}

\subsection{Концевые точки ОФП в системе нефть-вода}

\includegraphics[width=\textwidth, page=85]{Kurs_OsnovyGDM_Kai_774_gorodovSV_v6_0.pdf}

\includegraphics[width=\textwidth, page=86]{Kurs_OsnovyGDM_Kai_774_gorodovSV_v6_0.pdf}

\subsection{Масштабирование ОФП}

\includegraphics[width=\textwidth, page=87]{Kurs_OsnovyGDM_Kai_774_gorodovSV_v6_0.pdf}

\includegraphics[width=\textwidth, page=88]{Kurs_OsnovyGDM_Kai_774_gorodovSV_v6_0.pdf}

\includegraphics[width=\textwidth, page=89]{Kurs_OsnovyGDM_Kai_774_gorodovSV_v6_0.pdf}

\includegraphics[width=\textwidth, page=90]{Kurs_OsnovyGDM_Kai_774_gorodovSV_v6_0.pdf}

\subsubsection{По горизонтали (по насыщенности)}

\includegraphics[width=\textwidth, page=91]{Kurs_OsnovyGDM_Kai_774_gorodovSV_v6_0.pdf}

\subsubsection{По вертикали}

\includegraphics[width=\textwidth, page=92]{Kurs_OsnovyGDM_Kai_774_gorodovSV_v6_0.pdf}

\subsection{Согласованность массивов в модели}

\includegraphics[width=\textwidth, page=93]{Kurs_OsnovyGDM_Kai_774_gorodovSV_v6_0.pdf}

\includegraphics[width=\textwidth, page=94]{Kurs_OsnovyGDM_Kai_774_gorodovSV_v6_0.pdf}

\includegraphics[width=\textwidth, page=95]{Kurs_OsnovyGDM_Kai_774_gorodovSV_v6_0.pdf}

\subsection{Ремасштабирование (2-х фазный апскелинг ОФП)}

\includegraphics[width=\textwidth, page=96]{Kurs_OsnovyGDM_Kai_774_gorodovSV_v6_0.pdf}

\includegraphics[width=\textwidth, page=97]{Kurs_OsnovyGDM_Kai_774_gorodovSV_v6_0.pdf}

\subsection{Типы флюидов}

\includegraphics[width=\textwidth, page=98]{Kurs_OsnovyGDM_Kai_774_gorodovSV_v6_0.pdf}

\includegraphics[width=\textwidth, page=99]{Kurs_OsnovyGDM_Kai_774_gorodovSV_v6_0.pdf}

\subsection{Определение типа залежи по составу УВ}

\includegraphics[width=\textwidth, page=100]{Kurs_OsnovyGDM_Kai_774_gorodovSV_v6_0.pdf}

\subsection{PVT-свойства}

\includegraphics[width=\textwidth, page=101]{Kurs_OsnovyGDM_Kai_774_gorodovSV_v6_0.pdf}

\subsection{PVT-свойства нефти}

\includegraphics[width=\textwidth, page=102]{Kurs_OsnovyGDM_Kai_774_gorodovSV_v6_0.pdf}

\subsection{PVT-свойства нефти. Корреляции}

\includegraphics[width=\textwidth, page=103]{Kurs_OsnovyGDM_Kai_774_gorodovSV_v6_0.pdf}

\subsection{PVT-свойства "<живой нефти">}

\includegraphics[width=\textwidth, page=104]{Kurs_OsnovyGDM_Kai_774_gorodovSV_v6_0.pdf}

\subsection{Варианты описания PVT в моделях Black Oil}

\includegraphics[width=\textwidth, page=105]{Kurs_OsnovyGDM_Kai_774_gorodovSV_v6_0.pdf}

\includegraphics[width=\textwidth, page=106]{Kurs_OsnovyGDM_Kai_774_gorodovSV_v6_0.pdf}

\subsection{Сжимаемость порового пространства}

\includegraphics[width=\textwidth, page=107]{Kurs_OsnovyGDM_Kai_774_gorodovSV_v6_0.pdf}

\includegraphics[width=\textwidth, page=108]{Kurs_OsnovyGDM_Kai_774_gorodovSV_v6_0.pdf}

\subsection{Сжимаемость порового пространства. Корреляции}

\includegraphics[width=\textwidth, page=109]{Kurs_OsnovyGDM_Kai_774_gorodovSV_v6_0.pdf}

\subsection{Упражнение 1. Упражнения на обработку и подготовку исходных данных}

\includegraphics[width=\textwidth, page=110]{Kurs_OsnovyGDM_Kai_774_gorodovSV_v6_0.pdf}

\includegraphics[width=\textwidth, page=111]{Kurs_OsnovyGDM_Kai_774_gorodovSV_v6_0.pdf}

\includegraphics[width=\textwidth, page=112]{Kurs_OsnovyGDM_Kai_774_gorodovSV_v6_0.pdf}

\includegraphics[width=\textwidth, page=113]{Kurs_OsnovyGDM_Kai_774_gorodovSV_v6_0.pdf}

\end{document}
