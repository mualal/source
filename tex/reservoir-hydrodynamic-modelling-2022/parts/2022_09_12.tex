\documentclass[main.tex]{subfiles}

\begin{document}

\section{Лекция 12.09.2022 (Кайгородов С.В.)}

\subsection{Цели курса}

\includegraphics[width=\textwidth, page=7]{Kurs_OsnovyGDM_Kai_774_gorodovSV_v6_0.pdf}

Курс построен таким образом, что сначала мы посмотрим, что такое модели, какие требования есть к моделям в общем виде, какие ограничения, типы моделей, типы симуляторов, вообще зачем нужно моделирование, в том числе в нефтедобыче.
И дальше перейдём непосредственно к созданию гидродинамических моделей, пошаговый переход от геологической модели, анализ исходных данных, PVT, ОФП, капиллярные давления.
В общем всё, что нужно для создания модели, как их (эти данные) преобразовывать и загружать в модель, как модель инициализировать, затем настраивать на историю эксплуатации, какие методы желательно не использовать, чтобы не испортить модель.
И как затем с помощью этой модели считать разные варианты прогнозов и оптимизации разработки, подбирать ГТМ и так далее.
И в конце поговорим про регламенты по моделированию.

Есть ещё отдельная презентация по самому софту, который используется сейчас у нас в Газпром-Нефти.
Это т-Навигатор, как корпоративный симулятор.
У нас безлимитная лицензия на него.
Это наш отечественный симулятор, который распространился по миру и уже теснит продукты Schlumberger и других вендоров, производителей программного обеспечения.
Такая история успеха, в которую и мы тоже приложили доля своего, так скажем, воздействия, да, когда тестировали этот симулятор, давали рекомендации по его доработке.

\includegraphics[width=\textwidth, page=10]{Kurs_OsnovyGDM_Kai_774_gorodovSV_v6_0.pdf}

Цели курса у нас: расширить знания в области инструментов управления разработкой месторождения; сформировать убеждение, что процесс создания модели достаточно простой, если в нём разобраться; сформировать понятия об основных требованиях к моделям, этапах создания моделей, целях и основных проблемах моделирования, а также развить навыки создания и адаптации моделей, расчёта прогнозных вариантов.

Презентация выстроена таким образом, что сначала немножко теория идёт, а потом какое-то практическое упражнение.
Но поскольку у нас сейчас нет доступа к симулятору (у вас непосредственно на месте, насколько я понимаю, нет симулятора), то мы сейчас практические упражнения сделать не сможем, но вместе с Ильдаром вы, наверное, это всё проделаете.

Так что эти 2 дня у нас будет теория, а дальше уже практика отдельно.
Надеюсь, что в голове ничего не перепутается.

Говорит Ильдар Шамилевич: я поэтому на курсе сейчас и присутствую, чтобы какую-то теорию повторить ещё раз перед тем, как давать упражнения.

Далее продолжает Сергей Владимирович.

Да, на практике, в общем, вы всё это закрепите, что я рассказываю, поэтому где-то себе там записывайте; если какие-то вопросы есть, тоже сразу можно задавать.
Не стесняться!
Задача разобраться в материале!
Сегодня до 12:30 у нас по плану, и завтра тоже также.
За 2 дня должны мы, в принципе, основные понятия разобрать.

\subsection{Каналы распространения знаний по ГДМ}

\includegraphics[width=\textwidth, page=12]{Kurs_OsnovyGDM_Kai_774_gorodovSV_v6_0.pdf}

\subsection{Что такое модель?}

\includegraphics[width=\textwidth, page=15]{Kurs_OsnovyGDM_Kai_774_gorodovSV_v6_0.pdf}

Давайте сразу. На группы делиться не будем. Онлайн это сложно сделать.
В общем виде, что такое модель? Безотносительно нефтянки, а вообще. 
Модель -- это по определению система, исследование которой служит средством для получения информации о другой системе, то есть это некое упрощённое представление реального устройства и/или протекающих в нём процессов и явлений.
Поскольку окружающий нас мир бесконечен (вследствие неисчерпаемости материи и форм её взаимодействия, как внутри себя так и с внешней средой) и сложен, то моделирование -- это на самом деле обязательная часть исследований и разработок, и неотъемлемая часть нашей жизни.
То есть мы, может быть, и не задумываемся, но на самом деле каждый день занимаемся моделированием.
Моделируем какие-то ситуации, представляем себе отклик какой-либо системы на наше воздействие, то есть, например, что будет, если мы потянем за ручку двери (она откроется).
Такое простейшее представление.

\subsection{Требования к моделям}

\includegraphics[width=\textwidth, page=16]{Kurs_OsnovyGDM_Kai_774_gorodovSV_v6_0.pdf}

Здесь представлены основные требования к моделям.

Например, желательно, чтобы модель можно было использовать как можно для более широкого круга задач (универсальность), но при этом не терять в точности модели.
Ведь если мы будем делать всеобъемлющую модель всего, то понятно, что это будет либо суперсложная модель, которую мы не сможем рассчитать, либо эта модель будет возпроизводить какие-то основные характеристики системы, а какие-то более тонкие эффекты не будет показывать.
Поэтому нужен баланс (целесообразная экономичность): с одной стороны у нас есть точность результатов, а с другой стороны у нас есть затраты на моделирование (время, данные, квалификация).
В итоге, необходимо соотносить затраты с требуемой детальностью модели.
Для простых задач можем строить простые модели, которые требуют меньше времени, меньше данных.

Когда появятся в общем доступе квантовые компьютеры, то, наверное, можно будет считать суперточные модели мгновенно и иметь полную информацию обо всём, что происходит.

\subsection{Точность моделей}

\includegraphics[width=\textwidth, page=17]{Kurs_OsnovyGDM_Kai_774_gorodovSV_v6_0.pdf}

При упрощении реальных систем мы пренебрегаем какими-либо связями или характеристиками системы, и соответственно это сразу же приводит к погрешностям.

Погрешность исходных данных может быть связана как с погрешностью приборов, так и с погрешностью проведения самих экспериментов (замеров значений этих параметров), так и с погрешностью интерпретации, связанной с погрешностью методики интерпретации и непосредственно применения этой методики.
То есть несколько таких направлений, которые могут приводить к погрешностям в исходных данных.

Далее, что посеешь, то и пожнёшь: качество результатов расчётов не может превысить точности исходных данных.

\subsection{Виды моделей}

\includegraphics[width=\textwidth, page=18]{Kurs_OsnovyGDM_Kai_774_gorodovSV_v6_0.pdf}

\subsection{Гидродинамическая модель}

\includegraphics[width=\textwidth, page=19]{Kurs_OsnovyGDM_Kai_774_gorodovSV_v6_0.pdf}

\subsection{Цели гидродинамического моделирования (ГДМ)}

\includegraphics[width=\textwidth, page=20]{Kurs_OsnovyGDM_Kai_774_gorodovSV_v6_0.pdf}

По сути гидродинамическая модель служит своего рода такой базой данных, в которой собираются все результаты исследований, интерпретации исследований.
И информация об одних и тех же свойствах может идти из разных исследований.
Когда мы всю эту информацию собираем воедино, то можем увидеть какие-то нестыковки, несоответствия и выявить погрешности в исходных данных, чтобы затем уточнить: действительно ли корректны рассматриваемые значения исходных параметров, полученных при интерпретации выбранных исследований.
Может быть, стоит их пересмотреть и разобраться в причинах несоответствия.

Лучше один раз увидеть, чем сто раз услышать -- визуализация объекта разработки важна (тогда легче представить, что с этим объектом можно делать и как этот объект будет реагировать на определённые воздействия; другими словами, легче понять отклик моделируемой системы на разные воздействия и обдумать, что необходимо делать для оптимизации процессов, имеющихся на месторождении, чтобы улучшить результаты в эксплуатации).
Визуализация позволяет представить, что происходит, а именно как в динамике меняются свойства пласта, провести анализ разработки, подобрать варианты оптимизации, варианты геолого-технических мероприятий и посчитать различные прогнозы.

На месторождении мы можем сделать мероприятие только 1 раз (и свойства пласта необратимо изменятся), а в модели можем сделать сколько угодно разных мероприятий (не оказывая при этом воздействие на реальный пласт).
Следовательно, модель является инструментом принятия решений и экономит нам средства (деньги и время) на то, чтобы подобрать оптимальный способ разработки (ведь можем перебрать много разных способов, выбрать лучший и его уже реализовывать на месторождении).

\subsection{Математическая основа ГДМ}

\includegraphics[width=\textwidth, page=21]{Kurs_OsnovyGDM_Kai_774_gorodovSV_v6_0.pdf}

Здесь приведены уравнения, на которых основана гидродинамическая модель. 
Это система уравнений, которая включает в себя уравнение неразрывности сплошной среды (по сути это закон сохранения массы).

Если у нас ещё есть какие-то тепловые методы (которые применяются на месторождении), то добавляются ещё уравнения сохранения энергии.
Но как правило, это делают достаточно редко, поэтому можно сказать, что в большинстве моделей это не учитывается, а именно мы считаем процессы изотермическими, никакого теплового воздействия в пласте не происходит (всё зависит только от давления).

Также в систему уравнений входит уравнение состояния сплошной среды, которое описывает, как изменяются свойства пласта, свойства флюидов при изменении давления и температуры (если всё-таки есть тепловое воздействие).

Ещё в систему уравнений входит закон движения (фильтрации), то есть по сути различные модификации закона Дарси.

Плюс начальные и граничные условия. На слайде всё приведено в дифференциальном виде. Можно поразбираться.
Более детально не стал рассказывать, это уже для тех, кому особо интересно есть отдельный курс (несколько часов рассказывается, как получаются эти уравнения, как их затем решать).
Но что можно отсюда заметить? Это нелинейное дифференциальное уравнение в частных производных.

\includegraphics[width=\textwidth, page=22]{Kurs_OsnovyGDM_Kai_774_gorodovSV_v6_0.pdf}

Как решать такие уравнения аналитически в общем виде, науке пока неизвестно.
Такие уравнения встречаются не только в гидродинамике, но и во всей физике.
Например, в теории относительности и гравитации есть уравнение Эйнштейна (тоже нелинейное дифференциальное уравнение в частных производных).

Но есть численные методы, которые позволяют нам решать такие уравнение приближённо.
Мы разбиваем пространство и время на отрезки конечных размеров. То есть в пространстве это будут такие ячейки (кусочки пространства), а по времени -- временные шаги. И мы говорим, что в одной этой ячейке на каждый определённый шаг по времени свойства имеют одно значение (т.е пористость, проницаемость, насыщенность фиксированы).
Но эти свойства могут меняться с каждым шагом по времени.
Наступил следующий временной шаг и свойства могут измениться в зависимости от потоков через грани ячеек.
Тогда мы можем сказать, что в соответсвии с этим упрощением мы можем уравнение аппроксимировать: производные по времени заменить конечными разностями, а интеграл по объёму ячейки заменить на интеграл по поверхности. Тогда у нас уравнения упрощаются, и получается система уже более простых уравнений, которую мы можем дальше решать.
Тоже есть определённая последовательность действий: линеаризация этих уравнений, решение СЛАУ и так далее.
Но сейчас детально рассматривать не будем.
Отсюда нужно только понять, что мы разбиваем пространство и время на элементарные отрезки, за счёт этого уравнения у нас упрощаются, и мы можем их решать на компьютере численными методами.
И получать за счёт этого приближённое решение.

\subsection{Типы сеток ГДМ}

\includegraphics[width=\textwidth, page=23]{Kurs_OsnovyGDM_Kai_774_gorodovSV_v6_0.pdf}

Как можно пространство разбить на ячейки?

Самое простое: нарезать параллелепипеды, тогда получится блочно-центрированная сетка ячеек.

Но пласт у нас неровный. Осадконакопление происходит неравномерно, либо происходят какие-то тектонические процессы после осадконакопления и формирования пласта.
Соответственно пласт какой-то изогнутый и с помощью блочно-центрированных ячеек эту изогнутость воспроизвести сложно, поэтому нужно придумать более гибкие ячейки, чтобы описать изгибы пласта под землёй.

Придумали сетки ячеек, которые называются геометрией угловой точки.
Для их построения задаются направляющие линии, и на этих направляющих линиях задаются глубины точек, которые являются вершинами для ячейки и таким образом плоскости граней ячейки могут быть повёрнуты куда угодно, т.е. ячейки становятся более гибкими.
На сегодняшний момент 3D геометрия угловой точки является самым популярным способом построения сетки для геологической/гидродинамической модели, чтобы описать строение пласта.

\includegraphics[width=\textwidth, page=24]{Kurs_OsnovyGDM_Kai_774_gorodovSV_v6_0.pdf}

Также есть так называемая сетка Вороного (или перпендикулярный бисектор).
Это локально ортогональная сетка, в которой грани соседних ячеек равноудалены от центров этих ячеек.
То есть если мы расставим точки центров ячеек и нарисуем грани этих ячеек так, чтобы они были равноудалены от этих точек центров, то получатся как раз шестиугольники (подобные пчелиным сотам).
Такая сетка позволяет более точно описать приток к скважине (дальше тоже это посмотрим).

\subsection{Типы сеток ГДМ. LGR}

\includegraphics[width=\textwidth, page=25]{Kurs_OsnovyGDM_Kai_774_gorodovSV_v6_0.pdf}

Сетку можно измельчать или укрупнять. Понятно, что если будем сетку измельчать, то их количество будет расти, для каждой из этих ячеек нам придётся решать уравнения фильтрации (как из одной ячейки в другую перетекает флюид), и это будет замедлять расчёт. Но с другой стороны можем более точно в какой-то области замоделировать течение флюидов.

Здесь (как всегда) приходится искать компромисс между точностью и скоростью.
Если нужно какие-то эффекты точно воспроизвести в заданной области, то можем сетку локально измельчить.
Но также могут быть ячейки, потоки в которых нам особо неинтересны (например, в тех ячейках, где течёт в основном вода) -- такие ячейки укрупняем (тем самым уменьшаем количество ячеек и сокращаем время расчёта).

Можем строить радиальную сетку, но на практике, честно говоря, ни разу не видел, чтобы кто-то пользовался.
На радиальной сетке проводят в основном теоретические расчёты, но на практике она не используется.

\subsection{Порядок нумерации ячеек сетки}

\includegraphics[width=\textwidth, page=26]{Kurs_OsnovyGDM_Kai_774_gorodovSV_v6_0.pdf}

Как происходит нумерация ячеек сетки?

Сначала изменяется координата по $x$, потом по $y$, потом по $z$.
Начинаем с левого верхнего угла (ячейка с координатами $\left(1,1,1\right)$), следующие ячейки $\left(2,1,1\right)$, $\left(3,1,1\right)$, $\left(4,1,1\right)$ и так далее. Здесь 8 ячеек по $x$.
Далее переходим ко второму ряду по $y$, начиная с ячейки $\left(1,2,1\right)$ переходим к $\left(2,2,1\right)$ и так далее.
После всех рядов по $y$ переходим на следующий слой по $z$.

Я это рассказываю, чтобы было понимание, в каком порядке номера ячеек меняются, чтобы можно было при визуализации найти какую-то ячейку, которая вам интересна. Например, если вы знаете, какую ячейку вскрывает скважина.

\subsection{Структура файла исходных данных для симулятора ECLIPSE}

\includegraphics[width=\textwidth, page=27]{Kurs_OsnovyGDM_Kai_774_gorodovSV_v6_0.pdf}

В т-Навигаторе тоже поддерживается формат Eclipse.
И в т-Навигаторе тоже считываются DATA-файлы, которые состоят из секций, в которые сгруппированы определённые ключевые слова, описывающие модель.
По сути это чем-то похоже на программирование: есть некая команда, которая воспринимается программой симулятором, и дальше идут некие параметры выполнения этой команды.

RUNSPEC = спецификация запуска. Eclipse создавали ещё в 80-е годы на Фортране и в это время ещё не было достаточного количества оперативной памяти, следовательно, нужно было заранее определять, сколько памяти потребуется модели для расчёта.
Поэтому в этой секции указывались основные характеристики: сколько в модели будет скважин, сколько моделируемых фаз, сколько разных PVT-таблиц. В общем, такие характеристики, чтобы под них забронировать оперативную память.
Сейчас таких проблем с оперативной памятью уже нет, но исторически такая секция RUNSPEC осталась.

В секции PROPS задаются PVT-свойства флюидов и SCAL свойства (special core analysis in laboratory) взаимодействия этих флюидов с пластом.
Для получения этих свойств проводится специальный анализ флюидов и керна в лаборатории.

Секция REGIONS используется, если нам нужно задать отдельные регионы, в каждом из которых свои свойства (например, свои свойства флюида).
Когда это нужно? Например, у нас есть несколько пластов на месторождении, и в каждом из этих пластов свойства отличаются, соответственно, можем записать их как разные регионы и для каждого региона задавать свои свойства.

Секция SOLUTION описывает инициализацию модели, т.е. начальные условия (до того, как начался расчёт): какое начальное состояние по насыщенности и так далее.

В секцию SUMMARY записываются те графики, которые хотим посмотреть по результатам расчёта.
Эта секция тоже относится к симулятору Eclipse, в т-Навигаторе эта секция необязательна (в нём настройка отображаемых графиков производится в самом интерфейсе программы -- галочками отмечаются графики, которые требуется отобразить).

\subsection{Справочники для симулятора ECLIPSE}

\includegraphics[width=\textwidth, page=28]{Kurs_OsnovyGDM_Kai_774_gorodovSV_v6_0.pdf}

Ключевые слова запоминать необязательно.
И для Eclipse, и для т-Навигатора, и для других симуляторов есть справочники, которые поставляются вместе с программой.
В этих справочниках есть технический мануал, в котором описаны уравнения, заложенные в расчёт, и есть мануал, который описывает сами ключевые слова (обычно сгруппированы по первым буквам).
Следовательно, можем найти необходимое ключевое слово и посмотреть, какие параметры нужны для этого ключевого слова.

Также есть примеры файлов-моделей с различными опциями.
Если хотим смоделировать какой-либо процесс (например, закачку полимера или водогазовое воздействие), то можем просто открыть папку с готовыми примерами (как правило, эта папка совпадает с корневой папкой, в которой лежит сам симулятор) и посмотреть, какие ключевые слова используются для моделирования этого процесса.
Затем вернуться в мануал и просмотреть эти ключевые слова, чтобы понять, что необходимо задавать для моделирования этих опций и воздействий.

\subsection{Задание свойств в ячейках}

\includegraphics[width=\textwidth, page=29]{Kurs_OsnovyGDM_Kai_774_gorodovSV_v6_0.pdf}

Свойства должны быть заданы для каждой ячейки, чтобы симулятор знал, как производить расчёт.
Как правило, эти значения присваиваются центру каждой ячейки; свойства можно задать явным перечислением и, если есть повторяющиеся значения, то их можно сгруппировать (т.е. записать, что свойство в $n$ ячейках имеет значение $a$).

Значения свойств ещё могут быть заданы в виде функции (в Eclipse ключевое слово OPERATE, в т-Навигаторе ключевое слово ARITHMETIC).

Schlumberger раньше поставлял FloViz и FloGrid. Сейчас они устарели, и Schlumberger их не продаёт.

Для того, чтобы сэкономить ресурсы, расчёт производится только в активных ячейках.
Активными считаются ячейки, в которых фактически происходит поток флюида.
То есть в ячейках с глинами (неколлекторами), где нет никаких потоков флюида, нет необходимости проводить какие-либо расчёты.
Соответственно, можем просто их исключить из расчёта (по-умолчанию неактивны ячейки с нулевыми пористостью (PORO) или песчанистостью (NTG, отношение количества эффективных толщин к общим толщинам)).

Также есть ключевое слово ACTNUM, которое непосредственно задаёт активные и неактивные ячейки. Т.е. мы или геолог с помощью этого ключевого слова можем самостоятельно отметить ячейки с коллектором (песчаником) или неколлектором (глинами).
 
\includegraphics[width=\textwidth, page=30]{Kurs_OsnovyGDM_Kai_774_gorodovSV_v6_0.pdf}

На этом слайде показаны примеры задания свойств в ячейках непосредственно по ячейкам, с группировкой ячеек, с помощью ключевого слова EQUALS, с помощью копирования COPY, а также с помощью арифметических операций (в Eclipse ключевое слово MULTIPLY, в т-Навигаторе можем использовать ключевое слово ARITHMETIC).

\includegraphics[width=\textwidth, page=31]{Kurs_OsnovyGDM_Kai_774_gorodovSV_v6_0.pdf}

На этом слайде показаны примеры задания свойств в ячейках с помощью ключевого слова BOX.

На слайде приведены примеры использования ключевого слова INCLUDE.
Файлы с большими массивами данных (кубами свойств) хранятся отдельно и подключаются к основному файлу с помощью ключевого слова INCLUDE.

\subsection{Поток через ячейку}

\includegraphics[width=\textwidth, page=32]{Kurs_OsnovyGDM_Kai_774_gorodovSV_v6_0.pdf}

Как рассчитывается поток через ячейку?
Поток определяется градиентом давления между ячейками и проводимостью (т.е. насколько легко будет течь флюид через границу).
Направление градиента и направление потока противоположны.
Градиент показывает направление возрастания какой-либо величины.

\includegraphics[width=\textwidth, page=33]{Kurs_OsnovyGDM_Kai_774_gorodovSV_v6_0.pdf}

Как считается проводимость?
На этом слайде показана формула расчёта проводимости для блочно-центрированной ячейки.
В эту формулу включаются песчанистость, размеры ячеек, проницаемость, разница глубин.
Дополнительно есть множитель MULTX (множитель проводимости).
Для чего нужен этот множитель? Посмотрим дальше.

\includegraphics[width=\textwidth, page=34]{Kurs_OsnovyGDM_Kai_774_gorodovSV_v6_0.pdf}

Для геометрии угловой точки используется чуть более сложная формула.
Запоминать эти формулы не нужно; они даны для информации: какие параметры влияют на поток через грани ячейки (а именно свойства самих ячеек и площадь граней, через которые происходит переток).

\subsection{Несоседние соединения NNC}

\includegraphics[width=\textwidth, page=35]{Kurs_OsnovyGDM_Kai_774_gorodovSV_v6_0.pdf}

Как правило переток происходит между ячейками, у которых индексы отличаются на единицу, но есть ряд случаев, когда необходимо, чтобы переток был между ячейками, у которых индексы отличаются больше чем на единицу.
Например, разлом со смещением, т.е. часть пласта у нас в результате тектонической активности сместилась относительно другой части пласта.
Для таких случаев симулятор создаёт так называемые несоседние соединения NNC (грубо говоря, прописывает взаимосвязи ячеек).
На самом деле об этом можно и не задумываться, так как такие соединения создаются в автоматическом режиме, но просто полезно для информации, что такое бывает.

В ячейках с радиальной геометрией идёт нумерация по часовой стрелке (вторая координата меняется по часовой стрелке), поэтому получается, что первая и последняя ячейки граничат друг с другом, но при этом их индексы отличаются больше чем на единицу (соответственно симулятор будет себе отмечать, что переток между этими ячейками должен быть).

То же самое для водоносных горизонтов.
Если они подключаются к каким-то неактивным ячейкам, то можно сделать так, чтобы были несоседние соединения, чтобы переток с водоносных горизонтов осуществлялся в модель.

Выклинивание: если какие-либо ячейки исключаются (из-за ключевых слов PINCH или MINPV), то, чтобы не создавать искусственный барьер, возникает (симулятор автоматически прописывает) несоседнее соединение между ячейками, примыкающими к исключённой ячейке.

\subsection{Проблемы пространственной дискретизации}

\includegraphics[width=\textwidth, page=36]{Kurs_OsnovyGDM_Kai_774_gorodovSV_v6_0.pdf}

За то, что мы прибегли к упрощению (а именно, воспользовались дискретным представлением пространства и времени), нам приходится платить точностью расчёта.
Из-за дискретизации пространства и времени возникает численная ошибка, которая называется численная дисперсия.
Она говорит о том, что, чем более грубые ячейки (чем более грубо мы разрезали месторождение на ячейки), тем менее точно будет описан процесс фильтрации.

Представим себе аналогию с разрешением картинки: если у нас есть картинка с разрешением 100 на 100 пикселей, то она чёткая; если же мы делаем меньше пикселей, то картинка становиться размазанной/размытой; и при определённом загрублении мы уже не можем понять, что изображено на картинке.
То же самое и в модели.
В каждой ячейке задаётся набор свойств.
Если мы сделаем слишком грубую сетку, то представления о распределении свойств под землёй будут искажены, и соответственно мы будем получать искажённое решение.

Как и везде в итоге необходимо искать баланс: и достаточно быстро, и достаточно точно.
Но бывает и не быстро, и не точно.

Помимо измельчения сетки есть ещё способ уменьшить численную дисперсию, а именно включить эту численную дисперсию в ОФП (получить при этом так называемую псевдо-ОФП), т.е. учесть что поток идёт более плавно по этим грубым ячейкам.
Про это расскажу более подробно чуть позже, когда будем рассматривать ОФП.

\includegraphics[width=\textwidth, page=37]{Kurs_OsnovyGDM_Kai_774_gorodovSV_v6_0.pdf}
Ещё один численный эффект, возникающий при дискретизации, это эффект ориентации сетки.
Он заключается в том, что время прихода флюида из одной точки в другую зависит от того, сколько ему нужно пройти ячеек.

Видим, что в случае, когда добывающие скважины расположены по диагонали ячеек сетки, вода к ним приходит позже.
Это такой чисто численный эффект, который нужно как-то исключить. 

\includegraphics[width=\textwidth, page=38]{Kurs_OsnovyGDM_Kai_774_gorodovSV_v6_0.pdf}

Для уменьшения/исключения эффекта ориентации сетки можно измельчить сетку, использовать альтернативные численные схемы (которые учитывают взаимодействие ячеек по диагонали; естественно эти вычислительные схемы усложняют расчёт и требуют дополнительных вычислительных ресурсов), можно использовать сетку Вороного (позволяет более точно смоделировать приток к скважине, т.е. уменьшить эффект ориентации сетки) или линии тока (но линии тока являются неким упрощением, когда мы решаем для насыщенности одномерную задачу; про линии тока поговорим ещё дальше по курсу).

Для ячеек Вороного (PEBI) тоже есть сложности с решением систем уравнений, ведь у PEBI самих граней, через которые течёт поток, больше (у прямоугольной ячейки 6 граней, у ячейки Вороного 8 граней), соответственно, и сами матрицы систем уравнений становятся сложнее для решения.
Углубляться не будем.

Вообще рекомендация такая: желательно ориентировать сетку ячеек по направлению основных потоков, которые в пласте происходят.
Эти потоки могут быть связаны как с сеткой скважин, так и с региональными стрессами (какими-либо разломами, трещиноватостями). Всё равно есть преимущественные направления фильтрации, и сетку желательно ориентировать так, чтобы она была в направлении этих потоков (в направлении фильтрации).

\subsection{Построение грида}

\includegraphics[width=\textwidth, page=39]{Kurs_OsnovyGDM_Kai_774_gorodovSV_v6_0.pdf}

\subsection{Гидродинамические модели (схема)}

\includegraphics[width=\textwidth, page=40]{Kurs_OsnovyGDM_Kai_774_gorodovSV_v6_0.pdf}

\subsection{Типы расчётных моделей}

\subsubsection{Модель нелетучей нефти}

\includegraphics[width=\textwidth, page=41]{Kurs_OsnovyGDM_Kai_774_gorodovSV_v6_0.pdf}

\subsubsection{Композиционная модель}

\includegraphics[width=\textwidth, page=42]{Kurs_OsnovyGDM_Kai_774_gorodovSV_v6_0.pdf}

\subsubsection{Термические модели}

\includegraphics[width=\textwidth, page=43]{Kurs_OsnovyGDM_Kai_774_gorodovSV_v6_0.pdf}

\subsubsection{Модель двойной или мульти-среды}

\includegraphics[width=\textwidth, page=44]{Kurs_OsnovyGDM_Kai_774_gorodovSV_v6_0.pdf}

\subsubsection{Модели линий тока}

\includegraphics[width=\textwidth, page=45]{Kurs_OsnovyGDM_Kai_774_gorodovSV_v6_0.pdf}

\subsubsection{Proxy-модели}

\includegraphics[width=\textwidth, page=46]{Kurs_OsnovyGDM_Kai_774_gorodovSV_v6_0.pdf}

\includegraphics[width=\textwidth, page=47]{Kurs_OsnovyGDM_Kai_774_gorodovSV_v6_0.pdf}

\subsubsection{Суррогатные (мета) модели}

\includegraphics[width=\textwidth, page=48]{Kurs_OsnovyGDM_Kai_774_gorodovSV_v6_0.pdf}

\includegraphics[width=\textwidth, page=49]{Kurs_OsnovyGDM_Kai_774_gorodovSV_v6_0.pdf}

\includegraphics[width=\textwidth, page=50]{Kurs_OsnovyGDM_Kai_774_gorodovSV_v6_0.pdf}

\subsection{Иерархия гидродинамических моделей}

\includegraphics[width=\textwidth, page=51]{Kurs_OsnovyGDM_Kai_774_gorodovSV_v6_0.pdf}

\subsection{Местоположение моделирования в цикле нефтедобычи}

\includegraphics[width=\textwidth, page=52]{Kurs_OsnovyGDM_Kai_774_gorodovSV_v6_0.pdf}

\includegraphics[width=\textwidth, page=53]{Kurs_OsnovyGDM_Kai_774_gorodovSV_v6_0.pdf}

\subsection{Источники геологической информации в масштабах месторождения}

\includegraphics[width=\textwidth, page=54]{Kurs_OsnovyGDM_Kai_774_gorodovSV_v6_0.pdf}

\subsection{Охват исследованием и погрешность}

\includegraphics[width=\textwidth, page=55]{Kurs_OsnovyGDM_Kai_774_gorodovSV_v6_0.pdf}

\subsection{Исходные данные для гидродинамического моделирования}

\includegraphics[width=\textwidth, page=56]{Kurs_OsnovyGDM_Kai_774_gorodovSV_v6_0.pdf}

\includegraphics[width=\textwidth, page=58]{Kurs_OsnovyGDM_Kai_774_gorodovSV_v6_0.pdf}

\subsection{Подходы к построению ПДГГДМ}

\includegraphics[width=\textwidth, page=57]{Kurs_OsnovyGDM_Kai_774_gorodovSV_v6_0.pdf}

\subsection{Ремасштабирование геомодели}

\includegraphics[width=\textwidth, page=59]{Kurs_OsnovyGDM_Kai_774_gorodovSV_v6_0.pdf}

\subsection{Ремасштабирование структуры (upgridding)}

\includegraphics[width=\textwidth, page=60]{Kurs_OsnovyGDM_Kai_774_gorodovSV_v6_0.pdf}

\includegraphics[width=\textwidth, page=61]{Kurs_OsnovyGDM_Kai_774_gorodovSV_v6_0.pdf}

\subsection{Ремасштабирование свойств}

\includegraphics[width=\textwidth, page=62]{Kurs_OsnovyGDM_Kai_774_gorodovSV_v6_0.pdf}

\subsection{Ремасштабирование проницаемости}

\includegraphics[width=\textwidth, page=63]{Kurs_OsnovyGDM_Kai_774_gorodovSV_v6_0.pdf}

\includegraphics[width=\textwidth, page=64]{Kurs_OsnovyGDM_Kai_774_gorodovSV_v6_0.pdf}

\subsection{Ремасштабирование геомодели. Контроль качества}

\includegraphics[width=\textwidth, page=65]{Kurs_OsnovyGDM_Kai_774_gorodovSV_v6_0.pdf}

\subsection{Поверхностное натяжение}

\includegraphics[width=\textwidth, page=66]{Kurs_OsnovyGDM_Kai_774_gorodovSV_v6_0.pdf}

Далее необходимо рассмотреть, каким образом флюиды взаимодействуют друг с другом и с пластом в ходе фильтрации. Это взаимодействие в основном связано с поверхностным натяжением. Считаем, что химического взаимодействия в пласте не происходит.

\subsection{Смачиваемость}

\includegraphics[width=\textwidth, page=67]{Kurs_OsnovyGDM_Kai_774_gorodovSV_v6_0.pdf}

\subsection{Капиллярное давление}

\includegraphics[width=\textwidth, page=68]{Kurs_OsnovyGDM_Kai_774_gorodovSV_v6_0.pdf}

Чем больше радиус капилляра, тем меньше капиллярное давление и на меньшую высоту поднимется вода от уровня равновесия (зеркала свободной воды).

Для одного и того же капиллярного давления: чем больше разница плотностей, тем на меньшую высоту флюид поднимется в капилляре.
Поэтому переходная зона между нефтью и водой значительно больше, чем переходная зона между нефтью и газом.
На самом деле, переходную зону между нефтью и газом моделируют очень редко: обычно просто задают газонефтяной контакт (ГНК) в пределах одной ячейки.

Нелинейная фильтрация связана с вязкостью жидкости и капиллярными эффектами (запирающий градиент / давление сдвига). Подумать об этом и почитать подробнее про нелинейную фильтрацию!

\includegraphics[width=\textwidth, page=69]{Kurs_OsnovyGDM_Kai_774_gorodovSV_v6_0.pdf}

Зеркало свободной воды = капиллярное давление равно нулю.
От этого уровня считается высота подъёма воды по капиллярам.

Определение ВНК (водонефтяного контакта) не так однозначно (есть несколько разных определений).

\includegraphics[width=\textwidth, page=70]{Kurs_OsnovyGDM_Kai_774_gorodovSV_v6_0.pdf}

\includegraphics[width=\textwidth, page=71]{Kurs_OsnovyGDM_Kai_774_gorodovSV_v6_0.pdf}

По виду капиллярной кривой можно судить об однородности коллектора и о размере пор.

Для узких пор полка (практически постоянное значение на графике) по капиллярному давлению находится выше, чем для широких пор.

Для неоднородного коллектора нет полки по капиллярному давлению (плавный переход).

\subsection{Капиллярное давление для разных типов породы}

\includegraphics[width=\textwidth, page=72]{Kurs_OsnovyGDM_Kai_774_gorodovSV_v6_0.pdf}

Во втором песчанике самый худший коллектор (самые узкие поры).

\subsection{J-функция Леверетта}

\includegraphics[width=\textwidth, page=73]{Kurs_OsnovyGDM_Kai_774_gorodovSV_v6_0.pdf}

$\sqrt{\dfrac{k}{\varphi}}$ характеризует извилистость поровых каналов.

Рассчитываем значения J-функции, и далее строим график, подобный представленному справа: отмечаем подсчитанные точки и аппроксимируем их некой зависимостью (которую в дальнейшем будем использовать в расчётах ГДМ).

На графике могут получиться не одно облако точек, а два или три (если есть несколько пластов с разными характеристиками или разные блоки на месторождении, в каждом из которых получился свой тип коллектора вследствие разных геологических процессов).
Тогда будет несколько аппроксимирующих кривых, которые можно использовать отдельно для каждого рассматриваемого блока или пласта соответственно.

\subsection{Капиллярное давление. Лабораторные исследования}

\includegraphics[width=\textwidth, page=74]{Kurs_OsnovyGDM_Kai_774_gorodovSV_v6_0.pdf}

Иногда проводятся керновые лабораторные исследования не с пластовыми флюидами.
Например, с ртутью и воздухом.
И полученные данные пытаются применить для пласта.
Но в наше время так делают только самые отсталые лаборатории.
Сейчас стараются извлекать флюид, имеющийся на месторождении, и использовать его в экспериментах с керном.
Если же исследование уже проведено в системе ртуть-воздух, то придётся их пересчитать в систему нефть-вода по формуле, представленной на слайде.
При этом понадобятся значения, представленные в таблице.

\subsection{ОФП}

\includegraphics[width=\textwidth, page=75]{Kurs_OsnovyGDM_Kai_774_gorodovSV_v6_0.pdf}

Поверхностное натяжение кроме капиллярного давления приводит ещё к взаимному сопротивлению фильтрации нескольких флюидов.

Относительная фазовая проницаемость (ОФП) флюида 1 в присутствии флюида 2 -- это некий множитель (зависящий от насыщенности флюида 1) перед абсолютной проницаемостью, который позволяет найти эффективную проницаемость флюида 1 в присутствии флюида 2.

В рассматриваемой на слайде ситуации (50\% воды и 50 \% нефти) из графиков зависимости ОФП от водонасыщенности видим, что эффективная проницаемость воды будет составлять 5\% от абсолютной проницаемости, а эффективная проницаемость нефти будет составлять 15\% от абсолютной проницаемости.

\subsection{Смачиваемость. Критерий Craig (1971)}

\includegraphics[width=\textwidth, page=76]{Kurs_OsnovyGDM_Kai_774_gorodovSV_v6_0.pdf}

По виду кривых ОФП можем сделать вывод о гидрофобности или гидрофильности рассматриваемой породы.

Для гидрофильной породы вода прилипает к стенкам поры. Следовательно, связанная водонасыщенность будет достаточно большой (как правило, больше 20\%) и максимальная ОФП по воде будет иметь небольшое значение (как правило меньше 0.3).
Кривая ОФП по воде прижата к оси абсцисс: точка пересечения кривых ОФП будет правее 50\% по насыщенности.

Для гидрофобной породы наоборот: нефть прилипает к порам, а вода нет. Следовательно, кривая ОФП по нефти более прижата, а по воде более поднята.
Связанная водонасыщенность меньше 15\%, максимальная ОФП по воде больше 50\%. Точка пересечения кривых ОФП будет левее 50\%.

\subsection{Гистерезис ОФП}

\includegraphics[width=\textwidth, page=77]{Kurs_OsnovyGDM_Kai_774_gorodovSV_v6_0.pdf}

В школьном курсе физики изучали гистерезис для упругих свойств (сжатие-растяжение) при преодолении определённого значения напряжения.

В рассматриваемом случае гистерезис наблюдается вследствие зависимости ОФП от направления фильтрации (вода вытесняет нефть или нефть воду).

\subsection{ОФП. Лабораторные исследования}

\includegraphics[width=\textwidth, page=78]{Kurs_OsnovyGDM_Kai_774_gorodovSV_v6_0.pdf}

Теория Баклея-Леверетта.
На основе ОФП можем рассчитать, каким образом будет происходить заводнение в пласте  (другими словами, как будет продвигаться фронт вытеснения).

ОФП совместно с соотношением вязкостей нефти и воды влияют на скорость распространения фронта заводнения и на величину скачка насыщенности.

$f_w(S_w)$ -- функция фракционного потока.

Графический анализ (по Уэлджу): зная угол наклона касательной к кривой фракционного потока (графику зависимости $f_w(S_w)$), можем найти скорость продвижения фронта заводнения.

Насыщенность в точке касания -- это насыщенность на фронте вытеснения.

Насыщенность в точке пересечения касательной и горизонтальной прямой $f_w=1$ -- это средняя насыщенность от нагнетательной скважины до края заводнения.

Таким образом, даже без построения модели, имея только ОФП и вязкости, можем многое рассказать о том, каким образом будет происходить вытеснение.

\includegraphics[width=\textwidth, page=79]{Kurs_OsnovyGDM_Kai_774_gorodovSV_v6_0.pdf}

Есть 2 режима лабораторных исследований: установившийся и неустановившийся.

\includegraphics[width=\textwidth, page=80]{Kurs_OsnovyGDM_Kai_774_gorodovSV_v6_0.pdf}

По стандартам все исследования должны проводиться на установившемся режиме.
Минус такого подхода: для низкопроницаемых образцов время установления может занимать месяц или даже несколько месяцев.
Это дорого.
Поэтому иногда проводят быстрые исследования на неустановившемся режиме, но это менее точно и не соответствует стандартам.

\includegraphics[width=\textwidth, page=81]{Kurs_OsnovyGDM_Kai_774_gorodovSV_v6_0.pdf}

\subsection{ОФП. Корреляции Corey и LET}

\includegraphics[width=\textwidth, page=82]{Kurs_OsnovyGDM_Kai_774_gorodovSV_v6_0.pdf}

Аппроксимация проводится с целью удобства: необходимо, чтобы ОФП были гладкими функциями.
Это позволяет легче находить решение при использовании численных схем.

Корреляция LET (появилась 15-20 лет) позволяет лучше описать лабораторные исследования: есть участки с разной выпуклостью/вогнутостью.

\subsection{Как задать ОФП в ГДМ, если есть несколько исследований?}

\includegraphics[width=\textwidth, page=83]{Kurs_OsnovyGDM_Kai_774_gorodovSV_v6_0.pdf}



\includegraphics[width=\textwidth, page=84]{Kurs_OsnovyGDM_Kai_774_gorodovSV_v6_0.pdf}

\subsection{Концевые точки ОФП в системе нефть-вода}

\includegraphics[width=\textwidth, page=85]{Kurs_OsnovyGDM_Kai_774_gorodovSV_v6_0.pdf}

\includegraphics[width=\textwidth, page=86]{Kurs_OsnovyGDM_Kai_774_gorodovSV_v6_0.pdf}

\subsection{Масштабирование ОФП}

\includegraphics[width=\textwidth, page=87]{Kurs_OsnovyGDM_Kai_774_gorodovSV_v6_0.pdf}

\includegraphics[width=\textwidth, page=88]{Kurs_OsnovyGDM_Kai_774_gorodovSV_v6_0.pdf}

\includegraphics[width=\textwidth, page=89]{Kurs_OsnovyGDM_Kai_774_gorodovSV_v6_0.pdf}

\includegraphics[width=\textwidth, page=90]{Kurs_OsnovyGDM_Kai_774_gorodovSV_v6_0.pdf}

\subsubsection{По горизонтали (по насыщенности)}

\includegraphics[width=\textwidth, page=91]{Kurs_OsnovyGDM_Kai_774_gorodovSV_v6_0.pdf}

\subsubsection{По вертикали}

\includegraphics[width=\textwidth, page=92]{Kurs_OsnovyGDM_Kai_774_gorodovSV_v6_0.pdf}

\subsection{Согласованность массивов в модели}

\includegraphics[width=\textwidth, page=93]{Kurs_OsnovyGDM_Kai_774_gorodovSV_v6_0.pdf}

\includegraphics[width=\textwidth, page=94]{Kurs_OsnovyGDM_Kai_774_gorodovSV_v6_0.pdf}

\includegraphics[width=\textwidth, page=95]{Kurs_OsnovyGDM_Kai_774_gorodovSV_v6_0.pdf}

\subsection{Ремасштабирование (2-х фазный апскелинг ОФП)}

\includegraphics[width=\textwidth, page=96]{Kurs_OsnovyGDM_Kai_774_gorodovSV_v6_0.pdf}

\includegraphics[width=\textwidth, page=97]{Kurs_OsnovyGDM_Kai_774_gorodovSV_v6_0.pdf}

\subsection{Типы флюидов}

\includegraphics[width=\textwidth, page=98]{Kurs_OsnovyGDM_Kai_774_gorodovSV_v6_0.pdf}

\includegraphics[width=\textwidth, page=99]{Kurs_OsnovyGDM_Kai_774_gorodovSV_v6_0.pdf}

\subsection{Определение типа залежи по составу УВ}

\includegraphics[width=\textwidth, page=100]{Kurs_OsnovyGDM_Kai_774_gorodovSV_v6_0.pdf}

\subsection{PVT-свойства}

\includegraphics[width=\textwidth, page=101]{Kurs_OsnovyGDM_Kai_774_gorodovSV_v6_0.pdf}

\subsection{PVT-свойства нефти}

\includegraphics[width=\textwidth, page=102]{Kurs_OsnovyGDM_Kai_774_gorodovSV_v6_0.pdf}

\subsection{PVT-свойства нефти. Корреляции}

\includegraphics[width=\textwidth, page=103]{Kurs_OsnovyGDM_Kai_774_gorodovSV_v6_0.pdf}

\subsection{PVT-свойства "<живой нефти">}

\includegraphics[width=\textwidth, page=104]{Kurs_OsnovyGDM_Kai_774_gorodovSV_v6_0.pdf}

\subsection{Варианты описания PVT в моделях Black Oil}

\includegraphics[width=\textwidth, page=105]{Kurs_OsnovyGDM_Kai_774_gorodovSV_v6_0.pdf}

\includegraphics[width=\textwidth, page=106]{Kurs_OsnovyGDM_Kai_774_gorodovSV_v6_0.pdf}

\subsection{Сжимаемость порового пространства}

\includegraphics[width=\textwidth, page=107]{Kurs_OsnovyGDM_Kai_774_gorodovSV_v6_0.pdf}

\includegraphics[width=\textwidth, page=108]{Kurs_OsnovyGDM_Kai_774_gorodovSV_v6_0.pdf}

\subsection{Сжимаемость порового пространства. Корреляции}

\includegraphics[width=\textwidth, page=109]{Kurs_OsnovyGDM_Kai_774_gorodovSV_v6_0.pdf}

\subsection{Упражнение 1. Упражнения на обработку и подготовку исходных данных}

\includegraphics[width=\textwidth, page=110]{Kurs_OsnovyGDM_Kai_774_gorodovSV_v6_0.pdf}

\includegraphics[width=\textwidth, page=111]{Kurs_OsnovyGDM_Kai_774_gorodovSV_v6_0.pdf}

\includegraphics[width=\textwidth, page=112]{Kurs_OsnovyGDM_Kai_774_gorodovSV_v6_0.pdf}

\includegraphics[width=\textwidth, page=113]{Kurs_OsnovyGDM_Kai_774_gorodovSV_v6_0.pdf}

\end{document}
