\documentclass[main.tex]{subfiles}

\begin{document}


\section{Лекция 12.09.2022 (Кайгородов С.В.)}

\subsection{Цели курса}

\includegraphics[width=\textwidth]{im7_new}

\subsection{Что такое модель?}

\includegraphics[width=\textwidth]{im15_new}

\subsection{Требования к моделям}

\includegraphics[width=\textwidth]{im16_new}

\subsection{Точность моделей}

\includegraphics[width=\textwidth]{im17_new}

\subsection{Виды моделей}

\includegraphics[width=\textwidth]{im18_new}

\subsection{Гидродинамическая модель}

\includegraphics[width=\textwidth]{im19_new}

\subsection{Цели гидродинамического моделирования (ГДМ)}

\includegraphics[width=\textwidth]{im20_new}

\subsection{Математическая основа ГДМ}

\includegraphics[width=\textwidth]{im21_new}

\includegraphics[width=\textwidth]{im22_new}

\subsection{Типы сеток ГДМ}

\includegraphics[width=\textwidth]{im23_new}

\includegraphics[width=\textwidth]{im24_new}

\subsection{Типы сеток ГДМ. LGR}

\includegraphics[width=\textwidth]{im25_new}

\subsection{Порядок нумерации ячеек сетки}

\includegraphics[width=\textwidth]{im26_new}

\subsection{Структура файла исходных данных для симулятора ECLIPSE}

\includegraphics[width=\textwidth]{im27_new}

\subsection{Справочники для симулятора ECLIPSE}

\includegraphics[width=\textwidth]{im28_new}

\subsection{Задание свойств в ячейках}

\includegraphics[width=\textwidth]{im29_new}

\includegraphics[width=\textwidth]{im30_new}

\includegraphics[width=\textwidth]{im31_new}

\subsection{Поток через ячейку}

\includegraphics[width=\textwidth]{im32_new}

\includegraphics[width=\textwidth]{im33_new}

\includegraphics[width=\textwidth]{im34_new}

\subsection{Несоседние соединения NNC}

\includegraphics[width=\textwidth]{im35_new}

\subsection{Проблемы пространственной дискретизации}

\includegraphics[width=\textwidth]{im36_new}

\includegraphics[width=\textwidth]{im37_new}

\includegraphics[width=\textwidth]{im38_new}

\subsection{Построение грида}

\includegraphics[width=\textwidth]{im39_new}

\subsection{Гидродинамические модели (схема)}

\includegraphics[width=\textwidth]{im40_new}

\subsection{Типы расчётных моделей}

\subsubsection{Модель нелетучей нефти}

\includegraphics[width=\textwidth]{im41_new}

\subsubsection{Композиционная модель}

\includegraphics[width=\textwidth]{im42_new}

\subsubsection{Термические модели}

\includegraphics[width=\textwidth]{im43_new}

\subsubsection{Модель двойной или мульти-среды}

\includegraphics[width=\textwidth]{im44_new}

\subsubsection{Модели линий тока}

\includegraphics[width=\textwidth]{im45_new}

\subsubsection{Proxy-модели}

\includegraphics[width=\textwidth]{im46_new}

\includegraphics[width=\textwidth]{im47_new}

\subsubsection{Суррогатные (мета) модели}

\includegraphics[width=\textwidth]{im48_new}

\includegraphics[width=\textwidth]{im49_new}

\includegraphics[width=\textwidth]{im50_new}

\subsection{Иерархия гидродинамических моделей}

\includegraphics[width=\textwidth]{im51_new}

\subsection{Местоположение моделирования в цикле нефтедобычи}

\includegraphics[width=\textwidth]{im52_new}

\includegraphics[width=\textwidth]{im53_new}

\subsection{Источники геологической информации в масштабе месторождения}

\includegraphics[width=\textwidth]{im54_new}

\subsection{Охват исследованием и погрешность}

\includegraphics[width=\textwidth]{im55_new}

\subsection{Исходные данные для гидродинамического моделирования}

\includegraphics[width=\textwidth]{im56_new}

\includegraphics[width=\textwidth]{im58_new}

\subsection{Подходы к построению ПДГГДМ}

\includegraphics[width=\textwidth]{im57_new}

\subsection{Ремасштабирование геомодели}

\includegraphics[width=\textwidth]{im59_new}

\subsection{Ремасштабирование структуры (upgridding)}

\includegraphics[width=\textwidth]{im60_new}

\includegraphics[width=\textwidth]{im61_new}

\subsection{Ремасштабирование свойств}

\includegraphics[width=\textwidth]{im62_new}

\subsection{Ремасштабирование проницаемости}

\includegraphics[width=\textwidth]{im63_new}

\includegraphics[width=\textwidth]{im64_new}

\subsection{Ремасштабирование геомодели. Контроль качества}

\includegraphics[width=\textwidth]{im65_new}

\subsection{Поверхностное натяжение}

\includegraphics[width=\textwidth]{im66_new}

\subsection{Смачиваемость}

\includegraphics[width=\textwidth]{im67_new}

\subsection{Капиллярное давление}

\includegraphics[width=\textwidth]{im68_new}

\includegraphics[width=\textwidth]{im69_new}

\includegraphics[width=\textwidth]{im70_new}

\includegraphics[width=\textwidth]{im71_new}

\subsection{Капиллярное давление для разных типов породы}

\includegraphics[width=\textwidth]{im72_new}

\subsection{J-функция Леверетта}

\includegraphics[width=\textwidth]{im73_new}

\subsection{Капиллярное давление. Лабораторные исследования}

\includegraphics[width=\textwidth]{im74_new}

\subsection{ОФП}

\includegraphics[width=\textwidth]{im75_new}

\subsection{Смачиваемость. Критерий Craig (1971)}

\includegraphics[width=\textwidth]{im76_new}

\subsection{Гистерезис ОФП}

\includegraphics[width=\textwidth]{im77_new}

\subsection{ОФП. Лабораторные исследования}

\includegraphics[width=\textwidth]{im78_new}

\includegraphics[width=\textwidth]{im79_new}

\includegraphics[width=\textwidth]{im80_new}

\includegraphics[width=\textwidth]{im81_new}

\includegraphics[width=\textwidth]{im82_new}

\includegraphics[width=\textwidth]{im83_new}

\includegraphics[width=\textwidth]{im84_new}

\subsection{Концевые точки ОФП в системе нефть-вода}

\includegraphics[width=\textwidth]{im85_new}

\includegraphics[width=\textwidth]{im86_new}

\subsection{Масштабирование ОФП}

\includegraphics[width=\textwidth]{im87_new}

\includegraphics[width=\textwidth]{im88_new}

\includegraphics[width=\textwidth]{im89_new}

\includegraphics[width=\textwidth]{im90_new}

\subsubsection{По горизонтали}

\includegraphics[width=\textwidth]{im91_new}

\subsubsection{По вертикали}

\includegraphics[width=\textwidth]{im92_new}

\subsection{Согласованность массивов в модели}

\includegraphics[width=\textwidth]{im93_new}

\includegraphics[width=\textwidth]{im94_new}

\includegraphics[width=\textwidth]{im95_new}

\subsection{Ремасштабирование (2-х фазный апскелинг ОФП)}

\includegraphics[width=\textwidth]{im96_new}

\includegraphics[width=\textwidth]{im97_new}

\subsection{Типы флюидов}

\includegraphics[width=\textwidth]{im98_new}

\includegraphics[width=\textwidth]{im99_new}

\subsection{Определение типа залежи по составу УВ}

\includegraphics[width=\textwidth]{im100_new}

\subsection{PVT-свойства}

\includegraphics[width=\textwidth]{im101_new}

\subsection{PVT-свойства нефти}

\includegraphics[width=\textwidth]{im102_new}

\subsection{PVT-свойства нефти. Корреляции}

\includegraphics[width=\textwidth]{im103_new}

\subsection{PVT-свойства "<живой нефти">}

\includegraphics[width=\textwidth]{im104_new}

\subsection{Варианты описания PVT в моделях Black Oil}

\includegraphics[width=\textwidth]{im105_new}

\includegraphics[width=\textwidth]{im106_new}

\subsection{Сжимаемость порового пространства}

\includegraphics[width=\textwidth]{im107_new}

\includegraphics[width=\textwidth]{im108_new}

\subsection{Сжимаемость порового пространства. Корреляции}

\includegraphics[width=\textwidth]{im109_new}

\subsection{Упражнения}

\includegraphics[width=\textwidth]{im110_new}

\includegraphics[width=\textwidth]{im111_new}

\includegraphics[width=\textwidth]{im112_new}

\includegraphics[width=\textwidth]{im113_new}

\end{document}
