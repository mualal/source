\documentclass[main.tex]{subfiles}

\begin{document}

% \textcolor{red}{Вводная лекция}

\section{Экзамен 12.01.2023 (Базыров И.Ш.)}

\subsection{Какие режимы течения Вы знаете? Раскрыть суть каждого режима.}

\newpage

\subsection{Какие режимы притока Вы знаете? Раскрыть суть каждого режима.}

\newpage

\subsection{Что такое уравнение Дарси, как выводится уравнение в общем виде?}

\newpage

\subsection{Вывод уравнения Дарси для линейного течения}

\newpage

\subsection{Вывод уравнения Дарси для радиального течения}

\newpage

\subsection{Раскрыть суть уравнения Дарси-Дюпюи, как выводится уравнение? Как получен коэффициент в формуле Дарси-Дюпюи, равный 18.41?}

\newpage

\subsection{Раскрыть суть уравнения пьезопроводности, как выводится уравнение пьезопроводности для "<неупругого пласта">? Что такое коэффициент пьезопроводности?}

Уравнение пьезопроводности -- основное уравнение гидродинамики, с помощью которого описывается процесс фильтрации жидкости в пористых средах.
\\

Уравнение пьезопроводности строится на трёх уравнениях: на уравнении неразрывности потока (т.е. уравнении переноса массы, записанном в дифференциальной форме), на законе Дарси и на соотношении для сжимаемости флюида.
\\

Набор исходных уравнений для вывода уравнения пьезопроводности:
\begin{itemize}
	\item неразрывность потока
	\beq\label{Continuity0}
	\frac{\partial\left(\rho_f\varphi\right)}{\partial t}+\pmb{\nabla}\cdot\left(\rho_f\varphi \pmb{v_f}\right)=q_f(\pmb{x})
	\eeq
	\item закон Дарси
	\beq\label{Darcy0}
	\pmb{W}=-\frac{k}{\mu_f}\cdot\pmb{\nabla} p
	\eeq
	\item сжимаемость флюида
	\beq\label{Compressibility0}
	p-p_0=K_f\frac{\rho_f-\rho_f^0}{\rho_f^0}
	\eeq
\end{itemize}
\ \\

\textbf{Вывод уравнения пьезопроводности в векторной форме (быстрый, но не совсем строгий вывод)}

В предположении неподвижности скелета ($\pmb{v_s}\approx \pmb{0}$ и $\varphi(t)=\textrm{const}$) верно равенство $\pmb{W}\approx\varphi \pmb{v_f}$.
Подставляя это выражение в закон Дарси \eqref{Darcy0}, получаем:
\beq\label{DarcyWithSkeletNotMoving0}
\varphi \pmb{v_f}=-\frac{k}{\mu_f}\cdot\pmb{\nabla} p
\eeq

Условие сжимаемости флюида \eqref{Compressibility0} перепишем в дифференциальной форме:
\beq\label{CompressibilityDiff0}
\frac{\partial p}{\partial t}=\frac{K_f}{\rho_f^0}\frac{\partial\rho_f}{\partial t}
\eeq

Учитывая предположение о неподвижности скелета, перепишем уравнение неразрывности потока:
\beq\label{ContinuityWithSkeletNotMoving0}
\varphi\frac{\partial\rho_f}{\partial t}+\pmb{\nabla}\cdot\left(\rho_f\varphi\pmb{v_f}\right)=q_f(\pmb{x})
\eeq

Подставляя \eqref{DarcyWithSkeletNotMoving0} и \eqref{CompressibilityDiff0} в \eqref{ContinuityWithSkeletNotMoving0}, при отсутствии источникового слагаемого ($q_f(\pmb{x})=0$) получаем:
\beq
\varphi\frac{\rho_f^0}{K_f}\frac{\partial p}{\partial t}-\pmb{\nabla}\cdot\left(\rho_f\frac{k}{\mu_f}\pmb{\nabla} p\right)=0
\eeq

При дополнительном условии слабосжимаемости флюида ($\rho_f\approx\rho_f^0=\textrm{const}$) получаем:
\beq
\frac{\partial p}{\partial t}=\frac{kK_f}{\mu_f\varphi}\pmb{\nabla}^2p
\eeq

Это уравнение пьезопроводности (без упругости пласта), полученное в приближении слабосжимаемого флюида, неподвижного и недеформируемого пласта.
\\

\textbf{Вывод уравнения пьезопроводности в покомпонентной форме с обезразмериванием (от Шеля Е.В.)}

Запишем ЗСМ для флюида:
\beq
\frac{\partial r_f}{\partial t}+\partial_i\left(r_f v_i^f\right)=0
\eeq

Закон Дарси в "<школьной"> форме:
\beq
Q=-\frac{\Delta p}{L}\frac{k}{\mu}S
\eeq

Закон Дарси в дифференциальной форме:
\beq\label{DarcyDiffShel0}
W_i=-\frac{k_{ij}}{\mu}\partial_j p,
\eeq
где $W_i=\varphi v_i^f$ -- потоковая относительная скорость флюида.

Учитывая связь эффективной и истинной плотностей ($r_f=\varphi\rho_f$), перепишем ЗСМ для флюида:
\beq\label{ContinuityShel0}
\frac{\partial\left(\rho_f\varphi\right)}{\partial t}+\partial_i\left(\rho_f\varphi v_i^f\right)=0
\eeq

Подставляя \eqref{DarcyDiffShel0} в \eqref{ContinuityShel0}, получаем:
\beq\label{GeneralPiezo0}
\frac{\partial\left(\rho_f\varphi\right)}{\partial t}-\partial_i\left(\rho_f\frac{k_{ij}}{\mu}\partial_j p\right)=0
\eeq

--------------------------------------------------------------------

Замыкающее соотношение (связь плотности флюида и давления):
\beq\label{Zam10}
\rho_f=\rho_f^0\left(1+c_f\left(p-p_0\right)\right),
\eeq
где $c_f$ -- сжимаемость флюида (1/Па).


Замыкающее соотношение (связь пористости и давления):
\beq\label{Zam20}
\varphi=\varphi^0+c_{\text{п}}\left(p-p_0\right),
\eeq
где $c_{\text{п}}$ -- сжимаемость пор (не равно сжимаемости породы).

--------------------------------------------------------------------

Продифференцируем по времени замыкающее соотношение \eqref{Zam10}:
\beq\label{DiffZam10}
\frac{\partial\rho_f}{\partial t}=c_f\rho_f^0\frac{\partial p}{\partial t}
\eeq

Продифференцируем по пространству замыкающее соотношение \eqref{Zam10}:
\beq\label{GradZam10}
\partial_i\rho_f=c_f\rho_f^0\partial_i p
\eeq

Продифференцируем по времени замыкающее соотношение \eqref{Zam20}:
\beq\label{DiffZam20}
\frac{\partial\varphi}{\partial t}=c_\text{п}\frac{\partial p}{\partial t}
\eeq

Продифференцируем по пространству замыкающее соотношение \eqref{Zam20}:
\beq\label{GradZam20}
\partial_i\varphi=c_\text{п}\partial_i p
\eeq

--------------------------------------------------------------------

Раскрывая производные произведений в \eqref{GeneralPiezo0}, получаем:
\beq\label{OpenGeneralContinuity0}
\frac{\partial\rho_f}{\partial t}\varphi+\rho_f\frac{\partial\varphi}{\partial t}-\frac{k_{ij}}{\mu}\partial_j p\,\partial_i\rho_f-\rho_f\partial_j p\,\partial_i\!\left(\frac{k_{ij}}{\mu}\right)-\rho_f\frac{k_{ij}}{\mu}\left(\partial_i\partial_j p\right)=0
\eeq

Подставляя \eqref{DiffZam10}, \eqref{GradZam10}, \eqref{DiffZam20} и \eqref{GradZam20} в \eqref{OpenGeneralContinuity0}, получаем:
\begin{multline}\label{Expanded0}
c_f\rho_f^0\frac{\partial p}{\partial t}\varphi+\rho_f c_\text{п}\frac{\partial p}{\partial t}-\frac{k_{ij}}{\mu}\partial_j p\,c_f\rho_f^0\,\partial_i p-\frac{\rho_f}{\mu}\partial_j p\,\partial_i k_{ij}+\\+\rho_f\,\partial_j p\,k_{ij}\frac{\partial\mu}{\partial p}\frac{1}{\mu^2}\partial_i p-\rho_f\frac{k_{ij}}{\mu}\left(\partial_i\partial_j p\right)=0
\end{multline}

--------------------------------------------------------------------

Перед анализом физических уравнений всегда делают масштабный анализ, чтобы понять, какие слагаемые в уравнении важны, а какие не важны (пример: уравнение Навье-Стокса с числами Струхаля, Эйлера, Рейнольдса, Фруда).

Спойлер: ГДМ симуляторы не решают уравнение пьезопроводности в классическом виде, а решают закон сохранения массы, в который они подставляют закон Дарси.

Далее необходимо выделить характерные масштабные факторы, обезразмерив каждую из функций в уравнении.

Введём безразмерное давление $\tilde{p}$ такое, что:
\beq
p=\tilde{p}\cdot p_0,
\eeq
где $p_0$ -- пластовое давление.

Введём безразмерное расстояние $\tilde{r}$ такое, что:
\beq
\vec{r}=\tilde{r}\cdot L,
\eeq
где $L$ -- некое характерное расстояние (например, между скважинами).

Введём безразмерную проницаемость $\tilde{k}_{ij}$ такую, что:
\beq
k_{ij}=\tilde{k}_{ij}\cdot k_0,
\eeq
где $k_0$ -- некая характерная проницаемость.

Введём безразмерную вязкость $\tilde{\mu}$ такую, что:
\beq
\mu=\tilde{\mu}\cdot\mu_0,
\eeq
где $\mu_0$ -- некая характерная вязкость.

Все безразмерные функции (с волной) порядка единицы.

--------------------------------------------------------------------

Перепишем \eqref{Expanded0} в введённых безразмерных величинах, разделив обе части этого уравнения на $\rho_f^0$:
\begin{multline}
\frac{\partial p}{\partial t}\left(\varphi c_f+\frac{\rho_f}{\rho_f^0}\cdot c_\text{п}\right)-\frac{k_0}{\mu_0}\frac{p_0^2}{L^2}c_f\frac{\tilde{k}_{ij}}{\tilde{\mu}}\,\tilde{\partial}_i\tilde{p}\,\tilde{\partial}_j\tilde{p}-\frac{\rho_f}{\rho_f^0}\frac{k_0\,p_0}{\mu_0L^2}\frac{\tilde{k}_{ij}}{\tilde{\mu}}\,\tilde{\partial}_j\tilde{p}\,\tilde{\partial}_i\tilde{k}_{ij}+\\+\frac{\rho_f}{\rho_f^0}\frac{p_0\,k_0}{L^2\mu_0}\,\tilde{\partial}_j\tilde{p}\,\tilde{k}_{ij}\frac{\partial\tilde{\mu}}{\partial\tilde{p}}\frac{1}{\tilde{\mu}^2}\,\tilde{\partial}_i\tilde{p}-\frac{\rho_f}{\rho_f^0}\frac{k_0}{\mu_0}\frac{p_0}{L^2}\,\frac{\tilde{k}_{ij}}{\tilde{\mu}}\left(\tilde{\partial}_i\tilde{\partial}_j\tilde{p}\right)=0
\end{multline}

Вынесли все масштабные множители. Далее делим обе части уравнения на множитель перед старшей производной $\left(\text{на }\frac{k_0\,p_0}{\mu_0\,L^2}\right)$, т.е. обезразмериваем уравнение:
\begin{multline}\label{PiezoEqDiv0}
\frac{\mu_0L^2}{k_0p_0}\cdot\frac{\partial p}{\partial t}\left(\varphi c_f+\frac{\rho_f}{\rho_f^0}\cdot c_\text{п}\right)-p_0c_f\frac{\tilde{k}_{ij}}{\tilde{\mu}}\,\tilde{\partial}_i\tilde{p}\,\tilde{\partial}_j\tilde{p}-\frac{\rho_f}{\rho_f^0}\frac{\tilde{k}_{ij}}{\tilde{\mu}}\,\tilde{\partial}_j\tilde{p}\,\tilde{\partial}_i\tilde{k}_{ij}+\\+\frac{\rho_f}{\rho_f^0}\frac{\partial\tilde{\mu}}{\partial\tilde{p}}\frac{1}{\tilde{\mu}^2}\tilde{k}_{ij}\,\tilde{\partial}_j\tilde{p}\,\tilde{\partial}_i\tilde{p}-\frac{\rho_f}{\rho_f^0}\frac{\tilde{k}_{ij}}{\tilde{\mu}}\left(\tilde{\partial}_i\tilde{\partial}_j\tilde{p}\right)=0
\end{multline}

--------------------------------------------------------------------

Сделаем 3 важных приближения:
\begin{enumerate}
	\item $p_0 c_f\ll 1$ (прикинем: сжимаемость воды порядка $10^{-5}\text{ атм}^{-1}=10^{-10}\text{ Па}^{-1}$; характерные значения давлений на глубинах, равных нескольким километрам, составляют сотни атмосфер; таким образом, произведение порядка $10^{-3}$, что много меньше единицы; но такое приближение не работает для газа: для него рассматриваемое произведение порядка единицы); это приближение фактически равносильно приближению $\rho_f\approx\rho_f^0$;
	\item $\tilde{\partial}_i\tilde{k}_{ij}\ll 1$ (считаем, что на характерном масштабе задачи по данному направлению проницаемость изменяется незначительно, не больше 10 процентов);
	\item $\dfrac{\partial\tilde{\mu}}{\partial\tilde{p}}\ll 1$ (считаем, что отмасштабированный график проницаемости от давления пологий -- этот факт подтверждается экспериментально -- вязкость слабо зависит от давления)
\end{enumerate}

Тогда уравнение \eqref{PiezoEqDiv0} перепишется в следующем виде (убрали слагаемые с пренебрежимо малыми множителями в рамках сделанных приближений и вернулись от безразмерных функций с волной к обычным функциям):
\beq
\frac{\partial p}{\partial t}\underbrace{\left(\varphi c_f+c_\text{п}\right)}_{c_t}-\frac{k_{ij}}{\mu}\partial_i\partial_j p=0
\eeq
(заметим, что если есть анизотропия проницаемости, то лапласиана в уравнении не будет).

Получаем классическое уравнение пьезопроводности:
\beq
\frac{\partial p}{\partial t}-\frac{k_{ij}}{\mu c_t}\partial_i\partial_j p=0,
\eeq
где $c_t$ -- это полная сжимаемость.

Замечание. Но есть литература, в которой $c_t=c_f+\frac{c_\text{п}}{\varphi}$, тогда уравнение пьезопроводности будет выглядеть так:
\beq
\frac{\partial p}{\partial t}-\frac{k_{ij}}{\mu\varphi c_t}\partial_i\partial_j p=0
\eeq

--------------------------------------------------------------------

Пусть тензор проницаемости изотропен $k_{ij}=k_0\cdot\delta_{ij}$, тогда:
\beq
\frac{\partial p}{\partial t}-\frac{k_0}{\mu c_t}\delta_{ij}\,\partial_i\partial_j p=0\Leftrightarrow\frac{\partial p}{\partial t}-\frac{k_0}{\mu c_t}\Delta p=0
\eeq
(получили всем известный вид уравнения пьезопроводности).

\newpage

\subsection{Вывод уравнения пьезопроводности для "<упругого пласта">.}

Для \textbf{упругого изотропного пласта} можем записать известные соотношения пороупругости:
\begin{itemize}[parsep=-5pt]
\item на тензор полных напряжений \\
\beq
\pmb{T}=\sigma^0\pmb{I}+\left(\lambda I_1(\pmb{\varepsilon})-b\Delta p\right)\pmb{I}+2\mu\pmb{\varepsilon},
\eeq
где $\pmb{T}$ -- тензор полных напряжений; $\pmb{I}$ -- единичный тензор; $\pmb{\varepsilon}$ -- тензор полных деформаций; $\lambda=K-2G/3$ и $\mu=G$ -- константы (параметры) Ляме; $K$ -- модуль всестороннего сжатия; $G$ -- модуль сдвига; $I_1(\pmb{\varepsilon})$ -- след тензора полных деформаций; $b$ -- константа Био; $\Delta p$ -- изменение давления; $\sigma^0$ -- начальное напряжение
\item на пористость \\
\beq
\varphi = \varphi_0+bI_1(\pmb{\varepsilon})+\dfrac{1}{N}\Delta p,
\eeq
где $\varphi_0$ -- начальная пористость; $b$ -- константа Био; $I_1(\pmb{\varepsilon})$ -- след тензора полных деформаций; $N$ -- модуль Био; $\Delta p$ -- изменение давления.
\item условие равновесия \\
\beq
\pmb{\nabla}\cdot\pmb{T}=\pmb{0}
\eeq
\end{itemize}

Для \textbf{флюида} запишем:
\begin{itemize}[parsep=-5pt]
	\item закон Дарси \\
	\beq
	\pmb{W}=-\frac{k}{\mu_f}\cdot\pmb{\nabla} p,
	\eeq
	где $\pmb{W}=\pmb{v_f}-\pmb{v_s}$; $k$ -- проницаемость пласта; $\mu_f$ -- вязкость флюида; $\pmb{\nabla} p$ -- градиент давления
	\item условие на сжимаемость флюида
	\beq
	p-p_0=K_f\frac{\rho_f-\rho_f^0}{\rho_f^0},
	\eeq
	где $K_f$ -- сжимаемость флюида
	\item уравнение неразрывности потока при отсутствии источникового слагаемого (уравнение переноса массы, записанное в дифференциальной форме):
	\beq
	\frac{\partial\left(\rho_f\varphi\right)}{\partial t}+\pmb{\nabla}\cdot\left(\rho_f\varphi\,\pmb{v_f}\right)=0
	\eeq
\end{itemize}

Из \textbf{уравнения неразрывности} получаем:
\beq
\varphi_0\frac{\partial\rho_f}{\partial t}+\rho_0\frac{\partial\varphi}{\partial t}+\rho_0\pmb{\nabla}\cdot\pmb{W}+\rho_0\varphi_0\frac{\partial I_1(\pmb{\varepsilon})}{\partial t}=0,
\eeq
где
\beq
\frac{\partial I_1(\pmb{\varepsilon})}{\partial t}\equiv\pmb{\nabla}\cdot\pmb{v_s}
\eeq

А дальше через ряд свёрток и всяких операций получаем:
\beq
b\dot{I}_1(\pmb{\varepsilon})+\left(\frac{1}{N}+\frac{\varphi_0}{K_f}\right)\dot{p}=\frac{k}{\mu_f}\pmb{\nabla}^2p
\eeq

В осесимметричном случае при условии отсутствия деформации на бесконечности получаем:
\beq
\dot{p}=a\pmb{\nabla}^2p,
\eeq
где
\beq
a=\frac{kM}{\mu_f}\text{ и } M=\frac{b\left(b+\varphi\right)}{\lambda+2\mu}+\frac{1}{N}+\frac{\varphi}{K_f}
\eeq

\newpage

\subsection{Опишите основные уравнения для моделирования фильтрации потока.}

\newpage

\subsection{Приведите примеры прямых и обратных задач уравнения пьезопроводности.}

\newpage

\subsection{Что такое капиллярное давление? Нарисовать кривую капиллярного давления от глубины. Нарисовать кривую капиллярного давления от насыщенности.}

\newpage

\subsection{Что такое ОФП? Раскрыть суть ОФП. Привести примеры наиболее популярных корреляцией для построения кривых относительных фазовых проницаемостей}

\includegraphics[width=\textwidth, page=75]{Kurs_OsnovyGDM_Kai_774_gorodovSV_v6_0.pdf}

Поверхностное натяжение (помимо капиллярного давления) приводит ещё к взаимному сопротивлению фильтрации нескольких флюидов.

Пример: у нас через образец только нефть течёт со скоростью 10 см$^3$/мин, а при пропускании через образец 50\% нефти и 50\% воды нефть течёт с гораздо меньшей скоростью 1.5 см$^3$/мин.

Это снижение проницаемости решили выражать некими функциями, которые зависят от насыщенности, и эти функции называются относительными фазовыми проницаемостями.
\\

Относительная фазовая проницаемость (ОФП) по флюиду 1 в присутствии флюида 2 -- это некий множитель (зависящий от насыщенности флюида 1) перед абсолютной проницаемостью, который позволяет найти эффективную проницаемость по флюиду 1 в присутствии флюида 2.
\\

В рассматриваемой на слайде ситуации (50\% воды и 50 \% нефти) из графиков зависимости ОФП от водонасыщенности видим, что эффективная проницаемость по воде будет составлять 5\% от абсолютной проницаемости, а эффективная проницаемость по нефти будет составлять 15\% от абсолютной проницаемости.
Видим, что общий поток тоже снизится по сравнению с пропусканием только одного флюида через образец.


\includegraphics[width=\textwidth, page=82]{Kurs_OsnovyGDM_Kai_774_gorodovSV_v6_0.pdf}

После проведения лабораторных исследований ОФП их обычно аппроксимируют некими функциями.
\\

Аппроксимация проводится с целью удобства: необходимо, чтобы ОФП были гладкими функциями.
Это позволяет легче находить решение при использовании численных схем.
\\

Наиболее популярные корреляции для построения кривых ОФП: корреляция Кори и корреляция LET.
Фактически подбираются настроечные параметры корреляций с целью наилучшего согласования с лабораторными исследованиями.
\\

Корреляция LET (появилась 15-20 лет; имеет 3 настроечных параметра) позволяет лучше описать лабораторные исследования, т.к. у неё есть участки с разной выпуклостью/вогнутостью.

\newpage

\subsection{Что такое функция Леверетта? Раскройте суть функции Леверетта.}

\includegraphics[width=\textwidth, page=73]{Kurs_OsnovyGDM_Kai_774_gorodovSV_v6_0.pdf}

Когда у нас есть несколько капиллярных кривых, то нужно как-то их перенести в модель.
Есть такая J-функция Леверетта, с помощью которой эти кривые можно нормализовать, осреднить и использовать дальше в расчёте.
Что делается? Капиллярные кривые взвешиваются на $\sigma \cos\theta$ и на корень из отношения проницаемости и пористости.
\\

$\sqrt{\dfrac{k}{\varphi}}$ характеризует извилистость поровых каналов.
Коллекторы примерно с одним и тем же строением (с одной и той же извилистостью) будут иметь похожее поведение, поэтому взвешивание капиллярных кривых на эту величину позволяет нам кривые осреднить и отслеживать только их характеристику, связанную с описанием толщины каналов.
\\

Рассчитываем значения J-функции Леверетта, и далее строим график в зависимости от водонасыщенности, подобный представленному справа: отмечаем подсчитанные точки и аппроксимируем их некой зависимостью (которую в дальнейшем будем использовать в расчётах ГДМ модели).
\\

На графике могут получиться не одно облако точек, а два или три (если есть несколько пластов с разными характеристиками или разные блоки на месторождении, в каждом из которых получился свой тип коллектора вследствие разных геологических процессов).
Тогда будет несколько аппроксимирующих кривых, которые можно использовать отдельно для каждого рассматриваемого пласта или блока соответственно.

\newpage

\subsection{Чем отличается связанная и критическая водонасыщенность. На что повлияет, если мы переместим связанную водонасыщенность S*L или максимальную водонасыщенность S*U. На что повлияет, если мы переместим критическую водонасыщенность S*CR.}

\newpage

\subsection{Что такое масштабирование ОФП? Зачем нужно нормировать ОФП?}

\newpage

\subsection{Какие лабораторные исследования проводят для измерения ОФП? Опишите суть исследований.}

\newpage

\subsection{Что такое фазовые диаграммы? Нарисуйте фазовые диаграммы (PVT диаграммы)}

\newpage

\subsection{Что такое PVT свойства, раскрыть суть PVT свойств.}

\newpage

\subsection{Какие PVT свойства жидкостей вы знаете? Какие основные PVT свойства необходимо задавать в модели? Что такое сжимаемость?}

\newpage

\subsection{Приведите примеры описания PVT моделей в симуляторе T-Navigator.}

\newpage

\subsection{Что такое давление насыщения? Что происходит при "<переходе"> через точку давления насыщения. Как будут меняться свойства жидкостей при "<переходе"> через  точку давления насыщения.}

Давление насыщения нефти -- давление, выше которого весь газ уже растворился, нефть остаётся недонасыщенной.
При снижении давления ниже давления насыщения из нефти начинает выделяться газ.

\includegraphics[width=\textwidth, page=102]{Kurs_OsnovyGDM_Kai_774_gorodovSV_v6_0.pdf}

На слайде показаны типичные кривые свойств нефти.
\\

Красная кривая (газосодержание нефти).

До давления насыщения при увеличении давления газосодержание растёт (газ растворяется и растворяется в нефти), при достижении давления насыщения весь газ растворился и дальше газосодержание остаётся постоянным.
\\

Зелёная кривая (объёмный коэффициент нефти).

От точки давления насыщения: если мы увеличиваем давление, то газа у нас нет, у нас только происходит сжатие нефти и объём нефти уменьшается с увеличением давления, поэтому объёмный коэффициент тоже снижается.

От точки давления насыщения: если мы снижаем давление ниже давления насыщения, то кроме уменьшения давления (т.е. увеличения объёма нефти за счёт расширения) из нефти начинает выделяться газ и соответственно объём нефти уменьшается, т.е. несмотря на то, что она расширяется из-за уменьшения давления, объёмный коэффициент начинает снижаться за счёт того, что газа много и он уходит из нефти.
\\

Синяя кривая (вязкость нефти).

Для вязкости картина наоборот: до давления насыщения при увеличении давления газ растворяется и растворяется в нефти, что приводит к снижению вязкости.
После достижения давления насыщения весь газ растворился в нефти и при дальнейшем увеличении давления происходит просто сжатие нефти, т.е. увеличение вязкости (т.к. вязкость -- это мера внутреннего сопротивления одних слоёв жидкости относительно других слоёв при их движении, а это сопротивление очевидно растёт при увеличении давления).

\newpage

\subsection{Раскрыть суть решения уравнения фильтрации явным методом. Опишите основные шаги}

\newpage

\subsection{Раскрыть суть решения уравнения фильтрации неявным методом. Опишите основные шаги}

\newpage

\subsection{Опишите способы решения системы линейных ур-ий. Приведите минимум 2 способа в качестве примеров.}

\newpage

\subsection{Что такое критерий стабильности? Раскройте суть критерия стабильности. Приведите минимум 2 примера критерия стабильности.}

\newpage

\subsection{Что такое начальные и граничные условия? Какое граничное условие называют Неймана, а какое Дирихле? Опишите применение граничных условий в решении задач в нефтяной индустрии.}

\newpage

\subsection{Что такое Box model? Раскройте суть Box model.}

\newpage

\subsection{Какие сетки используем в гидродинамическом моделировании? Приведите примеры различных сеток и опишите их специфику.}

\newpage

\subsection{Какими способами можно задать кубы свойств?}

\newpage

\subsection{Какие способы инициализации модели Вы знаете? Раскройте суть хотя бы одного из способов. Какие ключевые слова применяются для инициализации ГДМ в симуляторе T-Navigator?}

\newpage

\subsection{Что будет происходить с моделью при инициализации модели различными способами (неравновесный, равновесный, равновесный + начальный куб насыщенности)?}

\newpage

\subsection{Что такое аквифер? Раскройте суть аквифера. Какое ключевое слово обозначает аквифер в T-Navigator?}

\newpage

\subsection{Опишите структуру Data файла для запуска модели в симуляторе T-Navigator. Опишите основные разделы Data файла.}

\newpage

\subsection{Как задавать параметры скважин в T-Navigator? Как можно задать контроль на скважинах в T-Navigator?}

\newpage

\subsection{Что такое ремасштабирование (UpScaling)? Раскройте суть ремасштабирования}

\textbf{Ремасштабирование геомодели}

\includegraphics[width=\textwidth, page=59]{Kurs_OsnovyGDM_Kai_774_gorodovSV_v6_0.pdf}

Теперь переходим непосредственно к созданию модели.
После того, как геолог создал свою статичную геологическую модель и передал его гидродинамику, бывают случаи, когда эту модель необходимо сделать более грубой, когда эта модель слишком детальная и эта детальность излишняя (только забирает ресурсы и никакой информации особо не несёт для гидродинамика).
\\

Возникает необходимость сделать процедуру ремасштабирования, т.е. укрупнение модели (точнее укрупнение ячеек модели).
Эта процедура состоит из двух этапов: первое это upgridding (ремасштабирование сетки; изменение размеров и количества ячеек) и второе это upscaling (ремасштабирование свойств; т.е. после того, как мы получили большие ячейки, нам в эти большие ячейки нужно записать свойства, а именно осреднить значения свойств из маленьких ячеек и перенести эти осреднённые значения в большую ячейку).
\\

Возникает такой вопрос: до какой степени нам модель можно укрупнять и когда следует остановиться, т.е. когда мы начнём терять в качестве?

Можно построить такой график (представлен на слайде): по оси $x$ откладываем количество слоёв по вертикали (здесь мы говорим про укрупнение по вертикали), а по оси $y$ откладываем погрешность в расчёте накопленной добычи нефти (FOPT), накопленной добычи воды (FLPT).
Т.е. мы сравниваем, насколько результаты вот этих накопленных показателей отличаются от модели с исходной геологической сеткой.
При уменьшении количества слоёв ошибка постепенно растёт, и в какой-то момент на графике возникает перегиб (ошибка начинает возрастать более резко), т.е. это является неким косвенным признаком того, что мы начинаем в этот момент терять какую-то информацию о геологическом строении, о неоднородности.
И соответственно можем сказать, что в этой точке перегиба у нас оптимум, дальше которого модель укрупнять не следует (стоит остановиться).
Итак, первым способом выбора степени укрупнения является нахождение оптимума (точки перегиба на графике).

Второй способ -- это просто ограничиться каким-то значением ошибки. Почему-то обычно привязываются к каким-то круглым значениям (5, 10 или 20 \%).

Но лучше всё-таки использовать способ, который показывает, в какой момент мы начинаем терять информацию о строении месторождения.
\\

\textbf{Ремасштабирование структуры (upgridding)}

\includegraphics[width=\textwidth, page=60]{Kurs_OsnovyGDM_Kai_774_gorodovSV_v6_0.pdf}

По горизонтали рекомендация следующая: рекомендуется, чтобы между скважинами было не менее 3-5 ячеек, чтобы описать фильтрацию между скважинами.
Но с другой стороны, если у нас количество данных по месторождению ограничено, то стремиться к излишней детализации тоже не стоит, потому что точности не добавится (ведь новых данных нет), а время расчёта увеличится.
Но как минимум 3-5 ячеек всё таки желательно оставлять.
Справа на слайде приведён пример модели, которую передавали мне на экспертизу: и даже сложно различить, где какая скважина находится (здесь представлены горизонтальные скважины; крестиками помечены перфорации), настолько близко они расположены (в соседних ячейках), что, честно, даже непонятно, где какой ствол идёт; что таким образом пытались смоделировать тоже непонятно, естественно экспертизу такая модель не прошла и было рекомендовано сделать более детальную модель, чтобы между скважинами корректно воспроизводить процесс фильтрации.

\includegraphics[width=\textwidth, page=61]{Kurs_OsnovyGDM_Kai_774_gorodovSV_v6_0.pdf}

По вертикали желательно следить за тем, чтобы сохранялась расчленённость, нарезка слоёв и глинистые перемычки не пропадали.

Здесь на слайде тоже показан пример одной из моделей, которая проходила на экспертизу (левый рисунок на слайде).
Как делать не надо.
Видим, что в геологической модели и верхний, и нижний пласты достаточно расчленённые (на ГИС есть много белых глинистых перемычек).
А после укрупнения видим, что верхний пласт вообще склеился в один однородный массив, и в нижнем пласте тоже расчленённость пропала.

На правом рисунке на слайде есть пример, в котором глинистые перемычки пропали не внутри одного пласта, а даже между пластами, т.е. пласты, которые вообще не сообщаются гидродинамически, вдруг стали гидродинамически связанными.
Такое ужасное нарушение; модель стала совсем непригодной для расчётов, поскольку появилась вертикальная связь.
\\

\textbf{Ремасштабирование свойств}

\includegraphics[width=\textwidth, page=62]{Kurs_OsnovyGDM_Kai_774_gorodovSV_v6_0.pdf}

После того, как мы укрупнили ячейки, нужно в эти ячейки перенести свойства.

Вот у нас пример здесь: были такие маленькие ячейки, теперь эти маленькие ячейки стали одной большой ячейкой. У маленьких ячеек были разные значения какого-то свойства. Какое значение теперь занести в большую ячейку?

Есть рекомендованные методы расчёта средних свойств и очерёдности расчёта: сначала мы рассчитываем среднее значение песчанистости (рассчитывается как среднее арифметическое, но взвешенное по объёму ячеек; формула представлена на слайде -- ячейки имеющие больший объём вносят больший вклад), следующей по очереди осредняется пористость (здесь среднее арифметическое, взвешенное на эффективный объём; эффективный объём -- это песчанистость, умноженная на геометрический объём), далее осредняется насыщенность (среднее арифметическое, взвешенное на поровый объём; поровый объём -- это пористость, умноженная на эффективный объём).
Это всё делается для того, чтобы воспроизвести запасы для модели с укрупнённой сеткой.

Если же делать всё-таки гидродинамически уравновешенную модель, то насыщенность можно не осреднять, а рассчитать по гидростатическому равновесию.
Про это (про расчёт насыщенностей) дальше мы тоже будем говорить, когда будем обсуждать инициализацию модели.
\\

\textbf{Ремасштабирование проницаемости}

\includegraphics[width=\textwidth, page=63]{Kurs_OsnovyGDM_Kai_774_gorodovSV_v6_0.pdf}

Для проницаемости методы осреднения более разнообразны, поскольку проницаемость не является объёмной характеристикой, а является характеристикой, зависящей от направления фильтрации.

И получается, что если поток идёт параллельно напластованию, то мы можем использовать среднее арифметическое для осреднения.

Если поток идёт перпендикулярно напластованию, то рекомендуется использовать среднее гармоническое.

Если же пласт сильно неоднородный (сложно выделить направление напластования), то можно использовать среднее геометрическое или хитроумную комбинацию арифметических и гармонических.

Но самый лучший способ -- это осреднение на основе решения уравнений однофазной или многофазной фильтрации.
Т.е. что делается?
Фактически производится расчёт потоков (по формуле Дарси, грубо говоря) на мелких ячейках и затем на крупных ячейках обратным пересчётом рассчитывается проницаемость так, чтобы потоки через грани крупных ячеек были такими же, как и сумма потоков через грани мелких ячеек, которые составляют эту крупную ячейку. Т.е. основная задача -- это сохранить потоки, и таким образом подбирается проницаемость, чтобы эти потоки сохранились.

\includegraphics[width=\textwidth, page=64]{Kurs_OsnovyGDM_Kai_774_gorodovSV_v6_0.pdf}

Здесь говорится о том, какие условия задавать на границах при таком способе осреднения.

Если поток идёт по горизонтали, то мы говорим, что вертикального перетока нет.

Если идёт косая слоистость, то вычисляется полный тензор проницаемости.

Если идёт и горизонтальный поток, и вертикальный, то можно задать изменение давления на верхних границах, чтобы был переток.
\\

\textbf{Ремасштабирование геомодели. Контроль качества}

\includegraphics[width=\textwidth, page=65]{Kurs_OsnovyGDM_Kai_774_gorodovSV_v6_0.pdf}

Как контролировать ремасштабирование?
Есть несколько способов.

Первое -- это геолого-статистический разрез (например, по песчанистости).
Здесь чисто визуально оценивается, сохранились ли глинистые перемычки, оцениваются доли коллектора с высоким и низким содержанием глины, т.е. такое визуальное сравнение графиков.

На слайде (на крайнем левом рисунке) красным показано среднее значение песчанистости в слоях по укрупнённой модели, а синей -- в исходной геологической модели.
Геолого-статистический разрез получается следующим образом: в каждом слое считается среднее арифметическое значение и наносится на график (по оси ординат -- слои, по оси абсцисс -- значения песчанистости).

Получается такая вот "<кардиограмма"> (крайний левый рисунок), на которой мы сопоставляем визуально, насколько хорошо сохранились глинистые перемычки.

Также можно сопоставить начальные запасы углеводородов (сохранились или не сохранились после ремасштабирования), эффективные толщины и ещё можно посмотреть гистограммы.

Гистограммы, конечно, один в один не совпадут, потому что количество крайних ячеек (точнее ячеек, которые имеют крайние значения, т.е. либо максимальные, либо минимальные) сократится, поскольку такие ячейки в ходе осреднения будут объединяться с ячейками с другими значениями.
Соответственно количество крайних значений уменьшится, и гистограмма как бы прижмётся к своему среднему значению, но при этом вид самой гистограммы должен быть одинаковый как до, так и после укрупнения ячеек.
Т.е. если мы видим какое-то смещение среднего значения или другой вид гистограммы, то это может говорить о том, что мы потеряли какую-то информацию о строении пласта в ходе этого укрупнения и нужно вернуться и проверить, всё ли правильно мы сделали, правильные ли методы осреднения использовали и не слишком ли грубо мы всё это сделали.

\newpage

\subsection{Что такое радиус Писмана? Раскройте суть радиуса Писмана.}

\newpage

\subsection{Что такое адаптация модели? Опишите основные шаги адаптации модели.}

\newpage

\subsection{Какие основные параметры модели изменяются при адаптации? Какие критерии показывают, что модель успешно адаптирована?}

\newpage

\subsection{Как можно садаптировать модель в случае отставания фронта заводнения в модели?}

\newpage

\subsection{Что такое уравнение Баклея-Леверетта? Подробно опишите суть уравнения Баклея-Леверетта.}

\newpage

\subsection{Что такое материальный баланс в гидродинамическом моделировании? Опишите суть материального баланса.}

\newpage

\end{document}