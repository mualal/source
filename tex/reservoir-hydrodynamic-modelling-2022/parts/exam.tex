\documentclass[main.tex]{subfiles}

\begin{document}

% \textcolor{red}{Вводная лекция}

\section{Экзамен 12.01.2023 (Базыров И.Ш.)}

\subsection{Какие режимы течения Вы знаете? Раскрыть суть каждого режима.}

\newpage

\subsection{Какие режимы притока Вы знаете? Раскрыть суть каждого режима.}

\newpage

\subsection{Что такое уравнение Дарси, как выводится уравнение в общем виде?}

\newpage

\subsection{Вывод уравнения Дарси для линейного течения}

\newpage

\subsection{Вывод уравнения Дарси для радиального течения}

\newpage

\subsection{Раскрыть суть уравнения Дарси-Дюпюи, как выводится уравнение? Как получен коэффициент в формуле Дарси-Дюпюи, равный 18.41?}

\newpage

\subsection{Раскрыть суть уравнения пьезопроводности, как выводится уравнение пьезопроводности для "<неупругого пласта">? Что такое коэффициент пьезопроводности?}

\newpage

\subsection{Вывод уравнения пьезопроводности для "<упругого пласта">.}

\newpage

\subsection{Опишите основные уравнения для моделирования фильтрации потока.}

\newpage

\subsection{Приведите примеры прямых и обратных задач уравнения пьезопроводности.}

\newpage

\subsection{Что такое капиллярное давление? Нарисовать кривую капиллярного давления от глубины. Нарисовать кривую капиллярного давления от насыщенности.}

\newpage

\subsection{Что такое ОФП? Раскрыть суть ОФП. Привести примеры наиболее популярных корреляцией для построения кривых относительных фазовых проницаемостей}

\newpage

\subsection{Что такое функция Леверетта. Раскройте суть функции Леверетта.}

\newpage

\subsection{Чем отличается связанная и критическая водонасыщенность. На что повлияет, если мы переместим связанную водонасыщенность S*L или максимальную водонасыщенность S*U. На что повлияет, если мы переместим критическую водонасыщенность S*CR.}

\newpage

\subsection{Что такое масштабирование ОФП? Зачем нужно нормировать ОФП?}

\newpage

\subsection{Какие лабораторные исследования проводят для измерения ОФП? Опишите суть исследований.}

\newpage

\subsection{Что такое фазовые диаграммы? Нарисуйте фазовые диаграммы (PVT диаграммы)}

\newpage

\subsection{Что такое PVT свойства, раскрыть суть PVT свойств.}

\newpage

\subsection{Какие PVT свойства жидкостей вы знаете? Какие основные PVT свойства необходимо задавать в модели? Что такое сжимаемость?}

\newpage

\subsection{Приведите примеры описания PVT моделей в симуляторе T-Navigator.}

\newpage

\subsection{Что такое давление насыщения? Что происходит при "<переходе"> через  точку давления насыщения. Как будут меняться свойства жидкостей при "<переходе"> через  точку давления насыщения.}

\newpage

\subsection{Раскрыть суть решения уравнение фильтрации явным методом. Опишите основные шаги}

\newpage

\subsection{Раскрыть суть решения уравнение фильтрации неявным методом. Опишите основные шаги}

\newpage

\subsection{Опишите способы решения системы линейных ур-ий. Приведите минимум 2 способа в качестве примеров.}

\newpage

\subsection{Что такое критерий стабильности? Раскройте суть критерия стабильности. Приведите минимум 2 примера критерия стабильности.}

\newpage

\subsection{Что такое начальные и граничные условия? Какое граничное условие называют Неймана, а какое Дирихле? Опишите применение граничных условий в решении задач в нефтяной индустрии.}

\newpage

\subsection{Что такое Box model? Раскройте суть Box model.}

\newpage

\subsection{Какие сетки используем в гидродинамическом моделировании? Приведите примеры различных сеток и опишите их специфику.}

\newpage

\subsection{Какими способами можно задать кубы свойств?}

\newpage

\subsection{Какие способы инициализации модели Вы знаете. Раскройте суть хотя бы одного из способов. Какие ключевые слова применяются для инициализации ГДМ симуляторе T-Navigator?}

\newpage

\subsection{Что будет происходить с моделей при инициализации модели различными способами (неравновесный, равновесный, равновесный + начальный куб насыщенности)?}

\newpage

\subsection{Что такое аквифер? Раскройте суть аквифера. Какое ключевое слово обозначает аквифер в T-Navigator?}

\newpage

\subsection{Опишите структуру Data файла для запуска модели в симуляторе в T-Navigator. Опишите основные разделы Data файла.}

\newpage

\subsection{Как задавать параметры скважин в T-Navigator? Как можно задать контроль на скважинах в T-Navigator?}

\newpage

\subsection{Что такое ремасштабирование (UpScaling)? Раскройте суть ремасштабирования}

\newpage

\subsection{Что такое радиус Писмана? Раскройте суть радиуса Писмана.}

\newpage

\subsection{Что такое адаптация модели? Опишите основные шаги адаптации модели.}

\newpage

\subsection{Какие основные параметры модели изменяются при адаптации? Какие критерии показывают, что модель успешно адаптирована?}

\newpage

\subsection{Как можно садаптировать модель в случае отставания фронта заводнения в модели?}

\newpage

\subsection{Что такое уравнение Баклея-Леверетта? Подробно опишите суть уравнения Баклея-Леверетта.}

\newpage

\subsection{Что такое материальный баланс в гидродинамическом моделировании? Опишите суть материального баланса.}

\newpage

\end{document}