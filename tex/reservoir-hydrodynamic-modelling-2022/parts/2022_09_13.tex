\documentclass[main.tex]{subfiles}

\begin{document}

\section{Лекция 13.09.2022 (Кайгородов С.В.)}

\subsection{Анализ разработки перед построением модели}

\includegraphics[width=\textwidth, page=114]{Kurs_OsnovyGDM_Kai_774_gorodovSV_v6_0.pdf}

Перед построением модели необходимо провести анализ разработки, чтобы понимать, какие скважины друг на друга влияют, откуда они обводняются, есть ли загрязнения призабойной зоны, есть ли трещины авто-ГРП на нагнетательных скважинах.

Другими словами, необходимо проанализировать, как работает месторождение, как работают скважины, чтобы это учесть при построении модели.

\subsection{Матбаланс}

\includegraphics[width=\textwidth, page=115]{Kurs_OsnovyGDM_Kai_774_gorodovSV_v6_0.pdf}


\includegraphics[width=\textwidth, page=116]{Kurs_OsnovyGDM_Kai_774_gorodovSV_v6_0.pdf}

\subsection{Матбаланс. Пример использования}

\includegraphics[width=\textwidth, page=117]{Kurs_OsnovyGDM_Kai_774_gorodovSV_v6_0.pdf}

Делали проект разработки группы месторождений.
Площадь большая, и поэтому возник вопрос: связаны ли все эти залежи нефти (отмечены зелёным на карте) между собой?
Красным на карте отмечены разломы.

Если строить одну большую модель, расчёт будет идти долго. Поэтому необходимо проверить, можно ли разрезать рассматриваемый участок на несколько отдельных участков, чтобы построить несколько отдельных моделей.

Построили несколько простых моделей материального баланса: каждая залежь представлялась отдельной бочкой и между этими бочками рисовалась связь (отмечена голубыми ромбиками). Далее проводился расчёт и производилась настройка параметров связности
между различными залежами.
На графиках синим обозначены рассчитанные значения динамики пластового давления, а красные квадратики -- фактические замеры.

Оказалось, что наилучшую настройку показали модели, в которых часть из этих залежей не связаны.
Это позволило разделить рассматриваемую группу месторождений на отдельные части, моделировать их отдельно и соответственно ускорить расчёты; другими словами, за ограниченное время проекта сделать больше расчётов.

\subsection{Анализ источников обводнения}

\includegraphics[width=\textwidth, page=118]{Kurs_OsnovyGDM_Kai_774_gorodovSV_v6_0.pdf}

По виду характеристических графиков Чена можно определить, откуда в скважину попала вода.
Строится график водонефтяного фактора (ВНФ, WOR = отношение добытой воды к добытой нефти) от времени в логарифмических координатах.

Замечание. За рубежом более популярен ВНФ, у нас обычно используют обводнённость.

\includegraphics[width=\textwidth, page=119]{Kurs_OsnovyGDM_Kai_774_gorodovSV_v6_0.pdf}

\includegraphics[width=\textwidth, page=120]{Kurs_OsnovyGDM_Kai_774_gorodovSV_v6_0.pdf}

\includegraphics[width=\textwidth, page=121]{Kurs_OsnovyGDM_Kai_774_gorodovSV_v6_0.pdf}

К сожалению, не всегда удаётся увидеть эти закономерности на фактических данных (т.к. фактические данные часто бывают зашумлены, есть погрешности измерений и т.д.).
Но иногда это срабатывает, и графики Чена дают действительно полезную информацию об источнике обводнения.

\subsection{Оценка загрязнения призабойной зоны}

\includegraphics[width=\textwidth, page=122]{Kurs_OsnovyGDM_Kai_774_gorodovSV_v6_0.pdf}

\subsection{Оценка наличия трещин авто-ГРП на нагнетательных скважинах}

\includegraphics[width=\textwidth, page=123]{Kurs_OsnovyGDM_Kai_774_gorodovSV_v6_0.pdf}

Если график отклоняется вверх от диагональной прямой (т.е. для закачки такого же объёма воды требуется большее давление), то это говорит нам об ухудшении свойств призабойной зоны.

Если график отклоняется вниз от диагональной прямой (т.е. для закачки такого же объёма воды требуется меньшее давление), то это говорит нам о трещине  авто-ГРП (улучшение проницаемости в призабойной зоне).

Дребезжание на графике Холла отражает поведение при циклической закачке.

Отклонения на графике Холла могут быть вызваны не только изменением свойств призабойной зоны, но и изменением диаметра штуцера (при смене штуцеров). Этот факт важно учитывать при анализе графиков Холла, чтобы не сделать ошибочные выводы. 

\subsection{Исходные данные по скважинам}

\includegraphics[width=\textwidth, page=124]{Kurs_OsnovyGDM_Kai_774_gorodovSV_v6_0.pdf}

Принадлежность к группе создаётся, если хотим отслеживать, как работают отдельные кусты скважин или скважины, относящиеся к одной какой-либо группе - например, к одной ДНС (дожимной насосной станции).

\subsection{Моделирование притока к скважине}

\includegraphics[width=\textwidth, page=125]{Kurs_OsnovyGDM_Kai_774_gorodovSV_v6_0.pdf}

\includegraphics[width=\textwidth, page=126]{Kurs_OsnovyGDM_Kai_774_gorodovSV_v6_0.pdf}

\subsection{Способы инициализации модели в симуляторах}

\includegraphics[width=\textwidth, page=127]{Kurs_OsnovyGDM_Kai_774_gorodovSV_v6_0.pdf}

Фактически при инициализации модели задаём начальные условия.

\subsubsection{Неравновесный}

\includegraphics[width=\textwidth, page=128]{Kurs_OsnovyGDM_Kai_774_gorodovSV_v6_0.pdf}

Неравновесный: задаются значения давления и насыщенности на начальный момент времени во всех ячейках.
На начальный момент времени залежь не находится в равновесии. После инициализации могут начаться перетоки даже в том случае, если в модели нет никаких скважин.

Такой способ инициализации на практике практически не встречается (ведь всегда предполагается, что залежь формировалась долгое время, за которое все флюиды пришли в гидростатическое равновесие).

\subsubsection{Равновесный}

\includegraphics[width=\textwidth, page=129]{Kurs_OsnovyGDM_Kai_774_gorodovSV_v6_0.pdf}

На практике обычно используют равновесный способ инициализации.

\includegraphics[width=\textwidth, page=130]{Kurs_OsnovyGDM_Kai_774_gorodovSV_v6_0.pdf}

\subsubsection{Равновесный с соблюдением начальной насыщенности}

\includegraphics[width=\textwidth, page=131]{Kurs_OsnovyGDM_Kai_774_gorodovSV_v6_0.pdf}

\subsection{Оценка корректности инициализации ГДМ}

\includegraphics[width=\textwidth, page=132]{Kurs_OsnovyGDM_Kai_774_gorodovSV_v6_0.pdf}

\begin{enumerate}
	\item Можно сравнить насыщенность на начальный шаг с заданной насыщенностью
	\item Можно оценить диапазоны изменения куба капиллярного давления после инициализации и сравнить их с теми, которые получали при исследовании на керне
	\item Можно оценить, совпадают ли запасы в ГДМ с геомоделью
	\item Можно провести расчёт модели без скважин и убедиться в отсутствии изменений насыщенности и давления (для равновесных инициализаций)
\end{enumerate}

\subsection{Аналитический аквифер}

\includegraphics[width=\textwidth, page=133]{Kurs_OsnovyGDM_Kai_774_gorodovSV_v6_0.pdf}

Переходим к граничным условиям.

На границах, если есть водоносный горизонт, то его можно задать в модели.

Есть точное решение Hurst van Everdingen, которое описывает приток из аквифера конечных размеров.

Но в модели такое точное решение не задать.

\includegraphics[width=\textwidth, page=134]{Kurs_OsnovyGDM_Kai_774_gorodovSV_v6_0.pdf}

Поэтому Carter-Tracy разработали модель притока из аквифера, результаты расчётов по которой близки к аналитическому решению Hurst van Everdingen.

Модель Carter-Tracy задаётся в ГДМ симуляторах с помощью ключевого слова AQUCT.

Модель Carter-Tracy рекомендуется использовать либо для больших залежей, либо для низкопроницаемых залежей (другими словами, для залежей, на которых режим течения устанавливается не быстро).

\includegraphics[width=\textwidth, page=135]{Kurs_OsnovyGDM_Kai_774_gorodovSV_v6_0.pdf}

Для высокопроницаемых или маленьких залежей есть более простая модель Fetkovich, которая моделирует приток из аквифера просто в виде произведения продуктивности аквифера и разницы давлений.

Для больших или низкопроницаемых залежей модель Fetkovich использовать не рекомендуется. В этом случае лучше использовать модель Carter-Tracy.

\subsection{Упражнение 2. Создание синтетической BOX-модели}

\includegraphics[width=\textwidth, page=136]{Kurs_OsnovyGDM_Kai_774_gorodovSV_v6_0.pdf}

\subsection{Упражнение 3. Инициализация ГДМ}

\includegraphics[width=\textwidth, page=137]{Kurs_OsnovyGDM_Kai_774_gorodovSV_v6_0.pdf}

\subsection{Задание истории работы скважин}

\includegraphics[width=\textwidth, page=138]{Kurs_OsnovyGDM_Kai_774_gorodovSV_v6_0.pdf}

Здесь перечислены ключевые слова для задания истории работы скважин.

\subsection{Упражнение 4. Подготовка SCHEDULE-секции}

\includegraphics[width=\textwidth, page=139]{Kurs_OsnovyGDM_Kai_774_gorodovSV_v6_0.pdf}

\subsection{Алгоритм работы в ПО SCHEDULE}

\includegraphics[width=\textwidth, page=140]{Kurs_OsnovyGDM_Kai_774_gorodovSV_v6_0.pdf}

На слайдах показан процесс работы в ПО SCHEDULE.
Раньше поставлялось в пакете Schlumberger вместе с Eclipse, сейчас практически не используется.

Информация на слайдах будет полезна, если вдруг появится необходимость работы с ПО SCHEDULE.

\includegraphics[width=\textwidth, page=141]{Kurs_OsnovyGDM_Kai_774_gorodovSV_v6_0.pdf}

\includegraphics[width=\textwidth, page=142]{Kurs_OsnovyGDM_Kai_774_gorodovSV_v6_0.pdf}

\includegraphics[width=\textwidth, page=143]{Kurs_OsnovyGDM_Kai_774_gorodovSV_v6_0.pdf}

\includegraphics[width=\textwidth, page=144]{Kurs_OsnovyGDM_Kai_774_gorodovSV_v6_0.pdf}

\subsection{Загрузка истории эксплуатации}

\includegraphics[width=\textwidth, page=145]{Kurs_OsnovyGDM_Kai_774_gorodovSV_v6_0.pdf}

Можно сформировать SCHEDULE секцию с помощью tNavigator.
Для этого необходимо воспользоваться опцией: Загрузить данные по скважинам...
И далее действовать по алгоритму на слайдах.

\includegraphics[width=\textwidth, page=146]{Kurs_OsnovyGDM_Kai_774_gorodovSV_v6_0.pdf}

\includegraphics[width=\textwidth, page=147]{Kurs_OsnovyGDM_Kai_774_gorodovSV_v6_0.pdf}

\includegraphics[width=\textwidth, page=148]{Kurs_OsnovyGDM_Kai_774_gorodovSV_v6_0.pdf}

\subsection{Адаптация модели}

\includegraphics[width=\textwidth, page=149]{Kurs_OsnovyGDM_Kai_774_gorodovSV_v6_0.pdf}

После указания всех данных в модели можем поставить её на расчёт и обнаружить, что результат расчёта не совпал с фактическими замерами, которые мы в модель занесли.
Почему это происходит?

Во-первых, данных недостаточно.

Во-вторых, имеющиеся данные обладают неопределённостью: у нас данные точечные (только по скважинам), а в межскважинном пространстве геолог стохастическими методами распределил свойства -- даже в скважинах измеренные данные обладают погрешностью, а в межскважинном пространстве эта погрешность тем более есть.

Перед тем как использовать построенную модель её нужно настроить на факт.
Другими словами, необходимо провести адаптацию модели.

Но существует бесконечное множество сочетаний параметров модели, при которых результат расчёта этой модели будет с заданной точностью совпадать с фактом, замеренным по скважинам.

Задача оптимизационных алгоритмов: варьируя параметры модели, устремить целевую функцию к нулю.

\subsubsection{Обратные задачи}

\includegraphics[width=\textwidth, page=150]{Kurs_OsnovyGDM_Kai_774_gorodovSV_v6_0.pdf}

Здесь ясно, как аппроксимировать имеющиеся замеры.

\includegraphics[width=\textwidth, page=151]{Kurs_OsnovyGDM_Kai_774_gorodovSV_v6_0.pdf}

Но если есть шум и несколько размерностей, то задача подбора нужной поверхности становится нетривиальной и может иметь бесконечное множество разумных решений.
В этом случае очень сложно вручную подобрать параметры; необходимо осуществлять автоадаптацию и проверять найденные решения на разумность и физичность с точки зрения рассматриваемой ГД-модели.

\subsubsection{Адаптация модели на разных стадиях разработки}

\includegraphics[width=\textwidth, page=152]{Kurs_OsnovyGDM_Kai_774_gorodovSV_v6_0.pdf}

На разных периодах разработки месторождения настраиваем разные параметры модели.

Период до начала добычи = blue field.

Период безводной добычи = green field.

Период обводнённой добычи = brown field.

\includegraphics[width=\textwidth, page=153]{Kurs_OsnovyGDM_Kai_774_gorodovSV_v6_0.pdf}

Обычно адаптация идёт от крупного к мелкому (от месторождения к скважинам).

Сначала настраиваем энергетическое состояние залежи: матбаланс по скважинам и пластовое давление.

После настройки энергетики, переходим к настройке по соотношению нефть/вода или нефть/газ.
Т.е. к настройке по отборам конкретных флюидов.

И финально производится настройка по коэффициентам продуктивности и забойным давлениям.

\subsubsection{По отборам жидкости и пластовому давлению}

\includegraphics[width=\textwidth, page=154]{Kurs_OsnovyGDM_Kai_774_gorodovSV_v6_0.pdf}

Если говорить о месторождениях в Западной Сибири, то там пласты имеют сжимаемость порядка $10^{-5} \text{атм}^{-1}$, что приводит к тому, что сжимаемость фактически не оказывает ощутимого влияния на динамику пластового давления.

\subsubsection{По соотношению нефть/вода}

\includegraphics[width=\textwidth, page=155]{Kurs_OsnovyGDM_Kai_774_gorodovSV_v6_0.pdf}

При варьировании остаточных насыщенностей гораздо легче испортить модель, чем при варьировании, например, абсолютной проницаемости.

\subsubsection{По коэффициенту продуктивности и Pзаб}

\includegraphics[width=\textwidth, page=156]{Kurs_OsnovyGDM_Kai_774_gorodovSV_v6_0.pdf}

\subsection{Уточнение распределений параметров при адаптации модели}

\includegraphics[width=\textwidth, page=157]{Kurs_OsnovyGDM_Kai_774_gorodovSV_v6_0.pdf}

Можно воспринимать адаптацию модели, как уточнение исходных распределений параметров, т.е. на начальный момент времени у нас есть параметры, обладающие неопределённостями в каком-то диапазоне, но сами виды распределений (какие значения параметра наиболее вероятны или менее вероятны) мы не знаем.

Из-за отсутствия информации о виде распределения обычно задают равномерное распределение возможных значений параметра в заданном диапазоне.
Далее проводится расчёт модели со значениями параметров в рассматриваемых диапазонах, и мы видим, что какие-то из результатов расчётов не будут соответствовать фактическим замерам (даже с учётом допустимой погрешности).
Это позволит нам сузить диапазоны вариации исходных данных и уточнить виды распределения.
Например, от равномерных распределений можем прийти к нормальным или треугольным распределениям.

Другими словами, адаптацию можно рассматривать в качестве проверки, в каких диапазонах исходные данные (значения параметров) могут находиться и какие значения этих параметров наиболее вероятны.

\subsubsection{Алгоритм проведения автоадаптации}

\includegraphics[width=\textwidth, page=158]{Kurs_OsnovyGDM_Kai_774_gorodovSV_v6_0.pdf}

\subsubsection{Программы автоадаптации}

\includegraphics[width=\textwidth, page=159]{Kurs_OsnovyGDM_Kai_774_gorodovSV_v6_0.pdf}

Сейчас программы автоадаптации используются в качестве вспомогательного инструмента, чтобы быстрее найти решение / сузить диапазоны поиска значений параметров.

После проведения автоадаптации всё равно необходимо проводить дополнительный анализ на физичность / геологичность найденных сочетаний параметров.
Другими словами, на данный момент программы автоадаптации решают чисто оптимизационную задачу и не способны самостоятельно учесть всевозможные нефизичности найденных сочетаний параметров.

Но есть проекты когнитивной автоадаптации, в которых пытаются контролировать физическую / геологическую обоснованность всех параметров и их сочетаний в автоматическом режиме.

Адаптация модели является самым времязатратным периодом работы с моделью (может занимать несколько месяцев работы до окончательной настройки модели).

\subsubsection{Критерии адаптации}

\includegraphics[width=\textwidth, page=160]{Kurs_OsnovyGDM_Kai_774_gorodovSV_v6_0.pdf}

Критерии адаптации, если смотрим в целом по месторождению (сумму по всем скважинам).

По дебитам воды, нефти, жидкости, газа ошибка не должна превышать 10\%.

По накопленной добыче воды, нефти, жидкости, газа ошибка не должна превышать 5\%.

По пластовым давлениям по регламенту ошибка не должна превышать 25\%, но обычно стараются добиться меньшего диапазона вариации.

\includegraphics[width=\textwidth, page=161]{Kurs_OsnovyGDM_Kai_774_gorodovSV_v6_0.pdf}

Критерии адаптации, если смотрим отдельно по скважинам.

Строятся кроссплоты расчёт-факт (отмечаются все скважины) по накопленной добыче нефти на определённую дату.

Допустимые ошибки: 20\% по нефти; 20\% по воде; 25\% по давлению; 5\% по жидкости; 5\% по закачке.

\includegraphics[width=\textwidth, page=162]{Kurs_OsnovyGDM_Kai_774_gorodovSV_v6_0.pdf}

Чтение представленного на слайде графика: видим, что доля фонда скважин с относительной погрешностью расчёт-факт, не превышающей 20\%, составляет около 63\%. И при этом эти 63\% скважин обеспечивают накопленную добычу нефти чуть больше 80\%.\\

Принцип Паретто: 20\% усилий дают 80\% результата; чтобы получить оставшиеся 20\% результата приходится приложить 80\% усилий.\\

Для задач, где не требуется настройка каждой скважины (необходимо понимать только поведение месторождения в целом, например, для проектно-технологических документов), обычно требуют настройку в пределах 20\% только для тех скважин, которые суммарно дают 80\% накопленной добычи по месторождению. 

\subsubsection{"<Запрещённые"> и нежелательные приёмы адаптации}

\includegraphics[width=\textwidth, page=163]{Kurs_OsnovyGDM_Kai_774_gorodovSV_v6_0.pdf}

Стоит помнить, что целью модели является прогноз дальнейшей динамики работы месторождения при различных сценариях.
Если будем использовать некорректные методы адаптации, то это сделает модель непригодной для дальнейших прогнозов.

\subsection{Упражнение 5. Расчёт моделей с разными наборами исходных данных}

\includegraphics[width=\textwidth, page=164]{Kurs_OsnovyGDM_Kai_774_gorodovSV_v6_0.pdf}

\subsection{Упражнение 6. Адаптация ГДМ}

\includegraphics[width=\textwidth, page=165]{Kurs_OsnovyGDM_Kai_774_gorodovSV_v6_0.pdf}

\subsection{Упражнение 6. Адаптация ГДМ. Обсуждение результатов}

\includegraphics[width=\textwidth, page=166]{Kurs_OsnovyGDM_Kai_774_gorodovSV_v6_0.pdf}

\subsection{Групповая дискуссия}

\includegraphics[width=\textwidth, page=167]{Kurs_OsnovyGDM_Kai_774_gorodovSV_v6_0.pdf}

\subsection{Инструменты для оптимизации разработки месторождения}

\includegraphics[width=\textwidth, page=168]{Kurs_OsnovyGDM_Kai_774_gorodovSV_v6_0.pdf}

После создания и настройки модели её можно использовать для подбора вариантов разработки, оптимизации текущей разработки месторождения.

Для этого нужно провести анализ этой настроенной модели, получить из неё карты остаточных подвижных запасов нефти, карты пластового давления, карты проницаемости.\\

С помощью линий тока можем оценить, насколько эффективно работает каждая из нагнетательных скважин (насколько эффективно она вытесняет нефть, воду; насколько эффективно поддерживает давление).

\subsection{Линии тока}

\includegraphics[width=\textwidth, page=169]{Kurs_OsnovyGDM_Kai_774_gorodovSV_v6_0.pdf}

\subsection{Оптимизация ППД на основе матриц дренирования}

\includegraphics[width=\textwidth, page=170]{Kurs_OsnovyGDM_Kai_774_gorodovSV_v6_0.pdf}



\includegraphics[width=\textwidth, page=171]{Kurs_OsnovyGDM_Kai_774_gorodovSV_v6_0.pdf}

\subsection{Прогнозные расчёты. Анализ таблиц дренирования}

\includegraphics[width=\textwidth, page=172]{Kurs_OsnovyGDM_Kai_774_gorodovSV_v6_0.pdf}

\includegraphics[width=\textwidth, page=173]{Kurs_OsnovyGDM_Kai_774_gorodovSV_v6_0.pdf}

\subsection{Подготовка и проведение прогнозных расчётов}

\includegraphics[width=\textwidth, page=174]{Kurs_OsnovyGDM_Kai_774_gorodovSV_v6_0.pdf}

\includegraphics[width=\textwidth, page=175]{Kurs_OsnovyGDM_Kai_774_gorodovSV_v6_0.pdf}

\includegraphics[width=\textwidth, page=176]{Kurs_OsnovyGDM_Kai_774_gorodovSV_v6_0.pdf}

\subsection{Создание рестартов из GUI tNavigator}

\includegraphics[width=\textwidth, page=177]{Kurs_OsnovyGDM_Kai_774_gorodovSV_v6_0.pdf}

\subsection{Вырезание сектора}

\includegraphics[width=\textwidth, page=178]{Kurs_OsnovyGDM_Kai_774_gorodovSV_v6_0.pdf}

\subsection{Подготовка и проведение прогнозных расчётов}

\includegraphics[width=\textwidth, page=179]{Kurs_OsnovyGDM_Kai_774_gorodovSV_v6_0.pdf}

\subsection{Прогнозные расчёты}

\subsubsection{Расстановка ВС}

\includegraphics[width=\textwidth, page=180]{Kurs_OsnovyGDM_Kai_774_gorodovSV_v6_0.pdf}

\subsubsection{Расстановка ГС/ННС}

\includegraphics[width=\textwidth, page=181]{Kurs_OsnovyGDM_Kai_774_gorodovSV_v6_0.pdf}

\subsubsection{Расстановка скважин по сетке}

\includegraphics[width=\textwidth, page=182]{Kurs_OsnovyGDM_Kai_774_gorodovSV_v6_0.pdf}

\subsubsection{Моделирование ГТМ}

\includegraphics[width=\textwidth, page=183]{Kurs_OsnovyGDM_Kai_774_gorodovSV_v6_0.pdf}

\includegraphics[width=\textwidth, page=184]{Kurs_OsnovyGDM_Kai_774_gorodovSV_v6_0.pdf}

\subsubsection{Задание ГРП}

\includegraphics[width=\textwidth, page=185]{Kurs_OsnovyGDM_Kai_774_gorodovSV_v6_0.pdf}

\subsubsection{Оценка адекватности результатов}

\includegraphics[width=\textwidth, page=186]{Kurs_OsnovyGDM_Kai_774_gorodovSV_v6_0.pdf}

\subsubsection{Анализ чувствительности}

\includegraphics[width=\textwidth, page=187]{Kurs_OsnovyGDM_Kai_774_gorodovSV_v6_0.pdf}

\includegraphics[width=\textwidth, page=188]{Kurs_OsnovyGDM_Kai_774_gorodovSV_v6_0.pdf}

\subsection{Упражнение 7. Прогнозные расчёты}

\includegraphics[width=\textwidth, page=189]{Kurs_OsnovyGDM_Kai_774_gorodovSV_v6_0.pdf}

\subsection{Регламенты по созданию ГДМ}

\includegraphics[width=\textwidth, page=190]{Kurs_OsnovyGDM_Kai_774_gorodovSV_v6_0.pdf}

\subsection{Сравнительная характеристика ПО для ГДМ}

\includegraphics[width=\textwidth, page=191]{Kurs_OsnovyGDM_Kai_774_gorodovSV_v6_0.pdf}

\subsubsection{Тесты SPE}

\includegraphics[width=\textwidth, page=192]{Kurs_OsnovyGDM_Kai_774_gorodovSV_v6_0.pdf}

\includegraphics[width=\textwidth, page=193]{Kurs_OsnovyGDM_Kai_774_gorodovSV_v6_0.pdf}

\includegraphics[width=\textwidth, page=194]{Kurs_OsnovyGDM_Kai_774_gorodovSV_v6_0.pdf}

\subsubsection{О недостатках тестов SPE}

\includegraphics[width=\textwidth, page=195]{Kurs_OsnovyGDM_Kai_774_gorodovSV_v6_0.pdf}

\includegraphics[width=\textwidth, page=196]{Kurs_OsnovyGDM_Kai_774_gorodovSV_v6_0.pdf}

\subsubsection{Тесты ЦКР}

\includegraphics[width=\textwidth, page=197]{Kurs_OsnovyGDM_Kai_774_gorodovSV_v6_0.pdf}

\subsection{Основные проблемы моделирования}

\includegraphics[width=\textwidth, page=198]{Kurs_OsnovyGDM_Kai_774_gorodovSV_v6_0.pdf}

\subsection{Выводы}

\includegraphics[width=\textwidth, page=199]{Kurs_OsnovyGDM_Kai_774_gorodovSV_v6_0.pdf}

\includegraphics[width=\textwidth, page=202]{Kurs_OsnovyGDM_Kai_774_gorodovSV_v6_0.pdf}

\includegraphics[width=\textwidth, page=203]{Kurs_OsnovyGDM_Kai_774_gorodovSV_v6_0.pdf}

\includegraphics[width=\textwidth, page=204]{Kurs_OsnovyGDM_Kai_774_gorodovSV_v6_0.pdf}

\includegraphics[width=\textwidth, page=205]{Kurs_OsnovyGDM_Kai_774_gorodovSV_v6_0.pdf}

\includegraphics[width=\textwidth, page=206]{Kurs_OsnovyGDM_Kai_774_gorodovSV_v6_0.pdf}


\end{document}
