\documentclass[main.tex]{subfiles}

\begin{document}

\section{Лекция 13.09.2022 (Кайгородов С.В.)}

\subsection{Анализ разработки перед построением модели}

\includegraphics[width=\textwidth, page=114]{Kurs_OsnovyGDM_Kai_774_gorodovSV_v6_0.pdf}

Перед построением модели необходимо провести анализ разработки, чтобы понимать, какие скважины друг на друга влияют, откуда они обводняются, есть ли загрязнения призабойной зоны, есть ли трещины авто-ГРП на нагнетательных скважинах.

Другими словами, необходимо проанализировать, как работает месторождение, как работают скважины, чтобы это учесть при построении модели.

\subsection{Матбаланс}

\includegraphics[width=\textwidth, page=115]{Kurs_OsnovyGDM_Kai_774_gorodovSV_v6_0.pdf}


\includegraphics[width=\textwidth, page=116]{Kurs_OsnovyGDM_Kai_774_gorodovSV_v6_0.pdf}

\subsection{Матбаланс. Пример использования}

\includegraphics[width=\textwidth, page=117]{Kurs_OsnovyGDM_Kai_774_gorodovSV_v6_0.pdf}

Делали проект разработки группы месторождений.
Площадь большая, и поэтому возник вопрос: связаны ли все эти залежи нефти (отмечены зелёным на карте) между собой?
Красным на карте отмечены разломы.

Если строить одну большую модель, расчёт будет идти долго. Поэтому необходимо проверить, можно ли разрезать рассматриваемый участок на несколько отдельных участков, чтобы построить несколько отдельных моделей.

Построили несколько простых моделей материального баланса: каждая залежь представлялась отдельной бочкой и между этими бочками рисовалась связь (отмечена голубыми ромбиками). Далее проводился расчёт и производилась настройка параметров связности
между различными залежами.
На графиках синим обозначены рассчитанные значения динамики пластового давления, а красные квадратики -- фактические замеры.

Оказалось, что наилучшую настройку показали модели, в которых часть из этих залежей не связаны.
Это позволило разделить рассматриваемую группу месторождений на отдельные части, моделировать их отдельно и соответственно ускорить расчёты; другими словами, за ограниченное время проекта сделать больше расчётов.

\subsection{Анализ источников обводнения}

\includegraphics[width=\textwidth, page=118]{Kurs_OsnovyGDM_Kai_774_gorodovSV_v6_0.pdf}

По виду характеристических графиков Чена можно определить, откуда в скважину попала вода.
Строится график водонефтяного фактора (ВНФ, WOR = отношение добытой воды к добытой нефти) от времени в логарифмических координатах.

Замечание. За рубежом более популярен ВНФ, у нас обычно используют обводнённость.

\includegraphics[width=\textwidth, page=119]{Kurs_OsnovyGDM_Kai_774_gorodovSV_v6_0.pdf}

\includegraphics[width=\textwidth, page=120]{Kurs_OsnovyGDM_Kai_774_gorodovSV_v6_0.pdf}

\includegraphics[width=\textwidth, page=121]{Kurs_OsnovyGDM_Kai_774_gorodovSV_v6_0.pdf}

К сожалению, не всегда удаётся увидеть эти закономерности на фактических данных (т.к. фактические данные часто бывают зашумлены, есть погрешности измерений и т.д.).
Но иногда это срабатывает, и графики Чена дают действительно полезную информацию об источнике обводнения.

\subsection{Оценка загрязнения призабойной зоны}

\includegraphics[width=\textwidth, page=122]{Kurs_OsnovyGDM_Kai_774_gorodovSV_v6_0.pdf}

\subsection{Оценка наличия трещин авто-ГРП на нагнетательных скважинах}

\includegraphics[width=\textwidth, page=123]{Kurs_OsnovyGDM_Kai_774_gorodovSV_v6_0.pdf}

Если график отклоняется вверх от диагональной прямой (т.е. для закачки такого же объёма воды требуется большее давление), то это говорит нам об ухудшении свойств призабойной зоны.

Если график отклоняется вниз от диагональной прямой (т.е. для закачки такого же объёма воды требуется меньшее давление), то это говорит нам о трещине  авто-ГРП (улучшение проницаемости в призабойной зоне).

Дребезжание на графике Холла отражает поведение при циклической закачке.

Отклонения на графике Холла могут быть вызваны не только изменением свойств призабойной зоны, но и изменением диаметра штуцера (при смене штуцеров). Этот факт важно учитывать при анализе графиков Холла, чтобы не сделать ошибочные выводы. 

\subsection{Исходные данные по скважинам}

\includegraphics[width=\textwidth, page=124]{Kurs_OsnovyGDM_Kai_774_gorodovSV_v6_0.pdf}

Принадлежность к группе создаётся, если хотим отслеживать, как работают отдельные кусты скважин или скважины, относящиеся к одной какой-либо группе - например, к одной ДНС (дожимной насосной станции).

\subsection{Моделирование притока к скважине}

\includegraphics[width=\textwidth, page=125]{Kurs_OsnovyGDM_Kai_774_gorodovSV_v6_0.pdf}

\includegraphics[width=\textwidth, page=126]{Kurs_OsnovyGDM_Kai_774_gorodovSV_v6_0.pdf}

\subsection{Способы инициализации модели в симуляторах}

\includegraphics[width=\textwidth, page=127]{Kurs_OsnovyGDM_Kai_774_gorodovSV_v6_0.pdf}

\subsubsection{Неравновесный}

\includegraphics[width=\textwidth, page=128]{Kurs_OsnovyGDM_Kai_774_gorodovSV_v6_0.pdf}

Неравновесный: задаются значения давления и насыщенности на начальный момент времени во всех ячейках.
На начальный момент времени залежь не находится в равновесии. После инициализации могут начаться перетоки даже в том случае, если в модели нет никаких скважин.

Такой способ инициализации на практике практически не встречается (ведь всегда предполагается, что залежь формировалась долгое время, за которое все флюиды пришли в гидростатическое равновесие).

\subsubsection{Равновесный}

\includegraphics[width=\textwidth, page=129]{Kurs_OsnovyGDM_Kai_774_gorodovSV_v6_0.pdf}

На практике обычно используют равновесный способ инициализации.

\includegraphics[width=\textwidth, page=130]{Kurs_OsnovyGDM_Kai_774_gorodovSV_v6_0.pdf}

\subsubsection{Равновесный с соблюдением начальной насыщенности}

\includegraphics[width=\textwidth, page=131]{Kurs_OsnovyGDM_Kai_774_gorodovSV_v6_0.pdf}

\subsection{Оценка корректности инициализации ГДМ}

\includegraphics[width=\textwidth, page=132]{Kurs_OsnovyGDM_Kai_774_gorodovSV_v6_0.pdf}

\begin{enumerate}
	\item Можно сравнить насыщенность на начальный шаг с заданной насыщенностью
	\item Можно оценить диапазоны изменения куба капиллярного давления после инициализации и сравнить их с теми, которые получали при исследовании на керне
	\item Можно оценить, совпадают ли запасы в ГДМ с геомоделью
	\item Можно провести расчёт модели без скважин и убедиться в отсутствии изменений насыщенности и давления (для равновесных инициализаций)
\end{enumerate}

\subsection{Аналитический аквифер}

\includegraphics[width=\textwidth, page=133]{Kurs_OsnovyGDM_Kai_774_gorodovSV_v6_0.pdf}

\includegraphics[width=\textwidth, page=134]{Kurs_OsnovyGDM_Kai_774_gorodovSV_v6_0.pdf}

\includegraphics[width=\textwidth, page=135]{Kurs_OsnovyGDM_Kai_774_gorodovSV_v6_0.pdf}

\subsection{Упражнение 2. Создание синтетической BOX-модели}

\includegraphics[width=\textwidth, page=136]{Kurs_OsnovyGDM_Kai_774_gorodovSV_v6_0.pdf}

\subsection{Упражнение 3. Инициализация ГДМ}

\includegraphics[width=\textwidth, page=137]{Kurs_OsnovyGDM_Kai_774_gorodovSV_v6_0.pdf}

\subsection{Задание истории работы скважин}

\includegraphics[width=\textwidth, page=138]{Kurs_OsnovyGDM_Kai_774_gorodovSV_v6_0.pdf}

\subsection{Упражнение 4. Подготовка SCHEDULE-секции}

\includegraphics[width=\textwidth, page=139]{Kurs_OsnovyGDM_Kai_774_gorodovSV_v6_0.pdf}

\subsection{Алгоритм работы в ПО SCHEDULE}

\includegraphics[width=\textwidth, page=140]{Kurs_OsnovyGDM_Kai_774_gorodovSV_v6_0.pdf}

\includegraphics[width=\textwidth, page=141]{Kurs_OsnovyGDM_Kai_774_gorodovSV_v6_0.pdf}

\includegraphics[width=\textwidth, page=142]{Kurs_OsnovyGDM_Kai_774_gorodovSV_v6_0.pdf}

\includegraphics[width=\textwidth, page=143]{Kurs_OsnovyGDM_Kai_774_gorodovSV_v6_0.pdf}

\includegraphics[width=\textwidth, page=144]{Kurs_OsnovyGDM_Kai_774_gorodovSV_v6_0.pdf}

\subsection{Загрузка истории эксплуатации}

\includegraphics[width=\textwidth, page=145]{Kurs_OsnovyGDM_Kai_774_gorodovSV_v6_0.pdf}

\includegraphics[width=\textwidth, page=146]{Kurs_OsnovyGDM_Kai_774_gorodovSV_v6_0.pdf}

\includegraphics[width=\textwidth, page=147]{Kurs_OsnovyGDM_Kai_774_gorodovSV_v6_0.pdf}

\includegraphics[width=\textwidth, page=148]{Kurs_OsnovyGDM_Kai_774_gorodovSV_v6_0.pdf}

\subsection{Адаптация модели}

\includegraphics[width=\textwidth, page=149]{Kurs_OsnovyGDM_Kai_774_gorodovSV_v6_0.pdf}

\subsubsection{Обратные задачи}

\includegraphics[width=\textwidth, page=150]{Kurs_OsnovyGDM_Kai_774_gorodovSV_v6_0.pdf}

\includegraphics[width=\textwidth, page=151]{Kurs_OsnovyGDM_Kai_774_gorodovSV_v6_0.pdf}

\subsubsection{Адаптация модели на разных стадиях разработки}

\includegraphics[width=\textwidth, page=152]{Kurs_OsnovyGDM_Kai_774_gorodovSV_v6_0.pdf}

\includegraphics[width=\textwidth, page=153]{Kurs_OsnovyGDM_Kai_774_gorodovSV_v6_0.pdf}

\subsubsection{По отборам жидкости и пластовому давлению}

\includegraphics[width=\textwidth, page=154]{Kurs_OsnovyGDM_Kai_774_gorodovSV_v6_0.pdf}

\subsubsection{По соотношению нефть/вода}

\includegraphics[width=\textwidth, page=155]{Kurs_OsnovyGDM_Kai_774_gorodovSV_v6_0.pdf}

\subsubsection{По коэффициенту продуктивности и Pзаб}

\includegraphics[width=\textwidth, page=156]{Kurs_OsnovyGDM_Kai_774_gorodovSV_v6_0.pdf}

\includegraphics[width=\textwidth, page=157]{Kurs_OsnovyGDM_Kai_774_gorodovSV_v6_0.pdf}

\subsubsection{Алгоритм проведения автоадаптации}

\includegraphics[width=\textwidth, page=158]{Kurs_OsnovyGDM_Kai_774_gorodovSV_v6_0.pdf}

\subsubsection{Программы автоадаптации}

\includegraphics[width=\textwidth, page=159]{Kurs_OsnovyGDM_Kai_774_gorodovSV_v6_0.pdf}

\subsubsection{Критерии адаптации}

\includegraphics[width=\textwidth, page=160]{Kurs_OsnovyGDM_Kai_774_gorodovSV_v6_0.pdf}

\includegraphics[width=\textwidth, page=161]{Kurs_OsnovyGDM_Kai_774_gorodovSV_v6_0.pdf}

\includegraphics[width=\textwidth, page=162]{Kurs_OsnovyGDM_Kai_774_gorodovSV_v6_0.pdf}

\subsubsection{"<Запрещённые"> и нежелательные приёмы адаптации}

\includegraphics[width=\textwidth, page=163]{Kurs_OsnovyGDM_Kai_774_gorodovSV_v6_0.pdf}

\subsection{Упражнение 5. Расчёт моделей с разными наборами исходных данных}

\includegraphics[width=\textwidth, page=164]{Kurs_OsnovyGDM_Kai_774_gorodovSV_v6_0.pdf}

\subsection{Упражнение 6. Адаптация ГДМ}

\includegraphics[width=\textwidth, page=165]{Kurs_OsnovyGDM_Kai_774_gorodovSV_v6_0.pdf}

\subsection{Упражнение 6. Адаптация ГДМ. Обсуждение результатов}

\includegraphics[width=\textwidth, page=166]{Kurs_OsnovyGDM_Kai_774_gorodovSV_v6_0.pdf}

\subsection{Групповая дискуссия}

\includegraphics[width=\textwidth, page=167]{Kurs_OsnovyGDM_Kai_774_gorodovSV_v6_0.pdf}

\subsection{Инструменты для оптимизации разработки месторождения}

\includegraphics[width=\textwidth, page=168]{Kurs_OsnovyGDM_Kai_774_gorodovSV_v6_0.pdf}

\subsection{Линии тока}

\includegraphics[width=\textwidth, page=169]{Kurs_OsnovyGDM_Kai_774_gorodovSV_v6_0.pdf}

\subsection{Оптимизация ППД на основе матриц дренирования}

\includegraphics[width=\textwidth, page=170]{Kurs_OsnovyGDM_Kai_774_gorodovSV_v6_0.pdf}

\includegraphics[width=\textwidth, page=171]{Kurs_OsnovyGDM_Kai_774_gorodovSV_v6_0.pdf}

\subsection{Прогнозные расчёты. Анализ таблиц дренирования}

\includegraphics[width=\textwidth, page=172]{Kurs_OsnovyGDM_Kai_774_gorodovSV_v6_0.pdf}

\includegraphics[width=\textwidth, page=173]{Kurs_OsnovyGDM_Kai_774_gorodovSV_v6_0.pdf}

\subsection{Подготовка и проведение прогнозных расчётов}

\includegraphics[width=\textwidth, page=174]{Kurs_OsnovyGDM_Kai_774_gorodovSV_v6_0.pdf}

\includegraphics[width=\textwidth, page=175]{Kurs_OsnovyGDM_Kai_774_gorodovSV_v6_0.pdf}

\includegraphics[width=\textwidth, page=176]{Kurs_OsnovyGDM_Kai_774_gorodovSV_v6_0.pdf}

\subsection{Создание рестартов из GUI tNavigator}

\includegraphics[width=\textwidth, page=177]{Kurs_OsnovyGDM_Kai_774_gorodovSV_v6_0.pdf}

\subsection{Вырезание сектора}

\includegraphics[width=\textwidth, page=178]{Kurs_OsnovyGDM_Kai_774_gorodovSV_v6_0.pdf}

\subsection{Подготовка и проведение прогнозных расчётов}

\includegraphics[width=\textwidth, page=179]{Kurs_OsnovyGDM_Kai_774_gorodovSV_v6_0.pdf}

\subsection{Прогнозные расчёты}

\subsubsection{Расстановка ВС}

\includegraphics[width=\textwidth, page=180]{Kurs_OsnovyGDM_Kai_774_gorodovSV_v6_0.pdf}

\subsubsection{Расстановка ГС/ННС}

\includegraphics[width=\textwidth, page=181]{Kurs_OsnovyGDM_Kai_774_gorodovSV_v6_0.pdf}

\subsubsection{Расстановка скважин по сетке}

\includegraphics[width=\textwidth, page=182]{Kurs_OsnovyGDM_Kai_774_gorodovSV_v6_0.pdf}

\subsubsection{Моделирование ГТМ}

\includegraphics[width=\textwidth, page=183]{Kurs_OsnovyGDM_Kai_774_gorodovSV_v6_0.pdf}

\includegraphics[width=\textwidth, page=184]{Kurs_OsnovyGDM_Kai_774_gorodovSV_v6_0.pdf}

\subsubsection{Задание ГРП}

\includegraphics[width=\textwidth, page=185]{Kurs_OsnovyGDM_Kai_774_gorodovSV_v6_0.pdf}

\subsubsection{Оценка адекватности результатов}

\includegraphics[width=\textwidth, page=186]{Kurs_OsnovyGDM_Kai_774_gorodovSV_v6_0.pdf}

\subsubsection{Анализ чувствительности}

\includegraphics[width=\textwidth, page=187]{Kurs_OsnovyGDM_Kai_774_gorodovSV_v6_0.pdf}

\includegraphics[width=\textwidth, page=188]{Kurs_OsnovyGDM_Kai_774_gorodovSV_v6_0.pdf}

\subsection{Упражнение 7. Прогнозные расчёты}

\includegraphics[width=\textwidth, page=189]{Kurs_OsnovyGDM_Kai_774_gorodovSV_v6_0.pdf}

\subsection{Регламенты по созданию ГДМ}

\includegraphics[width=\textwidth, page=190]{Kurs_OsnovyGDM_Kai_774_gorodovSV_v6_0.pdf}

\subsection{Сравнительная характеристика ПО для ГДМ}

\includegraphics[width=\textwidth, page=191]{Kurs_OsnovyGDM_Kai_774_gorodovSV_v6_0.pdf}

\subsubsection{Тесты SPE}

\includegraphics[width=\textwidth, page=192]{Kurs_OsnovyGDM_Kai_774_gorodovSV_v6_0.pdf}

\includegraphics[width=\textwidth, page=193]{Kurs_OsnovyGDM_Kai_774_gorodovSV_v6_0.pdf}

\includegraphics[width=\textwidth, page=194]{Kurs_OsnovyGDM_Kai_774_gorodovSV_v6_0.pdf}

\subsubsection{О недостатках тестов SPE}

\includegraphics[width=\textwidth, page=195]{Kurs_OsnovyGDM_Kai_774_gorodovSV_v6_0.pdf}

\includegraphics[width=\textwidth, page=196]{Kurs_OsnovyGDM_Kai_774_gorodovSV_v6_0.pdf}

\subsubsection{Тесты ЦКР}

\includegraphics[width=\textwidth, page=197]{Kurs_OsnovyGDM_Kai_774_gorodovSV_v6_0.pdf}

\subsection{Основные проблемы моделирования}

\includegraphics[width=\textwidth, page=198]{Kurs_OsnovyGDM_Kai_774_gorodovSV_v6_0.pdf}

\subsection{Выводы}

\includegraphics[width=\textwidth, page=199]{Kurs_OsnovyGDM_Kai_774_gorodovSV_v6_0.pdf}

\includegraphics[width=\textwidth, page=202]{Kurs_OsnovyGDM_Kai_774_gorodovSV_v6_0.pdf}

\includegraphics[width=\textwidth, page=203]{Kurs_OsnovyGDM_Kai_774_gorodovSV_v6_0.pdf}

\includegraphics[width=\textwidth, page=204]{Kurs_OsnovyGDM_Kai_774_gorodovSV_v6_0.pdf}

\includegraphics[width=\textwidth, page=205]{Kurs_OsnovyGDM_Kai_774_gorodovSV_v6_0.pdf}

\includegraphics[width=\textwidth, page=206]{Kurs_OsnovyGDM_Kai_774_gorodovSV_v6_0.pdf}


\end{document}
