\documentclass[main.tex]{subfiles}

\begin{document}

\section{Лекция 13.09.2022 (Кайгородов С.В.)}

\subsection{Анализ разработки перед построением модели}

\includegraphics[width=\textwidth, page=114]{Kurs_OsnovyGDM_Kai_774_gorodovSV_v6_0.pdf}

Перед построением модели необходимо провести анализ разработки, чтобы понимать, какие скважины друг на друга влияют, откуда они обводняются, есть ли загрязнения призабойной зоны, есть ли трещины авто-ГРП на нагнетательных скважинах.

Другими словами, необходимо проанализировать, как работает месторождение, как работают скважины, чтобы это учесть при построении модели.

\subsection{Матбаланс}

\includegraphics[width=\textwidth, page=115]{Kurs_OsnovyGDM_Kai_774_gorodovSV_v6_0.pdf}

Один из способов анализа разработки месторождения -- это материальный баланс.

По факту представляем в виде бочки, в которую что-то втекает, что-то вытекает, то что осталось в бочке либо расширяется, либо сжимается в зависимости от того, как изменилось давление.
Дополнительно может выделяться газ и так далее.

На слайде записана формула в общем виде: то, что относится к газу, к нефти, к воде.
С одной стороны то, что было, что расширилось, закачалось.

\includegraphics[width=\textwidth, page=116]{Kurs_OsnovyGDM_Kai_774_gorodovSV_v6_0.pdf}

Обычно есть либо в Экселе какие-то готовые формулы, через которые считается матбаланс, либо специальные программы (MBAL и ещё какие-то).

На вход подаются начальное пластовое давление, PVT-свойства, точная история отбора и закачки, свойства коллектора и аквифера, объём начальных запасов.

И на основе этих данных рассчитывается динамика пластового давления, которая затем сопоставляется с фактическими замерами: если расчёт не совпал с фактом, то значит где-то исходные данные неточные.
В пределах имеющихся неопределённостей исходных данных можем их поварьировать и таким образом добиться совпадения расчётной динамики пластового давления и фактической.

За счёт такой вариации можно проанализировать, насколько неопределённые входные параметры, т.е., например, начальные запасы, свойства пласта или аквифера.
И таким образом как бы осуществить анализ на основе матбаланса.

\subsection{Матбаланс. Пример использования}

\includegraphics[width=\textwidth, page=117]{Kurs_OsnovyGDM_Kai_774_gorodovSV_v6_0.pdf}

Здесь приведён пример использования матбаланса.
\\

Делали проект разработки группы месторождений.
Площадь большая, и поэтому возник вопрос: связаны ли все эти залежи нефти (отмечены зелёным на карте) между собой?
Красным на карте отмечены разломы.

Если строить одну большую модель, расчёт будет идти долго. Поэтому необходимо проверить, можно ли разрезать рассматриваемый участок на несколько отдельных участков, чтобы построить несколько отдельных моделей.

Построили несколько простых моделей материального баланса: каждая залежь представлялась отдельной бочкой и между этими бочками рисовалась связь (отмечена голубыми ромбиками). Далее проводился расчёт и производилась настройка параметров связности между различными залежами.
На графиках синим обозначены рассчитанные значения динамики пластового давления, а красные квадратики -- фактические замеры.

Оказалось, что наилучшую настройку показали модели, в которых часть из этих залежей не связаны.
Другими словами, в результате настройки модели матбаланса оказалось, что какие-то связанности оказались нулевыми или близкими к нулю.

Это позволило разделить рассматриваемую группу месторождений на отдельные части, моделировать их отдельно и соответственно ускорить расчёты; другими словами, за ограниченное время проекта сделать больше расчётов.

\subsection{Анализ источников обводнения}

\includegraphics[width=\textwidth, page=118]{Kurs_OsnovyGDM_Kai_774_gorodovSV_v6_0.pdf}

Следующее, что можно использовать для анализа работы скважин, -- это анализ источников обводнения.
Есть такие характеристические графики Чена.
\\

По виду характеристических графиков Чена можно определить, откуда в скважину попала вода.
Строится график водонефтяного фактора (ВНФ, WOR = отношение добытой воды к добытой нефти) от времени в логарифмических координатах.
\\

\textbf{Замечание.} За рубежом более популярен ВНФ, у нас обычно используют обводнённость.
\\

На слайдах показаны характеристические графики Чена для разных источников обводнения.

\includegraphics[width=\textwidth, page=119]{Kurs_OsnovyGDM_Kai_774_gorodovSV_v6_0.pdf}

\includegraphics[width=\textwidth, page=120]{Kurs_OsnovyGDM_Kai_774_gorodovSV_v6_0.pdf}

Если вода попала в добывающую скважину от нагнетательной по пропласткам (постепенно какие-то пропластки обводнялись, и вода по ним попадала в добывающую скважину), то график Чена будет выглядит следующим образом: сначала на графике наблюдается резкий рост ВНФ и потом значительное замедление и рост более плавный.
Производная ВНФ тоже будет выглядеть как рост и затем стабильная кривая с небольшим ростом.

\includegraphics[width=\textwidth, page=121]{Kurs_OsnovyGDM_Kai_774_gorodovSV_v6_0.pdf}

Если источником обводнения служит конус воды, то на графике Чена это будет выглядеть как достаточно плавный рост водонефтяного фактора, а производная будет сначала расти и потом снижаться.
\\

К сожалению, не всегда удаётся увидеть эти закономерности на фактических данных (т.к. фактические данные часто бывают зашумлены, есть погрешности измерений, человеческий фактор и т.д.).
Но иногда это срабатывает, и графики Чена дают действительно полезную информацию об источнике обводнения.

\subsection{Оценка загрязнения призабойной зоны}

\includegraphics[width=\textwidth, page=122]{Kurs_OsnovyGDM_Kai_774_gorodovSV_v6_0.pdf}

Полезно также оценивать загрязнение призабойной зоны.
Для этого посмотреть на динамику работы скважин: если наблюдается снижение дебитов, коэффициента продуктивности, но при этом на скважине стабильная обводнённость и пластовое давление, то скорее всего это связано с загрязнением призабойной зоны.
\\

Можно ещё посмотреть на проведённые ГТМ, ремонты скважины.
Если проводили какие-то обработки призабойной зоны, ГРП и после этого дебит вырос, то можно говорить, что это загрязнение (скин) можно занулить или сделать отрицательным.
Если же скважина подвергалась ремонту, тогда её точно глушили, чтобы не допустить нефтегазоводопроявления на устье, и это могло вызвать кольматацию призабойной зоны.
Т.е. если после ремонта видим снижение продуктивности, снижение дебитов, то это может говорить нам о кольматации; в модели это будет воспроизводиться с помощью скин-фактора (его придётся подбирать, чтобы настроить добычу и коэффициент продуктивности).
\\

Также можно посмотреть по исследованиям скважин из какого интервала сколько жидкости притекает, какой интервал принимает нагнетательную скважину.
\\

Ещё косвенным признаком служит сопоставление профиля притока (или профиля приёмистости) с профилем проницаемости.
Если профиль приёмистости не совпадает с профилем проницаемости, то скорее всего высокопроницаемые пропластки загрязнились и поэтому стали меньше принимать: в итоге, принимают воду больше те пропластки, которые казалось бы должны принимать её меньше.
Такой вот косвенный признак того, что какие-то из пропластков загрязнились и соответственно в модели это тоже можно воспроизвести, чтобы корректно моделировать приток по пропласткам.

\subsection{Оценка наличия трещин авто-ГРП на нагнетательных скважинах}

\includegraphics[width=\textwidth, page=123]{Kurs_OsnovyGDM_Kai_774_gorodovSV_v6_0.pdf}

На нагнетательных скважинах можно оценить наличие ухудшения призабойной зоны или наоборот трещины авто-ГРП по графику Холла.
В идеале все точки должны лежать на диагональной кривой и, если они отклоняются вверх или вниз, то это служит диагностическим критерием для определения того, что со скважиной что-то произошло (что-то пошло не так).
\\

Если график отклоняется вверх от диагональной прямой (т.е. для закачки такого же объёма воды требуется большее давление), то это говорит нам об ухудшении свойств призабойной зоны, т.е. скважины закольматировалась и поэтому стала хуже принимать.

Если график отклоняется вниз от диагональной прямой (т.е. для закачки такого же объёма воды требуется меньшее давление), то это говорит нам о трещине  авто-ГРП (об улучшении проницаемости в призабойной зоне).

Дребезжание на графике Холла отражает поведение при циклической закачке.

Отклонения на графике Холла могут быть вызваны не только изменением свойств призабойной зоны, но и изменением диаметра штуцера (при смене штуцеров).
Этот факт важно учитывать при анализе графиков Холла, чтобы не сделать ошибочные выводы. 

\subsection{Исходные данные по скважинам}

\includegraphics[width=\textwidth, page=124]{Kurs_OsnovyGDM_Kai_774_gorodovSV_v6_0.pdf}

Принадлежность к группе создаётся, если хотим отслеживать, как работают отдельные кусты скважин или скважины, относящиеся к одной какой-либо группе - например, к одной ДНС (дожимной насосной станции).

\subsection{Моделирование притока к скважине}

\includegraphics[width=\textwidth, page=125]{Kurs_OsnovyGDM_Kai_774_gorodovSV_v6_0.pdf}

Формула для расчёта притока к скважине в модели немного отличается от стандартной формулы расчёта радиального притока.
Отличается в плане того, что в обычной формуле мы берём радиус контура питания и давление на этом контуре питания, а в модели мы рассчитываем приток в скважину из той ячейки (или из тех ячеек), которые скважина вскрывает.
И соответственно вместо радиуса контура питания берётся такой радиус от скважины, на котором давление равно среднему давлению в ячейке.
Также в этой формуле будет немого отличаться скин-фактор.
В итоге, эти 3 параметра (радиус контура питания, давление на контуре питания, скин-фактор) отличаются, но так, чтобы дебит, который рассчитан по формуле компьютерной модели, совпадал с дебитом, рассчитанном по формуле радиального притока.

\includegraphics[width=\textwidth, page=126]{Kurs_OsnovyGDM_Kai_774_gorodovSV_v6_0.pdf}

На слайде представлены формулы для радиуса Писмана.
\\

Обычно у нас пласт изотропный, проницаемости по $x$ и по $y$ одинаковы, размеры ячеек тоже примерно одинаковы, поэтому можно считать, что радиус Писмана $r_p\approx0.2\Delta x$, т.е. для 100-метровой ячейки на 20 метрах от скважины давление будет равно среднему давлению в ячейке.

\subsection{Способы инициализации модели в симуляторах}

\includegraphics[width=\textwidth, page=127]{Kurs_OsnovyGDM_Kai_774_gorodovSV_v6_0.pdf}

Фактически при инициализации модели задаём начальные условия.

Есть 2 способа инициализации: равновесный и неравновесный.
У равновесного есть ещё 1 подспособ, а именно равновесный с соблюдением начальной насыщенности.  

\subsubsection{Неравновесный}

\includegraphics[width=\textwidth, page=128]{Kurs_OsnovyGDM_Kai_774_gorodovSV_v6_0.pdf}

В неравновесном способе задаются значения давления и насыщенности на начальный момент времени во всех ячейках.
Из названия ясно, что на начальный момент времени залежь не находится в равновесии.
После инициализации могут начаться перетоки даже в том случае, если в модели нет никаких скважин.

Такой способ инициализации на практике практически не встречается, ведь всегда предполагается, что залежь формировалась долгое время, за которое все флюиды пришли в гидростатическое равновесие, поэтому для инициализации обычно используется равновесный способ.

\subsubsection{Равновесный}

\includegraphics[width=\textwidth, page=129]{Kurs_OsnovyGDM_Kai_774_gorodovSV_v6_0.pdf}

В равновесном способе инициализации задаётся ключевое слово EQUIL, в котором указываются опорная (отсчётная) глубина, пластовое давление на этой глубине, глубина ВНК, капиллярное давление на ВНК (если равно нулю, то это зеркало свободной воды FWL), глубина ГНК, капиллярное давление на ГНК, дальше 2 настроечных параметра по умолчанию (см. мануал) и параметр, определяющий точность расчёта запасов (если ячейка большая, то может оказаться так, что уровень ВНК может идти где-то между нижней гранью ячейки и её центром; в этом случае центр находится выше ВНК, поэтому вся ячейка по умолчанию будет иметь ненулевую нефтенасыщенность (даже часть, находящаяся ниже ВНК), что будет немного завышать запасы -- если есть необходимость точно воспроизвести запасы, то можно изменить последний параметр -- для этого он и нужен).
\\

В примере на слайде задано 3 строки: показано, что есть 3 региона, в которых разные уровни зеркала свободной воды.
Это может быть 3 пласта или 3 блока месторождения, которые не сообщаются между собой.
Для каждого мы соответственно задаём свои условия равновесия, свои условия инициализации.
\\

На практике обычно используют равновесный способ инициализации.

\includegraphics[width=\textwidth, page=130]{Kurs_OsnovyGDM_Kai_774_gorodovSV_v6_0.pdf}

Как происходит инициализация модели в случае равновесного способа?

Сначала вычисляется давление в нефтяной фазе (по формуле $\rho_o gh$) вверх и вниз от точки отсчёта (т.е. от уровня, равного значению первого параметра EQUIL).

Таким образом, получаем давление на заданном контакте.

Давление в водяной фазе на контакте получается отниманием капиллярного давления, заданного на контакте.

После этого вычисляем давление в водяной фазе (по формуле $\rho_w gh$) вверх и вниз от точки контакта.

Таким образом, в каждой ячейке есть давление в нефтяной фазе и давление в водяной фазе, а разница между этими давлениями -- это фактически капиллярное давление.

Дальше симулятор идёт в ключевое слово SWOF.
В этом ключевом слове заданы зависимости фазовых проницаемостей и капиллярного давления от насыщенности.
И симулятор таким образом находит для соответствующего значения капиллярного давления насыщенность и эту насыщенность задаёт в ячейках.

Т.е. в ячейках у симулятора рассчитаны давления в водяной и нефтяной фазах, их разница это капиллярные давления, а этим капиллярным давлениям можно сопоставить насыщенности (из таблицы), что и происходит.

\subsubsection{Равновесный с соблюдением начальной насыщенности}

\includegraphics[width=\textwidth, page=131]{Kurs_OsnovyGDM_Kai_774_gorodovSV_v6_0.pdf}

А что происходит, если у нас кроме ключевого слова EQUIL задаётся ещё куб начальной водонасыщенности (например, мы делаем проектно-технологическую документацию ПТД и нам строго нужно соблюдать запасы, которые есть в геологической модели)?
\\

Геолог передаёт нам куб водонасыщенности, и мы его подключаем в модель с помощью SWATINIT.
При таком способе инициализации у симулятора получается две насыщенности: которую он сам рассчитал через условие равновесия и которая у него есть в кубе SWATINIT.
Что делать, если они не совпали?
Симулятор говорит, что он будет стараться настроить насыщенность так, чтобы она совпала с тем, что задано в кубе SWATINIT.
Для этого он будет масштабировать кривую капиллярного давления (т.е. просто растягивать или сжимать её по вертикали) таким образом, чтобы насыщенность в данной ячейке совпала с той, которая задана в ключевом слове SWATINIT.

Если насыщенность геологом рассчитана некорректно (т.е. неравновесно), то это может привести к тому, что масштабирования приведут к тому, что капиллярное давление будет слишком большим или слишком маленьким (и это один из критериев для проверки корректности инициализации, т.е. можно оценить диапазоны изменения куба капиллярного давления после инициализации и сравнить его с тем, что мы получали по исследованиям на керне).

\subsection{Оценка корректности инициализации ГДМ}

\includegraphics[width=\textwidth, page=132]{Kurs_OsnovyGDM_Kai_774_gorodovSV_v6_0.pdf}

Как убедиться в корректности инициализации?

\begin{enumerate}
	\item Можно сравнить насыщенность на начальный (нулевой) шаг с заданной насыщенностью
	\item Можно оценить диапазоны изменения куба капиллярного давления после инициализации и сравнить их с теми, которые получали при исследовании на керне
	\item Можно оценить, совпадают ли запасы в ГДМ с геомоделью
	\item Можно провести расчёт модели без скважин на долгий период и убедиться в отсутствии изменений насыщенности и давления (для равновесных инициализаций)
\end{enumerate}

\subsection{Аналитический аквифер}

\includegraphics[width=\textwidth, page=133]{Kurs_OsnovyGDM_Kai_774_gorodovSV_v6_0.pdf}

Итак, начальные условия мы задали. 
Теперь переходим к граничным условиям.

На границах, если есть водоносный горизонт, то его можно задать в модели.

Есть точное решение Hurst van Everdingen, которое описывает приток из аквифера конечных размеров (формула представлена на слайде).

Но в модели такое точное решение не задать, поэтому была разработана модель притока из аквифера.

\includegraphics[width=\textwidth, page=134]{Kurs_OsnovyGDM_Kai_774_gorodovSV_v6_0.pdf}

Carter-Tracy разработали модель притока из аквифера, результаты расчётов по которой близки к аналитическому решению Hurst van Everdingen.

Модель Carter-Tracy задаётся в ГДМ симуляторах с помощью ключевого слова AQUCT.

Формулы представлены на слайде.
Фактически задаются параметры, которые описывают аквифер: мощность (толщина), пористость, сжимаемость, радиус и так называемый угол влияния.
В данной модели пласт представляется кругом или частью круга, а аквифер присоединяется к краям залежи.
И соответственно угол $\theta$, который говорит, какой частью круга является пласт, в формулу и входит.

Эта модель описывает переход из неустановившегося в псевдо-установившийся режим течения. 

Модель Carter-Tracy рекомендуется использовать либо для больших залежей, либо для низкопроницаемых залежей (другими словами, для залежей, на которых режим течения устанавливается не быстро).

\includegraphics[width=\textwidth, page=135]{Kurs_OsnovyGDM_Kai_774_gorodovSV_v6_0.pdf}

Для высокопроницаемых или маленьких залежей есть более простая модель Fetkovich-а, которая моделирует приток из аквифера просто в виде произведения продуктивности аквифера и разницы давлений в аквифере и нефтеносном пласте.

Для больших или низкопроницаемых залежей модель Fetkovich-а использовать не рекомендуется. В этом случае лучше использовать модель Carter-Tracy.

\subsection{Упражнение 2. Создание синтетической BOX-модели}

\includegraphics[width=\textwidth, page=136]{Kurs_OsnovyGDM_Kai_774_gorodovSV_v6_0.pdf}

\subsection{Упражнение 3. Инициализация ГДМ}

\includegraphics[width=\textwidth, page=137]{Kurs_OsnovyGDM_Kai_774_gorodovSV_v6_0.pdf}

\subsection{Задание истории работы скважин}

\includegraphics[width=\textwidth, page=138]{Kurs_OsnovyGDM_Kai_774_gorodovSV_v6_0.pdf}

Здесь перечислены ключевые слова для задания истории работы скважин.
\\

Всё, что написано после ключевого слова END, симулятор не читает.

\subsection{Упражнение 4. Подготовка SCHEDULE-секции}

\includegraphics[width=\textwidth, page=139]{Kurs_OsnovyGDM_Kai_774_gorodovSV_v6_0.pdf}

Далее показано, как формируется секция SCHEDULE с описанием работы скважин.

\subsection{Алгоритм работы в ПО SCHEDULE}

\includegraphics[width=\textwidth, page=140]{Kurs_OsnovyGDM_Kai_774_gorodovSV_v6_0.pdf}

На слайдах показан процесс работы в ПО SCHEDULE.
Раньше поставлялось в пакете Schlumberger вместе с Eclipse, сейчас практически не используется.

Информация на слайдах будет полезна, если вдруг появится необходимость работы с ПО SCHEDULE (можно по слайдам пройтись и посмотреть, какие кнопки нужно нажимать).

\includegraphics[width=\textwidth, page=141]{Kurs_OsnovyGDM_Kai_774_gorodovSV_v6_0.pdf}

\includegraphics[width=\textwidth, page=142]{Kurs_OsnovyGDM_Kai_774_gorodovSV_v6_0.pdf}

\includegraphics[width=\textwidth, page=143]{Kurs_OsnovyGDM_Kai_774_gorodovSV_v6_0.pdf}

\includegraphics[width=\textwidth, page=144]{Kurs_OsnovyGDM_Kai_774_gorodovSV_v6_0.pdf}

\subsection{Загрузка истории эксплуатации в т-Навигаторе}

\includegraphics[width=\textwidth, page=145]{Kurs_OsnovyGDM_Kai_774_gorodovSV_v6_0.pdf}

Можно сформировать SCHEDULE секцию с помощью tNavigator.
Для этого необходимо воспользоваться опцией: Загрузить данные по скважинам...
И далее действовать по алгоритму на слайдах (пройтись по закладкам и в каждой добавить соответствующий файл).

\includegraphics[width=\textwidth, page=146]{Kurs_OsnovyGDM_Kai_774_gorodovSV_v6_0.pdf}

\includegraphics[width=\textwidth, page=147]{Kurs_OsnovyGDM_Kai_774_gorodovSV_v6_0.pdf}

Если не поставить галочку "<Присвоить нулевые значения для дат, отсутствующих в загружаемой истории">, то пропущенным датам будут присвоены значения из предыдущей даты (т.е. будут просто протягиваться те дебиты и те приёмистости, которые записаны на прошлую дату).

\includegraphics[width=\textwidth, page=148]{Kurs_OsnovyGDM_Kai_774_gorodovSV_v6_0.pdf}

\subsection{Адаптация модели}

\includegraphics[width=\textwidth, page=149]{Kurs_OsnovyGDM_Kai_774_gorodovSV_v6_0.pdf}

После указания всех данных в модели можем поставить её на расчёт и обнаружить, что результат расчёта не совпал с фактическими замерами, которые мы в модель занесли.

Почему это происходит?

Во-первых, данных недостаточно.

Во-вторых, имеющиеся данные обладают неопределённостью: у нас данные точечные (только по скважинам), а в межскважинном пространстве геолог стохастическими методами (или ещё как-то) распределил свойства -- даже в скважинах измеренные данные обладают погрешностью, а в межскважинном пространстве эта погрешность тем более есть, что приводит к несовпадению результатов расчёта с фактом.

Перед тем как использовать построенную модель для прогнозов её нужно настроить на факт.
Другими словами, необходимо провести адаптацию модели, т.е. изменить значения параметров модели, обладающих неопределённостью, с целью минимизации отклонений расчётных значений параметров работы скважин и месторождения от фактически замеренных.

Но существует бесконечное множество сочетаний параметров модели, при которых результат расчёта этой модели будет с заданной точностью совпадать с фактом, замеренным по скважинам.

Обычный подход к адаптации -- это ручная корректировка параметров модели на основе инженерного опыта, представления о физике пласта. Но есть и программные комплексы, позволяющие осуществлять автоматизированную адаптацию на основе оптимизационных алгоритмов.

Задача оптимизационных алгоритмов: варьируя параметры модели, устремить целевую функцию к нулю.
Тоже есть много нюансов, как эти алгоритмы настроить, как задать целевую функцию и так далее.
И ещё сами алгоритмы не контролируют физическую адекватность решения (полученные решения нужно перепроверять).

\subsubsection{Обратные задачи}

\includegraphics[width=\textwidth, page=150]{Kurs_OsnovyGDM_Kai_774_gorodovSV_v6_0.pdf}

Есть несколько слайдов про обратные задачи.

Есть несколько точек. Вопрос: как построить аппроксимацию?

Здесь ясно, как аппроксимировать имеющиеся замеры.

\includegraphics[width=\textwidth, page=151]{Kurs_OsnovyGDM_Kai_774_gorodovSV_v6_0.pdf}

Но если есть шум (данные обладают погрешностью) и несколько размерностей (у модели много параметров), то задача подбора нужной поверхности становится нетривиальной и может иметь бесконечное множество разумных решений.

В этом случае очень сложно вручную подобрать параметры; необходимо осуществлять автоадаптацию и проверять найденные решения на разумность и физичность с точки зрения рассматриваемой гидродинамической модели.

\subsubsection{Адаптация модели на разных стадиях разработки}

\includegraphics[width=\textwidth, page=152]{Kurs_OsnovyGDM_Kai_774_gorodovSV_v6_0.pdf}

На разных периодах разработки месторождения настраиваем разные параметры модели.

Период до начала добычи = blue field.

Период безводной добычи = green field.

Период обводнённой добычи (зрелые месторождения) = brown field.

\includegraphics[width=\textwidth, page=153]{Kurs_OsnovyGDM_Kai_774_gorodovSV_v6_0.pdf}

Обычно адаптация идёт от крупного к мелкому (от месторождения к скважинам).

Сначала настраиваем энергетическое состояние залежи: матбаланс по скважинам (сколько отобрали жидкости / сколько закачали воды) и пластовое давление, которое получилось в результате работы всех скважин.

После настройки энергетики, переходим к настройке по соотношению нефть/вода или нефть/газ.
Т.е. к настройке по отборам конкретных флюидов.

И финально производится настройка по коэффициентам продуктивности и забойным давлениям.

\subsubsection{По отборам жидкости и пластовому давлению}

\includegraphics[width=\textwidth, page=154]{Kurs_OsnovyGDM_Kai_774_gorodovSV_v6_0.pdf}

При адаптации обычно меняют те параметры, которые обладают наибольшей неопределённостью.
\\

Про сжимаемость порового пространства: если говорить о месторождениях в Западной Сибири, то там пласты имеют сжимаемость порядка $10^{-5} \text{ атм}^{-1}$, что приводит к тому, что сжимаемость фактически не оказывает ощутимого влияния на динамику пластового давления.

\subsubsection{По соотношению нефть/вода}

\includegraphics[width=\textwidth, page=155]{Kurs_OsnovyGDM_Kai_774_gorodovSV_v6_0.pdf}

При варьировании остаточных насыщенностей гораздо легче испортить модель, чем при варьировании, например, абсолютной проницаемости.
\\

Необходимо помнить, что модель мы делаем для того, чтобы считать на ней какие-то прогнозы, оценивать как поведёт себя месторождение в случае какого-то воздействия, т.е. мы хотим получить адекватный инструмент и соответственно должны использовать физически корректные диапазоны вариации параметров модели.

\subsubsection{По коэффициенту продуктивности и Pзаб}

\includegraphics[width=\textwidth, page=156]{Kurs_OsnovyGDM_Kai_774_gorodovSV_v6_0.pdf}

По коэффициенту продуктивности и забойному давлению настройка простая: варьируем абсолютную проницаемость вблизи скважины или скин-фактор.

\subsection{Уточнение распределений параметров при адаптации модели}

\includegraphics[width=\textwidth, page=157]{Kurs_OsnovyGDM_Kai_774_gorodovSV_v6_0.pdf}

Можно воспринимать адаптацию модели, как уточнение исходных распределений параметров, т.е. на начальный момент времени у нас есть параметры, обладающие неопределённостями в каком-то диапазоне, но сами виды распределений (какие значения параметра наиболее вероятны или менее вероятны) мы не знаем.

Из-за отсутствия информации о виде распределения обычно задают равномерное распределение возможных значений параметра в заданном диапазоне.
Далее проводится расчёт модели со значениями параметров в рассматриваемых диапазонах, и мы видим, что какие-то из результатов расчётов не будут соответствовать фактическим замерам (даже с учётом допустимой погрешности).
Это позволит нам сузить диапазоны вариации исходных данных и уточнить виды распределения.
Например, от равномерных распределений можем прийти к нормальным или треугольным распределениям.

Другими словами, адаптацию можно рассматривать в качестве проверки, в каких диапазонах исходные данные (значения параметров) могут находиться и какие значения этих параметров наиболее вероятны.

\subsubsection{Алгоритм проведения автоадаптации}

\includegraphics[width=\textwidth, page=158]{Kurs_OsnovyGDM_Kai_774_gorodovSV_v6_0.pdf}

На этом слайде представлен алгоритм проведения автоматизированной адаптации.

Сначала мы производим расчёт базовой модели, выбираем диапазоны изменения параметров, которые обладают неопределённостью.
Делаем несколько расчётов со значениями параметров в этих диапазонах и смотрим, какие из этих параметров оказывают наибольшее влияние на результат расчёта и какие из диапазонов мы можем сузить по результатам этих первых нескольких расчётов.

Таким образом, дальше мы сокращаем количество параметров, которые будут участвовать в оптимизации (это делается для того, чтобы сократить количество необходимых расчётов, поскольку чем больше параметров будут участвовать, тем более многомерное пространство поиска решения у нас будет -- большее количество расчётов модели потребуется и большее время для того, чтобы оптимизационные алгоритмы сошлись).
Т.е. мы по результатам первых нескольких расчётов сокращаем количество параметров и сужаем их диапазоны.

Дальше задаём целевую функцию, задаём оптимизационные алгоритмы и запускаем на расчёт (идём в отпуск или на обед -- смотря сколько времени считается модель).
Будет проведено около нескольких десятков или сотен расчётов для того, чтобы целевая функция устремилась к нашим минимальным значениям.

\subsubsection{Программы автоадаптации}

\includegraphics[width=\textwidth, page=159]{Kurs_OsnovyGDM_Kai_774_gorodovSV_v6_0.pdf}

Сейчас программы автоадаптации используются в качестве вспомогательного инструмента, чтобы быстрее найти решение / сузить диапазоны поиска значений параметров (а для абсолютно полной автоматической адаптации такие программы обычно сейчас не используются).

После проведения автоадаптации всё равно необходимо проводить дополнительный анализ на физичность / геологичность найденных сочетаний параметров.
Другими словами, на данный момент программы автоадаптации решают чисто оптимизационную задачу и не способны самостоятельно учесть всевозможные нефизичности найденных сочетаний параметров.
\\

Но есть проекты когнитивной автоадаптации, в которых пытаются контролировать физическую / геологическую обоснованность всех параметров и их сочетаний в автоматическом режиме.

Адаптация модели является самым времязатратным периодом работы с моделью (может занимать несколько месяцев работы до окончательной настройки модели).

\subsubsection{Критерии адаптации}

\includegraphics[width=\textwidth, page=160]{Kurs_OsnovyGDM_Kai_774_gorodovSV_v6_0.pdf}

Представлены критерии адаптации в случае, если смотрим в целом по всему месторождению (сумму по всем скважинам).

По дебитам воды, нефти, жидкости, газа ошибка не должна превышать 10\%.

По накопленной добыче воды, нефти, жидкости, газа ошибка не должна превышать 5\%.

По пластовым давлениям по регламенту ошибка не должна превышать 25\%, но обычно стараются добиться меньшего диапазона вариации.

\includegraphics[width=\textwidth, page=161]{Kurs_OsnovyGDM_Kai_774_gorodovSV_v6_0.pdf}

Представлены критерии адаптации в случае, если смотрим отдельно по скважинам.

Строятся кроссплоты расчёт-факт (отмечаются все скважины) по накопленной добыче нефти на определённую дату.

Допустимые ошибки: 20\% по нефти; 20\% по воде; 25\% по давлению; 5\% по жидкости; 5\% по закачке.

Симулятор в первую очередь ориентируется на добычу по жидкости и на закачку, поэтому сильных отклонений по жидкости и по закачке (как правило) не бывает.
Все отклонения, как правило, бывают связаны именно с распределением флюидов (нефти, воды, газа) в пределах той жидкости, которую скважина добыла.

\includegraphics[width=\textwidth, page=162]{Kurs_OsnovyGDM_Kai_774_gorodovSV_v6_0.pdf}

Иногда строят такой график, который показывает, какая доля скважин обеспечивает какую долю накопленной добычи нефти и какую погрешность расчёт-факт при этом имеет.
\\

Чтение представленного на слайде графика: видим, что доля фонда скважин с относительной погрешностью расчёт-факт, не превышающей 20\%, составляет около 63\%. И при этом эти 63\% скважин обеспечивают накопленную добычу нефти чуть больше 80\%.\\

Принцип Паретто: 20\% усилий дают 80\% результата; чтобы получить оставшиеся 20\% результата приходится приложить 80\% усилий.\\

Для задач, где не требуется настройка каждой скважины (необходимо понимать только поведение месторождения в целом, например, для проектно-технологических документов), обычно требуют настройку в пределах 20\% только для тех скважин, которые суммарно дают 80\% накопленной добычи по месторождению. 

\subsubsection{"<Запрещённые"> и нежелательные приёмы адаптации}

\includegraphics[width=\textwidth, page=163]{Kurs_OsnovyGDM_Kai_774_gorodovSV_v6_0.pdf}

Стоит помнить, что целью модели является прогноз дальнейшей динамики работы месторождения при различных сценариях.

Если будем использовать некорректные методы адаптации, то это сделает модель непригодной для дальнейших прогнозов (т.к. непонятно, что в действительности в межскважинном пространстве происходит).

Все регионы, которые вводятся в модель, должны быть обоснованы: например, есть несколько пластов, у них свойства отличаются, поэтому задали регионы. А внутри одного пласта достаточно сложно обосновать, почему используем разные регионы (можно, конечно, если есть какие-то несообщающиеся блоки).
Если есть разные петротипы, какие-то гидравлические единицы потока или фации, то к ним можно привязать разные регионы ОФП.

Но не нужно к каждой скважине присваивать свой регион ОФП!
\\

Нельзя рисовать необоснованные барьеры.
Например, в случае, когда в модели к какой-то скважине прорывается слишком много воды, нельзя ограничивать эту воду выдуманным барьером.

Нужно вспомнить анализ источников обводнения (откуда и какие скважины обводняются), который мы проводили до создания модели, и в соответствии с этим анализом стараться настраиваться (пытаться разумными способами переориентировать потоки).
\\

Историю работы скважин тоже не нужно менять (совпадение графиков будет красивым, но модель будет липовая).
\\

Расположение скважин тоже не нужно менять.
Ошибки в расположении, конечно, могут быть, но они обычно небольшие.

\subsection{Упражнение 5. Расчёт моделей с разными наборами исходных данных}

\includegraphics[width=\textwidth, page=164]{Kurs_OsnovyGDM_Kai_774_gorodovSV_v6_0.pdf}

\subsection{Упражнение 6. Адаптация ГДМ}

\includegraphics[width=\textwidth, page=165]{Kurs_OsnovyGDM_Kai_774_gorodovSV_v6_0.pdf}

\subsection{Упражнение 6. Адаптация ГДМ. Обсуждение результатов}

\includegraphics[width=\textwidth, page=166]{Kurs_OsnovyGDM_Kai_774_gorodovSV_v6_0.pdf}

\subsection{Групповая дискуссия}

\includegraphics[width=\textwidth, page=167]{Kurs_OsnovyGDM_Kai_774_gorodovSV_v6_0.pdf}

Сейчас 10:47. Давайте до 11:00 сделаем перерыв.

\subsection{Инструменты для оптимизации разработки месторождения}

\includegraphics[width=\textwidth, page=168]{Kurs_OsnovyGDM_Kai_774_gorodovSV_v6_0.pdf}

Давайте продолжать.

После создания и настройки модели её можно использовать для подбора вариантов разработки, оптимизации текущей разработки месторождения.

Для этого нужно провести анализ этой настроенной модели, получить из неё карты остаточных подвижных запасов нефти, карты пластового давления, карты проницаемости (которые получились после настройки, мы же её варьировали при адаптации).\\

Мы смотрим, где в пласте осталась подвижная нефть, почему на там осталась (неэффективная работа системы ППД или низкая проницаемость, или просто неразбуренная область), и в соответствии с этим принимаем решение, что нужно сделать, чтобы оставшуюся подвижную нефть добыть.\\

С помощью линий тока можем оценить, насколько эффективно работает каждая из нагнетательных скважин (насколько эффективно она вытесняет нефть, воду; насколько эффективно поддерживает давление; какая доля добычи нефти и воды обеспечивается закачкой данной нагнетательной скважины).

\subsection{Линии тока}

\includegraphics[width=\textwidth, page=169]{Kurs_OsnovyGDM_Kai_774_gorodovSV_v6_0.pdf}

На каждый шаг времени распределение линий тока может меняться, так как распределение давления в пласте меняется (изменились давления $\Rightarrow$ изменились градиенты давления $\Rightarrow$ изменились направления потоков).

\subsection{Оптимизация ППД на основе матриц дренирования}

\includegraphics[width=\textwidth, page=170]{Kurs_OsnovyGDM_Kai_774_gorodovSV_v6_0.pdf}

Здесь рассказывается, как можно рассчитать эффективность работы нагнетательных скважин.

\includegraphics[width=\textwidth, page=171]{Kurs_OsnovyGDM_Kai_774_gorodovSV_v6_0.pdf}

\subsection{Прогнозные расчёты. Анализ таблиц дренирования}

\includegraphics[width=\textwidth, page=172]{Kurs_OsnovyGDM_Kai_774_gorodovSV_v6_0.pdf}

В т-Навигаторе тоже есть функционал анализа линий тока.

\includegraphics[width=\textwidth, page=173]{Kurs_OsnovyGDM_Kai_774_gorodovSV_v6_0.pdf}

Таблицу дренирования можно группировать по добывающим или нагнетательным скважинам: смотреть сколько воды закачано в нагнетательные скважины и какой приток нефти/воды/жидкости обеспечен данной нагнетательной скважиной.
\\

Анализ таблиц дренирования проводится с целью оптимизации системы ППД (системы заводнения).

\subsection{Подготовка и проведение прогнозных расчётов}

\includegraphics[width=\textwidth, page=174]{Kurs_OsnovyGDM_Kai_774_gorodovSV_v6_0.pdf}

Как считать прогнозы после того, как мы придумали, куда располагать новые скважины, новые ЗБС (другими словами, придумали, как оптимизировать систему заводнения)?

Итак, хотим считать прогнозные варианты и хотим оценить, насколько наше придуманное решение эффективно по сравнению с базовым вариантом.

Если у месторождения большая история разработки, то, чтобы каждый раз (на каждый прогноз) её не пересчитывать, можно делать рестарты, т.е. начинать расчёт с того момента, когда у нас закончилась история, т.е. мы считаем на будущее и каждый раз делать расчёт предыдущей истории не обязательно.

Есть 2 способа рестарта: один гибкий, другой быстрый.
Гибкий позволяет вносить какие-то изменения в базовую модель, но при этом требует перечитывания всей модели, пересчёта каких-то данных (например, проводимости) -- это занимает некоторое время, т.е. если модель большая, она достаточно долго может инициализироваться.

Если хочется сделать быстрый рестарт, то такая возможность тоже есть, но есть ограничение, что никаких изменений в базовую модель внести нельзя (они не будут отражаться на прогнозе).

\includegraphics[width=\textwidth, page=175]{Kurs_OsnovyGDM_Kai_774_gorodovSV_v6_0.pdf}

Здесь рассказывается, какие ключевые слова используются для гибкого рестарта в симуляторе ECLIPSE.

А в т-Навигаторе это всё делается с помощью интерфейса: просто ставятся галочки, на какие шаги нужно записывать результаты расчёта.
\\

После того, как расчёт базовой модели завершён, мы создаём data-файл, в который копируем базовую модель, далее в этом файле в секции SOLUTION удаляем всё, что связано с инициализацией и аквиферами, вставляем RESTART с именем базовой модели и номером шага, с которого будет начинаться расчёт.
В секцию SCHEDULE записываем ключевое слово SKIPREST -- тогда симулятор ECLIPSE будет пропускать все шаги до начала прогнозного расчёта.

Дальше возможно добавить дополнительные даты и какие-то модификации в секцию SCHEDULE, а именно что вносим на прогноз (какие-то новые скважины, новые параметры работы скважин, перевод каких-то скважин в ППД и так далее).

\includegraphics[width=\textwidth, page=176]{Kurs_OsnovyGDM_Kai_774_gorodovSV_v6_0.pdf}

В быстром рестарте примерно то же самое, только в начале базовой модели нужно прописать ключевое слово SAVE, и тогда будет создан специальный файл, который будет подключаться ключевым словом LOAD в data-файле.

\subsection{Создание рестартов из GUI tNavigator}

\includegraphics[width=\textwidth, page=177]{Kurs_OsnovyGDM_Kai_774_gorodovSV_v6_0.pdf}

А в т-Навигаторе просто заходим в меню и выбираем "<Создать прогноз...">

Всплывает окно, в котором просто указываем имя модели, с какого шага нужно начать прогноз, на какой период, с какой периодичностью записывать результаты (месяц/день/год и так далее) и какие контрольные параметры на скважинах установить.
Затем нажимаем кнопку OK, и автоматически создастся data-файл и SCHEDULE с режимами работы скважин на прогноз, т.е. если мы просто после настройки скважины нажмём "<Создать прогноз..."> и поставим галочки, что скважины будут работать с текущим забойным давлением, то фактически это будет такой базовый вариант (т.е. как месторождение будет работать, если мы будем эксплуатировать скважины на забойном давлении на последнюю дату, т.е. никаких изменений не вносим).
С этим базовым вариантом будем потом сравнивать модели, в которые внесены модификации, чтобы оценить их эффективность.

Т.е. мы создаём базовый расчёт и потом уже какие-то модификационные другие дополнительные модели, эффективность которых хотим проверить.

\subsection{Вырезание сектора}

\includegraphics[width=\textwidth, page=178]{Kurs_OsnovyGDM_Kai_774_gorodovSV_v6_0.pdf}

Вырезание сектора полезно в том случае, если ввод новых скважин производится только на небольшом участке модели.
Тогда можем вырезать этот участок, проводить расчёты только на нём и сравнивать с этим же участком базовой модели с целью оценки эффективности ГТМ.
\\

Есть ещё ситуация, когда может быть полезно вырезание сектора: например, имеется большая модель и есть несколько человек, которые могут заниматься настройкой этой модели.
Чтобы каждый не таскал всю эту модель, её можно разделить на несколько частей, каждый будет свою часть считать отдельно, и периодически эту модель нужно будет сшивать.

В т-Навигаторе есть кнопка "<Разрезать">, но перед этим нужно сначала её раскрасить (нарисовать регионы, по которым эту модель будем разрезать; грубо говоря, присвоить ячейкам модели номер региона, к которому данная ячейка принадлежит).
После разрезания в папке с исходной моделью будут созданы папки с секторными моделями.
После этого нужно ещё записать граничные условия в эти сектора, т.е. открыть модель с расширением .patterns, рассчитать её и тогда граничные условия будут автоматически записаны в эти сектора.
После этого можно каждый сектор отдельно использовать.

Чтобы синхронизировать друг с другом изменения, вносимые в каждый из секторов (т.е. свести всё в единую модель), нужно опять в эту же папку все эти сектора сложить, снова открыть модель с расширением .patterns, пересчитать её, и тогда все изменения будут автоматически перенесены в модель, и граничные условия на секторах пересчитаны.
\\

Вопрос: переиндексируются ли ячейки при обрезании модели?

Нет, симулятор просто остальные ячейки (за пределами рассматриваемого сектора) делает неактивными.
Т.е. модель-сектор может по-прежнему инициализироваться достаточно долго, но считаться будет быстро.

\subsection{Подготовка и проведение прогнозных расчётов}

\includegraphics[width=\textwidth, page=179]{Kurs_OsnovyGDM_Kai_774_gorodovSV_v6_0.pdf}

На данном слайде указано, как задавать контроль по скважинам в случае проведения прогнозных расчётов.
Когда адаптировали на историю, были ключевые слова WCONHIST для добывающих скважин и WCONINJH для нагнетательных скважин.
А здесь (на прогноз) для добывающих скважин используется ключевое слово WCONPROD, для нагнетательных скважин используется ключевое слово WCONINJE.

Можно контроль на скважинах задавать по дебитам, по забойному давлению; на нагнетательных соответственно -- по приёмистости.
Также можно осуществлять контроль по устьевому давлению, но тогда нужно будет ещё добавить VFP-таблицы, в которых описано, как меняется давление по стволу скважины при различных режимах течения, т.е. грубо говоря какие потери давления будут по стволу скважины (это нужно для того, чтобы симулятор пересчитал забойное давление; от устьевого до забойного пересчитал потери).
Есть специальные программы, в которых формируются VFP-таблицы, т.е. можно их заранее создать.
\\

Можно также указывать групповой контроль GCONPROD.
Например, это может быть полезно, если у нас заданы какие-то ограничения по отборам на дожимной насосной станции ДНС.
Т.е. эксплуатируем скважины так: сколько притечёт, столько притечёт, но сверху есть ограничение, что группа скважин не может добывать больше какой-то величины (соответственно это можно в групповом контроле указать).

Также и для нагнетательных скважин можно указать групповой контроль GCONINJE.
Здесь есть ещё опция compens (обеспечить компенсацию), т.е. можно задать, чтобы какая-то группа нагнетательных скважин обеспечивала компенсацию по какой-то группе добывающих скважин.
Можно также сделать и по месторождению, что все нагнетательные скважины должны нагнетать столько, чтобы обеспечить 100\% (или 120\% или сколько укажете) компенсацию (тогда просто режим эксплуатации нагнетательных скважин будет подбираться так, чтобы эту компенсацию обеспечить -- вручную не нужно будет подбирать -- всё будет сделано автоматически).

\subsection{Прогнозные расчёты}

\subsubsection{Расстановка ВС}

\includegraphics[width=\textwidth, page=180]{Kurs_OsnovyGDM_Kai_774_gorodovSV_v6_0.pdf}

Показан полезный функционал: как добавлять скважину в интерфейсе т-Навигатора.

ЛКМ = левая кнопка мыши.

Просто зажимаем клавишу Alt и левой кнопкой мыши в 2D или 3D обзоре щёлкаем на ячейки, куда хотим добавить скважину.
Всплывёт окно, в котором нужно будет задать параметры вскрытия (по умолчанию все ячейки, которые отображаются в окне, будут вскрыты; но можно удалить те интервалы, которые не хотим вскрывать).

Во вкладке "<Управление"> указываем тип скважины (добывающая или нагнетательная), режим работы (если она на забойном давлении, то указываете забойное давление; если на расходе жидкости, то указываете расход).

Во вкладке "<Эконом. пределы"> можно указать экономические ограничения, когда скважина будет останавливаться (в проектных документах обычно ставят ограничение 99\% обводнённости или 1 м$^3$/сут по жидкости).

После этого нажимаем "<Добавить скважину"> и она будет в модель добавлена, и её параметры прописаны в отдельном файле, который сохранён в папке USER.

Если ещё не начали расчёт, то можно из интерфейса щёлкнуть на этой скважине правой кнопкой мыши и всплывёт окно с предложением редактировать или удалить скважину.
Если же расчёт уже начали, то удалить скважину с помощью интерфейса будет нельзя, но можно будет зайти в папку USER и вручную в текстовом файле параметры скважины изменить и модель просто заново перечитать.

Если хотим, чтобы скважина вскрывала только какие-то определённые ячейки (например, ячейки с насыщенностью больше какого-то значения), то можно заранее настроить фильтр в т-Навигаторе, отфильтровать ячейки по этому условию и потом уже только расставлять скважины.
Тогда скважина будет вскрывать только эти отфильтрованные ячейки (и вручную не нужно будет их удалять, прореживать и так далее -- очень удобно).

\subsubsection{Расстановка ГС/ННС}

\includegraphics[width=\textwidth, page=181]{Kurs_OsnovyGDM_Kai_774_gorodovSV_v6_0.pdf}

То же самое для горизонтальных скважин.

Но теперь можно провести её прямо по горизонтальным ячейкам, через которые она должна пройти.
Для этого строим разрез (сечение или профиль -- зелёная кнопка на правой боковой панели) по ячейкам, где хотим провести горизонтальную скважину.
Потом этот профиль открываем.
С зажатой клавишой Alt выбираем ячейку входа скважины в пласт.
Появляется окно с настройками вскрытия скважиной.
В нём ставим галочку, что скважина наклонная.
Окно не закрываем.
И дальше, держа нажатой клавишу Alt, продолжаем щёлкать мышкой по тем ячейкам, через которые мы хотим, чтобы горизонтальный ствол прошёл.
После того, как прощёлкали можно указать какие-то параметры работы скважины, экономические ограничения и так далее.
Нажимаем "<Добавить скважину">, и в папке USER тоже будет создан файл, в котором прописаны параметры этой скважины и её траектория.
Эту траекторию можно потом передать буровикам, чтобы они проверили, возможно ли под таким углом туда скважину пробурить.
Мы можем нащёлкать как угодно (хоть спиралеобразную скважину) в модели, но не факт, что её возможно будет пробурить.
В общем, есть ограничения на скорость набора угла, т.е. с какой скоростью скважина может искривляться (обычно 3 градуса на 100 метров).

\subsubsection{Расстановка скважин по сетке}

\includegraphics[width=\textwidth, page=182]{Kurs_OsnovyGDM_Kai_774_gorodovSV_v6_0.pdf}

Можно скважины расставлять по сетке.
Для этого на боковой панели выбираем "<Действия для скважин">, "<Шаблон скважин"> и настраиваем этот шаблон.

Здесь можно расставить добывающие скважины, нагнетательные скважины, сам шаблон можно побольше сделать (на рисунке показывается, как в ячейках будут располагаться скважины, когда мы будем щёлкать).
Можно задать префикс группы, который будет перед названием скважин указываться.

После того, как настроили шаблон, нажали OK: дальше, когда будем щёлкать на ячейки левой кнопкой мыши в 2D или 3D обзоре с зажатой клавишей Alt, будет появляться не одна скважина, а набор скважин, который мы задали в шаблоне.

\subsubsection{Моделирование ГТМ}

\includegraphics[width=\textwidth, page=183]{Kurs_OsnovyGDM_Kai_774_gorodovSV_v6_0.pdf}

Моделировать ГРП можно с помощью плагина EasyFrac для Petrel.
В нём задаются параметры трещины и после работы плагина на выходе мы получаем SCHEDULE секцию с ключевыми словами COMPDAT, в которых трещина имитируется дополнительными перфорациями, т.е. как будто бы скважина проперфорирована в тех ячейках, через которые проходит трещина.

Можно также задать вырождение трещины, т.е. то, что трещина постепенно ухудшает свои свойства.
Это тоже в автоматическом режиме через плагин EasyFrac формируются ключевые слова ACTIONX и WPIMULT, которые со временем ухудшают свойства трещины.

\includegraphics[width=\textwidth, page=184]{Kurs_OsnovyGDM_Kai_774_gorodovSV_v6_0.pdf}

В т-Навигаторе есть возможность задать трещины через интерфейс.
Здесь просто щёлкаем на скважине правой клавишой мыши, нажимаем редактировать, дальше во всплывающем окне просто щёлкаем кнопку ГРП и указываем параметры ГРП.
И тогда тоже будут созданы виртуальные перфорации с помощью ключевых слов WFRAC и WFRACP.

Но есть проблема (мы её обнаружили недавно), что это ключевое слово WFRACP не очень точно позволяет описать трещину, особенно для низкопроницаемых коллекторов.
Для высокопроницаемых коллекторов вроде неплохо работает, а для низкопроницаемых приходится ставить большой множитель на продуктивность трещины, чтобы она воспроизводила фактическую историю.
Т.е. если на скважине провести ГДИС, то чтобы воспроизвести историю замеров давления, придётся в ключевом слове WFRACP подбирать множитель mult.

Там сейчас есть другие ключевые слова, но их через интерфейс можно по-моему только в дизайнере моделей добавить (или вручную в текстовом файле).
В обычном графическом интерфейсе (GUI) не получится.

\subsubsection{Задание ГРП}

\includegraphics[width=\textwidth, page=185]{Kurs_OsnovyGDM_Kai_774_gorodovSV_v6_0.pdf}

Здесь как раз показано, как через ключевое слово WFRACP добавлять трещины.
Когда мы щёлкнули на скважине, всплывёт такое окно и мы выбираем те ячейки через которые трещина пройдёт, затем нажимаем "<Сгенерировать трещины..."> и указываем параметры, после этого опять таки будет создан дополнительный файл, в котором прописаны параметры трещины.

\subsubsection{Оценка адекватности результатов}

\includegraphics[width=\textwidth, page=186]{Kurs_OsnovyGDM_Kai_774_gorodovSV_v6_0.pdf}

После того, как рассчитали прогноз, необходимо оценить, насколько этот расчёт получился адекватным.
Можно сопоставить прогнозные дебиты и обводнённости новых скважин с работой окружения, т.е. если скважина находится в таких же геологических условиях, то она не должна в разы (или десятки раз) отличаться по дебитам от окружения.
Если такое будет в модели, то сразу возникнет вопрос: почему так произошло, почему другие скважины, которые мы пробурили ранее, не дают таких дебитов?
Действительно ли возможно достичь таких показателей на новой скважине?
Поскольку все эти параметры идут в бизнес-план и потом за этот бизнес-план главному геологу приходится отчитываться, поэтому сразу возникнет вопрос: стоит ли такие большие параметры показывать?

В принципе бывают такие случаи: если мы разбуриваем какую-то новую зону и в ней могут быть совсем другие свойства, т.е. в принципе скважина может работать и с более высокими дебитами, чем другие.
В этом ничего страшного нет, просто нужно ещё раз проверить и убедиться, что в модели нет каких-то ошибок, случайных погрешностей, которые привели к таким аномально высоким значениям.
\\

Ещё необходимо отслеживать отсутствие скачков дебитов на первый шаг прогноза на графиках для "<старых"> скважин.
Это может быть следствием того, что скважины не настроены по продуктивности и тогда, если мы поставим контроль по забойному давлению, то старые скважины начнут работать не с той продуктивностью, которая была по факту, а с какой-то расчётной, и будет наблюдаться скачок дебита.
Это такой индикатор настройки продуктивности на последний шаг расчёта до прогноза.
\\

Также необходимо посмотреть, как меняется добыча по скважинам со временем: если наблюдается рост дебитов, то нужно проверить, действительно ли это так.
Почему при стабильном забойном давлении наблюдается рост? Может быть мы слишком много качаем. Возможно ли это в реальности на месторождении?

Либо может быть слишком сильное падение добычи, здесь тоже нужно сопоставлять со скважинами в аналогичных условиях (как они себя вели); в общем, ещё раз убедиться, что все параметры работы как прогнозных скважин, так и скважин окружения заданы правильно.
\\

При групповом контроле необходимо следить, чтобы было адекватное распределение дебитов.
Потому что бывает такое, что дебиты распределяются просто пропорционально продуктивности и какие-то скважины могут иметь высокую продуктивность (значительно отличающуюся от других), соответственно тогда скважины с низкой продуктивностью будут работать на каких-то низких дебитах, а в реальности никто такие скважины (с такими низкими дебитами) в эксплуатацию вводить не будет.
Если у скважин дебит ниже экономически рентабельного, то просто их не будут бурить, поэтому нужно за этим следить, чтобы не было того, что одни скважины пережимают другие, если мы управляем с помощью группового контроля.

\subsubsection{Анализ чувствительности}

\includegraphics[width=\textwidth, page=187]{Kurs_OsnovyGDM_Kai_774_gorodovSV_v6_0.pdf}

Здесь показаны графики, как после расчёта прогноза просто проводить анализ чувствительности к каким-то параметрам неопределённости.
Мы рассчитали профиля и потом оцениваем, насколько наше решение получилось устойчивым, т.е. варьируем какие-то параметры модели и с помощью "<Spider"> диаграммы или графика "<торнадо"> показываем, насколько сильно неопределённые параметры модели могут влиять на результат прогноза.

Это представлен случай, если мы варьируем по одному параметру модели за раз, но на самом деле у нас же есть взаимовлияние параметров, поэтому нужно придумать, как это оценить.

\includegraphics[width=\textwidth, page=188]{Kurs_OsnovyGDM_Kai_774_gorodovSV_v6_0.pdf}

Придуманы методы дизайна эксперимента (DoE), которые позволяют за небольшое число расчётов оценить разные сочетания варьируемых параметров.
Но возникает вопрос: как анализировать результаты?
У нас получилось несколько моделей.
Один из способов анализа -- это построить карты вероятности превышения выбранным параметром порогового значения.
Допустим мы хотим оценить, будет ли у нас прорываться газ в скважину, пробуренную в какую-то точку.
Мы рассчитали несколько моделей и дальше просто делим количество реализаций, в которых рассматриваемый параметр превышает пороговое значение, к общему числу реализаций.
И таким образом получается некое подобие вероятности того, например, что в данной точке будет газ.

На рисунке показана карта вероятности того, что газонасыщенность будет больше 50\%, т.е. мы следим за выполнением/невыполнением условия (например, больше ли 50\% газонасыщенность) в каждой ячейке для каждой из моделей и делим количество моделей, в которых условие выполняется, на общее количество.
Получаем что-то подобное вероятности.

\subsection{Упражнение 7. Прогнозные расчёты}

\includegraphics[width=\textwidth, page=189]{Kurs_OsnovyGDM_Kai_774_gorodovSV_v6_0.pdf}

\subsection{Регламенты по созданию ГДМ}

\includegraphics[width=\textwidth, page=190]{Kurs_OsnovyGDM_Kai_774_gorodovSV_v6_0.pdf}

Дальше небольшой рассказ про регламенты.
\\

Есть регламент по созданию постоянно действующих геолого-технологических моделей.
Можно его в интернете найти.
\\

Есть положение от ГКЗ и регламент от ЦКР.
Тоже можно в интернете почитать, какие критерии к моделям применяются для того, чтобы считать их адекватными.

\subsection{Сравнительная характеристика ПО для ГДМ}

\includegraphics[width=\textwidth, page=191]{Kurs_OsnovyGDM_Kai_774_gorodovSV_v6_0.pdf}

И ещё 1 вопрос.

Есть много разных симуляторов, они могут выдавать отличающиеся результаты.
Какой из этих результатов правильный?

У тех, кто создавал симуляторы, в самом начале тоже возникали эти вопросы, и придумали ряд сравнительных тестов для оценки результатов расчёта с некоторым набором эталонных расчётов.

Есть так называемые тесты SPE, ЦКР тоже придумал свои тесты, и в регламенте говорится, что симулятор обязательно должен пройти тесты SPE1 и SPE7, чтобы его результаты расчёта можно было принимать для проектных документов.

\subsubsection{Тесты SPE}

\includegraphics[width=\textwidth, page=192]{Kurs_OsnovyGDM_Kai_774_gorodovSV_v6_0.pdf}

Здесь представлен список тестов SPE, их автор и год, когда они были предложены.
Видно, что большинство тестов предложены на заре гидродинамического моделирования, когда ещё особо не было больших моделей месторождений.

Т.е. в тестах просто проверяется работа различных опций, корректно ли они в данном симуляторе считаются.

\includegraphics[width=\textwidth, page=193]{Kurs_OsnovyGDM_Kai_774_gorodovSV_v6_0.pdf}

Здесь большая таблица с мелкими буквами.
Показано, какие из симуляторов участвовали в расчётах тестов.
Результаты расчётов этих симуляторов стали эталонными (с этими результатами сравниваются расчёты симуляторов, которые появляются позже).

Какие-то из этих симуляторов уже не существуют, какие-то вообще представлены без названия (noname; просто собственные разработки некоторых нефтяных компаний)

\includegraphics[width=\textwidth, page=194]{Kurs_OsnovyGDM_Kai_774_gorodovSV_v6_0.pdf}

Здесь показаны графики различных параметров по результатам расчётов этих SPE тестов.

\subsubsection{О недостатках тестов SPE}

\includegraphics[width=\textwidth, page=195]{Kurs_OsnovyGDM_Kai_774_gorodovSV_v6_0.pdf}

Здесь указаны недостатки тестов SPE.
\\

Точность эталонных симуляторов остаётся неизвестной, т.е. нельзя сказать, что те симуляторы, которые использовались в 80-х годах, более точные, чем те, которые появились позже (ведь теперь есть новые более корректные методики).

\includegraphics[width=\textwidth, page=196]{Kurs_OsnovyGDM_Kai_774_gorodovSV_v6_0.pdf}

Новые симуляторы могут включать в себя новые (усовершенствованные) способы аппроксимации продуктивности, течения жидкости и газов в стволе скважины, потерь давления из-за неполного вскрытия и т.д., поэтому фактически новые симуляторы могут более точно рассчитывать параметры работы скважин, но эти результаты могут не совпадать с эталонными (и это не значит, что это плохо, просто эталон устарел).

\subsubsection{Тесты центральной комиссии по разработке (ЦКР)}

\includegraphics[width=\textwidth, page=197]{Kurs_OsnovyGDM_Kai_774_gorodovSV_v6_0.pdf}

В ЦКР тоже придумали свой набор тестов для проверки различных опций расчёта симулятора.
Но здесь сравнение результатов производится не с другими симуляторами, а с известными точными аналитическими решениями или с эталонными физическими решениями, которые получены в лаборатории.
Это позволяет исключить предвзятость, что какие-то симуляторы считаются эталонными, а какие-то неэталонными.

\subsection{Основные проблемы моделирования}

\includegraphics[width=\textwidth, page=198]{Kurs_OsnovyGDM_Kai_774_gorodovSV_v6_0.pdf}

\subsection{Выводы}

\includegraphics[width=\textwidth, page=199]{Kurs_OsnovyGDM_Kai_774_gorodovSV_v6_0.pdf}

При создании модели нужно исходить из принципов целесообразной экономичности, подбирать тип моделей, исходя из той задачи, которая решается, из того количества данных, которые имеются, и из доступных программных продуктов, которые позволяют моделировать месторождение.

Перед созданием модели необходимо провести анализ разработки месторождения, верификацию данных.

На каждом этапе создания модели необходимо валидировать результаты на данные фактической эксплуатации так, чтобы в конце концов модель получилась непротиворечивой.

Методы, используемые при настройке модели на историю разработки, должны быть обоснованы с точки зрения геологии и физики пласта.
Поскольку мы решаем обратную задачу, то в пределах допустимого диапазона неопределённости существует множество правильных вариантов настройки модели.

В итоге, результаты моделирования должны соответствовать критериям точности и адекватности.

\subsection{Термины, определения, сокращения}

\includegraphics[width=\textwidth, page=202]{Kurs_OsnovyGDM_Kai_774_gorodovSV_v6_0.pdf}

\includegraphics[width=\textwidth, page=203]{Kurs_OsnovyGDM_Kai_774_gorodovSV_v6_0.pdf}

\includegraphics[width=\textwidth, page=204]{Kurs_OsnovyGDM_Kai_774_gorodovSV_v6_0.pdf}

\includegraphics[width=\textwidth, page=205]{Kurs_OsnovyGDM_Kai_774_gorodovSV_v6_0.pdf}

\subsection{Список литературы и источников информации}

\includegraphics[width=\textwidth, page=206]{Kurs_OsnovyGDM_Kai_774_gorodovSV_v6_0.pdf}


\end{document}
