\documentclass[main.tex]{subfiles}

\begin{document}


\section{Практика 14.09.2022 (Базыров И.Ш.)}

\subsection{Закон Дарси}

В XIX веке наука во Франции была передовой.
В 1856 году в работе Дарси "<Les fontaines publiques de la ville de Dijon. Paris 1856"> (Общественные фонтаны города Дижон. Париж 1856) опубликованы результаты опытов по фильтрации воды в песке.
Опубликован закон, связывающий скорость фильтрации жидкости в пористой среде с градиентом давления.
Является основополагающим законом, который используется в гидродинамике.

До Дарси считалось, что поток в трубе не зависит от диаметра трубы и шероховатости её стенок. Это большое заблуждение, которое опровергли Дарси и Вейсбах.
На самом деле, потери напора в трубе связаны со скоростью в квадрате и есть коэффициент местного сопротивления (коэффициент потерь), который показывает изменение потерей напора на всём протяжении трубы (эти потери прямо пропорциональны длине трубы и обратно пропорциональны диаметру трубы).

Закон Дарси применим для фильтрации жидкостей, подчиняющихся закону вязкого трения Ньютона (закону Навье-Стокса).
Для фильтрации неньютоновских жидкостей (например, некоторых нефтей) связь между градиентом давления и скоростью фильтрации может быть нелинейной или вообще неалгебраической (например, дифференциальной).

Для ньютоновских жидкостей область применения закона Дарси ограничивается малыми скоростями фильтрации (числа Рейнольдса, рассчитанные по характерному размеру пор, меньше или порядка единицы).
При больших скоростях зависимость между градиентом давления и скоростью фильтрации нелинейна (хорошее совпадение с экспериментальными данными даёт квадратичная зависимость -- закон фильтрации Форхгеймера).

Основные допущения закона Дарси:
\begin{itemize}
\item постоянный дебит
\item ламинарное течение
\item гомогенная среда фильтрации
\item поровое пространство насыщенно одной фазой
\item отсутствие химического взаимодействия между породой и флюидом
\end{itemize}

\subsubsection{Линейное течение}


\subsubsection{Радиальное течение. Формула Дюпюи}

Дюпюи решил дифференциальное уравнение для случая границы в виде цилиндрической области (для радиального режима течения).

\beq
\frac{Q}{A}=\frac{k}{\mu}\frac{dP}{dx}\Rightarrow\frac{Q}{2\pi h}\int\limits_{r_w}^{r_e}{\frac{dr}{r}}=\frac{k}{\mu}\int\limits_{P_w}^{P_e}{dp}\Rightarrow Q=\frac{2\pi kh}{\mu}\frac{P_e-P_w}{\ln{\left(\dfrac{r_e}{r_w}\right)}}
\eeq

\subsection{Скин-фактор}

Для корректной оценки притока (калибровки модели к реальным данным) необходимо также учесть дополнительный перепад давления в призабойной зоне, то есть скин-фактор:
\beq
S=\dfrac{\nabla P_s}{\dfrac{Q\mu}{2\pi kh}}
\eeq

\beq
P_{wf}=P_e-\frac{Q\mu}{2\pi kh}\left(\ln{\left(\frac{r_e}{r_w}\right)}+S\right)
\eeq
В дальнейшем скин-фактор используется инженерами для учёта не только перепада давления в призабойной зоне.


\subsection{Формула Дюпюи с учётом скин-эффекта}

\beq
Q=\frac{kh}{18,41\cdot\mu}\frac{P_e-P_w}{\ln{\left(\dfrac{r_e}{r_w}\right)}+S}
\eeq

\subsection{Определение дебита по формуле Дюпюи, анализ чувствительности}



\end{document}
