\begin{document}

\subsection{Что такое ремасштабирование (UpScaling)? Раскройте суть ремасштабирования}

\textbf{Ремасштабирование геомодели}

\includegraphics[width=\textwidth, page=59]{Kurs_OsnovyGDM_Kai_774_gorodovSV_v6_0.pdf}

Теперь переходим непосредственно к созданию модели.
После того, как геолог создал свою статичную геологическую модель и передал его гидродинамику, бывают случаи, когда эту модель необходимо сделать более грубой, когда эта модель слишком детальная и эта детальность излишняя (только забирает ресурсы и никакой информации особо не несёт для гидродинамика).
\\

Возникает необходимость сделать процедуру ремасштабирования, т.е. укрупнение модели (точнее укрупнение ячеек модели).
Эта процедура состоит из двух этапов: первое это UpGridding (ремасштабирование сетки; изменение размеров и количества ячеек) и второе это UpScaling (ремасштабирование свойств; т.е. после того, как мы получили большие ячейки, нам в эти большие ячейки нужно записать свойства, а именно осреднить значения свойств из маленьких ячеек и перенести эти осреднённые значения в большую ячейку).
\\

Возникает такой вопрос: до какой степени нам модель можно укрупнять и когда следует остановиться, т.е. когда мы начнём терять в качестве?

Можно построить такой график (представлен на слайде): по оси $x$ откладываем количество слоёв по вертикали (здесь мы говорим про укрупнение по вертикали), а по оси $y$ откладываем погрешность в расчёте накопленной добычи нефти (FOPT), накопленной добычи воды (FLPT).
Т.е. мы сравниваем, насколько результаты вот этих накопленных показателей отличаются от модели с исходной геологической сеткой.
При уменьшении количества слоёв ошибка постепенно растёт, и в какой-то момент на графике возникает перегиб (ошибка начинает возрастать более резко), т.е. это является неким косвенным признаком того, что мы начинаем в этот момент терять какую-то информацию о геологическом строении, о неоднородности.
И соответственно можем сказать, что в этой точке перегиба у нас оптимум, дальше которого модель укрупнять не следует (стоит остановиться).
Итак, первым способом выбора степени укрупнения является нахождение оптимума (точки перегиба на графике).

Второй способ -- это просто ограничиться каким-то значением ошибки. Почему-то обычно привязываются к каким-то круглым значениям (5, 10 или 20 \%).

Но лучше всё-таки использовать способ, который показывает, в какой момент мы начинаем терять информацию о строении месторождения.
\\

\textbf{Ремасштабирование структуры (upgridding)}

\includegraphics[width=\textwidth, page=60]{Kurs_OsnovyGDM_Kai_774_gorodovSV_v6_0.pdf}

По горизонтали рекомендация следующая: рекомендуется, чтобы между скважинами было не менее 3-5 ячеек, чтобы описать фильтрацию между скважинами.
Но с другой стороны, если у нас количество данных по месторождению ограничено, то стремиться к излишней детализации тоже не стоит, потому что точности не добавится (ведь новых данных нет), а время расчёта увеличится.
Но как минимум 3-5 ячеек всё таки желательно оставлять.
Справа на слайде приведён пример модели, которую передавали мне на экспертизу: и даже сложно различить, где какая скважина находится (здесь представлены горизонтальные скважины; крестиками помечены перфорации), настолько близко они расположены (в соседних ячейках), что, честно, даже непонятно, где какой ствол идёт; что таким образом пытались смоделировать тоже непонятно, естественно экспертизу такая модель не прошла и было рекомендовано сделать более детальную модель, чтобы между скважинами корректно воспроизводить процесс фильтрации.

\includegraphics[width=\textwidth, page=61]{Kurs_OsnovyGDM_Kai_774_gorodovSV_v6_0.pdf}

По вертикали желательно следить за тем, чтобы сохранялась расчленённость, нарезка слоёв и глинистые перемычки не пропадали.

Здесь на слайде тоже показан пример одной из моделей, которая проходила на экспертизу (левый рисунок на слайде).
Как делать не надо.
Видим, что в геологической модели и верхний, и нижний пласты достаточно расчленённые (на ГИС есть много белых глинистых перемычек).
А после укрупнения видим, что верхний пласт вообще склеился в один однородный массив, и в нижнем пласте тоже расчленённость пропала.

На правом рисунке на слайде есть пример, в котором глинистые перемычки пропали не внутри одного пласта, а даже между пластами, т.е. пласты, которые вообще не сообщаются гидродинамически, вдруг стали гидродинамически связанными.
Такое ужасное нарушение; модель стала совсем непригодной для расчётов, поскольку появилась вертикальная связь.
\\

\begin{figure}[H]
\textbf{Ремасштабирование свойств}

\includegraphics[width=\textwidth, page=62]{Kurs_OsnovyGDM_Kai_774_gorodovSV_v6_0.pdf}
\end{figure}

После того, как мы укрупнили ячейки, нужно в эти ячейки перенести свойства.

Вот у нас пример здесь: были такие маленькие ячейки, теперь эти маленькие ячейки стали одной большой ячейкой. У маленьких ячеек были разные значения какого-то свойства. Какое значение теперь занести в большую ячейку?

Есть рекомендованные методы расчёта средних свойств и очерёдности расчёта: сначала мы рассчитываем среднее значение песчанистости (рассчитывается как среднее арифметическое, но взвешенное по объёму ячеек; формула представлена на слайде -- ячейки имеющие больший объём вносят больший вклад), следующей по очереди осредняется пористость (здесь среднее арифметическое, взвешенное на эффективный объём; эффективный объём -- это песчанистость, умноженная на геометрический объём), далее осредняется насыщенность (среднее арифметическое, взвешенное на поровый объём; поровый объём -- это пористость, умноженная на эффективный объём).
Это всё делается для того, чтобы воспроизвести запасы для модели с укрупнённой сеткой.

Если же делать всё-таки гидродинамически уравновешенную модель, то насыщенность можно не осреднять, а рассчитать по гидростатическому равновесию.
Про это (про расчёт насыщенностей) дальше мы тоже будем говорить, когда будем обсуждать инициализацию модели.
\\

\begin{figure}[H]
\textbf{Ремасштабирование проницаемости}

\includegraphics[width=\textwidth, page=63]{Kurs_OsnovyGDM_Kai_774_gorodovSV_v6_0.pdf}
\end{figure}

Для проницаемости методы осреднения более разнообразны, поскольку проницаемость не является объёмной характеристикой, а является характеристикой, зависящей от направления фильтрации.

И получается, что если поток идёт параллельно напластованию, то мы можем использовать среднее арифметическое для осреднения.

Если поток идёт перпендикулярно напластованию, то рекомендуется использовать среднее гармоническое.

Если же пласт сильно неоднородный (сложно выделить направление напластования), то можно использовать среднее геометрическое или хитроумную комбинацию арифметических и гармонических.

Но самый лучший способ -- это осреднение на основе решения уравнений однофазной или многофазной фильтрации.
Т.е. что делается?
Фактически производится расчёт потоков (по формуле Дарси, грубо говоря) на мелких ячейках и затем на крупных ячейках обратным пересчётом рассчитывается проницаемость так, чтобы потоки через грани крупных ячеек были такими же, как и сумма потоков через грани мелких ячеек, которые составляют эту крупную ячейку. Т.е. основная задача -- это сохранить потоки, и таким образом подбирается проницаемость, чтобы эти потоки сохранились.

\includegraphics[width=\textwidth, page=64]{Kurs_OsnovyGDM_Kai_774_gorodovSV_v6_0.pdf}

Здесь говорится о том, какие условия задавать на границах при таком способе осреднения.

Если поток идёт по горизонтали, то мы говорим, что вертикального перетока нет.

Если идёт косая слоистость, то вычисляется полный тензор проницаемости.

Если идёт и горизонтальный поток, и вертикальный, то можно задать изменение давления на верхних границах, чтобы был переток.
\\

\textbf{Ремасштабирование геомодели. Контроль качества}

\includegraphics[width=\textwidth, page=65]{Kurs_OsnovyGDM_Kai_774_gorodovSV_v6_0.pdf}

Как контролировать ремасштабирование?
Есть несколько способов.

Первое -- это геолого-статистический разрез (например, по песчанистости).
Здесь чисто визуально оценивается, сохранились ли глинистые перемычки, оцениваются доли коллектора с высоким и низким содержанием глины, т.е. такое визуальное сравнение графиков.

На слайде (на крайнем левом рисунке) красным показано среднее значение песчанистости в слоях по укрупнённой модели, а синей -- в исходной геологической модели.
Геолого-статистический разрез получается следующим образом: в каждом слое считается среднее арифметическое значение и наносится на график (по оси ординат -- слои, по оси абсцисс -- значения песчанистости).

Получается такая вот "<кардиограмма"> (крайний левый рисунок), на которой мы сопоставляем визуально, насколько хорошо сохранились глинистые перемычки.

Также можно сопоставить начальные запасы углеводородов (сохранились или не сохранились после ремасштабирования), эффективные толщины и ещё можно посмотреть гистограммы.

Гистограммы, конечно, один в один не совпадут, потому что количество крайних ячеек (точнее ячеек, которые имеют крайние значения, т.е. либо максимальные, либо минимальные) сократится, поскольку такие ячейки в ходе осреднения будут объединяться с ячейками с другими значениями.
Соответственно количество крайних значений уменьшится, и гистограмма как бы прижмётся к своему среднему значению, но при этом вид самой гистограммы должен быть одинаковый как до, так и после укрупнения ячеек.
Т.е. если мы видим какое-то смещение среднего значения или другой вид гистограммы, то это может говорить о том, что мы потеряли какую-то информацию о строении пласта в ходе этого укрупнения и нужно вернуться и проверить, всё ли правильно мы сделали, правильные ли методы осреднения использовали и не слишком ли грубо мы всё это сделали.

\end{document}