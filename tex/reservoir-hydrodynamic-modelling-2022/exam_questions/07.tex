\begin{document}

\subsection{Раскрыть суть уравнения пьезопроводности, как выводится уравнение пьезопроводности для "<неупругого пласта">? Что такое коэффициент пьезопроводности?}

Уравнение пьезопроводности -- основное уравнение гидродинамики, с помощью которого описывается процесс фильтрации жидкости в пористых средах.
\\

Уравнение пьезопроводности строится на трёх уравнениях: на уравнении неразрывности потока (т.е. уравнении переноса массы, записанном в дифференциальной форме), на законе Дарси и на соотношении для сжимаемости флюида.
\\

Набор исходных уравнений для вывода уравнения пьезопроводности:
\begin{itemize}
	\item неразрывность потока
	\beq\label{Continuity0}
	\frac{\partial\left(\rho_f\varphi\right)}{\partial t}+\pmb{\nabla}\cdot\left(\rho_f\varphi \pmb{v_f}\right)=q_f(\pmb{x})
	\eeq
	\item закон Дарси
	\beq\label{Darcy0}
	\pmb{W}=-\frac{k}{\mu_f}\cdot\pmb{\nabla} p
	\eeq
	\item сжимаемость флюида
	\beq\label{Compressibility0}
	p-p_0=K_f\frac{\rho_f-\rho_f^0}{\rho_f^0}
	\eeq
\end{itemize}
\ \\

\textbf{Вывод уравнения пьезопроводности в векторной форме (быстрый, но не совсем строгий вывод)}

В предположении неподвижности скелета ($\pmb{v_s}\approx \pmb{0}$ и $\varphi(t)=\textrm{const}$) верно равенство $\pmb{W}\approx\varphi \pmb{v_f}$.
Подставляя это выражение в закон Дарси \eqref{Darcy0}, получаем:
\beq\label{DarcyWithSkeletNotMoving0}
\varphi \pmb{v_f}=-\frac{k}{\mu_f}\cdot\pmb{\nabla} p
\eeq

Условие сжимаемости флюида \eqref{Compressibility0} перепишем в дифференциальной форме:
\beq\label{CompressibilityDiff0}
\frac{\partial p}{\partial t}=\frac{K_f}{\rho_f^0}\frac{\partial\rho_f}{\partial t}
\eeq

Учитывая предположение о неподвижности скелета, перепишем уравнение неразрывности потока:
\beq\label{ContinuityWithSkeletNotMoving0}
\varphi\frac{\partial\rho_f}{\partial t}+\pmb{\nabla}\cdot\left(\rho_f\varphi\pmb{v_f}\right)=q_f(\pmb{x})
\eeq

Подставляя \eqref{DarcyWithSkeletNotMoving0} и \eqref{CompressibilityDiff0} в \eqref{ContinuityWithSkeletNotMoving0}, при отсутствии источникового слагаемого ($q_f(\pmb{x})=0$) получаем:
\beq
\varphi\frac{\rho_f^0}{K_f}\frac{\partial p}{\partial t}-\pmb{\nabla}\cdot\left(\rho_f\frac{k}{\mu_f}\pmb{\nabla} p\right)=0
\eeq

При дополнительном условии слабосжимаемости флюида ($\rho_f\approx\rho_f^0=\textrm{const}$) получаем:
\beq
\frac{\partial p}{\partial t}=\frac{kK_f}{\mu_f\varphi}\pmb{\nabla}^2p
\eeq

Это уравнение пьезопроводности (без упругости пласта), полученное в приближении слабосжимаемого флюида, неподвижного и недеформируемого пласта.
\\

\textbf{Вывод уравнения пьезопроводности в покомпонентной форме с обезразмериванием (от Шеля Е.В.)}

Запишем ЗСМ для флюида:
\beq
\frac{\partial r_f}{\partial t}+\partial_i\left(r_f v_i^f\right)=0
\eeq

Закон Дарси в "<школьной"> форме:
\beq
Q=-\frac{\Delta p}{L}\frac{k}{\mu}S
\eeq

Закон Дарси в дифференциальной форме:
\beq\label{DarcyDiffShel0}
W_i=-\frac{k_{ij}}{\mu}\partial_j p,
\eeq
где $W_i=\varphi v_i^f$ -- потоковая относительная скорость флюида.

Учитывая связь эффективной и истинной плотностей ($r_f=\varphi\rho_f$), перепишем ЗСМ для флюида:
\beq\label{ContinuityShel0}
\frac{\partial\left(\rho_f\varphi\right)}{\partial t}+\partial_i\left(\rho_f\varphi v_i^f\right)=0
\eeq

Подставляя \eqref{DarcyDiffShel0} в \eqref{ContinuityShel0}, получаем:
\beq\label{GeneralPiezo0}
\frac{\partial\left(\rho_f\varphi\right)}{\partial t}-\partial_i\left(\rho_f\frac{k_{ij}}{\mu}\partial_j p\right)=0
\eeq

--------------------------------------------------------------------

Замыкающее соотношение (связь плотности флюида и давления):
\beq\label{Zam10}
\rho_f=\rho_f^0\left(1+c_f\left(p-p_0\right)\right),
\eeq
где $c_f$ -- сжимаемость флюида (1/Па).


Замыкающее соотношение (связь пористости и давления):
\beq\label{Zam20}
\varphi=\varphi^0+c_{\text{п}}\left(p-p_0\right),
\eeq
где $c_{\text{п}}$ -- сжимаемость пор (не равно сжимаемости породы).

--------------------------------------------------------------------

Продифференцируем по времени замыкающее соотношение \eqref{Zam10}:
\beq\label{DiffZam10}
\frac{\partial\rho_f}{\partial t}=c_f\rho_f^0\frac{\partial p}{\partial t}
\eeq

Продифференцируем по пространству замыкающее соотношение \eqref{Zam10}:
\beq\label{GradZam10}
\partial_i\rho_f=c_f\rho_f^0\partial_i p
\eeq

Продифференцируем по времени замыкающее соотношение \eqref{Zam20}:
\beq\label{DiffZam20}
\frac{\partial\varphi}{\partial t}=c_\text{п}\frac{\partial p}{\partial t}
\eeq

Продифференцируем по пространству замыкающее соотношение \eqref{Zam20}:
\beq\label{GradZam20}
\partial_i\varphi=c_\text{п}\partial_i p
\eeq

--------------------------------------------------------------------

Раскрывая производные произведений в \eqref{GeneralPiezo0}, получаем:
\beq\label{OpenGeneralContinuity0}
\frac{\partial\rho_f}{\partial t}\varphi+\rho_f\frac{\partial\varphi}{\partial t}-\frac{k_{ij}}{\mu}\partial_j p\,\partial_i\rho_f-\rho_f\partial_j p\,\partial_i\!\left(\frac{k_{ij}}{\mu}\right)-\rho_f\frac{k_{ij}}{\mu}\left(\partial_i\partial_j p\right)=0
\eeq

Подставляя \eqref{DiffZam10}, \eqref{GradZam10}, \eqref{DiffZam20} и \eqref{GradZam20} в \eqref{OpenGeneralContinuity0}, получаем:
\begin{multline}\label{Expanded0}
c_f\rho_f^0\frac{\partial p}{\partial t}\varphi+\rho_f c_\text{п}\frac{\partial p}{\partial t}-\frac{k_{ij}}{\mu}\partial_j p\,c_f\rho_f^0\,\partial_i p-\frac{\rho_f}{\mu}\partial_j p\,\partial_i k_{ij}+\\+\rho_f\,\partial_j p\,k_{ij}\frac{\partial\mu}{\partial p}\frac{1}{\mu^2}\partial_i p-\rho_f\frac{k_{ij}}{\mu}\left(\partial_i\partial_j p\right)=0
\end{multline}

--------------------------------------------------------------------

Перед анализом физических уравнений всегда делают масштабный анализ, чтобы понять, какие слагаемые в уравнении важны, а какие не важны (пример: уравнение Навье-Стокса с числами Струхаля, Эйлера, Рейнольдса, Фруда).

Спойлер: ГДМ симуляторы не решают уравнение пьезопроводности в классическом виде, а решают закон сохранения массы, в который они подставляют закон Дарси.

Далее необходимо выделить характерные масштабные факторы, обезразмерив каждую из функций в уравнении.

Введём безразмерное давление $\tilde{p}$ такое, что:
\beq
p=\tilde{p}\cdot p_0,
\eeq
где $p_0$ -- пластовое давление.

Введём безразмерное расстояние $\tilde{r}$ такое, что:
\beq
\vec{r}=\tilde{r}\cdot L,
\eeq
где $L$ -- некое характерное расстояние (например, между скважинами).

Введём безразмерную проницаемость $\tilde{k}_{ij}$ такую, что:
\beq
k_{ij}=\tilde{k}_{ij}\cdot k_0,
\eeq
где $k_0$ -- некая характерная проницаемость.

Введём безразмерную вязкость $\tilde{\mu}$ такую, что:
\beq
\mu=\tilde{\mu}\cdot\mu_0,
\eeq
где $\mu_0$ -- некая характерная вязкость.

Все безразмерные функции (с волной) порядка единицы.

--------------------------------------------------------------------

Перепишем \eqref{Expanded0} в введённых безразмерных величинах, разделив обе части этого уравнения на $\rho_f^0$:
\begin{multline}
\frac{\partial p}{\partial t}\left(\varphi c_f+\frac{\rho_f}{\rho_f^0}\cdot c_\text{п}\right)-\frac{k_0}{\mu_0}\frac{p_0^2}{L^2}c_f\frac{\tilde{k}_{ij}}{\tilde{\mu}}\,\tilde{\partial}_i\tilde{p}\,\tilde{\partial}_j\tilde{p}-\frac{\rho_f}{\rho_f^0}\frac{k_0\,p_0}{\mu_0L^2}\frac{\tilde{k}_{ij}}{\tilde{\mu}}\,\tilde{\partial}_j\tilde{p}\,\tilde{\partial}_i\tilde{k}_{ij}+\\+\frac{\rho_f}{\rho_f^0}\frac{p_0\,k_0}{L^2\mu_0}\,\tilde{\partial}_j\tilde{p}\,\tilde{k}_{ij}\frac{\partial\tilde{\mu}}{\partial\tilde{p}}\frac{1}{\tilde{\mu}^2}\,\tilde{\partial}_i\tilde{p}-\frac{\rho_f}{\rho_f^0}\frac{k_0}{\mu_0}\frac{p_0}{L^2}\,\frac{\tilde{k}_{ij}}{\tilde{\mu}}\left(\tilde{\partial}_i\tilde{\partial}_j\tilde{p}\right)=0
\end{multline}

Вынесли все масштабные множители. Далее делим обе части уравнения на множитель перед старшей производной $\left(\text{на }\frac{k_0\,p_0}{\mu_0\,L^2}\right)$, т.е. обезразмериваем уравнение:
\begin{multline}\label{PiezoEqDiv0}
\frac{\mu_0L^2}{k_0p_0}\cdot\frac{\partial p}{\partial t}\left(\varphi c_f+\frac{\rho_f}{\rho_f^0}\cdot c_\text{п}\right)-p_0c_f\frac{\tilde{k}_{ij}}{\tilde{\mu}}\,\tilde{\partial}_i\tilde{p}\,\tilde{\partial}_j\tilde{p}-\frac{\rho_f}{\rho_f^0}\frac{\tilde{k}_{ij}}{\tilde{\mu}}\,\tilde{\partial}_j\tilde{p}\,\tilde{\partial}_i\tilde{k}_{ij}+\\+\frac{\rho_f}{\rho_f^0}\frac{\partial\tilde{\mu}}{\partial\tilde{p}}\frac{1}{\tilde{\mu}^2}\tilde{k}_{ij}\,\tilde{\partial}_j\tilde{p}\,\tilde{\partial}_i\tilde{p}-\frac{\rho_f}{\rho_f^0}\frac{\tilde{k}_{ij}}{\tilde{\mu}}\left(\tilde{\partial}_i\tilde{\partial}_j\tilde{p}\right)=0
\end{multline}

--------------------------------------------------------------------

Сделаем 3 важных приближения:
\begin{enumerate}
	\item $p_0 c_f\ll 1$ (прикинем: сжимаемость воды порядка $10^{-5}\text{ атм}^{-1}=10^{-10}\text{ Па}^{-1}$; характерные значения давлений на глубинах, равных нескольким километрам, составляют сотни атмосфер; таким образом, произведение порядка $10^{-3}$, что много меньше единицы; но такое приближение не работает для газа: для него рассматриваемое произведение порядка единицы); это приближение фактически равносильно приближению $\rho_f\approx\rho_f^0$;
	\item $\tilde{\partial}_i\tilde{k}_{ij}\ll 1$ (считаем, что на характерном масштабе задачи по данному направлению проницаемость изменяется незначительно, не больше 10 процентов);
	\item $\dfrac{\partial\tilde{\mu}}{\partial\tilde{p}}\ll 1$ (считаем, что отмасштабированный график проницаемости от давления пологий -- этот факт подтверждается экспериментально -- вязкость слабо зависит от давления)
\end{enumerate}

Тогда уравнение \eqref{PiezoEqDiv0} перепишется в следующем виде (убрали слагаемые с пренебрежимо малыми множителями в рамках сделанных приближений и вернулись от безразмерных функций с волной к обычным функциям):
\beq
\frac{\partial p}{\partial t}\underbrace{\left(\varphi c_f+c_\text{п}\right)}_{c_t}-\frac{k_{ij}}{\mu}\partial_i\partial_j p=0
\eeq
(заметим, что если есть анизотропия проницаемости, то лапласиана в уравнении не будет).

Получаем классическое уравнение пьезопроводности:
\beq
\frac{\partial p}{\partial t}-\frac{k_{ij}}{\mu c_t}\partial_i\partial_j p=0,
\eeq
где $c_t$ -- это полная сжимаемость.

--------------------------------------------------------------------

\textbf{Замечание.} Но есть литература, в которой $c_t=c_f+\dfrac{c_\text{п}}{\varphi}=c_f+c_{\text{породы}}$, тогда уравнение пьезопроводности будет выглядеть так:
\beq
\frac{\partial p}{\partial t}-\frac{k_{ij}}{\mu\varphi c_t}\partial_i\partial_j p=0
\eeq

--------------------------------------------------------------------

Пусть тензор проницаемости изотропен $k_{ij}=k_0\cdot\delta_{ij}$, тогда:
\beq
\frac{\partial p}{\partial t}-\frac{k_0}{\mu c_t}\delta_{ij}\,\partial_i\partial_j p=0\Leftrightarrow\frac{\partial p}{\partial t}-\frac{k_0}{\mu c_t}\Delta p=0
\eeq
(получили всем известный вид уравнения пьезопроводности).

\end{document}