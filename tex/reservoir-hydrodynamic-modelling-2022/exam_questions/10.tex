\begin{document}

\subsection{Приведите примеры прямых и обратных задач уравнения пьезопроводности.}

\textbf{Прямые задачи уравнения пьезопроводности:} при известных параметрах пласта, начальных и граничных условиях найти искомую функцию (зависимость давления от координаты и времени).
\\


\textbf{Примеры прямых задач уравнения пьезопроводности.}

Начально-краевая задача для уравнения пьезопроводности в случае радиального притока к скважине и поддержания постоянного давления на внешней границе пласта (условие Дирихле):
\beq
\begin{cases}
	\dfrac{1}{\kappa}\dfrac{\partial p(r,t)}{\partial t}=\dfrac{1}{r}\dfrac{\partial p(r,t)}{\partial r}+\dfrac{\partial^2p(r,t)}{\partial r^2}\\
	p(r,0)=p_0, r\in\left(r_w,r_e\right]\\
	p(r_w,t)=p_w\\
	p(r_e,t)=p_e
\end{cases}
\eeq
(описывает неустановившийся и установившийся режимы)


Начально-краевая задача для уравнения пьезопроводности в случае радиального притока к скважине и отсутствия перетока на внешней границе пласта (условие Неймана):
\beq
\begin{cases}
	\dfrac{1}{\kappa}\dfrac{\partial p(r,t)}{\partial t}=\dfrac{1}{r}\dfrac{\partial p(r,t)}{\partial r}+\dfrac{\partial^2p(r,t)}{\partial r^2}\\
	p(r,0)=p_0, r\in\left(r_w,r_e\right]\\
	p(r_w,t)=p_w\\
	\dfrac{\partial p}{\partial r}\bigg|_{r=r_e}=0
\end{cases}
\eeq
(описывает неустановившийся и псевдоустановившийся режимы)
\\

Пример решения прямых задач уравнения пьезопроводности доступен по ссылке \href{https://github.com/mualal/source/blob/main/well_productivity/productivity.ipynb}{GO TO МОДЕЛИРОВАНИЕ РЕЖИМОВ ПРИТОКА К СКВАЖИНЕ}
\\


\textbf{Обратные задачи уравнения пьезопроводности:} при проведении ГДИС уже знаем распределения давлений, начальные и граничные условия; требуется уточнить значения параметров пласта (например, проницаемость).


\end{document}