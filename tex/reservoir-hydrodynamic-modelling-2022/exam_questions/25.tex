\begin{document}

\subsection{Что такое критерий стабильности? Раскройте суть критерия стабильности. Приведите минимум 2 примера критерия стабильности.}

Численная схема решения дифференциального уравнения неустойчива, если приводит к возникновению хаотических решений, не имеющих отношения к решению дифференциального уравнения.
\\

Другими словами, при численном решении дифференциальных уравнений мы делаем аппроксимации (разбиваем пространство и время на конечные интервалы -- ячейки сетки), которые могут вызвать нефизичность получаемых решений.
\\

Математически: схема устойчива, если малые изменения краевых условий и правой части приводят к малым изменениям решения.
Численное решение устойчиво, если непрерывно зависит от входных данных.
\\

Устойчивость называется безусловной, если определение справедливо при любых малых шагах интегрирования (нет связи между величинами шагов по различным переменным).
\\

Устойчивость считается условной, если шаги по разным координатам (например, по пространству и по времени) должны удовлетворять дополнительным соотношениям.
\\

Явная схема решения уравнения фильтрации условно устойчива, т.е. существуют критерии стабильности (связь между величинами шагов по различным переменным), при невыполнении которых явная схема разваливается (не даёт физически адекватного решения).

Анализ стабильности вон Ньюмана показал, что явное решение уравнения фильтрации имеет данный критерий стабильности:
\beq
\Delta t\leqslant\frac{1}{2}\left(\frac{\varphi\mu c}{k}\right)\Delta x^2
\eeq
\ \\

Неявная схема решения уравнения фильтрации безусловно устойчива.

Но после решения всё равно лучше проверять невязки и делать проверку по матбалансу.

\end{document}