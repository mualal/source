\begin{document}

\subsection{Что такое уравнение Баклея-Леверетта? Подробно опишите суть уравнения Баклея-Леверетта.}

Уравнение Баклея-Леверетта показывает, что скорость продвижения фронта с постоянной водонасыщенностью $S_w$ пропорциональна производной функции Баклея-Леверетта.

\includegraphics[width=\textwidth, page=78]{Kurs_OsnovyGDM_Kai_774_gorodovSV_v6_0.pdf}

\textbf{Теория Баклея-Леверетта.}

На основе ОФП можем рассчитать, каким образом будет происходить заводнение в пласте  (другими словами, как будет продвигаться фронт вытеснения).

ОФП совместно с соотношением вязкостей нефти и воды влияют на скорость распространения фронта заводнения и на величину скачка насыщенности.

$f_w(S_w)$ -- функция фракционного потока.

На слайде представлена формула для фракционного потока.
По сути это обводнённость, т.е. сколько воды мы добываем по отношению к сумме всей добытой жидкости (действительно, числитель и знаменатель функции Баклея-Леверетта можем умножить на абсолютную проницаемость и градиент давления, тогда получим отношение дебита воды к суммарному дебиту жидкости, т.е. обводнённость).
\\

Графический анализ (по Уэлджу): зная угол наклона касательной к кривой фракционного потока (графику зависимости $f_w(S_w)$; есть зависимость от насыщенности, так как в формулу фракционного потока входят ОФП, которые зависят от насыщенности), можем найти скорость продвижения фронта заводнения.
\\

Насыщенность в точке касания (на графике) -- это насыщенность на фронте вытеснения.
\\

Насыщенность в точке пересечения касательной и горизонтальной прямой $f_w=1$ -- это средняя насыщенность от нагнетательной скважины до края заводнения.
\\

Скорость продвижения фронта заводнения мы можем посчитать через производную, а производная этого фракционного потока -- это фактически угол наклона, т.е. тангенс угла наклона касательной будет определять производную.
\\

Таким образом, даже без построения модели, имея только ОФП и вязкости, можем многое рассказать о том, каким образом будет происходить вытеснение в пласте.
\\

\textbf{Далее представлены более подробные слайды про задачу Баклея-Леверетта.}

\includegraphics[width=\textwidth, page=74]{1МАТЕРИАЛЫ_ПО_КУРСУ_ДОБЫЧА_2022.pdf}

\includegraphics[width=\textwidth, page=75]{1МАТЕРИАЛЫ_ПО_КУРСУ_ДОБЫЧА_2022.pdf}

\includegraphics[width=\textwidth, page=76]{1МАТЕРИАЛЫ_ПО_КУРСУ_ДОБЫЧА_2022.pdf}

\includegraphics[width=\textwidth, page=77]{1МАТЕРИАЛЫ_ПО_КУРСУ_ДОБЫЧА_2022.pdf}

\end{document}