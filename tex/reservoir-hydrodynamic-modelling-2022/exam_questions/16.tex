\begin{document}

\subsection{Какие лабораторные исследования проводят для измерения ОФП? Опишите суть исследований.}

\includegraphics[width=\textwidth, page=80]{Kurs_OsnovyGDM_Kai_774_gorodovSV_v6_0.pdf}

По стандартам все лабораторные исследования ОФП должны проводиться на установившемся режиме.
Минус такого подхода: для низкопроницаемых образцов время установления может занимать месяц или даже несколько месяцев.
Это дорого.
Поэтому иногда проводят быстрые исследования на неустановившемся режиме (расчёт по формуле Баклея-Леверетта), но это менее точно и не соответствует стандартам.

\includegraphics[width=\textwidth, page=81]{Kurs_OsnovyGDM_Kai_774_gorodovSV_v6_0.pdf}

На слайде представлены схема и фотография установки.


\textbf{Далее представлены более подробные слайды о лабораторных методах определения проницаемости.}

\includegraphics[width=\textwidth, page=4]{Permeability.pdf}

\includegraphics[width=\textwidth, page=5]{Permeability.pdf}

\includegraphics[width=\textwidth, page=6]{Permeability.pdf}

\includegraphics[width=\textwidth, page=7]{Permeability.pdf}

\includegraphics[width=\textwidth, page=8]{Permeability.pdf}

\includegraphics[width=\textwidth, page=9]{Permeability.pdf}

\includegraphics[width=\textwidth, page=10]{Permeability.pdf}

\includegraphics[width=\textwidth, page=11]{Permeability.pdf}

\includegraphics[width=\textwidth, page=12]{Permeability.pdf}

\includegraphics[width=\textwidth, page=13]{Permeability.pdf}

\end{document}