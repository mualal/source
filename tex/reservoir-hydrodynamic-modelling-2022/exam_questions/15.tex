\begin{document}

\subsection{Что такое масштабирование ОФП? Зачем нужно нормировать ОФП?}

\textbf{Как задать ОФП в ГДМ, если есть несколько исследований?}

\includegraphics[width=\textwidth, page=83]{Kurs_OsnovyGDM_Kai_774_gorodovSV_v6_0.pdf}

Нормировать ОФП необходимо для того, чтобы привести их к единым диапазонам; после нормализации ОФП будут отличаться только выпуклостью и их можно будет аппроксимировать, подобрав степень кривизны корреляций Кори или LET (найти среднюю кривую).
\\

Для остаточных насыщенностей и концевых точек ОФП необходимо найти корреляции со свойствами образца (с пористостью или проницаемостью).

Это необходимо, чтобы симулятор масштабировал кривые ОФП в каждой ячейке модели в зависимости от свойств этой ячейки.

Масштабирование ОФП позволяет задать более точные кривые ОФП для каждой ячейки модели (в зависимости от свойств ячейки), если у нас есть несколько (большое количество) исследований на керне.

\includegraphics[width=\textwidth, page=84]{Kurs_OsnovyGDM_Kai_774_gorodovSV_v6_0.pdf}

Двухточечное и трёхточечное масштабирования отличаются по тому, используется ли критическая насыщенность по противоположной фазе для того, чтобы масштабировать кривые: при двухточечном масштабировании не используется, при трёхточечном масштабировании используется.
\\

\textbf{Концевые точки ОФП в системе нефть-вода}

\includegraphics[width=\textwidth, page=85]{Kurs_OsnovyGDM_Kai_774_gorodovSV_v6_0.pdf}

\includegraphics[width=\textwidth, page=86]{Kurs_OsnovyGDM_Kai_774_gorodovSV_v6_0.pdf}
\ \\

\textbf{Масштабирование ОФП}

\includegraphics[width=\textwidth, page=87]{Kurs_OsnovyGDM_Kai_774_gorodovSV_v6_0.pdf}

\includegraphics[width=\textwidth, page=88]{Kurs_OsnovyGDM_Kai_774_gorodovSV_v6_0.pdf}

\includegraphics[width=\textwidth, page=89]{Kurs_OsnovyGDM_Kai_774_gorodovSV_v6_0.pdf}

\includegraphics[width=\textwidth, page=90]{Kurs_OsnovyGDM_Kai_774_gorodovSV_v6_0.pdf}
\ \\

\textbf{По горизонтали (по насыщенности)}

\includegraphics[width=\textwidth, page=91]{Kurs_OsnovyGDM_Kai_774_gorodovSV_v6_0.pdf}
\ \\

\begin{figure}[H]
\textbf{По вертикали}

\includegraphics[width=\textwidth, page=92]{Kurs_OsnovyGDM_Kai_774_gorodovSV_v6_0.pdf}
\end{figure}

\end{document}