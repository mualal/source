\begin{document}

\subsection{Опишите структуру Data файла для запуска модели в симуляторе T-Navigator. Опишите основные разделы Data файла.}

\textbf{Структура файла исходных данных для симуляторов ECLIPSE и т-Навигатор}

\includegraphics[width=\textwidth, page=27]{Kurs_OsnovyGDM_Kai_774_gorodovSV_v6_0.pdf}

В т-Навигаторе тоже поддерживается формат Eclipse.
И в т-Навигаторе тоже считываются DATA-файлы, которые состоят из секций, в которые сгруппированы определённые ключевые слова, описывающие модель.
По сути это чем-то похоже на программирование: есть некая команда, которая воспринимается программой симулятором, и дальше идут некие параметры выполнения этой команды.

RUNSPEC = спецификация запуска. Eclipse создавали ещё в 80-е годы на Фортране и в это время ещё не было достаточного количества оперативной памяти, следовательно, нужно было заранее определять, сколько памяти потребуется модели для расчёта.
Поэтому в этой секции указывались основные характеристики: сколько в модели будет скважин, сколько моделируемых фаз, сколько разных PVT-таблиц. В общем, такие характеристики, чтобы под них забронировать оперативную память.
Сейчас таких проблем с оперативной памятью уже нет, но исторически такая секция RUNSPEC осталась.

В секции PROPS задаются PVT-свойства флюидов и SCAL свойства (special core analysis in laboratory) взаимодействия этих флюидов с пластом.
Для получения этих свойств проводится специальный анализ флюидов и керна в лаборатории.

Секция REGIONS используется, если нам нужно задать отдельные регионы, в каждом из которых свои свойства (например, свои свойства флюида).
Когда это нужно? Например, у нас есть несколько пластов на месторождении, и в каждом из этих пластов свойства отличаются, соответственно, можем записать их как разные регионы и для каждого региона задавать свои свойства.

Секция SOLUTION описывает инициализацию модели, т.е. начальные условия (до того, как начался расчёт): какое начальное состояние по насыщенности и так далее.

В секцию SUMMARY записываются те графики, которые хотим посмотреть по результатам расчёта.
Эта секция тоже относится к симулятору Eclipse, в т-Навигаторе эта секция необязательна (в нём настройка отображаемых графиков производится в самом интерфейсе программы -- галочками отмечаются графики, которые требуется отобразить).
\\

\textbf{Справочники для симуляторов ECLIPSE и т-Навигатор}

\includegraphics[width=\textwidth, page=28]{Kurs_OsnovyGDM_Kai_774_gorodovSV_v6_0.pdf}

Ключевые слова запоминать необязательно.
И для Eclipse, и для т-Навигатора, и для других симуляторов есть справочники, которые поставляются вместе с программой.
В этих справочниках есть технический мануал, в котором описаны уравнения, заложенные в расчёт, и есть мануал, который описывает сами ключевые слова (обычно сгруппированы по первым буквам).
Следовательно, можем найти необходимое ключевое слово и посмотреть, какие параметры нужны для этого ключевого слова.

Также есть примеры файлов-моделей с различными опциями.
Если хотим смоделировать какой-либо процесс (например, закачку полимера или водогазовое воздействие), то можем просто открыть папку с готовыми примерами (как правило, эта папка совпадает с корневой папкой, в которой лежит сам симулятор) и посмотреть, какие ключевые слова используются для моделирования этого процесса.
Затем вернуться в мануал и просмотреть эти ключевые слова, чтобы понять, что необходимо задавать для моделирования этих опций и воздействий.


\end{document}