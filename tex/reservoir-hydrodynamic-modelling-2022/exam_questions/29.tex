\begin{document}

\subsection{Какими способами можно задать кубы свойств?}

\includegraphics[width=\textwidth, page=29]{Kurs_OsnovyGDM_Kai_774_gorodovSV_v6_0.pdf}

Свойства должны быть заданы для каждой ячейки, чтобы симулятор знал, как производить расчёт.
Как правило, эти значения присваиваются центру каждой ячейки; свойства можно задать явным перечислением и, если есть повторяющиеся значения, то их можно сгруппировать (т.е. записать, что свойство в $n$ ячейках имеет значение $a$).

Значения свойств ещё могут быть заданы в виде функции (в Eclipse ключевое слово OPERATE, в т-Навигаторе ключевое слово ARITHMETIC).

Schlumberger раньше поставлял FloViz и FloGrid. Сейчас они устарели, и Schlumberger их не продаёт.

Для того, чтобы сэкономить ресурсы, расчёт производится только в активных ячейках.
Активными считаются ячейки, в которых фактически происходит поток флюида.
То есть в ячейках с глинами (неколлекторами), где нет никаких потоков флюида, нет необходимости проводить какие-либо расчёты.
Соответственно, можем просто их исключить из расчёта (по-умолчанию неактивны ячейки с нулевыми пористостью (PORO) или песчанистостью (NTG, отношение количества эффективных толщин к общим толщинам)).

Также есть ключевое слово ACTNUM, которое непосредственно задаёт активные и неактивные ячейки. Т.е. мы или геолог с помощью этого ключевого слова можем самостоятельно отметить ячейки с коллектором (песчаником) или неколлектором (глинами).
 
\includegraphics[width=\textwidth, page=30]{Kurs_OsnovyGDM_Kai_774_gorodovSV_v6_0.pdf}

На этом слайде показаны примеры задания свойств в ячейках непосредственно по ячейкам, с группировкой ячеек, с помощью ключевого слова EQUALS, с помощью копирования COPY, а также с помощью арифметических операций (в Eclipse ключевое слово MULTIPLY, в т-Навигаторе можем использовать ключевое слово ARITHMETIC).

\includegraphics[width=\textwidth, page=31]{Kurs_OsnovyGDM_Kai_774_gorodovSV_v6_0.pdf}

На этом слайде показаны примеры задания свойств в ячейках с помощью ключевого слова BOX.

На слайде приведены примеры использования ключевого слова INCLUDE.
Файлы с большими массивами данных (кубами свойств) хранятся отдельно и подключаются к основному файлу с помощью ключевого слова INCLUDE.

\end{document}