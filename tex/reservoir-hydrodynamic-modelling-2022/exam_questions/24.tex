\begin{document}

\subsection{Опишите способы решения системы линейных ур-ий. Приведите минимум 2 способа в качестве примеров.}

Прямые способы решения системы линейных алгебраических уравнений (СЛАУ):

1) метод Крамера;

2) метод Гаусса;

3) метод обратной матрицы;

4) метод прогонки

(перечислены известные названия, но одни из них могут быть частными случаями других: например, метод прогонки -- частный случай метода Гаусса)
\\

\textbf{СЛАУ в общем виде}

\includegraphics[width=\textwidth]{slau}

\begin{figure}[H]
\textbf{Подробное описание метода Гаусса}
\center
\includegraphics[width=\textwidth]{slau_gauss1}
\end{figure}


\begin{figure}[H]
\center
\includegraphics[width=\textwidth]{slau_gauss2}
\end{figure}
\ \\


\begin{figure}[H]
\textbf{Пример решения СЛАУ методом Гаусса}
\center
\includegraphics[width=\textwidth]{slau_gauss_example1}
\end{figure}

\begin{figure}[H]
\center
\includegraphics[width=.85\textwidth]{slau_gauss_example2}
\end{figure}

\begin{figure}[H]
\textbf{Описание метода обратной матрицы}
\center
\includegraphics[width=.85\textwidth]{slau_inv_matrix}
\end{figure}
\ \\


\begin{figure}[H]
\textbf{Подробное описание метода прогонки}
\center
\includegraphics[width=.85\textwidth]{slau_marching}
\end{figure}
\ \\

Итерационные способы решения СЛАУ:

1) метод простой итерации (метод Якоби);

2) метод Зейделя

(недостаток итерационных методов: решение получаем с заданной точностью в результате последовательных приближений (в сравнении: а при использовании прямых методов в предположении отсутствия ошибок округления получаем точное решение за конечное число арифметических действий); преимущество итерационных методов: могут быть легко расширены (приспособлены) для решения нелинейных уравнений)


%Метод прогонки применяется для решения систем уравнений с трёхдиагональной (ленточной) матрицей.
%Такая система уравнений записывается в виде:
%\beq
%a_ix_{i-1}+b_ix_i+c_ix_{i+1}=d_i,
%\eeq 
%где $i=1,2,3,...,n; a_1=0; c_n=0$

%Является частным случаем метода Гаусса и состоит из прямого и обратного хода.



\end{document}