\begin{document}

\subsection{Что такое аквифер? Раскройте суть аквифера. Какое ключевое слово обозначает аквифер в T-Navigator?}

\textbf{Аналитический аквифер}

\includegraphics[width=\textwidth, page=133]{Kurs_OsnovyGDM_Kai_774_gorodovSV_v6_0.pdf}

С помощью аквифера можно задать граничные условия в модели.

Другими словами, если есть водоносный горизонт на границах, то его можно задать в модели с помощью модели аквифера.
\\

Также аквифер является естественным источником поддержания пластового давления в модели.
\\

Есть точное решение Hurst van Everdingen, которое описывает приток из аквифера конечных размеров (формула представлена на слайде).

Но в модели такое точное решение не задать, поэтому была разработана модель притока из аквифера (модель Carter-Tracy).

\includegraphics[width=\textwidth, page=134]{Kurs_OsnovyGDM_Kai_774_gorodovSV_v6_0.pdf}

Carter-Tracy разработали модель притока из аквифера, результаты расчётов по которой близки к аналитическому решению Hurst van Everdingen.

Модель Carter-Tracy задаётся в ГДМ симуляторах с помощью ключевого слова AQUCT.

Формулы представлены на слайде.
Фактически задаются параметры, которые описывают аквифер: мощность (толщина), пористость, сжимаемость, радиус и так называемый угол влияния.
В данной модели пласт представляется кругом или частью круга, а аквифер присоединяется к краям залежи.
И соответственно угол $\theta$ (угол влияния), который говорит, какой частью круга является пласт, в формулу и входит.
\\

Модель Carter-Tracy описывает переход из неустановившегося в псевдо-установившийся режим течения. 
\\

Модель Carter-Tracy рекомендуется использовать либо для больших залежей, либо для низкопроницаемых залежей (другими словами, для залежей, на которых режим течения устанавливается не быстро).

\includegraphics[width=\textwidth, page=135]{Kurs_OsnovyGDM_Kai_774_gorodovSV_v6_0.pdf}

Для высокопроницаемых или маленьких залежей есть более простая модель Fetkovich-а, которая моделирует приток из аквифера просто в виде произведения продуктивности аквифера и разницы давлений в аквифере и нефтеносном пласте.
\\

Для больших или низкопроницаемых залежей модель Fetkovich-а использовать не рекомендуется. В этом случае лучше использовать модель Carter-Tracy.


\end{document}