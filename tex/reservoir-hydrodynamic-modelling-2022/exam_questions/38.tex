\begin{document}

\subsection{Какие основные параметры модели изменяются при адаптации? Какие критерии показывают, что модель успешно адаптирована?}

\textbf{Адаптация модели на разных стадиях разработки}

\includegraphics[width=\textwidth, page=152]{Kurs_OsnovyGDM_Kai_774_gorodovSV_v6_0.pdf}

На разных периодах разработки месторождения настраиваем разные параметры модели.

Период до начала добычи = blue field.

Период безводной добычи = green field.

Период обводнённой добычи (зрелые месторождения) = brown field.

\includegraphics[width=\textwidth, page=153]{Kurs_OsnovyGDM_Kai_774_gorodovSV_v6_0.pdf}

Обычно адаптация идёт от крупного к мелкому (от месторождения к скважинам).

Сначала настраиваем энергетическое состояние залежи: матбаланс по скважинам (сколько отобрали жидкости / сколько закачали воды) и пластовое давление, которое получилось в результате работы всех скважин.

После настройки энергетики, переходим к настройке по соотношению нефть/вода или нефть/газ.
Т.е. к настройке по отборам конкретных флюидов.

И финально производится настройка по коэффициентам продуктивности и забойным давлениям.
\\

\textbf{Адаптация по отборам жидкости и пластовому давлению}

\includegraphics[width=\textwidth, page=154]{Kurs_OsnovyGDM_Kai_774_gorodovSV_v6_0.pdf}

При адаптации обычно меняют те параметры, которые обладают наибольшей неопределённостью.
\\

Про сжимаемость порового пространства: если говорить о месторождениях в Западной Сибири, то там пласты имеют сжимаемость порядка $10^{-5} \text{ атм}^{-1}$, что приводит к тому, что сжимаемость фактически не оказывает ощутимого влияния на динамику пластового давления.

\textbf{Адаптация по соотношению нефть/вода}

\includegraphics[width=\textwidth, page=155]{Kurs_OsnovyGDM_Kai_774_gorodovSV_v6_0.pdf}

При варьировании остаточных насыщенностей гораздо легче испортить модель, чем при варьировании, например, абсолютной проницаемости.
\\

Необходимо помнить, что модель мы делаем для того, чтобы считать на ней какие-то прогнозы, оценивать как поведёт себя месторождение в случае какого-то воздействия, т.е. мы хотим получить адекватный инструмент и соответственно должны использовать физически корректные диапазоны вариации параметров модели.

\textbf{Адаптация по коэффициенту продуктивности и Pзаб}

\includegraphics[width=\textwidth, page=156]{Kurs_OsnovyGDM_Kai_774_gorodovSV_v6_0.pdf}

По коэффициенту продуктивности и забойному давлению настройка простая: варьируем абсолютную проницаемость вблизи скважины или скин-фактор.
\\

\textbf{Критерии адаптации}

\includegraphics[width=\textwidth, page=160]{Kurs_OsnovyGDM_Kai_774_gorodovSV_v6_0.pdf}

Представлены критерии адаптации в случае, если смотрим в целом по всему месторождению (сумму по всем скважинам).

По дебитам воды, нефти, жидкости, газа ошибка не должна превышать 10\%.

По накопленной добыче воды, нефти, жидкости, газа ошибка не должна превышать 5\%.

По пластовым давлениям по регламенту ошибка не должна превышать 25\%, но обычно стараются добиться меньшего диапазона вариации.

\includegraphics[width=\textwidth, page=161]{Kurs_OsnovyGDM_Kai_774_gorodovSV_v6_0.pdf}

Представлены критерии адаптации в случае, если смотрим отдельно по скважинам.

Строятся кроссплоты расчёт-факт (отмечаются все скважины) по накопленной добыче нефти на определённую дату.

Допустимые ошибки: 20\% по нефти; 20\% по воде; 25\% по давлению; 5\% по жидкости; 5\% по закачке.

Симулятор в первую очередь ориентируется на добычу по жидкости и на закачку, поэтому сильных отклонений по жидкости и по закачке (как правило) не бывает.
Все отклонения, как правило, бывают связаны именно с распределением флюидов (нефти, воды, газа) в пределах той жидкости, которую скважина добыла.

\includegraphics[width=\textwidth, page=162]{Kurs_OsnovyGDM_Kai_774_gorodovSV_v6_0.pdf}

Иногда строят такой график, который показывает, какая доля скважин обеспечивает какую долю накопленной добычи нефти и какую погрешность расчёт-факт при этом имеет.
\\

Чтение представленного на слайде графика: видим, что доля фонда скважин с относительной погрешностью расчёт-факт, не превышающей 20\%, составляет около 63\%. И при этом эти 63\% скважин обеспечивают накопленную добычу нефти чуть больше 80\%.\\

Принцип Паретто: 20\% усилий дают 80\% результата; чтобы получить оставшиеся 20\% результата приходится приложить 80\% усилий.\\

Для задач, где не требуется настройка каждой скважины (необходимо понимать только поведение месторождения в целом, например, для проектно-технологических документов), обычно требуют настройку в пределах 20\% только для тех скважин, которые суммарно дают 80\% накопленной добычи по месторождению. 


\end{document}