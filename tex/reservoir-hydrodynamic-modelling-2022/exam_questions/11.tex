\begin{document}

\subsection{Что такое капиллярное давление? Нарисовать кривую капиллярного давления от глубины. Нарисовать кривую капиллярного давления от насыщенности.}

\textbf{Капиллярное давление}

\includegraphics[width=\textwidth, page=68]{Kurs_OsnovyGDM_Kai_774_gorodovSV_v6_0.pdf}

Из-за смачивания возникает капиллярное давление, т.е. разница между давлениями смачивающей и несмачивающей фаз на границе их раздела.

Если представить себе гидрофильный капилляр, то вода стремится по нему растечься и бесконечно бы растекалась по нему, но действует сила гравитации, препятствующая растеканию воды по капилляру.
Сила тяжести и капиллярные силы балансируют.

Запишем баланс: сила гравитации обуславливается разницей давлений столба жидкости, а сила действующая вверх обуславливается поверхностным натяжением.

Из баланса получаем капиллярное давление. Эту формулу необходимо запомнить или уметь вывести.

Чем больше радиус капилляра, тем меньше капиллярное давление и на меньшую высоту поднимется вода от уровня равновесия (зеркала свободной воды -- дальше на слайдах про него расскажем) до уровня равновесия с нефтью. Т.е. переходная зона (уровень поднятия воды в капилляре) будет зависеть от размера капилляра.

Для одного и того же капиллярного давления: чем больше разница плотностей, тем на меньшую высоту флюид поднимется в капилляре.
Поэтому переходная зона между нефтью и водой значительно больше, чем переходная зона между нефтью и газом.
На самом деле, переходную зону между нефтью и газом моделируют очень редко: обычно просто задают газонефтяной контакт (ГНК) в пределах одной ячейки.
\\

\textbf{Дополнение.}
Нелинейная фильтрация связана с вязкостью жидкости и капиллярными эффектами (запирающий градиент / давление сдвига). Подумать об этом и почитать подробнее про нелинейную фильтрацию (могут ли капиллярные эффекты оказать существенное влияние на нелинейную фильтрацию)!

\includegraphics[width=\textwidth, page=69]{Kurs_OsnovyGDM_Kai_774_gorodovSV_v6_0.pdf}

Если нарисовать график капиллярного давления (здесь по оси $x$ откладывается водонасыщенность, а по оси $y$ можно отложить высоту от уровня зеркала свободной воды).

Справа на слайде показаны градиенты давления: давление $\rho_{o} g h$ для нефти и $\rho_{w} g h$ для воды; соответственно разница между ними -- это капиллярное давление.
\\

Зеркало свободной воды = капиллярное давление равно нулю (пересекаются графики давления в нефтяной и водной фазах).
От этого уровня считается высота подъёма воды $h$ по капиллярам.
\\

Определение ВНК (водонефтяного контакта) не так однозначно (есть несколько разных определений).
\\

В гидродинамике для однозначности используют уровень зеркала свободной воды (уровень, где капиллярное давление равно нулю).

\includegraphics[width=\textwidth, page=70]{Kurs_OsnovyGDM_Kai_774_gorodovSV_v6_0.pdf}

Как капиллярное давление влияет на добычу?
Ниже ВНК и в подошве переходной зоны добывается только вода.
В переходной зоне добывается и нефть, и вода.
Когда капиллярка выходит на асимптоту, то это означает, что на этой глубине в пласте находится только связанная вода, т.е. будет идти добыча только нефти (безводной продукции).

\includegraphics[width=\textwidth, page=71]{Kurs_OsnovyGDM_Kai_774_gorodovSV_v6_0.pdf}

По виду капиллярной кривой можно судить об однородности коллектора и о размере пор.

Если поры достаточно широкие, то капиллярное давление поднимет флюид на небольшую высоту от зеркала свободной воды.

Для узких пор полка (практически постоянное значение на графике) по капиллярному давлению находится выше, чем для широких пор.

Для неоднородного коллектора нет полки по капиллярному давлению (плавный переход).

Т.е. по виду кривой капиллярного давления и по ступеньке на этой кривой можно судить о том, насколько пласт проницаемый или однородный/неоднородный по распределению пор по размерам.
\\

\textbf{Капиллярное давление для разных типов породы}

\includegraphics[width=\textwidth, page=72]{Kurs_OsnovyGDM_Kai_774_gorodovSV_v6_0.pdf}

Здесь ещё раз нарисованы 4 капиллярные кривые.
Представим, что у нас есть 4 пропластка с разными свойствами (песчаники 1, 2, 3 и 4).
Мы пробурили скважину и вскрыли эти 4 песчаника.
Показано, на какую высоту в этих пропластках поднялась вода. 

Вопрос: в каком из этих пропластков самые плохие свойства, т.е. самые тонкие капилляры?

Во втором песчанике мы видим, что вода поднялась на самый высокий уровень. Вспоминаем, что капиллярное давление (и соответственно уровень поднятия) обратно пропорционально радиусу капилляра. Т.е. во втором песчанике у нас самые тонкие поры.
Следовательно, во втором песчанике самый худший коллектор (самые узкие поры).


\end{document}