\begin{document}

\subsection{Что такое фазовые диаграммы? Нарисуйте фазовые диаграммы (PVT диаграммы)}

\includegraphics[width=\textwidth, page=98]{Kurs_OsnovyGDM_Kai_774_gorodovSV_v6_0.pdf}

Здесь представлена условная таблица (условное деление) на типы флюидов.
Есть разные типы: газ, жирный газ, газоконденсат, лёгкая нефть, тяжёлая нефть и т.д.

Фазовая диаграмма -- графическое отображение равновесного состояния физико-химической системы при условиях, отвечающих координатам рассматриваемой точки на диаграмме.

Другими словами, по фазовой диаграмме можно определить, в каком состоянии будет находиться определённый тип флюида при заданных значениях давления и температуры.

Вся область фазовой диаграммы делится на однофазные и двухфазные зоны.   

\includegraphics[width=\textwidth, page=99]{Kurs_OsnovyGDM_Kai_774_gorodovSV_v6_0.pdf}

На слайде представлена фазовая диаграмма.
\\

Сухой газ и в пласте находится в газообразном виде, и на забое, и по движению по стволу скважины он так и остаётся газом, и на сепараторе по-прежнему газом.
\\

Если и в пласте, и на забое газ, а при движении по стволу начинает выпадать жидкая фаза, то это жирный газ.
\\

Дальше жирность повышается -- от молока к сметане :)

От жирного газа переходим к газоконденсату.

Газоконденсат -- это такой флюид, который в пласте может находиться в газообразном состоянии, но при снижении давления вблизи скважины уже начинает происходить выделение жидкой фазы, начинает выпадать конденсат (и при движении по стволу скважины его всё больше и больше выделяется).
\\

По мере смещения налево переходим за критическую точку.
Соответственно, в пласте флюид находится в жидком состоянии, а по мере движения по скважине или вблизи забоя могут начинать выделяться газообразные составляющие.
\\

Чем левее находится кривая на диаграмме, тем более тяжёлая нефть.
\\

\textbf{Определение типа залежи по составу УВ}

\includegraphics[width=\textwidth, page=100]{Kurs_OsnovyGDM_Kai_774_gorodovSV_v6_0.pdf}

Если построим зависимость величины OGR, обратной к газосодержанию, от отношения лёгких компонент к тяжёлым, то внизу на графике будут находиться газовые и газоконденсатные залежи, а наверху -- нефтяные.


\end{document}