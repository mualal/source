\begin{document}

\subsection{Что такое ОФП? Раскрыть суть ОФП. Привести примеры наиболее популярных корреляцией для построения кривых относительных фазовых проницаемостей}

\includegraphics[width=\textwidth, page=75]{Kurs_OsnovyGDM_Kai_774_gorodovSV_v6_0.pdf}

Поверхностное натяжение (помимо капиллярного давления) приводит ещё к взаимному сопротивлению фильтрации нескольких флюидов.

Пример: у нас через образец только нефть течёт со скоростью 10 см$^3$/мин, а при пропускании через образец 50\% нефти и 50\% воды нефть течёт с гораздо меньшей скоростью 1.5 см$^3$/мин.

Это снижение проницаемости решили выражать некими функциями, которые зависят от насыщенности, и эти функции называются относительными фазовыми проницаемостями.
\\

Относительная фазовая проницаемость (ОФП) по флюиду 1 в присутствии флюида 2 -- это некий множитель (зависящий от насыщенности флюида 1) перед абсолютной проницаемостью, который позволяет найти эффективную проницаемость по флюиду 1 в присутствии флюида 2.
\\

В рассматриваемой на слайде ситуации (50\% воды и 50 \% нефти) из графиков зависимости ОФП от водонасыщенности видим, что эффективная проницаемость по воде будет составлять 5\% от абсолютной проницаемости, а эффективная проницаемость по нефти будет составлять 15\% от абсолютной проницаемости.
Видим, что общий поток тоже снизится по сравнению с пропусканием только одного флюида через образец.


\includegraphics[width=\textwidth, page=82]{Kurs_OsnovyGDM_Kai_774_gorodovSV_v6_0.pdf}

После проведения лабораторных исследований ОФП их обычно аппроксимируют некими функциями.
\\

Аппроксимация проводится с целью удобства: необходимо, чтобы ОФП были гладкими функциями.
Это позволяет легче находить решение при использовании численных схем.
\\

Наиболее популярные корреляции для построения кривых ОФП: корреляция Кори и корреляция LET.
Фактически подбираются настроечные параметры корреляций с целью наилучшего согласования с лабораторными исследованиями.
\\

Корреляция LET (появилась 15-20 лет; имеет 3 настроечных параметра) позволяет лучше описать лабораторные исследования, т.к. у неё есть участки с разной выпуклостью/вогнутостью.

\end{document}