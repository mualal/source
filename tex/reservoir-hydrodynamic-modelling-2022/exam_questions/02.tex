\begin{document}

\subsection{Какие режимы притока Вы знаете? Раскрыть суть каждого режима.}

\textbf{Режимы притока к скважине.}

\includegraphics[width=\textwidth, page=13]{1_Производительность_скважин_на_отправку.pdf}
\ \\

1) \textbf{Неустановившийся} режим притока к скважине -- давление и/или дебит изменяются во времени, т.е. случай, когда перераспределение давления ещё не достигло границ пласта и/или пока не проявляется влияние соседних скважин.

На неустановившемся режиме:
\beq
p=f(r,t)
\eeq
\ \\

2) \textbf{Установившийся} режим притока к скважине -- распределение давления в пласте, забойные давления и дебиты постоянны во времени.

Данный тип притока возможен только при поддержании постоянного давления на границе пласта, т.е. при наличии большой газовой шапки, активной законтурной области или в случае проведения мероприятий по поддержанию пластового давления.

На установившемся режиме:
\beq
\frac{\partial p}{\partial t}=0\text{ и }p_e=\text{const}
\eeq
\ \\


3) \textbf{Псевдоустановившийся} режим притока к скважине -- форма профиля давления постоянна во времени (профиль давления в пласте движется параллельно и линейно по времени).
Давление на границе снижается.
Данный режим притока характерен для изолированных пластов с непроницаемыми границами.

На псевдоустановившемся режиме:
\beq
\frac{\partial p}{\partial r}=0\left(\text{при } r=r_e\right)\text{ и }\frac{\partial p}{\partial t}=\text{const}
\eeq
\ \\

ГДИС почти всегда выполняются на неустановившемся режиме притока, даже если и начинает проявляться влияние границ пласта.
\\

Моделирование режимов притока к скважине доступно по ссылке: \href{https://github.com/mualal/source/blob/main/well_productivity/productivity.ipynb}{GO TO МОДЕЛИРОВАНИЕ РЕЖИМОВ ПРИТОКА К СКВАЖИНЕ}

\end{document}