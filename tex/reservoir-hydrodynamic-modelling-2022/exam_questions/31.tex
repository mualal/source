\begin{document}

\subsection{Что будет происходить с моделью при инициализации модели различными способами (неравновесный, равновесный, равновесный + начальный куб насыщенности)?}

Теория подробно раскрыта при ответе на предыдущий вопрос 30.
\\

При \textbf{неравновесном способе} с помощью ключевого слова SWAT задаются значения насыщенности на начальный момент времени во всех ячейках.
На начальный момент времени залежь не находится в равновесии, и после инициализации могут начаться перетоки даже в случае отсутствия скважин в модели.
\\

На практике обычно используют \textbf{равновесный способ инициализации}.

\includegraphics[width=\textwidth, page=130]{Kurs_OsnovyGDM_Kai_774_gorodovSV_v6_0.pdf}

Как происходит инициализация модели в случае равновесного способа?

Сначала вычисляется давление в нефтяной фазе (по формуле $\rho_o gh$) вверх и вниз от точки отсчёта (т.е. от уровня, равного значению первого параметра EQUIL).

Таким образом, получаем давление на заданном контакте.

Давление в водяной фазе на контакте получается отниманием капиллярного давления, заданного на контакте.

После этого вычисляем давление в водяной фазе (по формуле $\rho_w gh$) вверх и вниз от точки контакта.

Таким образом, в каждой ячейке есть давление в нефтяной фазе и давление в водяной фазе, а разница между этими давлениями -- это фактически капиллярное давление.

Дальше симулятор идёт в ключевое слово SWOF.
В этом ключевом слове заданы зависимости фазовых проницаемостей и капиллярного давления от насыщенности.
И симулятор таким образом находит для соответствующего значения капиллярного давления насыщенность и эту насыщенность задаёт в ячейках.

Т.е. в ячейках у симулятора рассчитаны давления в водяной и нефтяной фазах, их разница это капиллярные давления, а этим капиллярным давлениям можно сопоставить насыщенности (из таблицы), что и происходит.
\\

\textbf{Равновесный способ инициализации с соблюдением начальной насыщенности}

\includegraphics[width=\textwidth, page=131]{Kurs_OsnovyGDM_Kai_774_gorodovSV_v6_0.pdf}

А что происходит, если у нас кроме ключевого слова EQUIL задаётся ещё куб начальной водонасыщенности (например, мы делаем проектно-технологическую документацию ПТД и нам строго нужно соблюдать запасы, которые есть в геологической модели)?
\\

Геолог передаёт нам куб водонасыщенности, и мы его подключаем в модель с помощью SWATINIT.
При таком способе инициализации у симулятора получается две насыщенности: которую он сам рассчитал через условие равновесия и которая у него есть в кубе SWATINIT.
Что делать, если они не совпали?
Симулятор говорит, что он будет стараться настроить насыщенность так, чтобы она совпала с тем, что задано в кубе SWATINIT.
Для этого он будет масштабировать кривую капиллярного давления (т.е. просто растягивать или сжимать её по вертикали) таким образом, чтобы насыщенность в данной ячейке совпала с той, которая задана в ключевом слове SWATINIT.

Если насыщенность геологом рассчитана некорректно (т.е. неравновесно), то это может привести к тому, что масштабирования приведут к тому, что капиллярное давление будет слишком большим или слишком маленьким (и это один из критериев для проверки корректности инициализации, т.е. можно оценить диапазоны изменения куба капиллярного давления после инициализации и сравнить его с тем, что мы получали по исследованиям на керне).
\\

\end{document}