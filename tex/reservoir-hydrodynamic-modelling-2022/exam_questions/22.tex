\begin{document}

\subsection{Раскрыть суть решения уравнения фильтрации явным методом. Опишите основные шаги}

Дифференциальное уравнение в частных производных для линейной одномерной горизонтальной фильтрации жидкости (с учётом постоянной проницаемости, вязкости и сжимаемости жидкости) имеет вид:
\beq
\frac{\partial^2 p}{\partial x^2}=\left(\frac{\varphi\mu c}{k}\right)\frac{\partial p}{\partial t}
\eeq

Данное уравнение можно решить численно, используя конечно-разностную аппроксимацию для двух дифференциальных составляющих:
\beq
\frac{\partial^2 p}{\partial x^2}\text{ и }\frac{\partial p}{\partial t}
\eeq

Для начала пространство по оси $x$ разделяем на некоторое количество ячеек, а время делим на некоторые конечные временные интервалы (шаги).
\\

При использовании явного метода: для того, чтобы получить значения давлений в ячейках на следующем временном шаге, достаточно воспользоваться явной формулой.
\\


Описание основных шагов явного метода доступно по ссылке: \href{https://docs.yandex.ru/docs/view?url=ya-disk-public\%3A\%2F\%2FWicxck3PRHBvGgXBKaNry\%2FRtiluR1m6cR6Bi14xdyyNQDBM0D9VXsHr3EpLffWemDqZvSgIch5AN9ddz7ydViQ\%3D\%3D\%3A\%2F\%D0\%92\%D1\%82\%D0\%BE\%D1\%80\%D0\%BE\%D0\%B9\%20\%D0\%BA\%D1\%83\%D1\%80\%D1\%81\%20\%D0\%BC\%D0\%B0\%D0\%B3\%D0\%B8\%D1\%81\%D1\%82\%D1\%80\%D0\%B0\%D1\%82\%D1\%83\%D1\%80\%D1\%8B\%2F\%D0\%93\%D0\%B8\%D0\%B4\%D1\%80\%D0\%BE\%D0\%B4\%D0\%B8\%D0\%BD\%D0\%B0\%D0\%BC\%D0\%B8\%D1\%87\%D0\%B5\%D1\%81\%D0\%BA\%D0\%BE\%D0\%B5\%20\%D0\%BC\%D0\%BE\%D0\%B4\%D0\%B5\%D0\%BB\%D0\%B8\%D1\%80\%D0\%BE\%D0\%B2\%D0\%B0\%D0\%BD\%D0\%B8\%D0\%B5\%2FExplicit\%20Numerical\%20Solution\%20of\%20One-Dimensional\%20Filtration\%20in\%20a\%20Reservoir.pdf&name=Explicit\%20Numerical\%20Solution\%20of\%20One-Dimensional\%20Filtration\%20in\%20a\%20Reservoir.pdf&nosw=1}{GO TO ЯВНЫЙ МЕТОД}

\textbf{Здесь тоже повторю эти шаги.}
\\

Конечно-разностные уравнения записываются таким образом, чтобы можно было получить значения зависимых параметров по всей сеточной области.
Уравнения в частных производных заменяют их конечно-разностными эквивалентами.
Получить конечно-разностные уравнения можно, используя метод разложения функции в ряд Тейлора в заданной точке и решая уравнения относительно искомой производной.

Аппроксимация функции $f(x)$ в точке $x$:
\beq
f(x+h)=f(x)+\frac{h}{1!}f'(x)+\frac{h^2}{2!}f''(x)+\frac{h^3}{3!}f'''(x)+...
\eeq
\ \\

\textbf{Аппроксимация дифференциальной составляющей по пространству.}

При постоянном времени $t$ можно записать аппроксимации для давления при движении вперёд и назад по оси $x$ в следующем виде:
\beq
p(x+\Delta x, t)=p(x,t)+\frac{\Delta x}{1!}p'(x,t)+\frac{(\Delta x)^2}{2!}p''(x,t)+\frac{(\Delta x)^3}{3!}p'''(x,t)+...
\eeq

\beq
p(x-\Delta x, t)=p(x,t)+\frac{-\Delta x}{1!}p'(x,t)+\frac{(-\Delta x)^2}{2!}p''(x,t)+\frac{(-\Delta x)^3}{3!}p'''(x,t)+...
\eeq

Складывая эти два равенства и выражая производную второго порядка, получаем:
\beq
p''(x,t)=\frac{p(x+\Delta x,t)-2p(x,t)+p(x-\Delta x,t)}{(\Delta x)^2}+\frac{(\Delta x)^2}{12}p''''(x,t)+...
\eeq

Или, применяя индексацию для ячеек и обозначив верхним индексом временной интервал, получим:
\beq
\left(\frac{\partial^2 p}{\partial x^2}\right)_{\!\!i}^{\!\!t}=\frac{p_{i+1}^t-2p_i^t+p_{i-1}^t}{(\Delta x)^2}+o(\Delta x^2)
\eeq
\ \\

\textbf{Аппроксимация дифференциальной составляющей по времени.}

В постоянной точке пространства $x$ можно записать аппроксимацию для давления вперёд по оси времени в следующем виде:
\beq
p(x, t+\Delta t)=p(x,t)+\frac{\Delta t}{1!}p'(x,t)+\frac{(\Delta t)^2}{2!}p''(x,t)+\frac{(\Delta t)^3}{3!}p'''(x,t)+...
\eeq

Выражая первую производную по времени, получим:
\beq
p'(x,t)=\frac{p(x,t+\Delta t)-p(x,t)}{\Delta t}+\frac{\Delta t}{2}p''(x,t)+...
\eeq

Или, применяя индексацию для ячеек:
\beq
\left(\frac{\partial p}{\partial t}\right)_{\!\!i}^{\!\!t}=\frac{p_i^{t+\Delta t}-p_i^t}{\Delta t}+o(\Delta t)
\eeq

\textbf{Явная запись решения ДУ.}

Подставляя полученные выше аппроксимации в первоначальное дифференциальное уравнение притока, получим:
\beq
\frac{p_{i+1}^t-2p_i^t+p_{i-1}^t}{(\Delta x)^2}=\left(\frac{\varphi\mu c}{k}\right)\frac{p_i^{t+\Delta t}-p_i^t}{\Delta t},
\eeq
где $i=2,...,N-1$.

Получена основная формула явного метода:
\beq
p_i^{t+\Delta t}=p_i^t+\left(\frac{\Delta t}{\Delta x^2}\right)\left(\frac{k}{\varphi\mu c}\right)\left(p_{i+1}^t-2p_i^t+p_{i-1}^t\right),
\eeq
где $i=2,...,N-1$.

Дополнительно из граничных условий получаем выражения при $i=1$ и $i=N$ (эти выражения получаются аналогичным способом из рядов Тейлора, но стоит учесть, что граничные условия заданы для граней ячейки, поэтому расчёт проводить для $h=\Delta x/2$):
\beq
\begin{cases}
p_1^{t+\Delta t}=p_1^t+\dfrac{4}{3}\left(\dfrac{\Delta t}{\Delta x^2}\right)\left(\dfrac{k}{\varphi\mu c}\right)\left(p_2^t-3p_1^t+2p_{\text{left}}\right)\\\\
p_N^{t+\Delta t}=p_N^t+\dfrac{4}{3}\left(\dfrac{\Delta t}{\Delta x^2}\right)\left(\dfrac{k}{\varphi\mu c}\right)\left(2p_{\text{right}}-3p_N^t+p_{N-1}^t\right)
\end{cases}
\!\!\!\!\!\!(\text{ГУ постоянного давления})
\eeq

\beq
\begin{cases}
p_1^{t+\Delta t}=p_1^t+\left(\dfrac{\Delta t}{\Delta x^2}\right)\left(\dfrac{k}{\varphi\mu c}\right)\left(p_2^t-p_1^t+Q_{\text{left}}\dfrac{\mu}{Ak}\Delta x\right)\\\\
p_N^{t+\Delta t}=p_N^t+\left(\dfrac{\Delta t}{\Delta x^2}\right)\left(\dfrac{k}{\varphi\mu c}\right)\left(p_{N-1}^t-p_N^t-Q_{\text{right}}\dfrac{\mu}{Ak}\Delta x\right)
\end{cases}
\!\!\!\!\!\!(\text{ГУ постоянного расхода})
\eeq


\end{document}