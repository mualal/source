\begin{document}

\subsection{Раскрыть суть решения уравнения фильтрации явным методом. Опишите основные шаги}

Дифференциальное уравнение в частных производных для линейной одномерной горизонтальной фильтрации жидкости (с учётом постоянной проницаемости, вязкости и сжимаемости жидкости) имеет вид:
\beq
\frac{\partial^2 p}{\partial x^2}=\left(\frac{\varphi\mu c}{k}\right)\frac{\partial p}{\partial t}
\eeq

Данное уравнение можно решить численно, используя конечно-разностную аппроксимацию для двух дифференциальных составляющих:
\beq
\frac{\partial^2 p}{\partial x^2}\text{ и }\frac{\partial p}{\partial t}
\eeq

Для начала пространство по оси $x$ разделяем на некоторое количество ячеек, а время делим на некоторые конечные временные интервалы (шаги).
\\

При использовании явного метода: для того, чтобы получить значения давлений в ячейках на следующем временном шаге, требуется воспользоваться явной формулой.
\\


Описание основных шагов явного метода доступно по ссылке: \href{https://docs.yandex.ru/docs/view?url=ya-disk-public\%3A\%2F\%2FWicxck3PRHBvGgXBKaNry\%2FRtiluR1m6cR6Bi14xdyyNQDBM0D9VXsHr3EpLffWemDqZvSgIch5AN9ddz7ydViQ\%3D\%3D\%3A\%2F\%D0\%92\%D1\%82\%D0\%BE\%D1\%80\%D0\%BE\%D0\%B9\%20\%D0\%BA\%D1\%83\%D1\%80\%D1\%81\%20\%D0\%BC\%D0\%B0\%D0\%B3\%D0\%B8\%D1\%81\%D1\%82\%D1\%80\%D0\%B0\%D1\%82\%D1\%83\%D1\%80\%D1\%8B\%2F\%D0\%93\%D0\%B8\%D0\%B4\%D1\%80\%D0\%BE\%D0\%B4\%D0\%B8\%D0\%BD\%D0\%B0\%D0\%BC\%D0\%B8\%D1\%87\%D0\%B5\%D1\%81\%D0\%BA\%D0\%BE\%D0\%B5\%20\%D0\%BC\%D0\%BE\%D0\%B4\%D0\%B5\%D0\%BB\%D0\%B8\%D1\%80\%D0\%BE\%D0\%B2\%D0\%B0\%D0\%BD\%D0\%B8\%D0\%B5\%2FExplicit\%20Numerical\%20Solution\%20of\%20One-Dimensional\%20Filtration\%20in\%20a\%20Reservoir.pdf&name=Explicit\%20Numerical\%20Solution\%20of\%20One-Dimensional\%20Filtration\%20in\%20a\%20Reservoir.pdf&nosw=1}{GO TO ЯВНЫЙ МЕТОД}

\end{document}