\begin{document}

\subsection{Что такое радиус Писмана? Раскройте суть радиуса Писмана.}

\includegraphics[width=\textwidth, page=125]{Kurs_OsnovyGDM_Kai_774_gorodovSV_v6_0.pdf}

Формула для расчёта притока к скважине в модели немного отличается от стандартной формулы расчёта радиального притока.
Отличается в плане того, что в обычной формуле мы берём радиус контура питания и давление на этом контуре питания, а в модели мы рассчитываем приток в скважину из той ячейки (или из тех ячеек), которые скважина вскрывает.
И соответственно вместо радиуса контура питания берётся такой радиус от скважины, на котором давление равно среднему давлению в ячейке (так называемый радиус Писмана).
Также в этой формуле будет немого отличаться скин-фактор.
В итоге, эти 3 параметра (радиус контура питания, давление на контуре питания, скин-фактор) отличаются, но так, чтобы дебит, который рассчитан по формуле компьютерной модели, совпадал с дебитом, рассчитанном по формуле радиального притока.

\includegraphics[width=\textwidth, page=126]{Kurs_OsnovyGDM_Kai_774_gorodovSV_v6_0.pdf}

На слайде представлены формулы для радиуса Писмана.
\\

Обычно у нас пласт изотропный, проницаемости по $x$ и по $y$ одинаковы, размеры ячеек тоже примерно одинаковы, поэтому можно считать, что радиус Писмана $r_p\approx0.2\Delta x$, т.е. для 100-метровой ячейки на 20 метрах от скважины давление будет равно среднему давлению в ячейке.


\end{document}