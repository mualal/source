\begin{document}

\subsection{Как задавать параметры скважин в T-Navigator? Как можно задать контроль на скважинах в T-Navigator?}

\includegraphics[width=\textwidth, page=138]{Kurs_OsnovyGDM_Kai_774_gorodovSV_v6_0.pdf}

Здесь перечислены ключевые слова для задания истории работы скважин (а также контроль на скважинах при воспроизведении истории -- т.е. расчёт производится для участка времени с уже имеющейся историей эксплуатации).
\\

Всё, что написано после ключевого слова END, симулятор не читает.
\\

\textbf{Подготовка и проведение прогнозных расчётов}

\includegraphics[width=\textwidth, page=179]{Kurs_OsnovyGDM_Kai_774_gorodovSV_v6_0.pdf}

На данном слайде указано, как задавать контроль по скважинам в случае проведения прогнозных расчётов (для будущего времени, для которого ещё нет исторических данных).

Когда адаптировали на историю, были ключевые слова WCONHIST для добывающих скважин и WCONINJH для нагнетательных скважин.

А здесь (на прогноз) для добывающих скважин используется ключевое слово WCONPROD, для нагнетательных скважин используется ключевое слово WCONINJE.

Можно контроль на скважинах задавать по дебитам, по забойному давлению; на нагнетательных соответственно -- по приёмистости.
Также можно осуществлять контроль по устьевому давлению, но тогда нужно будет ещё добавить VFP-таблицы, в которых описано, как меняется давление по стволу скважины при различных режимах течения, т.е. грубо говоря, какие потери давления будут по стволу скважины (это нужно для того, чтобы симулятор пересчитал забойное давление; от устьевого до забойного пересчитал потери).
Есть специальные программы, в которых формируются VFP-таблицы, т.е. можно их заранее создать.
\\

Можно также указывать групповой контроль GCONPROD.
Например, это может быть полезно, если у нас заданы какие-то ограничения по отборам на дожимной насосной станции ДНС.
Т.е. эксплуатируем скважины так: сколько притечёт, столько притечёт, но сверху есть ограничение, что группа скважин не может добывать больше какой-то величины (соответственно это можно в групповом контроле указать).

Также и для нагнетательных скважин можно указать групповой контроль GCONINJE.
Здесь есть ещё опция compens (обеспечить компенсацию), т.е. можно задать, чтобы какая-то группа нагнетательных скважин обеспечивала компенсацию по какой-то группе добывающих скважин.
Можно также сделать и по месторождению, что все нагнетательные скважины должны нагнетать столько, чтобы обеспечить 100\% (или 120\% или сколько укажете) компенсацию (тогда просто режим эксплуатации нагнетательных скважин будет подбираться так, чтобы эту компенсацию обеспечить -- вручную не нужно будет подбирать -- всё будет сделано автоматически).


\end{document}