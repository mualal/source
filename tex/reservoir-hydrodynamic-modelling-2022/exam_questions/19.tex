\begin{document}

\subsection{Какие PVT свойства жидкостей вы знаете? Какие основные PVT свойства необходимо задавать в модели? Что такое сжимаемость?}

\includegraphics[width=\textwidth, page=101]{Kurs_OsnovyGDM_Kai_774_gorodovSV_v6_0.pdf}

Для модели нелетучей нефти (Black Oil) задаются PVT-свойства, представленные на слайде.

Давление насыщения нефти -- давление, выше которого весь газ уже растворился, нефть остаётся недонасыщенной.
При снижении давления ниже давления насыщения из нефти начинает выделяться газ.

Газосодержание характеризует количество газа, растворённого в нефти.

Объёмный коэффицент -- это отношение объёма флюида в пласте к его объёму на поверхности.

Вязкость показывает, как хорошо флюид течёт (какие у него силы внутреннего трения, которые мешают ему течь).
\\

Сжимаемость флюида характеризует, насколько сильно изменяется его объём при воздействии на него давлением.

Сжимаемость:
\beq
c=-\frac{1}{V}\left(\frac{\partial V}{\partial p}\right)_{T=\text{const}}
\eeq

Сжимаемость нефти:
\beq
c_o=-\frac{1}{V_o}\left(\frac{\partial V_o}{\partial p}\right)_{T=\text{const}}=-\frac{1}{\rho_o}\left(\frac{\partial \rho_o}{\partial p}\right)_{T=\text{const}}=-\frac{1}{B_o}\left(\frac{\partial B_o}{\partial p}\right)_{T=\text{const}}
\eeq

Общая сжимаемость:
\beq
c_t=S_oc_o+S_wc_w+S_gc_g+c_{\text{породы}}
\eeq

В представленных выше формулах:
$V$ -- объём флюида в пласте (м$^3$);
$\rho$ -- плотность флюида (кг/м$^3$);
$B$ -- объёмный фактор флюида;
$c_w$ -- сжимаемость воды (атм$^{-1}$);
$c_o$ -- сжимаемость нефти (атм$^{-1}$);
$c_g$ -- сжимаемость газа (атм$^{-1}$);
$S_w$ -- водонасыщенность;
$S_o$ -- нефтенасыщенность;
$S_g$ -- газонасыщенность;
$c_\text{породы}$ -- сжимаемость породы;
$S_w+S_o+S_g=1$.

Сжимаемость $c$ -- относительное изменение объёма флюида на единицу изменения давления.
Единица измерения сжимаемости -- величина обратная давлению.
\\

Изотермическая сжимаемость нефти выше давления насыщения $c_o$ всегда величина положительная, так как объём недонасыщенной жидкости уменьшается при увеличении давления.

$c_o$ определяется в лаборатории по экспериментальным данным или с помощью корреляции Трубэ.

Для жидкости $c_f$ приблизительно можно считать постоянной, т.е. $c_f$ не зависит от давления.
\\

Общая сжимаемость системы $c_t$, кроме изменения объёма нефти, учитывает расширение пластовой воды и свободного газа, а также уменьшение объёма пор (за счёт сжатия породы).
\\

Типичные величины сжимаемости для трёх фазовых компонент при среднем давлении $p=136\text{ атм}$:

$c_o=2.2\cdot 10^{-4} 1/\text{атм}$

$c_w=4.4\cdot 10^{-5} 1/\text{атм}$

$c_g=7.3\cdot 10^{-3} 1/\text{атм}$
\\

Сжимаемость газа на порядок выше, чем сжимаемость жидкостей или породы.
В газовых залежах принято считать, что $c_t\approx c_g$.



\end{document}