\begin{document}

\subsection{Какие PVT свойства жидкостей вы знаете? Какие основные PVT свойства необходимо задавать в модели? Что такое сжимаемость?}

\includegraphics[width=\textwidth, page=101]{Kurs_OsnovyGDM_Kai_774_gorodovSV_v6_0.pdf}

Для модели нелетучей нефти (Black Oil) задаются PVT-свойства, представленные на слайде.

Давление насыщения нефти -- давление, выше которого весь газ уже растворился, нефть остаётся недонасыщенной.
При снижении давления ниже давления насыщения из нефти начинает выделяться газ.

Газосодержание характеризует количество газа, растворённого в нефти.

Объёмный коэффицент -- это отношение объёма флюида в пласте к его объёму на поверхности.

Вязкость показывает, как хорошо флюид течёт (какие у него силы внутреннего трения, которые мешают ему течь).

Сжимаемость флюида характеризует, насколько сильно изменяется его объём при воздействии на него давлением.

\end{document}