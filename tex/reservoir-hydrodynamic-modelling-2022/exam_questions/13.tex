\begin{document}

\subsection{Что такое функция Леверетта? Раскройте суть функции Леверетта.}

\includegraphics[width=\textwidth, page=73]{Kurs_OsnovyGDM_Kai_774_gorodovSV_v6_0.pdf}

Когда у нас есть несколько капиллярных кривых, то нужно как-то их перенести в модель.
Есть такая J-функция Леверетта, с помощью которой эти кривые можно нормализовать, осреднить и использовать дальше в расчёте.
Что делается? Капиллярные кривые взвешиваются на $\sigma \cos\theta$ и на корень из отношения проницаемости и пористости.
\\

$\sqrt{\dfrac{k}{\varphi}}$ характеризует извилистость поровых каналов.
Коллекторы примерно с одним и тем же строением (с одной и той же извилистостью) будут иметь похожее поведение, поэтому взвешивание капиллярных кривых на эту величину позволяет нам кривые осреднить и отслеживать только их характеристику, связанную с описанием толщины каналов.
\\

Рассчитываем значения J-функции Леверетта, и далее строим график в зависимости от водонасыщенности, подобный представленному справа: отмечаем подсчитанные точки и аппроксимируем их некой зависимостью (которую в дальнейшем будем использовать в расчётах ГДМ модели).
\\

На графике могут получиться не одно облако точек, а два или три (если есть несколько пластов с разными характеристиками или разные блоки на месторождении, в каждом из которых получился свой тип коллектора вследствие разных геологических процессов).
Тогда будет несколько аппроксимирующих кривых, которые можно использовать отдельно для каждого рассматриваемого пласта или блока соответственно.


\end{document}