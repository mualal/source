\begin{document}

\subsection{Что такое уравнение Дарси, как выводится уравнение в общем виде?}

Уравнение Дарси (закон Дарси) -- закон фильтрации жидкостей и газов в пористой среде.

Исторически закон был получен Анри Дарси экспериментально, но может быть получен с помощью осреднения уравнений Навье-Стокса, описывающих течение в масштабе пор (в настоящее время есть доказательства для пористых сред с периодической и случайной микроструктурой).

Уравнение Дарси выражает зависимость скорости фильтрации флюида от градиента напора:
\beq
\frac{q}{A}=-\frac{k}{\mu}\frac{dp}{dx},
\eeq
где $q$ -- расход (м$^3$/с), $A$ -- площадь (м$^2$), $k$ -- проницаемость (м$^2\approx 10^{12}$ Д), $\mu$ -- вязкость (Па$\cdot$с),

$p$ -- давление (Па), $x$ -- расстояние вдоль направления фильтрации (м).
В скобках указаны единицы измерения в СИ.

\textbf{Первое упоминание.}
В 1856 году в работе Дарси "<Les fontaines publiques de la ville de Dijon. Paris 1856"> (Общественные фонтаны города Дижон. Париж 1856) опубликованы результаты опытов по фильтрации воды в песке.
\\

Закон Дарси в интегральной форме:
\beq
Q=-\frac{kA}{\mu L}\Delta p
\eeq

Закон Дарси в дифференциальной форме:
\beq\label{DiffDarcyExam}
dp=-\frac{q}{A}\frac{\mu}{k}dx
\eeq

Основные допущения закона Дарси:

1) постоянный дебит;

2) ламинарное течение;

3) гомогенная среда фильтрации;

4) поровое пространство насыщено одной фазой;

5) отсутствие химического взаимодействия между породой и флюидом
\\

\textbf{Область применимости закона Дарси.}
Закон Дарси применим для фильтрации жидкостей, подчиняющихся закону вязкого трения Ньютона (закону Навье-Стокса).

Для фильтрации неньютоновских жидкостей (например, некоторых нефтей) связь между градиентом давления и скоростью фильтрации может быть нелинейной или вообще неалгебраической (например, дифференциальной).

Для ньютоновских жидкостей область применения закона Дарси ограничивается малыми скоростями фильтрации (числа Рейнольдса, рассчитанные по характерному размеру пор, меньше или порядка единицы).
При больших скоростях зависимость между градиентом давления и скоростью фильтрации нелинейна (хорошее совпадение с экспериментальными данными даёт квадратичная зависимость -- закон фильтрации Форхгеймера).

\end{document}