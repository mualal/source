\begin{document}

\subsection{Вывод уравнения пьезопроводности для "<упругого пласта">.}

Для \textbf{упругого изотропного пласта} можем записать известные соотношения пороупругости:
\begin{itemize}
\item на тензор полных напряжений \\
\beq
\pmb{T}=\sigma^0\pmb{I}+\left(\lambda I_1(\pmb{\varepsilon})-b\Delta p\right)\pmb{I}+2\mu\pmb{\varepsilon},
\eeq
где $\pmb{T}$ -- тензор полных напряжений; $\pmb{I}$ -- единичный тензор; $\pmb{\varepsilon}$ -- тензор полных деформаций; $\lambda=K-2G/3$ и $\mu=G$ -- константы (параметры) Ляме; $K$ -- модуль всестороннего сжатия; $G$ -- модуль сдвига; $I_1(\pmb{\varepsilon})$ -- след тензора полных деформаций; $b$ -- константа Био; $\Delta p$ -- изменение давления; $\sigma^0$ -- начальное напряжение
\item на пористость \\
\beq
\varphi = \varphi_0+bI_1(\pmb{\varepsilon})+\dfrac{1}{N}\Delta p,
\eeq
где $\varphi_0$ -- начальная пористость; $b$ -- константа Био; $I_1(\pmb{\varepsilon})$ -- след тензора полных деформаций; $N$ -- модуль Био; $\Delta p$ -- изменение давления.
\item условие равновесия \\
\beq
\pmb{\nabla}\cdot\pmb{T}=\pmb{0}
\eeq
\end{itemize}

Для \textbf{флюида} запишем:
\begin{itemize}
	\item закон Дарси \\
	\beq
	\pmb{W}=-\frac{k}{\mu_f}\cdot\pmb{\nabla} p,
	\eeq
	где $\pmb{W}=\varphi\left(\pmb{v_f}-\pmb{v_s}\right)$; $k$ -- проницаемость пласта; $\mu_f$ -- вязкость флюида; $\pmb{\nabla} p$ -- градиент давления
	\item соотношение на сжимаемость флюида
	\beq
	p-p_0=K_f\frac{\rho_f-\rho_f^0}{\rho_f^0},
	\eeq
	где $\dfrac{1}{K_f}$ -- сжимаемость флюида ($K_f$ -- объёмный модуль упругости флюида)
	\item уравнение неразрывности потока при отсутствии источникового слагаемого (уравнение переноса массы, записанное в дифференциальной форме):
	\beq
	\frac{\partial\left(\rho_f\varphi\right)}{\partial t}+\pmb{\nabla}\cdot\left(\rho_f\varphi\,\pmb{v_f}\right)=0
	\eeq
\end{itemize}

Из \textbf{уравнения неразрывности} получаем:
\beq
\varphi_0\frac{\partial\rho_f}{\partial t}+\rho_0\frac{\partial\varphi}{\partial t}+\rho_0\pmb{\nabla}\cdot\pmb{W}+\rho_0\varphi_0\frac{\partial I_1(\pmb{\varepsilon})}{\partial t}=0,
\eeq
где
\beq
\frac{\partial I_1(\pmb{\varepsilon})}{\partial t}\equiv\pmb{\nabla}\cdot\pmb{v_s}
\eeq

А дальше через ряд свёрток и всяких операций получаем:
\beq
b\dot{I}_1(\pmb{\varepsilon})+\left(\frac{1}{N}+\frac{\varphi_0}{K_f}\right)\dot{p}=\frac{k}{\mu_f}\pmb{\nabla}^2p
\eeq

В осесимметричном случае при условии отсутствия деформации на бесконечности получаем:
\beq
\dot{p}=a\pmb{\nabla}^2p,
\eeq
где
\beq
a=\frac{kM}{\mu_f}\text{ и } M=\frac{b\left(b+\varphi\right)}{\lambda+2\mu}+\frac{1}{N}+\frac{\varphi}{K_f}
\eeq


\end{document}