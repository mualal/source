\begin{document}

\subsection{Что такое адаптация модели? Опишите основные шаги адаптации модели.}

\textbf{Адаптация модели}

\includegraphics[width=\textwidth, page=149]{Kurs_OsnovyGDM_Kai_774_gorodovSV_v6_0.pdf}

После указания всех данных в модели можем поставить её на расчёт и обнаружить, что результат расчёта не совпал с фактическими замерами, которые мы в модель занесли.

Почему это происходит?

Во-первых, данных недостаточно.

Во-вторых, имеющиеся данные обладают неопределённостью: у нас данные точечные (только по скважинам), а в межскважинном пространстве геолог стохастическими методами (или ещё как-то) распределил свойства -- даже в скважинах измеренные данные обладают погрешностью, а в межскважинном пространстве эта погрешность тем более есть, что приводит к несовпадению результатов расчёта с фактом.

Перед тем как использовать построенную модель для прогнозов её нужно настроить на факт.
Другими словами, необходимо провести адаптацию модели, т.е. изменить значения параметров модели, обладающих неопределённостью, с целью минимизации отклонений расчётных значений параметров работы скважин и месторождения от фактически замеренных.

Но существует бесконечное множество сочетаний параметров модели, при которых результат расчёта этой модели будет с заданной точностью совпадать с фактом, замеренным по скважинам.

Обычный подход к адаптации -- это ручная корректировка параметров модели на основе инженерного опыта, представления о физике пласта. Но есть и программные комплексы, позволяющие осуществлять автоматизированную адаптацию на основе оптимизационных алгоритмов.

Задача оптимизационных алгоритмов: варьируя параметры модели, устремить целевую функцию к нулю.
Тоже есть много нюансов, как эти алгоритмы настроить, как задать целевую функцию и так далее.
И ещё сами алгоритмы не контролируют физическую адекватность решения (полученные решения нужно перепроверять).
\\

\textbf{Обратные задачи}

\includegraphics[width=\textwidth, page=150]{Kurs_OsnovyGDM_Kai_774_gorodovSV_v6_0.pdf}

Есть несколько слайдов про обратные задачи.

Есть несколько точек. Вопрос: как построить аппроксимацию?

Здесь ясно, как аппроксимировать имеющиеся замеры.

\includegraphics[width=\textwidth, page=151]{Kurs_OsnovyGDM_Kai_774_gorodovSV_v6_0.pdf}

Но если есть шум (данные обладают погрешностью) и несколько размерностей (у модели много параметров), то задача подбора нужной поверхности становится нетривиальной и может иметь бесконечное множество разумных решений.

В этом случае очень сложно вручную подобрать параметры; необходимо осуществлять автоадаптацию и проверять найденные решения на разумность и физичность с точки зрения рассматриваемой гидродинамической модели.
\\

\textbf{Уточнение распределений параметров при адаптации модели}

\includegraphics[width=\textwidth, page=157]{Kurs_OsnovyGDM_Kai_774_gorodovSV_v6_0.pdf}

Можно воспринимать адаптацию модели, как уточнение исходных распределений параметров, т.е. на начальный момент времени у нас есть параметры, обладающие неопределённостями в каком-то диапазоне, но сами виды распределений (какие значения параметра наиболее вероятны или менее вероятны) мы не знаем.

Из-за отсутствия информации о виде распределения обычно задают равномерное распределение возможных значений параметра в заданном диапазоне.
Далее проводится расчёт модели со значениями параметров в рассматриваемых диапазонах, и мы видим, что какие-то из результатов расчётов не будут соответствовать фактическим замерам (даже с учётом допустимой погрешности).
Это позволит нам сузить диапазоны вариации исходных данных и уточнить виды распределения.
Например, от равномерных распределений можем прийти к нормальным или треугольным распределениям.

Другими словами, адаптацию можно рассматривать в качестве проверки, в каких диапазонах исходные данные (значения параметров) могут находиться и какие значения этих параметров наиболее вероятны.

\textbf{Алгоритм проведения автоадаптации}

\includegraphics[width=\textwidth, page=158]{Kurs_OsnovyGDM_Kai_774_gorodovSV_v6_0.pdf}

На этом слайде представлен алгоритм проведения автоматизированной адаптации.

Сначала мы производим расчёт базовой модели, выбираем диапазоны изменения параметров, которые обладают неопределённостью.
Делаем несколько расчётов со значениями параметров в этих диапазонах и смотрим, какие из этих параметров оказывают наибольшее влияние на результат расчёта и какие из диапазонов мы можем сузить по результатам этих первых нескольких расчётов.

Таким образом, дальше мы сокращаем количество параметров, которые будут участвовать в оптимизации (это делается для того, чтобы сократить количество необходимых расчётов, поскольку чем больше параметров будут участвовать, тем более многомерное пространство поиска решения у нас будет -- большее количество расчётов модели потребуется и большее время для того, чтобы оптимизационные алгоритмы сошлись).
Т.е. мы по результатам первых нескольких расчётов сокращаем количество параметров и сужаем их диапазоны.

Дальше задаём целевую функцию, задаём оптимизационные алгоритмы и запускаем на расчёт (идём в отпуск или на обед -- смотря сколько времени считается модель).
Будет проведено около нескольких десятков или сотен расчётов для того, чтобы целевая функция устремилась к нашим минимальным значениям.
\\

\textbf{Программы автоадаптации}

\includegraphics[width=\textwidth, page=159]{Kurs_OsnovyGDM_Kai_774_gorodovSV_v6_0.pdf}

Сейчас программы автоадаптации используются в качестве вспомогательного инструмента, чтобы быстрее найти решение / сузить диапазоны поиска значений параметров (а для абсолютно полной автоматической адаптации такие программы обычно сейчас не используются).

После проведения автоадаптации всё равно необходимо проводить дополнительный анализ на физичность / геологичность найденных сочетаний параметров.
Другими словами, на данный момент программы автоадаптации решают чисто оптимизационную задачу и не способны самостоятельно учесть всевозможные нефизичности найденных сочетаний параметров.
\\

Но есть проекты когнитивной автоадаптации, в которых пытаются контролировать физическую / геологическую обоснованность всех параметров и их сочетаний в автоматическом режиме.

Адаптация модели является самым времязатратным периодом работы с моделью (может занимать несколько месяцев работы до окончательной настройки модели).

\end{document}