\begin{document}

\subsection{Раскрыть суть уравнения Дарси-Дюпюи, как выводится уравнение? Как получен коэффициент в формуле Дарси-Дюпюи, равный 18.41?}

Приравняв значение потоковой скорости $\dfrac{Q}{2\pi rh}$, найденное из цилиндрической геометрии пласта, к значению $\dfrac{k}{\mu}\dfrac{dP}{dr}$, найденному из закона Дарси, получим дифференциальное уравнение радиального притока флюида к скважине.

Дюпюи тоже составил и решил это дифференциальное уравнение для случая границы в виде цилиндрической области (для радиального режима течения):
\beq\label{DupuyExam}
\frac{Q}{A}=\frac{k}{\mu}\frac{dP}{dx}\Rightarrow\frac{Q}{2\pi h}\int\limits_{r_w}^{r_e}{\frac{dr}{r}}=\frac{k}{\mu}\int\limits_{P_w}^{P_e}{dp}\Rightarrow Q=\frac{2\pi kh}{\mu}\frac{P_e-P_w}{\ln{\left(\dfrac{r_e}{r_w}\right)}}
\eeq

Формула получена в СИ.
При пересчёте в промысловые единицы измерения формула Дюпюи примет следующий вид:
\beq
Q=\frac{kh}{18.41\cdot\mu}\,\frac{P_e-P_w}{\ln{\left(\dfrac{r_e}{r_w}\right)}}
\eeq

Т.е. коэффициент 18.41 появляется при пересчёте из СИ в промысловые единицы измерения.
\\

Как найти этот коэффициент?

Зададим значения параметров:
\begin{itemize}[parsep=-5pt]
	\item $k=1\text{ м}^2=10^{12}\text{ Дарси}\approx10^{15}\text{ мДарси}$
	\item $h=1\text{ м}$
	\item $\mu=1\text{ Па}\cdot\text{с}=10\text{ Пуаз}=10^3\text{ сПуаз}$
	\item $P_e-P_w=1\text{ Па}\approx10^{-5}\text{ атм}$
	\item $r_e=e\text{ м}$
	\item $r_w=1\text{ м}$
\end{itemize}

Приравняем формулы в СИ и в промысловых единицах (с неизвестным коэффициентом пересчёта $\alpha$) при заданных значениях параметров (учитывая, что в СИ получаем результат в м$^3$/с, а в промысловых единицах -- в м$^3$/сут):
\beq
86400\cdot\underbrace{\frac{2\cdot3.141593\cdot1\cdot 1}{1}\cdot\frac{1}{1}}_{\text{в СИ}}\approx\underbrace{\frac{10^{15}\cdot 1}{\alpha\cdot 10^3}\cdot\frac{10^{-5}}{1}}_{\substack{\text{в промысловых}\\\text{единицах}}}
\eeq

Тогда
\beq
\alpha\approx\frac{10^{7}}{86400\cdot2\cdot3.141593}\approx18.42
\eeq

Получили небольшое отклонение во втором знаке после запятой, так как на самом деле

$1\text{ атм}=101325\text{ Па}$ (а не $10^5\text{ Па}$) и $1\text{ Дарси}=1.02\text{ мкм}^2$ (а не $1\text{ мкм}^2$).

%Подставим единицы измерения СИ в формулу \eqref{DupuyExam}:
%\beq
%\frac{\text{м}^3}{\text{сут}}\approx\frac{2\cdot3.141593\cdot\text{мД}\cdot \text{м}\cdot\text{атм}}{\text{сП}}
%\eeq

%\beq
%\frac{\text{м}^3}{\text{сут}}\approx\frac{2\cdot3.141593\cdot10^{-3}\cdot10^{-12}\text{ м}^2\cdot \text{м}\cdot10^5\text{ Па}}{10^{-3}\text{ Па}\cdot\text{с}}
%\eeq

%\beq
%\frac{\text{м}^3}{\text{сут}}\approx628318.6 \frac{\text{м}^3}{\text{с}}
%\eeq

%Далее переведём все единицы измерения в промысловые единицы:
%\beq
%86400\frac{\text{м}^3}{\text{сут}}\approx\frac{2\cdot3.141593\cdot10^3\text{ мД}\cdot \text{м}\cdot10^{-5}\text{ атм}}{10^{-5}\text{ атм}\cdot\text{с}}
%\eeq

\end{document}