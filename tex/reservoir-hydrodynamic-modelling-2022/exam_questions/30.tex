\begin{document}

\subsection{Какие способы инициализации модели Вы знаете? Раскройте суть хотя бы одного из способов. Какие ключевые слова применяются для инициализации ГДМ в симуляторе T-Navigator?}

\includegraphics[width=\textwidth, page=127]{Kurs_OsnovyGDM_Kai_774_gorodovSV_v6_0.pdf}

Фактически при инициализации модели задаём начальные условия.

Есть 2 способа инициализации: равновесный и неравновесный.
У равновесного есть ещё 1 подспособ, а именно равновесный с соблюдением начальной насыщенности.  
\\

\textbf{Неравновесный способ инициализации}

\includegraphics[width=\textwidth, page=128]{Kurs_OsnovyGDM_Kai_774_gorodovSV_v6_0.pdf}

В неравновесном способе задаются значения давления и насыщенности на начальный момент времени во всех ячейках.
Из названия ясно, что на начальный момент времени залежь не находится в равновесии.
После инициализации могут начаться перетоки даже в том случае, если в модели нет никаких скважин.

Такой способ инициализации на практике практически не встречается, ведь всегда предполагается, что залежь формировалась долгое время, за которое все флюиды пришли в гидростатическое равновесие, поэтому для инициализации обычно используется равновесный способ.
\\

\textbf{Равновесный способ инициализации}

\includegraphics[width=\textwidth, page=129]{Kurs_OsnovyGDM_Kai_774_gorodovSV_v6_0.pdf}

В равновесном способе инициализации задаётся ключевое слово EQUIL, в котором указываются опорная (отсчётная) глубина, пластовое давление на этой глубине, глубина ВНК, капиллярное давление на ВНК (если равно нулю, то это зеркало свободной воды FWL), глубина ГНК, капиллярное давление на ГНК, дальше 2 настроечных параметра по умолчанию (см. мануал) и параметр, определяющий точность расчёта запасов (если ячейка большая, то может оказаться так, что уровень ВНК может идти где-то между нижней гранью ячейки и её центром; в этом случае центр находится выше ВНК, поэтому вся ячейка по умолчанию будет иметь ненулевую нефтенасыщенность (даже часть, находящаяся ниже ВНК), что будет немного завышать запасы -- если есть необходимость точно воспроизвести запасы, то можно изменить последний параметр -- для этого он и нужен).
\\

В примере на слайде задано 3 строки: показано, что есть 3 региона, в которых разные уровни зеркала свободной воды.
Это может быть 3 пласта или 3 блока месторождения, которые не сообщаются между собой.
Для каждого мы соответственно задаём свои условия равновесия, свои условия инициализации.
\\

На практике обычно используют равновесный способ инициализации.

\includegraphics[width=\textwidth, page=130]{Kurs_OsnovyGDM_Kai_774_gorodovSV_v6_0.pdf}

Как происходит инициализация модели в случае равновесного способа?

Сначала вычисляется давление в нефтяной фазе (по формуле $\rho_o gh$) вверх и вниз от точки отсчёта (т.е. от уровня, равного значению первого параметра EQUIL).

Таким образом, получаем давление на заданном контакте.

Давление в водяной фазе на контакте получается отниманием капиллярного давления, заданного на контакте.

После этого вычисляем давление в водяной фазе (по формуле $\rho_w gh$) вверх и вниз от точки контакта.

Таким образом, в каждой ячейке есть давление в нефтяной фазе и давление в водяной фазе, а разница между этими давлениями -- это фактически капиллярное давление.

Дальше симулятор идёт в ключевое слово SWOF.
В этом ключевом слове заданы зависимости фазовых проницаемостей и капиллярного давления от насыщенности.
И симулятор таким образом находит для соответствующего значения капиллярного давления насыщенность и эту насыщенность задаёт в ячейках.

Т.е. в ячейках у симулятора рассчитаны давления в водяной и нефтяной фазах, их разница это капиллярные давления, а этим капиллярным давлениям можно сопоставить насыщенности (из таблицы), что и происходит.
\\

\textbf{Равновесный способ инициализации с соблюдением начальной насыщенности}

\includegraphics[width=\textwidth, page=131]{Kurs_OsnovyGDM_Kai_774_gorodovSV_v6_0.pdf}

А что происходит, если у нас кроме ключевого слова EQUIL задаётся ещё куб начальной водонасыщенности (например, мы делаем проектно-технологическую документацию ПТД и нам строго нужно соблюдать запасы, которые есть в геологической модели)?
\\

Геолог передаёт нам куб водонасыщенности, и мы его подключаем в модель с помощью SWATINIT.
При таком способе инициализации у симулятора получается две насыщенности: которую он сам рассчитал через условие равновесия и которая у него есть в кубе SWATINIT.
Что делать, если они не совпали?
Симулятор говорит, что он будет стараться настроить насыщенность так, чтобы она совпала с тем, что задано в кубе SWATINIT.
Для этого он будет масштабировать кривую капиллярного давления (т.е. просто растягивать или сжимать её по вертикали) таким образом, чтобы насыщенность в данной ячейке совпала с той, которая задана в ключевом слове SWATINIT.

Если насыщенность геологом рассчитана некорректно (т.е. неравновесно), то это может привести к тому, что масштабирования приведут к тому, что капиллярное давление будет слишком большим или слишком маленьким (и это один из критериев для проверки корректности инициализации, т.е. можно оценить диапазоны изменения куба капиллярного давления после инициализации и сравнить его с тем, что мы получали по исследованиям на керне).
\\

\textbf{Оценка корректности инициализации ГДМ}

\includegraphics[width=\textwidth, page=132]{Kurs_OsnovyGDM_Kai_774_gorodovSV_v6_0.pdf}

Как убедиться в корректности инициализации?

\begin{enumerate}
	\item Можно сравнить насыщенность на начальный (нулевой) шаг с заданной насыщенностью
	\item Можно оценить диапазоны изменения куба капиллярного давления после инициализации и сравнить их с теми, которые получали при исследовании на керне
	\item Можно оценить, совпадают ли запасы в ГДМ с геомоделью
	\item Можно провести расчёт модели без скважин на долгий период и убедиться в отсутствии изменений насыщенности и давления (для равновесных инициализаций)
\end{enumerate}


\end{document}