\begin{document}

\subsection{Приведите примеры описания PVT моделей в симуляторе T-Navigator.}

\textbf{Варианты описания PVT в моделях Black Oil}

\includegraphics[width=\textwidth, page=105]{Kurs_OsnovyGDM_Kai_774_gorodovSV_v6_0.pdf}

Как задаются разные модели с разными флюидами?
\\

Если отсутствует свободный газ, то можно модель рассматривать, как модель мёртвой нефти (т.е. у нас есть только нефть и вода; и газосодержание задаётся постоянным с помощью ключевого слова RSCONST; для нефти задаётся таблица PVDO зависимости объёмного коэффициента и вязкости от давления при одном и том же газосодержании).
\\

Если есть свободный сухой газ, то мы задаём фазу газа GAS, растворённый газ DISGAS.
А для нефти задаём теперь несколько таблиц PVTO: как будут меняться свойства нефти в зависимости от разного газосодержания.
Для газа тоже задаём таблицу PVDG, как будет меняться объёмный коэффициент и вязкость газа от давления.

\includegraphics[width=\textwidth, page=106]{Kurs_OsnovyGDM_Kai_774_gorodovSV_v6_0.pdf}

Если у нас жирный газ, то ещё добавляется фаза испарённой нефти VAPOIL, т.е. часть нефти испаряется и содержится в газообразном состоянии.
Тогда нам нужно описывать не только нефть набором таблиц, но и газ тоже описывать набором таблиц (для газа при разном содержании испарённой нефти свойства газа тоже будут разными).
\\

Т.е. это такие вот упрощённые способы, как описать изменение свойств нефти и газа в зависимости от давления и от разного содержания флюидов.


\end{document}