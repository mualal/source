\begin{document}

\subsection{Приведите примеры описания PVT-моделей в симуляторе T-Navigator.}

PVT-свойства задаются в секции PROPS входного файла симулятора.


Пример задания PVT-свойств воды.
Задаются опорное давление и при опорном давлении коэффициент объёмного расширения воды, коэффициент сжимаемости воды, вязкость воды, производная вязкости воды по давлению.
\begin{eclrun}
PVTW          
--P    Bw      cw        muw     dmuw/dpres  
159.6  1.0264  4.32E-05  0.3253  0  /
\end{eclrun}
\ \\

Пример задания упругих свойств породы.
Задаются опорное давление, сжимаемость породы, сжимаемость скелета породы, сжимаемость блока (блока, содержащего смесь), значение пористости при опорном давлении, значение коэффициента Пуассона при опорном давлении.
В данном примере укажем значения только для опорного давления и сжимаемости породы; остальные параметры оставим по умолчанию (см. руководство).
\begin{eclrun}
ROCK
--Pref  Compr 
159.6  4.765E-05  /
\end{eclrun}
\ \\

Пример задания плотностей нефти и воды в поверхностных условиях.
\begin{eclrun}
DENSITY 
860  1007.5  /
\end{eclrun}
\ \\

Пример задания постоянной и однородной концентрации растворённого газа в нефти.
Это обеспечивает наиболее эффективное моделирование системы чёрная нефть, где нет отдельной газовой фазы и давление никогда не опускается ниже точки давления насыщения.

Указаны концентрация растворённого газа и давление насыщения (расчёт будет завершён, если давление в каком-либо блоке сетки опустится ниже этого значения).
\begin{eclrun}
RSCONST 
--Rs  Pb
  20  34  /
\end{eclrun}
\ \\

Пример задания PVT-свойств нелетучей нефти для рассматриваемых PVT-регионов.

Необходимо ввести следующие параметры: давление насыщения, коэффициент объёмного расширения нефти, вязкость нефти при давлении насыщения.

PVT свойства нелетучей нефти для каждого PVT региона вводятся в таблицы.
Количество таблиц равно количеству регионов, определённых в TABDIMS.

В рассматриваемом примере одна таблица для единственного региона.

\begin{eclrun}
PVDO 
--P     Bo      muo
34     1.06000  9.00000
40     1.05926  9.08157
60     1.05681  9.35346
80     1.05436  9.62535
100    1.05192  9.89724
120    1.04948  10.16913
140    1.04705  10.44102
159.6  1.04467  10.70748
180    1.04220  10.98481
200    1.03978  11.25670
/
\end{eclrun}
\ \\

Если в пласте присутствуют разные типы флюидов (с разным содержанием атомов углерода в молекуле), то лучше использовать композиционные модели.

Но такие модели более сложные, дольше считаются, поэтому нередко используют различные "<костыли"> для описания PVT в моделях Black Oil.
\\

\textbf{Варианты описания PVT в моделях Black Oil}

\includegraphics[width=\textwidth, page=105]{Kurs_OsnovyGDM_Kai_774_gorodovSV_v6_0.pdf}

Как задаются разные модели с разными флюидами?
\\

Если отсутствует свободный газ, то можно модель рассматривать, как модель мёртвой нефти (т.е. у нас есть только нефть и вода; и газосодержание задаётся постоянным с помощью ключевого слова RSCONST; для нефти задаётся таблица PVDO зависимости объёмного коэффициента и вязкости от давления при одном и том же газосодержании).
\\

Если есть свободный сухой газ, то мы задаём фазу газа GAS, растворённый газ DISGAS.
А для нефти задаём теперь несколько таблиц PVTO: как будут меняться свойства нефти в зависимости от разного газосодержания.
Для газа тоже задаём таблицу PVDG, как будет меняться объёмный коэффициент и вязкость газа от давления.

\includegraphics[width=\textwidth, page=106]{Kurs_OsnovyGDM_Kai_774_gorodovSV_v6_0.pdf}

Если у нас жирный газ, то ещё добавляется фаза испарённой нефти VAPOIL, т.е. часть нефти испаряется и содержится в газообразном состоянии.
Тогда нам нужно описывать не только нефть набором таблиц, но и газ тоже описывать набором таблиц (для газа при разном содержании испарённой нефти свойства этого газа тоже будут разными).
\\

В итоге, у нас может быть ситуация, когда в пласте есть нефть, в которой растворён газ (используем ключевое слово DISGAS), и дополнительно есть газ, в котором испарена нефть (используем ключевое слово VAPOIL).

Используются разные ключевые слова, так как состав флюидов отличается, т.е. если из нефти выделится газ, то его свойства будут отличаться от того газа, который содержится в пласте, потому что в нём есть испарённые компоненты нефти.

Немного сложно, но это "<костыль">, который позволяет описать флюиды, находящиеся в приграничном состоянии (т.е. когда и газ растворяется в нефти, и нефть испаряется в газ).
\\

Т.е. это такие вот упрощённые способы, как описать изменение свойств нефти и газа в зависимости от давления и от разного содержания флюидов.


\end{document}