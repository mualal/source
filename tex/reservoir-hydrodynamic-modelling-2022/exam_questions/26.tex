\begin{document}

\subsection{Что такое начальные и граничные условия? Какое граничное условие называют Неймана, а какое Дирихле? Опишите применение граничных условий в решении задач в нефтяной индустрии.}

Начальные и граничные условия -- дополнение к основному дифференциальному уравнению (обыкновенному или в частных производных), задающее его поведение в начальный момент времени или на границе рассматриваемой области соответственно.\\

Граничное условие Неймана определяет значение производной искомой функции на границе рассматриваемой области.\\

Граничное условие Дирихле определяет значение искомой функции на границе рассматриваемой области.\\

При решении задач в нефтяной индустрии (при решении уравнения пьезопроводности) граничное условие Неймана ставится, если на границе рассматриваемой области известен расход жидкости (например, нулевой расход на границе -- другими словами, условие неперетока).\\

При решении задач в нефтяной индустрии (при решении уравнения пьезопроводности) граничное условие Дирихле ставится, если на границе рассматриваемой области давление поддерживается постоянным (например, при наличии большой газовой шапки, активной законтурной области или в случае проведения мероприятий по поддержанию пластового давления).\\

Начально-краевая задача для уравнения пьезопроводности в случае радиального притока к скважине и поддержания постоянного давления на внешней границе пласта (условие Дирихле):

\beq
\begin{cases}
	\dfrac{1}{\kappa}\dfrac{\partial p(r,t)}{\partial t}=\dfrac{1}{r}\dfrac{\partial p(r,t)}{\partial r}+\dfrac{\partial^2p(r,t)}{\partial r^2}\\
	p(r,0)=p_0, r\in\left(r_w,r_e\right]\\
	p(r_w,t)=p_w\\
	p(r_e,t)=p_e
\end{cases}
\eeq
(описывает неустановившийся и установившийся режимы)

Начально-краевая задача для уравнения пьезопроводности в случае радиального притока к скважине и отсутствия перетока на внешней границе пласта (условие Неймана):

\beq
\begin{cases}
	\dfrac{1}{\kappa}\dfrac{\partial p(r,t)}{\partial t}=\dfrac{1}{r}\dfrac{\partial p(r,t)}{\partial r}+\dfrac{\partial^2p(r,t)}{\partial r^2}\\
	p(r,0)=p_0, r\in\left(r_w,r_e\right]\\
	p(r_w,t)=p_w\\
	\dfrac{\partial p}{\partial r}\bigg|_{r=r_e}=0
\end{cases}
\eeq
(описывает неустановившийся и псевдоустановившийся режимы)
\\

\textbf{Дополнение.}

Граничное условие Дирихле является граничным условием первого рода.

Граничное условие Неймана является граничным условием второго рода.

Также существует смешанное граничное условие (граничное условие третьего рода), которое задаёт связь функции и её производной на границе:
\beq
\left(ap+b\frac{\partial p}{\partial r}\right)\bigg|_{r=r_e}=0
\eeq

И существует условие идеального контакта (граничное условие четвёртого рода), которое предусматривает равенство температур и потоков на граничной поверхности двух сред:
\beq
\begin{cases}
T|_A=T_c|_A, \\
-\kappa\dfrac{\partial T}{\partial n}\bigg|_A=-\kappa_c\dfrac{\partial T_c}{\partial n}\bigg|_A,
\end{cases}
\eeq
где $T, T_c$ -- температура тела и соприкасающейся среды (или тела) соответственно; $\kappa, \kappa_c$ -- коэффициент пьезопроводности тела и соприкасающейся среды (или тела) соответственно, $n$ -- нормаль к поверхности $A$.


\end{document}