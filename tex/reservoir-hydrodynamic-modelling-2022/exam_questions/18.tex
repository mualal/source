\begin{document}

\subsection{Что такое PVT-свойства? Раскрыть суть PVT-свойств.}

\includegraphics[width=\textwidth, page=98]{Kurs_OsnovyGDM_Kai_774_gorodovSV_v6_0.pdf}

В разных месторождениях залегают разные типы нефти.
Они отличаются друг от друга по цвету, плотности, газосодержанию и т.д.
Газы могут иметь различную удельную плотность и содержать различные объёмы примесей.

На слайде выше представлена условная таблица (условное деление) на типы флюидов.
\\

PVT свойства -- это свойства данного типа флюида, которые зависят от давления и температуры. 
\\

Для точного определения PVT-свойств необходим лабораторный анализ полученных надлежащим образом образцов флюидов с каждого месторождения.
\\

Более простой метод: определить хотя бы некоторые ключевые параметры (тип флюида) и использовать известные корреляции (Standing, Beal и т.д.).
\\

Более подробно тема PVT-свойств раскрыта в следующих двух вопросах.

\end{document}