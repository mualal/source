\begin{document}

\subsection{Раскрыть суть решения уравнения фильтрации неявным методом. Опишите основные шаги}

В неявном методе конечно-разностное приближение для второй производной по координате выписывается через значения давлений в ячейках на момент времени $t+\Delta t$ (в отличие от явного метода, где конечно-разностное приближение строится на основе значений давления на текущем временном шаге $t$).
\\

При использовании неявного метода: для того, чтобы получить значения давлений в ячейках на следующем временном шаге, требуется решить систему линейных алгебраических уравнений (СЛАУ).
\\

Описание основных шагов неявного метода доступно по ссылке: \href{https://docs.yandex.ru/docs/view?url=ya-disk-public\%3A\%2F\%2FWicxck3PRHBvGgXBKaNry\%2FRtiluR1m6cR6Bi14xdyyNQDBM0D9VXsHr3EpLffWemDqZvSgIch5AN9ddz7ydViQ\%3D\%3D\%3A\%2F\%D0\%92\%D1\%82\%D0\%BE\%D1\%80\%D0\%BE\%D0\%B9\%20\%D0\%BA\%D1\%83\%D1\%80\%D1\%81\%20\%D0\%BC\%D0\%B0\%D0\%B3\%D0\%B8\%D1\%81\%D1\%82\%D1\%80\%D0\%B0\%D1\%82\%D1\%83\%D1\%80\%D1\%8B\%2F\%D0\%93\%D0\%B8\%D0\%B4\%D1\%80\%D0\%BE\%D0\%B4\%D0\%B8\%D0\%BD\%D0\%B0\%D0\%BC\%D0\%B8\%D1\%87\%D0\%B5\%D1\%81\%D0\%BA\%D0\%BE\%D0\%B5\%20\%D0\%BC\%D0\%BE\%D0\%B4\%D0\%B5\%D0\%BB\%D0\%B8\%D1\%80\%D0\%BE\%D0\%B2\%D0\%B0\%D0\%BD\%D0\%B8\%D0\%B5\%2F\%D0\%97\%D0\%B0\%D0\%B4\%D0\%B0\%D0\%BD\%D0\%B8\%D1\%8F\%2F\%D0\%9D\%D0\%B5\%D0\%B2\%D0\%BD\%D0\%BE\%D0\%B5_\%D1\%80\%D0\%B5\%D1\%88\%D0\%B5\%D0\%BD\%D0\%B8\%D0\%B5_\%D1\%83\%D1\%80\%D0\%B0\%D0\%B2\%D0\%BD\%D0\%B5\%D0\%BD\%D0\%B8\%D1\%8F_\%D1\%84\%D0\%B8\%D0\%BB\%D1\%8C\%D1\%82\%D1\%80\%D0\%B0\%D1\%86\%D0\%B8\%D0\%B8.pdf&name=\%D0\%9D\%D0\%B5\%D0\%B2\%D0\%BD\%D0\%BE\%D0\%B5_\%D1\%80\%D0\%B5\%D1\%88\%D0\%B5\%D0\%BD\%D0\%B8\%D0\%B5_\%D1\%83\%D1\%80\%D0\%B0\%D0\%B2\%D0\%BD\%D0\%B5\%D0\%BD\%D0\%B8\%D1\%8F_\%D1\%84\%D0\%B8\%D0\%BB\%D1\%8C\%D1\%82\%D1\%80\%D0\%B0\%D1\%86\%D0\%B8\%D0\%B8.pdf}{GO TO НЕЯВНЫЙ МЕТОД}

\end{document}