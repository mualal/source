\begin{document}

\subsection{Раскрыть суть решения уравнения фильтрации неявным методом. Опишите основные шаги}

В неявном методе конечно-разностное приближение для второй производной по координате выписывается через значения давлений в ячейках на момент времени $t+\Delta t$ (в отличие от явного метода, где конечно-разностное приближение строится на основе значений давления на текущем временном шаге $t$).
\\

При использовании неявного метода: для того, чтобы получить значения давлений в ячейках на следующем временном шаге, требуется решить систему линейных алгебраических уравнений (СЛАУ).
\\

Описание основных шагов неявного метода доступно по ссылке: \href{https://docs.yandex.ru/docs/view?url=ya-disk-public\%3A\%2F\%2FWicxck3PRHBvGgXBKaNry\%2FRtiluR1m6cR6Bi14xdyyNQDBM0D9VXsHr3EpLffWemDqZvSgIch5AN9ddz7ydViQ\%3D\%3D\%3A\%2F\%D0\%92\%D1\%82\%D0\%BE\%D1\%80\%D0\%BE\%D0\%B9\%20\%D0\%BA\%D1\%83\%D1\%80\%D1\%81\%20\%D0\%BC\%D0\%B0\%D0\%B3\%D0\%B8\%D1\%81\%D1\%82\%D1\%80\%D0\%B0\%D1\%82\%D1\%83\%D1\%80\%D1\%8B\%2F\%D0\%93\%D0\%B8\%D0\%B4\%D1\%80\%D0\%BE\%D0\%B4\%D0\%B8\%D0\%BD\%D0\%B0\%D0\%BC\%D0\%B8\%D1\%87\%D0\%B5\%D1\%81\%D0\%BA\%D0\%BE\%D0\%B5\%20\%D0\%BC\%D0\%BE\%D0\%B4\%D0\%B5\%D0\%BB\%D0\%B8\%D1\%80\%D0\%BE\%D0\%B2\%D0\%B0\%D0\%BD\%D0\%B8\%D0\%B5\%2F\%D0\%97\%D0\%B0\%D0\%B4\%D0\%B0\%D0\%BD\%D0\%B8\%D1\%8F\%2F\%D0\%9D\%D0\%B5\%D0\%B2\%D0\%BD\%D0\%BE\%D0\%B5_\%D1\%80\%D0\%B5\%D1\%88\%D0\%B5\%D0\%BD\%D0\%B8\%D0\%B5_\%D1\%83\%D1\%80\%D0\%B0\%D0\%B2\%D0\%BD\%D0\%B5\%D0\%BD\%D0\%B8\%D1\%8F_\%D1\%84\%D0\%B8\%D0\%BB\%D1\%8C\%D1\%82\%D1\%80\%D0\%B0\%D1\%86\%D0\%B8\%D0\%B8.pdf&name=\%D0\%9D\%D0\%B5\%D0\%B2\%D0\%BD\%D0\%BE\%D0\%B5_\%D1\%80\%D0\%B5\%D1\%88\%D0\%B5\%D0\%BD\%D0\%B8\%D0\%B5_\%D1\%83\%D1\%80\%D0\%B0\%D0\%B2\%D0\%BD\%D0\%B5\%D0\%BD\%D0\%B8\%D1\%8F_\%D1\%84\%D0\%B8\%D0\%BB\%D1\%8C\%D1\%82\%D1\%80\%D0\%B0\%D1\%86\%D0\%B8\%D0\%B8.pdf}{GO TO НЕЯВНЫЙ МЕТОД}

\textbf{Здесь тоже повторю эти шаги.}
\\

Вспомним явную запись конечно-разностного уравнения притока:

\beq
\frac{p_{i+1}^t-2p_i^t+p_{i-1}^t}{(\Delta x)^2}=\left(\frac{\varphi\mu c}{k}\right)\frac{p_i^{t+\Delta t}-p_i^t}{\Delta t},
\eeq
где $i=2,...,N-1$.

Так как вывод левой части производился при постоянном времени, можно переписать эту часть на момент времени $t+\Delta t$:

\beq
\frac{p_{i+1}^{t+\Delta t}-2p_i^{t+\Delta t}+p_{i-1}^{t+\Delta t}}{(\Delta x)^2}=\left(\frac{\varphi\mu c}{k}\right)\frac{p_i^{t+\Delta t}-p_i^t}{\Delta t},
\eeq
где $i=2,...,N-1$.

Граничные условия постоянного давления:
\beq
\begin{cases}
\dfrac{p_2^{t+\Delta t}-3p_1^{t+\Delta t}+2p_{\text{left}}}{\dfrac{3}{4}\Delta x^2}=\left(\dfrac{\varphi\mu c}{k}\right)\dfrac{p_1^{t+\Delta t}-p_1^t}{\Delta t}\\\\
\dfrac{2p_{\text{right}}-3p_N^{t+\Delta t}+p_{N-1}^{t+\Delta t}}{\dfrac{3}{4}\Delta x^2}=\left(\dfrac{\varphi\mu c}{k}\right)\dfrac{p_N^{t+\Delta t}-p_N^t}{\Delta t}
\end{cases}
\eeq

Граничные условия постоянного расхода:
\beq
\begin{cases}
\dfrac{p_2^{t+\Delta t}-p_1^{t+\Delta t}}{(\Delta x)^2}+Q_{\text{left}}\dfrac{\mu}{\Delta xAk}=\left(\dfrac{\varphi\mu c}{k}\right)\dfrac{p_1^{t+\Delta t}-p_1^t}{\Delta t}\\\\
\dfrac{p_{N-1}^{t+\Delta t}-p_N^{t+\Delta t}}{(\Delta x)^2}-Q_{\text{right}}\dfrac{\mu}{\Delta x Ak}=\left(\dfrac{\varphi\mu c}{k}\right)\dfrac{p_N^{t+\Delta t}-p_N^t}{\Delta t}
\end{cases}
\eeq
\ \\

\textbf{Неявная запись решения ДУ.}

Теперь возможно записать набор из $N$ уравнений с $N$ количеством неизвестных, которые должны быть решены одновременно (СЛАУ).
Для простоты данный набор уравнений может быть записан в данной форме:
\beq
a_i p_{i-1}^{t+\Delta t}+b_ip_i^{t+\Delta t}+c_i p_{i+1}^{t+\Delta t}=d_i,
\eeq
где $i=1,...,N$.

Введём обозначение:
\beq
\gamma=\left(\frac{\varphi\mu c}{k}\right)\left(\frac{\Delta x^2}{\Delta t}\right)
\eeq

Коэффициенты СЛАУ:

$a_i=1,\text{ при } i=2,...,N-1;$

$b_i=-2-\gamma,\text{ при } i=2,...,N-1;$

$c_i=1,\text{ при } i=2,...,N-1;$

$d_i=-\gamma p_i^t,\text{ при } i=2,...,N-1$
\ \\

Коэффициенты для левого граничного условия постоянного давления:

$a_1=0;$

$b_1=-3-\dfrac{3}{4}\gamma;$

$c_1=1;$

$d_1=-\dfrac{3}{4}\gamma p_1^t-2p_{\text{left}}$
\ \\

Коэффициенты для правого граничного условия постоянного давления:

$a_N=1;$

$b_N=-3-\dfrac{3}{4}\gamma;$

$c_N=0;$

$d_N=-\dfrac{3}{4}\gamma p_N^t-2p_{\text{right}}$
\ \\

Коэффициенты для левого граничного условия постоянного расхода:

$a_1=0;$

$b_1=-1-\gamma;$

$c_1=1;$

$d_1=-Q_{\text{left}}\dfrac{\Delta x \mu}{Ak}-\gamma p_1^t$
\ \\

Коэффициенты для правого граничного условия постоянного расхода:

$a_N=1;$

$b_N=-1-\gamma;$

$c_N=0;$

$d_N=Q_{\text{right}}\dfrac{\Delta x \mu}{Ak}-\gamma p_N^t$

\end{document}