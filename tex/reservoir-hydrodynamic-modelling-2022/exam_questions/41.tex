\begin{document}

\subsection{Что такое материальный баланс в гидродинамическом моделировании? Опишите суть материального баланса.}

Перед построением модели необходимо провести анализ разработки, чтобы понимать, какие скважины друг на друга влияют, откуда они обводняются, есть ли загрязнения призабойной зоны, есть ли трещины авто-ГРП на нагнетательных скважинах.

Другими словами, необходимо проанализировать, как работает месторождение, как работают скважины, чтобы это учесть при построении модели.
\\

\textbf{Матбаланс}

\includegraphics[width=\textwidth, page=115]{Kurs_OsnovyGDM_Kai_774_gorodovSV_v6_0.pdf}

Один из способов анализа разработки месторождения -- это материальный баланс.

По факту представляем в виде бочки, в которую что-то втекает, что-то вытекает, то что осталось в бочке либо расширяется, либо сжимается в зависимости от того, как изменилось давление.
Дополнительно может выделяться газ и так далее.

На слайде записана формула в общем виде: то, что относится к газу, к нефти, к воде.
С одной стороны то, что было, что расширилось, закачалось.

\includegraphics[width=\textwidth, page=116]{Kurs_OsnovyGDM_Kai_774_gorodovSV_v6_0.pdf}

Обычно есть либо в Экселе какие-то готовые формулы, через которые считается матбаланс, либо специальные программы (MBAL и ещё какие-то).
\\

На вход подаются начальное пластовое давление, PVT-свойства, точная история отбора и закачки, свойства коллектора и аквифера, объём начальных запасов.

И на основе этих данных рассчитывается динамика пластового давления, которая затем сопоставляется с фактическими замерами: если расчёт не совпал с фактом, то значит где-то исходные данные неточные.
В пределах имеющихся неопределённостей исходных данных можем их поварьировать и таким образом добиться совпадения расчётной динамики пластового давления и фактической.

За счёт такой вариации можно проанализировать, насколько неопределённые входные параметры, т.е., например, начальные запасы, свойства пласта или аквифера.
И таким образом как бы осуществить анализ на основе матбаланса.
\\

\textbf{Матбаланс. Пример использования}

\includegraphics[width=\textwidth, page=117]{Kurs_OsnovyGDM_Kai_774_gorodovSV_v6_0.pdf}

Здесь приведён пример использования матбаланса.
\\

Делали проект разработки группы месторождений.
Площадь большая, и поэтому возник вопрос: связаны ли все эти залежи нефти (отмечены зелёным на карте) между собой?
Красным на карте отмечены разломы.

Если строить одну большую модель, расчёт будет идти долго. Поэтому необходимо проверить, можно ли разрезать рассматриваемый участок на несколько отдельных участков, чтобы построить несколько отдельных моделей.

Построили несколько простых моделей материального баланса: каждая залежь представлялась отдельной бочкой и между этими бочками рисовалась связь (отмечена голубыми ромбиками). Далее проводился расчёт и производилась настройка параметров связности между различными залежами.
На графиках синим обозначены рассчитанные значения динамики пластового давления, а красные квадратики -- фактические замеры.

Оказалось, что наилучшую настройку показали модели, в которых часть из этих залежей не связаны.
Другими словами, в результате настройки модели матбаланса оказалось, что какие-то связанности оказались нулевыми или близкими к нулю.

Это позволило разделить рассматриваемую группу месторождений на отдельные части, моделировать их отдельно и соответственно ускорить расчёты; другими словами, за ограниченное время проекта сделать больше расчётов.
\\

\textbf{Далее более подробные слайды про матбаланс.}

\includegraphics[width=\textwidth, page=57]{1МАТЕРИАЛЫ_ПО_КУРСУ_ДОБЫЧА_2022.pdf}

\includegraphics[width=\textwidth, page=58]{1МАТЕРИАЛЫ_ПО_КУРСУ_ДОБЫЧА_2022.pdf}

\includegraphics[width=\textwidth, page=59]{1МАТЕРИАЛЫ_ПО_КУРСУ_ДОБЫЧА_2022.pdf}

\includegraphics[width=\textwidth, page=60]{1МАТЕРИАЛЫ_ПО_КУРСУ_ДОБЫЧА_2022.pdf}

\includegraphics[width=\textwidth, page=61]{1МАТЕРИАЛЫ_ПО_КУРСУ_ДОБЫЧА_2022.pdf}

\includegraphics[width=\textwidth, page=62]{1МАТЕРИАЛЫ_ПО_КУРСУ_ДОБЫЧА_2022.pdf}

\includegraphics[width=\textwidth, page=63]{1МАТЕРИАЛЫ_ПО_КУРСУ_ДОБЫЧА_2022.pdf}

\includegraphics[width=\textwidth, page=64]{1МАТЕРИАЛЫ_ПО_КУРСУ_ДОБЫЧА_2022.pdf}

\includegraphics[width=\textwidth, page=65]{1МАТЕРИАЛЫ_ПО_КУРСУ_ДОБЫЧА_2022.pdf}

\end{document}