\begin{document}

\subsection{Что такое давление насыщения? Что происходит при "<переходе"> через точку давления насыщения? Как будут меняться свойства жидкостей при "<переходе"> через  точку давления насыщения?}

Давление насыщения нефти -- давление, выше которого весь газ уже растворился, нефть остаётся недонасыщенной.
При снижении давления ниже давления насыщения из нефти начинает выделяться газ.

\includegraphics[width=\textwidth, page=102]{Kurs_OsnovyGDM_Kai_774_gorodovSV_v6_0.pdf}

На слайде показаны типичные кривые свойств нефти.
\\

Красная кривая (газосодержание нефти).

До давления насыщения при увеличении давления газосодержание растёт (газ растворяется и растворяется в нефти), при достижении давления насыщения весь газ растворился и дальше газосодержание остаётся постоянным.
\\

Зелёная кривая (объёмный коэффициент нефти).

От точки давления насыщения: если мы увеличиваем давление, то газа у нас нет, у нас только происходит сжатие нефти и объём нефти уменьшается с увеличением давления, поэтому объёмный коэффициент тоже снижается.

От точки давления насыщения: если мы снижаем давление ниже давления насыщения, то кроме уменьшения давления (т.е. увеличения объёма нефти за счёт расширения) из нефти начинает выделяться газ и соответственно объём нефти уменьшается, т.е. несмотря на то, что она расширяется из-за уменьшения давления, объёмный коэффициент начинает снижаться за счёт того, что газа много и он уходит из нефти.
\\

Синяя кривая (вязкость нефти).

Для вязкости картина наоборот: до давления насыщения при увеличении давления газ растворяется и растворяется в нефти, что приводит к снижению вязкости.
После достижения давления насыщения весь газ растворился в нефти и при дальнейшем увеличении давления происходит просто сжатие нефти, т.е. увеличение вязкости (т.к. вязкость -- это мера внутреннего сопротивления одних слоёв жидкости относительно других слоёв при их движении, а это сопротивление очевидно растёт при увеличении давления).


\end{document}