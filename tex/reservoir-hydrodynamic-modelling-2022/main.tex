% !TeX spellcheck = en_US
% !TeX program = xelatex

\documentclass[a4paper,12pt]{article}
\renewcommand{\baselinestretch}{1.2}
\usepackage[utf8]{inputenc}
\usepackage[T2A, T1]{fontenc}
\usepackage[english, russian]{babel}

\usepackage{fontspec}
\setmainfont{Times New Roman}
\usepackage{setspace, amsmath}
\usepackage{amssymb}
\usepackage{dsfont}
\usepackage{epsfig}
\usepackage{pdfpages}

\makeatletter
\let\@fnsymbol\@arabic
\makeatother

\usepackage{geometry}
\geometry{
a4paper,
total={170mm, 257mm},
left=20mm,
top=20mm,
}

\usepackage{systeme}
\usepackage{skak}
\usepackage{mathtools}
\usepackage{unicode-math}
\usepackage{array}
\usepackage{makecell}
\usepackage{subfiles}
\usepackage{hyperref}
\hypersetup{pdfstartview=FitH, linkcolor=black, urlcolor=blue, colorlinks=true}
\usepackage{framed}
\usepackage{graphicx}
\usepackage{caption}
\usepackage{subcaption}
\usepackage{color}
\usepackage{chngcntr}
\usepackage{tikz}
\usepackage{csquotes}
\usepackage{fancyhdr}
\usepackage{fancyvrb}
\usepackage{comment}
\usepackage{adjustbox}
\usepackage[breakable, skins]{tcolorbox}

\pagestyle{fancy}
\fancyhf{}
\fancyhead[L]{\leftmark}
\fancyhead[R]{\hyperlink{page.1}{\textcolor{violet}{Вернуться к содержанию}}}
%\fancyhead[L]{\leftmark \\ \rightmark}
%\fancyhead[R]{\thepage}
%\fancyhead[R]{\rightmark}
\fancyfoot[C]{\thepage}
%\fancyfoot[L]{\rightmark}

\usepackage{float}
\floatstyle{plaintop}
\usepackage{enumitem}
\setlength{\parindent}{0pt}
%\setcounter{section}{-1}

\graphicspath{{./img/}}
\newcommand{\myPictWidth}{.95\textwidth}
\newcommand{\phm}{\phantom{-}}
\newcommand{\beq}{\begin{equation}}
\newcommand{\eeq}{\end{equation}}

\newenvironment{eclrun}
{\VerbatimEnvironment
\begin{tcolorbox}[breakable,boxrule=0.5pt,colframe=gray!50]
\begin{Verbatim}
}
{
\end{Verbatim}	
\end{tcolorbox}
}

\begin{document}
	\subfile{parts/general_info}
	\tableofcontents
	\title{Гидродинамическое моделирование\\Конспект лекций и семинаров}
	\author{Муравцев А.А.\thanks{конспектирует; email: almuravcev@yandex.ru}
	\and
	Базыров И.Ш.\thanks{лектор, Высшая школа теоретической механики, Санкт-Петербургский Политехнический университет. Дополнительные материалы к лекциям \href{https://csspbstu-my.sharepoint.com/:f:/g/personal/muravtsev_aa_edu_spbstu_ru/Epiacj6WFMBHqIF6E3YQgCMB7yi5NAA1ycqFLqrTZMhJ4w?e=i2agP0}{доступны по ссылке}.}
	\and
	Кайгородов С.В.\thanks{лектор, Высшая школа теоретической механики, Санкт-Петербургский Политехнический университет. Дополнительные материалы к лекциям \href{https://csspbstu-my.sharepoint.com/:f:/g/personal/muravtsev_aa_edu_spbstu_ru/Epiacj6WFMBHqIF6E3YQgCMB7yi5NAA1ycqFLqrTZMhJ4w?e=i2agP0}{доступны по ссылке}.}}
	\maketitle
	\subfile{parts/exam}
	\newpage
	\subfile{parts/2022_09_07}
	\newpage
	\subfile{parts/2022_09_12}
	\newpage
	\subfile{parts/2022_09_13}
	\newpage
	\subfile{parts/2022_09_14}
	\newpage
	\subfile{parts/2022_09_21}
	\newpage
	%\subfile{parts/2022_09_28}
	\subfile{parts/2022_09_28_modified}
	\newpage
	\subfile{parts/2022_10_05}
	\newpage
	\subfile{parts/2022_10_12}
	\newpage
	\subfile{parts/2022_10_19}
	\newpage
	\subfile{parts/2022_10_26}
	\newpage
	\subfile{parts/2022_11_02}
	\newpage
	\subfile{parts/2022_11_09}
	\newpage
	\subfile{parts/2022_11_16}
	\newpage
	\subfile{parts/2022_11_30}
	\newpage
	\subfile{parts/2022_12_07}
	\newpage
	\subfile{parts/2022_12_14}
	\newpage
	\subfile{parts/2022_12_21}
	\newpage
	\subfile{parts/2022_12_23}
	\newpage
	\subfile{parts/2022_12_28}
	
\end{document}
