\documentclass[a4paper, 11pt]{article}
\usepackage[a4paper, total={7in, 10in}]{geometry}
\usepackage{amsmath,amssymb}
\usepackage[utf8]{inputenc}
\usepackage[russian]{babel}
\usepackage{xcolor}
\usepackage{mdframed}
\usepackage{hyperref}
\usepackage{tikz}

\usepackage{enumitem}
\setlength{\parindent}{0pt}

\newcommand{\beq}{\begin{equation}}
\newcommand{\eeq}{\end{equation}}
\newcommand{\T}[0]{\overline{T}}
\newcommand{\I}[2]{\,I_{#1}\!\left(#2\right)}

\newenvironment{problem}[2][Problem]
    { \begin{mdframed}[backgroundcolor=gray!10] \textbf{#1 #2} \\}
    {  \end{mdframed}}

\newenvironment{solution}
    {\textit{}}
    {}

\begin{document}
\large\textbf{Муравцев Александр 5040103/10401} \hfill \textbf{Расчётное задание 1}   \\
Email: \href{mailto:muravtsev.aa@edu.spbstu.ru}{muravtsev.aa@edu.spbstu.ru} \hfill Вариант: а \\
\rule{7in}{2.8pt}

\begin{problem}[Задача]{(несвязанная динамическая задача термоупругости)}
 
 Рассматривается полубесконечный стержень с модулем Юнга $E$ и плотностью $\rho$, для которого справедливо соотношение Дюамеля-Неймана. Объёмный источник в уравнении теплопроводности задан в виде
 $$
 	Q=J_0(H(t)-H(t-\tau))e^{-\gamma x},
 $$
 где $H(t)$ -- функция Хевисайда.\\
 
 Пренебрегая теплопроводностью материала, получить термоупругий импульс на расстоянии, существенно превышающем глубину проникновения теплового источника.\\
 
 Принять, что время действия теплового импульса $\tau$ много меньше времени пробега акустической волны до координаты, в которой производится регистрация сигнала.

\end{problem}
\begin{solution}\\
\textbf{Постановка задачи}\\

По условию пренебрегаем теплопроводностью материала стержня (не учитываем распространение тепла вдоль стержня), поэтому уравнение теплопроводности можем записать в виде
\beq
\frac{\partial\theta}{\partial t}=Q\Leftrightarrow\frac{\partial\theta}{\partial t}=J_0(H(t)-H(t-\tau))e^{-\gamma x}
\eeq
Тогда распределение температуры по стержню (см. рис.\ref{fig:integrate}):
\beq\label{Temperature}
\theta=J_0\left(tH(t)-(t-\tau)H(t-\tau)\right)e^{-\gamma x}
\eeq
\begin{figure}[h]
\centering
\begin{tikzpicture}[line width=0.5]
\draw (3,4) -- (5,4) -- (5,5) -- (6.5,5) -- (6.5,4) -- (8.5,4);
\draw (3,1.5) -- (5,1.5) -- (6.5,3) -- (8.5,3);
\draw[dashed] (5,4) -- (5,1.5);
\draw[dashed] (6.5,4) -- (6.5,1.5);
\draw[dashed] (3,5) -- (5,5);
\draw[dashed] (3,3) -- (6.5,3);
\node at (5,1.2) {$0$};
\node at (6.5,1.2) {$\tau$};
\node at (2.7,5) {$1$};
\node at (2.7,3) {$\tau$};
\end{tikzpicture}
\caption{Интегрирование ступенчатого импульса}
\label{fig:integrate}
\end{figure}


Далее подключаем уравнение баланса импульса:
\beq
\frac{\partial\sigma}{\partial x}-\rho\ddot{u}=0,
\eeq
где $\sigma=E\left(\varepsilon-\alpha\theta\right)$ (соотношение Дюамеля-Неймана) и $\displaystyle{}\varepsilon=\frac{\partial u}{\partial x}$.

После подстановки:
\beq
\frac{\partial^2u}{\partial x^2}-\frac{1}{c_0^2}\frac{\partial^2u}{\partial t^2}=\alpha\frac{\partial\theta}{\partial x},
\eeq
где $\displaystyle{}c_0=\sqrt{\frac{E}{\rho}}$ -- скорость звука в стержне.\\\\

Далее подставляем выражение для распределения температуры \eqref{Temperature}:
\beq
\frac{\partial^2u}{\partial x^2}-\frac{1}{c_0^2}\frac{\partial^2u}{\partial t^2}=-J_0\cdot\alpha\cdot\left(tH(t)-(t-\tau)H(t-\tau)\right)\cdot\gamma e^{-\gamma x}
\eeq
Левый торец стержня свободен:
\beq
\sigma|_{x=0}=0\Leftrightarrow\frac{\partial u}{\partial x}\bigg|_{x=0}=\alpha\cdot J_0\cdot\left(tH(t)-(t-\tau)H(t-\tau)\right)
\eeq
На бесконечности ставим условие излучения:
\beq
u|_{x\rightarrow\infty}<\infty
\eeq
Ставим нулевые начальные условия:
\beq
u(0,x)=0\text{ и }\dot{u}(0,x)=0
\eeq
Таким образом, получаем постановку задачи:
\beq
\begin{cases}
	\dfrac{\partial^2u}{\partial x^2}-\dfrac{1}{c_0^2}\dfrac{\partial^2u}{\partial t^2}=-J_0\cdot\alpha\cdot\left(tH(t)-(t-\tau)H(t-\tau)\right)\cdot\gamma e^{-\gamma x}\\\\
	\dfrac{\partial u}{\partial x}\bigg|_{x=0}=\alpha\cdot J_0\cdot\left(tH(t)-(t-\tau)H(t-\tau)\right)\\\\
	u|_{x\rightarrow\infty}<\infty\\\\
	u(0,x)=0\\
	\dot{u}(0,x)=0
\end{cases}
\eeq
\\\\
\textbf{Преобразование Лапласа}\\

В пространстве Лапласа постановка задачи перепишется в следующем виде:
\beq
\begin{cases}
\dfrac{d^2u^L}{dx^2}-\dfrac{p^2}{c_0^2}u^L=-J_0\cdot\alpha\cdot\dfrac{1-e^{-\tau p}}{p^2}\cdot \gamma e^{-\gamma x}\\\\
\dfrac{du^L}{dx}\bigg|_{x=0}=\alpha\cdot J_0\cdot\dfrac{1-e^{-\tau p}}{p^2}\\\\
u^L|_{x\rightarrow\infty}<\infty\\
\end{cases}
\eeq
\\\\\\\\\\\
\textbf{Решение в пространстве Лапласа}\\

Общее решение полученного дифференциального уравнения:
\beq
u^L=C_1e^{\,px/c}+C_2e^{-px/c}+\frac{J_0\cdot\alpha\cdot\gamma\cdot\left(1-e^{-\tau p}\right)}{p^2\left(\frac{p^2}{c^2}-\gamma^2\right)}e^{-\gamma x}
\eeq
Из второго граничного условия:
\beq
u^L|_{x\rightarrow\infty}<\infty \Rightarrow C_1=0
\eeq
Из первого граничного условия:
\beq
-\frac{p}{c}C_2-\frac{J_0\cdot\alpha\cdot\gamma^2\left(1-e^{-\tau p}\right)}{p^2\left(\frac{p^2}{c^2}-\gamma^2\right)}=\alpha\cdot J_0\cdot\frac{1-e^{-\tau p}}{p^2}
\eeq
\\
\beq
-\frac{p}{c}C_2=\frac{\alpha\cdot J_0\left(1-e^{-\tau p}\right)}{p^2}\left(1+\frac{\gamma^2}{\frac{p^2}{c^2}-\gamma^2}\right)
\eeq
\\
\beq
-\frac{p}{c}C_2=\frac{p^2}{c^2}\frac{\alpha\cdot J_0\left(1-e^{-\tau p}\right)}{p^2\left(\frac{p^2}{c^2}-\gamma^2\right)}
\eeq
\\
\beq
C_2=-\frac{p}{c}\frac{\alpha\cdot J_0\cdot\left(1-e^{-\tau p}\right)}{p^2\left(\frac{p^2}{c^2}-\gamma^2\right)}
\eeq
\end{solution}
Тогда
\beq
u^L=\frac{\alpha\cdot J_0\cdot\left(1-e^{-\tau p}\right)}{p^2\left(\frac{p^2}{c^2}-\gamma^2\right)}\left(\gamma e^{-\gamma x}-\frac{p}{c}\cdot e^{-\frac{px}{c}}\right)
\eeq
\\\\
\textbf{Переход из пространства Лапласа к оригиналу}\\

\end{document}
