\documentclass[a4paper, 11pt]{article}
\usepackage[a4paper, total={7in, 10in}]{geometry}
\usepackage[fleqn]{amsmath}
\usepackage[utf8]{inputenc}
\usepackage[russian]{babel}
\usepackage{amssymb,amsthm}
\usepackage{xcolor}
\usepackage{mdframed}
\usepackage{hyperref}

\usepackage{enumitem}
\setlength{\parindent}{0pt}

\newcommand{\T}[0]{\overline{T}}
\newcommand{\I}[2]{\,I_{#1}\!\left(#2\right)}

\newenvironment{problem}[2][Problem]
    { \begin{mdframed}[backgroundcolor=gray!10] \textbf{#1 #2} \\}
    {  \end{mdframed}}

\newenvironment{solution}
    {\textit{}}
    {}

\begin{document}
\large\textbf{Муравцев Александр 5040103/10401} \hfill \textbf{Расчётное задание 1}   \\
Email: \href{mailto:muravtsev.aa@edu.spbstu.ru}{muravtsev.aa@edu.spbstu.ru} \hfill Вариант: а \\
\rule{7in}{2.8pt}

\begin{problem}[Задача]{(несвязанная динамическая задача термоупругости)}
 
 Рассматривается полубесконечный стержень с модулем Юнга $E$ и плотностью $\rho$, для которого справедливо соотношение Дюамеля-Неймана. Объёмный источник в уравнении теплопроводности задан в виде
 $$
 	Q=J_0(H(t)-H(t-\tau))e^{-\gamma x},
 $$
 где $H(t)$ -- функция Хевисайда.\\
 
 Пренебрегая теплопроводностью материала, получить термоупругий импульс на расстоянии, существенно превышающем глубину проникновения теплового источника.\\
 
 Принять, что время действия теплового импульса $\tau$ много меньше времени пробега акустической волны до координаты, в которой производится регистрация сигнала.

\end{problem}
\begin{solution}\\
\textbf{Постановка задачи}\\
\textbf{Преобразование Лапласа}

\end{solution}

\end{document}
