\documentclass[main.tex]{subfiles}

\begin{document}

% \textcolor{red}{Вводная лекция}

\section{Лекция 08.04.2021 (Валов А.В.)}

\subsection{Модель Planar3D ILSA: дискретизация, поиск фронта, алгоритм}

Продолжаем разрабатывать численный алгоритм для модели Planar3D ILSA.

\subsubsection{Дискретизация уравнений. Продолжение}

В прошлый раз остановились на дискретизации уравнения Рейнольдса.
А именно проинтегрировали его по одному шагу по времени и, воспользовавшись формулой Ньютона-Лейбница и формулой правых прямоугольников, получили:
\beq
w(t)-w(t-\Delta t)-\Delta t\left[\text{div}{\left(\frac{w^3}{\mu'}\nabla p\right)}\right]_t=\int\limits_{t-\Delta t}^{t}{\varphi(\tau)d\tau}
\eeq


\textbf{Дискретизация уравнения Рейнольдса. Продолжение}

Далее проинтегрируем полученное уравнение по элементам.
Запишем:
\beq\label{RaynoldsIntegrateOnElements1}
\int\limits_{A_{i,j}}{\left(w(t)-w(t-\Delta t)\right)dA}-\Delta t\left[\,\,\int\limits_{A_{i,j}}{\text{div}{\left(\frac{w^3}{\mu'}\nabla p\right)}dA}\right]_t=\int\limits_{t-\Delta t}^{t}{\psi(\tau)d\tau},
\eeq
где
\beq
\psi(t)=\int\limits_{A_{i,j}}{\left(Q_0(t)\delta(x-x_0,y-y_0)-\frac{C'}{\sqrt{t-t_0(x,y)}}\right)dA}
\eeq

Вспомним теорему Гаусса-Остроградского
$$\int\limits_{V}\text{div}\vec{f}\,dV=\int\limits_{\partial V}\vec{f}\cdot\vec{n}\,ds$$
и перепишем уравнение \eqref{RaynoldsIntegrateOnElements} в следующем виде:
\beq\label{RaynoldsIntegrateOnElements2}
\int\limits_{A_{i,j}}{\left(w(t)-w(t-\Delta t)\right)dA}-\Delta t\left[\,\,\int\limits_{C_{i,j}}{\frac{w^3}{\mu'}\frac{\partial p}{\partial\vec{n}}dC}\right]_t=\int\limits_{t-\Delta t}^{t}{\psi(\tau)d\tau}
\eeq
Здесь дополнительно воспользовались равенством: $\nabla p\cdot\vec{n}=\dfrac{\partial p}{\partial\vec{n}}$.

С полученным уравнением \eqref{RaynoldsIntegrateOnElements2} уже можно работать.

Напомню, что раскрытие трещины $w(x,y,t)$ аппроксимируем кусочно-постоянно:
\beq
w(x,y,t)=\sum_{m,n}w_{m,n}(t)H_{m,n}(x,y),
\eeq
где
\beq
H_{i,j}(x,y)=
\begin{cases}
1,\,\,\,(x,y)\in A_{i,j}\\
0,\,\,\,(x,y)\notin A_{i,j}	
\end{cases}
\eeq

Тогда (после подстановки кусочно-постоянного раскрытия $w$) уравнение \eqref{RaynoldsIntegrateOnElements2} перепишется в следующем виде:




\end{document}