\documentclass[main.tex]{subfiles}

\begin{document}

% \textcolor{red}{Вводная лекция}

\section{Лекция 11.05.2021 (Байкин А.Н.)}

\subsection{Моделирование течения жидкости в скважине}


Мы с вами движемся дальше.
Сегодня у нас будут тема про моделирование скважин.
Т.е. до этого мы рассматривали преимущественно именно процессы в самой трещине, процессы в окружающем пласте (такие как утечки или деформация породы), а закачка всегда предполагалась на забое скважины (расход задавался на входе в саму трещину).
Но вообще говоря у нас с вами есть скважины и то давление на входе в трещину, которое мы получаем в расчётах, не совпадает с давлением, которое мы получаем при измерении какими-то приборами (как на поверхности, так и когда мы опускаем датчик давления вниз к забою).
\\

Зачем нам необходимо моделировать скважину?

Во-первых, чтобы знать, какое давление на забое скважины и как оно соотносится с давлением на устье скважины.
Почему это важно?
С одной стороны, можно было бы сказать: давайте поставим датчик давления на забое и всё будет классно, но этот датчик давления будет стоять не на самой трещине (т.е. между датчиком и трещиной будет либо участок трубы, либо как минимум участок с перфораций вдоль которых возникает падение давления из-за трения; в итоге, измеряемое давление BHP будет немного выше, чем давление на входе в трещину).
Датчик забойного давления BHP практически никогда не ставят, потому что это дорого; обычно мы знаем только давление на устье WHP.

Чтобы осуществить пересчёт BHP через известное WHP нам необходимо учесть падение давления за счёт трения жидкости при движении по трубе и гидростатическое давление.
Т.е. получаем, что за счёт трения давление на забое снижается (относительно WHP), а за счёт гидростатики давление на забое увеличивается (относительно WHP).

Кроме того, скважину интересно моделировать, чтобы объяснить наблюдаемый hammer effect: при резком закрытии скважины наблюдаются колебательные движения жидкости между скважиной и трещиной, которые имеют форму затухающих колебаний.
И по этим затухающим колебаниям пытаются проводить диагностику. Как минимум говорят, что если есть hammer effect, то связь между скважиной и трещиной достаточно хорошая (т.е. перфорацию сделали достаточно качественно).
Дальше по hammer эффекту пытаются оценить размеры трещины (ширину, длину).

Ещё дальше пытаются понять, какой порт заработал (если есть несколько портов) -- правда это уже немного другая технология, которая называется tube waves от компании Schlumberger.
\\

Для чего ещё моделировать скважину?

Если мы запустили пульсы проппанта (с определённой концентрацией) наверху (на устье), то никто не говорит, что они в таком же виде дойдут до забоя.
Вообще говоря, они могут размазаться.
Сегодня мы размазывание не будем рассматривать, потому что для этого нужна двухскоростная модель, а сегодня мы рассмотрим только односкоростную модель.
Но вообще говоря из-за того, что у нас есть профиль скорости, частички проппанта будут собираться (проваливаться) к центру.

Размазывание концентрации проппанта важно моделировать, чтобы понимать, какое значение концентрации будет на входе в трещину.


\underline{Замечание аудитории.}
Слышали, что при движении по круглой трубе частички проппанта будут собираться в кольцо на расстоянии 0.6 радиуса от центра.
Говорят, что это связано с тем, что сами частички проппанта могут крутиться вокруг своей оси.


Сегодня рассмотрим модель попроще, чтобы вы поняли общую схему, а дальше уже можно придумывать более сложные модели (главное понять, какой эффект хочется описать).
\\

Теперь давайте приступим к самой модели.

Основные предположения модели:

1) наклонная скважина переменного радиуса $R$ (на рисунке я специально нарисовал 2 цилиндра, т.к. скважина вообще говоря может иметь переменное сечение, но в реальности оно обычно кусочно постоянное);

2) односкоростная модель $\vec{u}_p=\vec{u}_f=\vec{u}_m$ (жидкость и проппант движутся с одинаковой скоростью и эта скорость равна усреднённой скорости, формула для которой была в прошлый раз) -- это оправдано, когда жидкость достаточно вязкая и частички проппанта как-бы заморожены в жидкость;

3) жидкость неньютоновская (т.е. жидкость со степенной реологией);

4) течение не расслаивается (не может быть такого , что проппант где-то внизу пошёл и течение расслоилось);

5) ламинарный, переходный, турбулентный режимы;

6) сжимаемостью пренебрегаем.
\\

Что нам нужно, чтобы описать течение рассматриваемой жидкости?

1) Закон сохранения объёма проппанта:
\beq\label{ZSM_proppant}
\frac{\partial\left(cS(x)\right)}{\partial t}+\frac{\partial\left(cS(x)u_p\right)}{\partial x}=0,\,\,\,\,\,c_p\equiv c
\eeq

2) Закон сохранения объёма жидкости:
\beq\label{ZSM_fluid}
\frac{\partial\left((1-c)S(x)\right)}{\partial t}+\frac{\partial\left((1-c)S(x)u_f\right)}{\partial x}=0,\,\,\,\,\,c_f\equiv 1-c
\eeq

Уравнения \eqref{ZSM_proppant} и \eqref{ZSM_fluid} уже усреднённые по сечению скважины, но они выводятся точно также, как и для течения проппанта в трещине (только сейчас вместо раскрытия трещины $w(x)$ используем площадь сечения $S(x)$ и сейчас нет утечек).

Далее используя предположение односкоростной модели $u_p=u_f=u_m$, где $u_m$ -- среднеобъёмная усреднённая скорость смеси по сечению $S(x)$, складываем уравнения \eqref{ZSM_proppant} и \eqref{ZSM_fluid}:
\beq\label{Q_const_t}
\frac{\partial\left(S(x)u_m\right)}{\partial x}=\frac{\partial\left(Q(t,x)\right)}{\partial x}=0,
\eeq
где $Q(t,x)=Su_m=const(t)=Q_{inlet}(t),\,\,\,\,S=\pi R^2$.

Полученное уравнение говорит нам о том, что расход через любое поперечное сечение скважины одинаков и зависит от расхода, закачиваемого в скважину сверху.
Если изменяется сечение скважины, то соответственно изменяется скорость течения так, чтобы расход оставался прежним.

3) Граничное условие (на концентрацию проппанта) на устье скважины:
\beq\label{proppant_concentration_gu}
c|_{x=0}=c_{inlet}(t)
\eeq

Если проводить аналогию с течением проппанта в трещине, то уравнение \eqref{Q_const_t} аналогично эллиптическому уравнению, в котором необходимо было искать давление.

В итоге: мы знаем расход $Q_{inlet}(t)$; знаем площадь $S(x)$; можем найти скорость $u_m(x,t)$; как только знаем скорость, мы можем подставить её в уравнение \eqref{ZSM_proppant}, решить это уравнение и с учётом граничного условия \eqref{proppant_concentration_gu} найти концентрацию проппанта.
\\

При решении данной задачи можно использовать тот же алгоритм, что и для переноса проппанта в трещине, т.е. взять одномерную разностную схему (например, Лакса-Вендроффа с лимитерами), но я хотел бы ещё показать другой численный алгоритм.
В данном случае, когда рассматриваем односкоростную модель, этот алгоритм проще, намного быстрее и точнее.
\\

Можно показать, что в случае односкоростной модели уравнение \eqref{ZSM_proppant} для переноса проппанта можно переписать в более простом виде:
\beq\label{ProppantTransfer}
\frac{\partial c}{\partial t}+u_m\frac{\partial c}{\partial x}=0
\eeq
(уравнение \eqref{ZSM_proppant} -- это закон сохранения в дивергентной форме; а уравнение \eqref{ProppantTransfer} -- это классическое уравнение переноса, которое всем известно с урматов).

Физический смысл уравнения \eqref{ProppantTransfer}: концентрация $c$ неизменно переносится векторным полем $\vec{u}$ (в данном случае со скоростью $u_m$ вдоль скважины).

Разобьём суммарное время закачки на $k-1$ временных интервалов:
\beq
\Delta t_k=t_k-t_{k-1},\,\,\,\,\,t_0=0
\eeq

Обозначение: $F_k$ -- это значение величины $F$ в момент времени $t_k$.

В лагранжевых координатах $(t,X)$ на интервале $[t_k, t_{k+1}]$ имеем решение вида:
\beq\label{LagrangePosition}
c(t,X(t))=c(t,X|_{t=t_k}),\,\,\,\,\,X(t)=X|_{t=t_k}+\int\limits_{0}^{t}u_m(X(s))ds
\eeq

Грубо говоря, мы в начале трубы выпускаем некоторую лагранжеву частицу со скоростью $u_m$, и формула \eqref{LagrangePosition} говорит нам, что эта частица через время $t$ дойдёт до положения с координатой $X$.

Соответственно мы можем рассматривать весь этот процесс в виде набора фронтов концентрации.
За каждый новый шаг по времени мы выпускаем новый фронт, далее он движется по трубе и мы фиксируем кусочно постоянный уровень концентрации проппанта на каком-то участке скважины.

Чтобы перейти в эйлерову сетку в точке $x$, мы просто смотрим между какими фронтами эта точка $x$ лежит и говорим, что концентрация равна этому значению.
\\

Нам нужна не только концентрация, а прежде всего нам нужно знать давление.
Если на забое будет слишком большое давление, то жидкость может просто порвать трубу.
Как будем считать давление?
Для этого берём уравнение Навье-Стокса.
Выводим аналогично выводу транспорта проппанта для трещины.
Но здесь в роли малого параметра будет
$$\varepsilon=\frac{2R}{L},$$
где $2R$ -- диаметр скважины; $L$ -- длина скважины.

Вдоль оси $Ox$:
\beq
0=-\frac{dp(x)}{dx}+\frac{1}{r}\frac{d}{dr}\left(r\tau_{rx}\right)+\rho g\sin\theta
\eeq

\beq
\tau_{ij}=2\mu_sD_{ij},\,\,\,\,\,D=\frac{1}{2}\left(\nabla u + (\nabla u)^T\right),\,\,\,\,\,\tau_{rx}=\mu_s\frac{\partial u_x}{\partial r}
\eeq

Есть зависимость вязкости смеси от концентрации проппанта.
Формула Нолти:
\beq
\mu_s(c)=\mu_f\left(1-\frac{c}{c_{max}}\right)^{-2.5n_{clean}},
\eeq
где $c_{max}=0.65$ -- максимальная концентрация упаковки, $\mu_f$ -- вязкость чистой жидкости (гель без проппанта), $n_{clean}$ -- индекс течения чистой жидкости (без проппанта).

По сути сейчас докажем формулу Пуазейля с учётом силы тяжести:
\beq
0=-\frac{dp(x)}{dx}+\frac{1}{r}\frac{d}{dr}\left(r\mu_s\frac{\partial u_x}{\partial r}\right)+\rho g \sin{\theta}
\eeq
\beq
\frac{d}{dr}\left(r\mu_s\frac{\partial u_x}{\partial r}\right)=\left(\frac{dp(x)}{dx}-\rho g\sin{\theta}\right)r
\eeq

С учётом $\tau_{rx}|_{r=0}=0$ (условие регулярности):
\beq
\mu_s\frac{\partial u_x}{\partial r}=\left(\frac{dp}{dx}-\rho g\sin{\theta}\right)\frac{r}{2}
\eeq

Интегрируем:
\beq
u_x(r)=\frac{1}{4\mu_s}\left(\frac{dp}{dx}-\rho g\sin{\theta}\right)r^2+C
\eeq

Используем граничное условие $u_x|_{r=R}=0$:
\beq
u_x(r)=\underbrace{-\frac{R^2}{4\mu_s}\left(\frac{dp}{dx}-\rho g\sin{\theta}\right)}_{u_{max}}\left(1-\frac{r^2}{R^2}\right)=u_{max}\left(1-\frac{r^2}{R^2}\right)
\eeq

Нашли максимальную скорость $u_{max}$, а уравнение переноса записано в терминах средней скорости, поэтому необходимо найти соотношение между средней скоростью и максимальной скоростью.

Возьмём и усредним найденный профиль скорости по сечению:
\begin{multline}
u_m=\frac{1}{|S|}\int\limits_S{u_xdS}=\frac{1}{\pi R^2}\int\limits_{0}^{2\pi}\int\limits_{0}^{R}u_{max}r\left(1-\frac{r^2}{R^2}\right)drd\varphi=\\=u_{max}\frac{2\pi}{\pi}\int\limits_0^1\frac{r}{R}\left(1-\frac{r^2}{R^2}\right)d\left(\frac{r}{R}\right)=\frac{u_{max}}{2}
\end{multline}

Таким образом, получаем среднюю скорость ламинарного течения.
В целом отсюда можно найти давление для ламинарного течения!

В тех же самых предположениях можем вывести профиль скорости для степенной жидкости:
\beq
\tau_{ij}=K_s\dot{\gamma}^{n-1}D_{ij},\,\,\,\,\,\dot{\gamma}=\sqrt{\frac{1}{2}\sum\limits_{i,j=1}^{3}D_{ij}^2}
\eeq
Профиль скорости:
\beq
u_x=u_{max}\left(1-\left(\frac{r}{R}\right)^{(n+1)/n}\right)
\eeq
\ \\

Сейчас выведем формулу для давления немного по-другому.
Опять стартуем с уравнения Навье-Стокса и сразу усредняем:
\beq
0=-\frac{dp(x)}{dx}+\frac{1}{r}\frac{d}{dr}\left(r\tau_{rx}\right)+\rho g \sin{\theta}\,\,\bigg|\,\,\,\,\,\frac{1}{\pi R^2}\int\limits_S(\cdot)dS
\eeq

Получаем:
\beq\label{PressureGeneral}
\frac{d\overline{p}}{dx}=-\frac{2\tau_w}{R}+\overline{\rho}g\sin{\theta},
\eeq
где $\tau_w=-\tau_{rx}|_{r=R}$ -- напряжение сдвига (трения) на стенке трубы.
Его можно измерить и в случае турбулентного течения (например, для известного перепада давления найти $\tau_w$ из \eqref{PressureGeneral}), поэтому этот вывод формулы для давления более общий (в предыдущем выводе не понятно, что такое профиль скорости в случае турбулентного течения).

Видим, что для определения давления профиль скорости нам и не нужен.

Обычно экспериментаторы работают с безразмерными величинами для того, чтобы можно было масштабировать результаты (измерить на одной трубе, а распространить результаты на трубы произвольного диаметра), поэтому вводят коэффициент трения Фаннинга:
\beq
f_s=\frac{\tau_w}{\rho u_m^2/2}
\eeq

Тогда уравнение \eqref{PressureGeneral} примет вид:
\beq
\frac{dp}{dx}=-\frac{\rho u_m^2}{R}f_s+\rho g\sin{\theta}
\eeq

Какой физический смысл у полученного уравнения?
Это баланс сил: есть сила давления, кроме того проталкивать жидкость нам помогает сила тяжести, а препятствует трение жидкости о стенки трубы.
\\

Давайте посчитаем коэффициент Фаннинга для ламинарного течения.

Профиль скорости:
\beq
u_x=2u_m\left(1-\left(\frac{r}{R}\right)^2\right)
\eeq

Подставляем профиль скорости в выражение для $\tau_w$:
\beq
\tau_w=-\mu_s\frac{\partial u_x}{\partial r}\bigg|_{r=R}=\frac{4\mu_s u_m}{R}
\eeq

Подставляем $\tau_w$ в выражение для коэффициента Фаннинга:
\beq
f_s=\frac{\tau_w}{\rho u_m^2/2}=\frac{4\mu_s u_m}{R\rho u_m^2/2}=\frac{8\cdot 2}{\rho u_m(2R)/\mu_s}=\frac{16}{Re},
\eeq
где
$$Re=\frac{\rho u_m(2R)}{\mu_s}$$

Для степенной жидкости можно показать, что
\beq
f_s=\frac{16}{Re'},
\eeq
где
$$Re'=\dfrac{\rho u_m^{2-n}(2R)^n}{K_s\left(\dfrac{3n+1}{4n}\right)^n8^{n-1}}-\text{обобщённое число Рейнольдса}$$

Для турбулентного течения явной формулы для коэффициента Фаннинга нет, но есть экспериментальные корреляции.

Теперь давайте перейдём к самому расчёту давления.
\beq
\frac{d\overline{p}}{dx}=-\frac{2\tau_w}{R}+\overline{\rho}g\sin{\theta}
\eeq

\beq
p_{bh}(t,x)=p_{wh}(t)+\Delta p_h(t,x)-\Delta p_{fric}(t,x)
\eeq

Гидростатика:
\beq
\Delta p_h(t,x)=\int\limits_{0}^{x}\rho_s(c(t,s))g\,\sin{\theta(s)}ds
\eeq

\beq
\rho_s(c)=\rho_p c+\rho_f\left(1-c\right)
\eeq

Трение:
\beq
\Delta p_{fric}(t,x)=\int\limits_0^x\frac{2\tau_w(t,s)}{R(s)}ds - \text{давление трения}
\eeq

Переходим к следующей теме.

\subsection{Разделение потоков между трещинами}

В предыдущем разделе рассмотрели скважину от устья до забоя.
Но у нас между трещиной и забоем могут быть участки скважины (во-первых, датчик забойного давления обычно выше трещины; во-вторых, у нас есть трение вдоль перфораций).
Это ещё усугубляется задачей многостадийного ГРП, когда у нас есть несколько портов.

Я вам рассказывал про технологию plug and purf, когда опускают перфорационный пистолет, который сразу может сделать несколько отверстий (портов).
В итоге при закачке растим несколько трещин и весь расход, который качаем в скважину, перераспределяется между трещинами.
В зависимости от чего?
Во-первых, в зависимости от трения по трубе и гидростатики.
Во-вторых (что более существенно), от давления на перфорациях.
Например, одна перфорация сделана хорошо и через неё будет хорошая проводимость.
Другая перфорация сделана плохо и через неё будет плохая проводимость.
Кроме того, есть эффект влияния соседних трещин друг на друга.
\\

Если есть 3 трещины, то боковые трещины пойдут криво, но мы это не учитываем (пока рассматриваем плоские трещины).

Если успеем добраться, то потом расскажу, что делать с кривыми трещинами.

Но даже если у нас 3 плоские трещины, то боковые трещины за счёт упругого воздействия через породу зажимают центральную трещину.
Соответственно в боковые трещины будет втекать больше жидкости и расход на них будет выше, чем в центральной части.


\subsection{Математическая модель гидроудара в вертикальной скважине}






\end{document}