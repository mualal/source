% !TeX spellcheck = en_US
% !TeX program = xelatex

\documentclass[a4paper,12pt]{article}
\renewcommand{\baselinestretch}{1.1}
\usepackage[utf8]{inputenc}
\usepackage[T2A, T1]{fontenc}
\usepackage[english, russian]{babel}

\usepackage{fontspec}
\setmainfont{Times New Roman}
\usepackage{setspace,amsmath}
\usepackage{amssymb}
\usepackage{dsfont}

\makeatletter
\let\@fnsymbol\@arabic
\makeatother

\usepackage{geometry}
\geometry{
a4paper,
total={190mm, 257mm},
left=10mm,
top=20mm,
}

\usepackage{systeme}
\usepackage{skak}
\usepackage{mathtools}
\usepackage{unicode-math}
\usepackage{array}
\usepackage{makecell}
\usepackage{subfiles}
\usepackage{hyperref}
\hypersetup{pdfstartview=FitH, linkcolor=black, urlcolor=blue, colorlinks=true}
\usepackage{framed}
\usepackage{graphicx}
\usepackage{caption}
\usepackage{subcaption}
\usepackage{color}
\usepackage{chngcntr}
\usepackage{tikz}
\usepackage{fancyhdr}
\usepackage{fancyvrb}
\usepackage{comment}
\usepackage{adjustbox}
\usepackage[breakable, skins]{tcolorbox}

\pagestyle{fancy}
\fancyhf{}
\rhead{\hyperlink{page.1}{\textcolor{violet}{Вернуться к содержанию}}}
\lhead{\leftmark}
\cfoot{\thepage}

\setsansfont{Arial}
\setmonofont{Courier New}
\usepackage[cache=false]{minted}
\usemintedstyle{borland}

\usepackage{float}
\floatstyle{plaintop}
\usepackage{enumitem}
\setlength{\parindent}{0pt}

\graphicspath{{./img/}}
\newcommand{\myPictWidth}{.95\textwidth}
\newcommand{\phm}{\phantom{-}}
\newcommand{\beq}{\begin{equation}}
\newcommand{\eeq}{\end{equation}}
\newcommand{\insertslide}[1]{\newpage\includegraphics[width=\textwidth]{#1}}
\newcommand{\insertproductivityslide}[1]{\newpage\includegraphics[width=\textwidth, page=#1]{WellProductivity.pdf}}
\newcommand{\insertwaterfloodingslide}[1]{\newpage\includegraphics[width=\textwidth, page=#1]{Waterflooding.pdf}}

\newenvironment{nohyphen}
  {\tolerance=1
   \emergencystretch=\maxdimen
   \hyphenpenalty=10000
   \hbadness=10000}
  {\par}


\begin{document}
	%\large
	\begin{nohyphen}

	\textbf{О курсах лекций.}
	
	Курсы лекций "<Методы математической физики в нефтяном инжиниринге"> и "<Заводнение терригенных и карбонатных коллекторов"> читаются на магистерской программе "<Математическое моделирование процессов нефтегазодобычи"> в Высшей Школе Теоретической Механики и Математической Физики в Санкт-Петербургском Политехническом университете Петра Великого.
	
	Автор курсов: Евгений Викторович Юдин.
	
	Видеозаписи лекций курсов доступны по ссылке: \href{https://csspbstu-my.sharepoint.com/:f:/g/personal/muravtsev_aa_edu_spbstu_ru/Epiacj6WFMBHqIF6E3YQgCMB7yi5NAA1ycqFLqrTZMhJ4w?e=i2agP0}{GO TO RECORDINGS}.
	\end{nohyphen}
	
	\tableofcontents
	\title{Заводнение терригенных и карбонатных коллекторов\\Конспект лекций}
	\author{Муравцев А.А.\thanks{конспектирует; email: almuravcev@yandex.ru}
	\and
	Юдин Е.В.\thanks{лектор и составитель слайдов, Высшая школа теоретической механики, Санкт-Петербургский Политехнический университет. Дополнительные материалы к лекциям \href{https://csspbstu-my.sharepoint.com/:f:/g/personal/muravtsev_aa_edu_spbstu_ru/Epiacj6WFMBHqIF6E3YQgCMB7yi5NAA1ycqFLqrTZMhJ4w?e=i2agP0}{доступны по ссылке}.}}
	\maketitle
	\subfile{parts/2022_02_04}
	\newpage
	\subfile{parts/2022_02_11}
	\newpage
	\subfile{parts/2022_03_04}
	\newpage
	\subfile{parts/2022_03_11}
	\newpage
	\subfile{parts/2022_03_18}
	\newpage
	\subfile{parts/2022_03_25}
	\newpage
	\subfile{parts/2022_04_01}
	\newpage
	\subfile{parts/2022_04_08}
	\newpage
	\subfile{parts/2022_04_15}
	\newpage
	\subfile{parts/2022_04_29}
	\newpage
	\subfile{parts/2022_05_06}
	\newpage
	\subfile{parts/2022_05_20}
	\newpage
	\subfile{parts/2022_05_20_test}
\end{document}
