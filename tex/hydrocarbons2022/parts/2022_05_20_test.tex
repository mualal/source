\documentclass[main.tex]{subfiles}

\begin{document}

\section*{Тест по курсу "<Заводнение терригенных и карбонатных коллекторов">\markboth{ТЕСТ ПО КУРСУ (ЮДИН Е.В.)}{}}
\addcontentsline{toc}{section}{ТЕСТ ПО КУРСУ (ЮДИН Е.В.)}

\begin{enumerate}
	\item Как зовут преподавателя? \textbf{Юдин Евгений Викторович}
	\item Давление в пласте с газовой шапкой: \textbf{Равно давлению насыщения}.
	\item В гидрофобном коллекторе уровень свободной воды: \textbf{Выше уровня ВНК}.
	\item Вязкость природного газа при высоких давлениях: \textbf{Снижается с ростом температуры}.
	\item Газовый фактор на скважине: \textbf{Зависит от режима работы пласта (может быть и выше газонасыщенности, и ниже газонасыщенности, и равен газонасыщенности)}.
	\item При росте степени Кори ОФП по воде и фиксированной ОФП по нефти: \textbf{Скорость обводнения скважин падает}.
	\item Чем характеризуются недиагональные элементы матрицы взаимных продуктивностей: \textbf{Это степень влияния депрессии одной скважины на дебит другой}.
	\item Оцените КИН в режиме истощения для пласта с пластовым давлением 250 атм и общей сжимаемостью $5\cdot 10^{-4}$ 1/атм: \textbf{10\%}.
	\item Соотношение добывающих и нагнетательных скважин в семиточечной системе разработки: \textbf{2 к 1}.
	\item В девятиточечной системе разработки забойное давление на добывающих скважинах равно 50 атм, на нагнетательных -- 450 атм. Оцените установившееся среднее пластовое давление для единичного соотношения подвижностей воды и нефти, если коэффициенты продуктивности всех скважин одинаковы, а начальное пластовое давление равно 250 атм. \textbf{150 атм}.
	\item В девятиточечной системе разработки забойное давление на добывающих скважинах равно 50 атм, на нагнетательных -- 450 атм. Оцените установившееся среднее пластовое давление для соотношения подвижностей воды и нефти равного 2, если коэффициенты продуктивности всех скважин одинаковы, а начальное пластовое давление равно 250 атм. \textbf{210 атм}.
	\item Поправка Вогеля это: \textbf{Корреляция для оценки снижения продуктивности скважины при работе ниже давления насыщения}.
	\item Скин-фактор -- это характеристика: \textbf{Степени изменения проницаемости призабойной зоны}.
	\item Коэффициент извлечения нефти это: \textbf{Отношение извлечённого количества нефти к геологическим запасам нефти на месторождении}.
	\item Закачка воды в пласт осуществляется для: \textbf{Все ответы верны (и для поддержания пластового давления, и для вытеснения нефти водой)}.
	\item Для расчета дебита скважины необходимо использовать: \textbf{Эффективную проницаемость}.
	\item Закон Дарси это? \textbf{Закон фильтрации жидкости в пористой среде}.
	\item Проницаемость это: \textbf{Способность породы пласта пропускать флюид при наличии перепада давления}.
\end{enumerate}
	
\end{document}
