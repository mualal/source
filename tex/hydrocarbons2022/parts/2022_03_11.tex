\documentclass[main.tex]{subfiles}

\begin{document}
\section{\textcolor{red}{Семинар 11.03.2022}}

\subsection{Производительность скважины в системе разработки}

\includegraphics[width=\textwidth]{slide1_00098}

Как считать производительность скважины, которая окружена другими скважинами?

Когда говорим о работе скважины в регулярной системе разработки, предполагается, что можем выбрать элемент симметрии и определённым количеством этого элемента замостить всю систему разработки.

Здесь представлены наиболее распространённые регулярные системы разработки. Одним из наиболее важных параметров, характеризующих данную систему разработки, является соотношение добывающих и нагнетательных скважин в её элементе симметрии.

В пятиточечной: 1 доб к 1 нагн. В семиточечной: 2 доб к 1 нагн. В девятиточечной: 3 доб к 1 нагн.

Блочная система, строго говоря, не является регулярной, если мы не делаем большое количество блоков (на практике такое не встречается).

Нефтяники любят девятиточечную систему разработки: если что-то случится с одной из добывающих скважин, то в элементе симметрии есть ещё две, с которыми можно продолжать работу); кроме того, девятиточечная система трансформируется в пятиточечную при обводнении (тоже очень удобно).\\

Систему разработки проектируют так, чтобы среднее пластовое давление, которое будем получать за счёт работы скважин, было бы равно начальному пластовому давлению, т.е. проектируют на целевой уровень компенсации 100\%.\\

Компенсация -- это отношение закаченной жидкости к добытой (необходимо различать понятия накопленной и текущей компенсаций). Как правило, значения компенсации для площадного заводнения больше 100\% (около 120\%), т.к. закаченная жидкость уходит в края или в ЗКЦ (заколонную циркуляцию).\\

Допустим, на одну нагнетательную скважину приходятся 3 добывающие. Пусть среднее пластовое давление 250 атм. Давление добычи 50 атм. Коэффициенты продуктивности на нагнетательной и добывающей скважинах равны (соотношение подвижностей равно 1). Почему получается, что одна нагнетательная скважина может компенсировать добычу от трёх добывающих и держать при этом начальное пластовое давление? (Так, действительно, получается, например центральная часть Приобского месторождения разбурена девятиточкой)\\

Первое предположение: у нагнетательной скважины больше репрессия. Но оценим тогда, какой она должна быть. Из условия депрессия на добывающей скважине 200 атм. У нагнетательной репрессия должна быть в 3 раза больше (т.е. 600 атм), т.е. забойное давление 850 атм, т.е. на устье должны держать около 600 атм (преполагаем, что скважина на глубине 2.5 км).
Но предельное значение водовода около 160 атм. И таких мощных насосов тоже нет. Первое предположение неверно.

На самом деле коэффициенты продуктивности на добывающей и нагнетательной скважинах не остаются равными. Когда под высоким давлением качаем воду в нагнетательную скважину, начинает расти трещина автоГРП. Некоторые трещины автоГРП достигают 1 км и более.\\

Трещины автоГРП могут как помочь, так и сыграть злую шутку.
Например в Западной Сибири были проблемы: есть региональный стресс (в направлении северо-запад), трещина будет раскрываться по региональному стрессу. На всех скважинах в системе разработки делаем ГРП (полудлина 150 метров): в том числе и на нагнетательных (т.к. они изначально добывающие и работают на естественном режиме).

Когда начинаем нагнетать воду, трещина начинает разрастаться и дорастает до угловой добывающей скважины в системе разработки. И это большая проблема. Например, на севере Приобки многие угловые скважины были под ударами трещин автоГРП и поэтому трансформировали систему из девятиточки в пятиточку.

\subsubsection{Маскет 1937}

\includegraphics[width=\textwidth]{slide1_00099}

Маскет (инженер по плотинам) пришёл в нефтянку и увидел кладезь нерешённых задач. Маскет основал нефтяной инжиниринг, которым мы сейчас и пользуемся.\\

Решил ряд задач, в том числе и для регулярных систем разработки: вывел формулы для производительности нагнетательной скважины, находящейся в рядной лобовой, в рядной шахматной в пятиточечной и семиточечной системах разработки. Вывел эти формулы с помощью ТФКП. Дробные значения в формулах -- это аппроксимационные значения сумм бесконечных рядов. Формулы очень похожи на формулу Дюпюи, но здесь $\Delta p$ -- это разница между забойным давлением нагнетания и забойным давлением добычи (Маскету так было удобнее).

\insertslide{slide1_00100}

Представлены формулы Маскета в промысловых единицах.

\subsubsection{Deppe 1961}

\includegraphics[width=\textwidth]{slide1_00101}

Чуть позже Deppe вывел формулу и для девятиточки. Почему же Маскет сразу не вывел эту формулу? Т.к. в девятиточке добывающие скважины находятся не в равноправных позициях: есть боковые и угловые по отношению к нагнетательным, поэтому и математические выкладки для вывода формулы оказываются сложнее.

\insertslide{slide1_00102}

Представлены формулы Deppe для девятиточки в промысловых единицах.

\insertslide{slide1_00104}

Представлены формулы Deppe для девятиточки в более удобном и правильном виде.

\insertslide{add_slide1}

В последнее время развиваются системы разработки с горизонтальными скважинами. Ключевое их отличие от систем с вертикальными скважинами заключается в том, что теперь заканчивание скважины является элементом системы разработки.\\

Когда работаем с вертикальными скважинами, то не важно заканчиваем открытым или закрытым хвостовиком, делаем ли ОПЗ (обработку призабойной зоны пласта), делаем ли малообъёмный ГРП -- всё это не оказывает существенного влияния на распределение фильтрационных потоков.\\

Когда же говорим про трещину ГРП с большой полудлиной (150-180 метров), то система заканчивания в этом случае является элементом системы разработки и нужно её учитывать при выборе системы разработки (её типа и плотности скважин), так как от этого зависит эффективность разработки в целом.\\

Изображена шахматная рядная система разработки с соотношением доб к нагн 1 к 1. Как посчитать продуктивность горизонтальной скважины, находящейся в такой системе разработки. По общей формуле с форм-фактором:
\beq
q_h=\dfrac{2\pi k_h h\left(\bar{p}-p_{prod}\right)}{\mu B\dfrac{1}{2}\ln{\left(\dfrac{4A}{\gamma_1 C_A \dfrac{L^2}{16}}\right)}},
\eeq
где $k_h$ -- проницаемость по горизонтали, $A$ -- площадь, приходящаяся на горизонтальную скважину.\\

При соотношении сторон
\beq\label{optimal}
\dfrac{L_x}{L_y}=2+1.15\dfrac{L^2}{A}
\eeq
достигается наибольшая продуктивность горизонтальной скважины.\\

Длина горизонтального ствола обычно составляет около 1000 метров.\\

Удивляет ценообразование: в случае горизонтальной скважины цена проходки (за 1 метр) примерно в 5 раз выше, чем для вертикальной скважины (хотя 60-70\% траектории совпадают, что для вертикальной, что и для горизонтальной скважин).

\insertslide{add_slide2}

Представлено выражение для форм фактора.\\

$a$, $b$ и $c$ -- корреляционные коэффициенты (нашли их подгоном к точному решению).

Полученная формула для производительности горизонтальной скважины в системе разработки верна в общем виде при любом соотношении сторон участка со скважиной. Но при соотношении сторон \eqref{optimal} получаем оптимальную систему разработки с наибольшей скоростью возврата инвестиций.

\insertslide{add_slide3}

\insertslide{slide1_00105}

Можно придумать очень много регулярных систем разработки, для которых нет формул. Что делать в этом случае?

\subsubsection{Обобщение Hansen 2003, единичное соотношение подвижностей}

\includegraphics[width=\textwidth]{slide1_00106}

Помним, что система разработки характеризуется следующими параметрами: целевая компенсация, соотношение добывающих и нагнетательных скважин, соотношение подвижностей воды и нефти (вытесняющей и вытесняемой фаз).\\

Если изобразим зависимость безразмерного давления от расстояния между скважинами для различных систем разработки, то увидим, что построенные зависимости практически совпадают.

Тогда можем ввести среднюю кривую для любой регулярной системы разработки будем использовать эту кривую.\\

Hansen решил обобщить формулы Маскета на общий случай двухфазного флюида и использовать в формуле вместо депрессии среднее пластовое давление (инженер мыслит в рамках среднего пластового давления).\\

Рассматривается установившийся режим и произвольная срегулярная система разработки. Суммируя ряды, Hansen показал, что безразмерное давление для добывающей и нагнетательной скважин совпадают.\\

Возьмём элемент симметрии системы разработки и рассчитаем его по материальному балансу (100\% компенсация). Пока рассматриваем случай с единичным соотношением подвижностей воды и нефти, поэтому коэффициенты продуктивности можем сократить.\\

Выражаем среднее пластовое давление и находим дебиты по формуле Дюпюи.\\

Примеры.
Пусть давление на забое нагнетательной скважины $p_{inj}=450$ атм, а на забое добывающей скважины $p_{prod}=50$ атм.

Для пятиточки среднее пластовое давление будет $\bar{p}=\dfrac{450+50}{2}=250$ атм.\\

Для семиточки среднее пластовое давление будет $\bar{p}=\dfrac{450+2\cdot50}{3}\approx183.3$ атм.\\

Для девятиточки среднее пластовое давление будет $\bar{p}=\dfrac{450+3\cdot50}{4}=150$ атм.

\subsubsection{Обобщение Hansen 2003, неединичное соотношение подвижностей}

\includegraphics[width=\textwidth]{slide1_00107}

Получены формулы при неединичном соотношении подвижностей (нефть обычно менее подвижна, чем вода).\\

Пример.
Дано: $M=2$, $p_{inj}=450$, $p_{prod}=50$.

Для пятиточки среднее пластовое давление будет $\bar{p}=\dfrac{2\cdot450+50}{3}\approx316.7$ атм.\\

Для семиточки среднее пластовое давление будет $\bar{p}=\dfrac{2\cdot450+2\cdot50}{4}=250$ атм.\\

Для девятиточки среднее пластовое давление будет $\bar{p}=\dfrac{2\cdot450+3\cdot50}{5}=210$ атм.\\

В данном случае получаем более высокие значения среднего пластового давления (чем в случае единичного соотношения подвижностей), так как в случае разных подвижностей добывающей скважине сложнее добывать.\\

Примечание. Трещина ГРП образуется не так, как трещина на льду, а более похоже на трещину при разрезании куска сливочного масла (диссипативная система).

\subsubsection{Проводимость элемента системы разработки}

\includegraphics[width=\textwidth]{slide1_00108}

Задача: подобрать такую систему разработки, чтобы максимизировать проводимость (чтобы быстрее отобрать и получить максимальную скорость возврата инвестиций)

\insertslide{slide1_00109}

\subsection{Оптимальная проводимость системы разработки}

\includegraphics[width=\textwidth]{slide1_00110}

\insertslide{slide1_00111}

\insertslide{slide1_00112}

\subsection{Анализ производительности методом индикаторных диаграмм}

\includegraphics[width=\textwidth]{slide1_00114}

\insertslide{slide1_00115}

\insertslide{slide1_00117}

\insertslide{slide1_00118}

\insertslide{slide1_00119}

\insertslide{slide1_00120}

\insertslide{slide1_00121}

\insertslide{slide1_00122}

\insertslide{slide1_00123}

\insertslide{slide1_00124}

\subsection{Понятие псевдоустановившегося режима в системе разработки}

\includegraphics[width=\textwidth]{slide1_00125}

\insertslide{slide1_00126}

\subsection{Обобщение модели Хансена на псевдоустановившийся режим}

\includegraphics[width=\textwidth]{slide1_00127}

\insertslide{slide1_00128}

\insertslide{slide1_00129}

\insertslide{slide1_00130}

\insertslide{slide1_00131}

\insertslide{slide1_00132}

\insertslide{slide1_00133}

\insertslide{slide1_00134}

\insertslide{slide1_00135}

\end{document}