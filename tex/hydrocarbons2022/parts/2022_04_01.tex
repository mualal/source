\documentclass[main.tex]{subfiles}

\begin{document}
%\section{\textcolor{red}{Семинар 01.04.2022}}
\section{Семинар 01.04.2022}

\subsection{Вытеснение на микроуровне}

%\includegraphics[width=\textwidth]{slide2_00008}
\includegraphics[width=\textwidth,page=8]{Waterflooding.pdf}

Необходимо понимать различные масштабы вытеснения.

%\insertslide{slide2_00009}
\insertwaterfloodingslide{9}

\subsection{Межфазное натяжение}

%\includegraphics[width=\textwidth]{slide2_00011}
\includegraphics[width=\textwidth,page=11]{Waterflooding.pdf}

\subsection{Смачиваемость}

%\includegraphics[width=\textwidth]{slide2_00012}
\includegraphics[width=\textwidth,page=12]{Waterflooding.pdf}

Одна и та же капля может как смачивать, так и не смачивать твёрдую поверхность в зависимости от того, в каком газе (или жидкости) эта капля находится.

%\insertslide{slide2_00013}
\insertwaterfloodingslide{13}

Острый угол $\Rightarrow$ жидкость смачивает породу. Тупой угол $\Rightarrow$ жидкость не смачивает породу.\\

От природы 99\% всех минералов, которые составляют нефтяные коллектора, являются гидрофильными.\\

Почему же порядка 50-60\% месторождений имеют промежуточную смачиваемость или являются гидрофобными? Это связано с тем, что в нефти обычно содержатся тяжёлые соединения (ненасыщенные асфальтены, из органической химии у них длинные хвосты; они одной стороной прилипают к известняку, а другой торчат в стороны), из-за которых поверхность известняка гидрофобизуется.
Поэтому (когда смотрим карбонатные коллектора), чтобы понять гидрофобный или гидрофильный коллектор, в первую очередь нужно понять состав нефти. Если в нефти есть тяжёлые компоненты, то у коллектора промежуточная смачиваемость.
А если у такого коллектора есть ещё и трещины, то заводнять строго не рекомендовано: вода прорвётся по трещинам и не зайдёт в матрицу промежуточной смачиваемости.

%\insertslide{slide2_00014}
\insertwaterfloodingslide{14}

%\insertslide{slide2_00015}
\insertwaterfloodingslide{15}

При уменьшении солёности воды порода становится более смачиваемой водой (более гидрофильной). На этом принципе основана закачка низкоминерализованной воды (smart water), чтобы изменить смачиваемость (сделать породу менее гидрофобной) и за этот счёт вытащить больше запасов. Но вряд ли это даёт какой-либо существенный прирост.\\

Есть интересные кейсы на шельфовом месторождении Вьетнама. 2 пласта. И проблема в том, что вода в первом пласте не совместима с водой во втором пласте. Если эти две воды смешиваются, то при термобарических условиях, которые есть на скважине, срезу выпадает большое количество солей. Глубинно-насосное оборудование встаёт, и скважина перестаёт добывать. Закачка кислоты для разъедания солей помогает ненадолго, и в итоге испортится подземное оборудование. А всего лишь в самом начале необходимо было проверить совместимость воды!\\

Но из-за низкоминерализованной (пресной) воды набухают глины, которые срезают эксплуатационную колонну. Поэтому необходимо закачивать воду примерно той же минерализации, что и уже есть в пласте (совместимую воду). Но если так произошло (обрезка), то качественно цементируют, ждут, когда цемент застынет, делают зарезку бокового ствола (ЗБС), кидают хвостовик с перекрытием и т.д. И начинают добывать. Но это всё по стоимости сопоставимо с новой скважиной.

%\insertslide{slide2_00016}
\insertwaterfloodingslide{16}

При движении капли краевой угол натекания больше, чем краевой угол оттекания $\Rightarrow$ гистерезис краевого угла смачивания.

%\insertslide{slide2_00017}
\insertwaterfloodingslide{17}

Интересный метод, который был опробован в советское время: проверили потенциал альфа-поляризации, закачали пресную воду и ещё раз проверили потенциал альфа-поляризации. И определили, что коллектор является гидрофобным. Одно из немногих направлений, когда измерили гидрофобность "<не в пробирке">, а на месторождениии.\\

Смачиваемость является крайне важным параметром, но к нему почему-то в настоящее время не присматриваются.

%\insertslide{slide2_00018}
\insertwaterfloodingslide{18}

Дренирование = увеличение насыщенности несмачивающей фазы\\

Пропитка = увеличение насыщенности смачивающей фазы

%\insertslide{slide2_00019}
\insertwaterfloodingslide{19}

Обсуждали ранее, что состав нефти есть главная причина изменения смачиваемости. 

%\insertslide{slide2_00020}
\insertwaterfloodingslide{20}

\subsection{Капиллярное давление}

%\includegraphics[width=\textwidth]{slide2_00023}
\includegraphics[width=\textwidth,page=23]{Waterflooding.pdf}

Пример: кусочек сахара прикасается к поверхности чая. Чай впитывается выше уровня.\\

Капиллярное давление определяем по высоте столба поднимающейся жидкости.

%\insertslide{slide2_00024}
\insertwaterfloodingslide{24}

%\insertslide{slide2_00025}
\insertwaterfloodingslide{25}

Влияние капиллярных сил на вытеснение (в том числе и на макромасштабе) критично. Почему?\\

Строим распределение от скважины к границам области дренирования. Ранее получали, что давление распределено по логарифму: быстро растёт вблизи скважины и медленно растёт вдали от скважины. Скорости зависят от градиента давления. Если этот градиент меньше капиллярного давления, то жидкость двигаться не будет. Поэтому вблизи скважины за счёт депрессии сможем нормально закачать воду и добыть запасы. Но вдали от скважины вытеснение определяется капиллярными силами, поэтому важно знать смачиваемость. В гидрофильном случае всё ОК. А в гидрофобном случае на расстоянии от скважины капиллярные силы не дадут воде зайти в поры матрицы!

%\insertslide{slide2_00026}
\insertwaterfloodingslide{26}

В трещиноватом гидрофобном коллекторе вода в матрицу не зайдёт (мешают капиллярные силы), а будет фильтроваться только по трещинам.\\

Не рекомендуется заводнять трещиноватые ГИДРОФОБНЫЕ коллектора! КИН будет точно ниже 10\%.\\

Изменять смачиваемость коллектора в таких случаях очень дорого, нужно качать специфическую химию, которая тоже НЕ всегда ПОМОЖЕТ! Нагревать пласт тоже дорого. С таких проектов лучше уходить; особенно, если от них есть завышенные ожидания.

%\insertslide{slide2_00027}
\insertwaterfloodingslide{27}

Капиллярное давление является причиной наличия переходной зоны.\\

ВНК = граница, на которой насыщенность нефтью равна нулю.\\

%\insertslide{slide2_00028}
\insertwaterfloodingslide{28}

%\insertslide{slide2_00029}
\insertwaterfloodingslide{29}

\subsection{J-функция Леверетта}

%\includegraphics[width=\textwidth]{slide2_00030}
\includegraphics[width=\textwidth,page=30]{Waterflooding.pdf}

%\insertslide{slide2_00031}
\insertwaterfloodingslide{31}

%\insertslide{slide2_00032}
\insertwaterfloodingslide{32}

%\insertslide{slide2_00033}
\insertwaterfloodingslide{33}

%\insertslide{slide2_00034}
\insertwaterfloodingslide{34}

%\insertslide{slide2_00035}
\insertwaterfloodingslide{35}

%\insertslide{slide2_00036}
\insertwaterfloodingslide{36}

%\insertslide{slide2_00037}
\insertwaterfloodingslide{37}

%\insertslide{slide2_00038}
\insertwaterfloodingslide{38}

%\insertslide{slide2_00039}
\insertwaterfloodingslide{39}

%\insertslide{slide2_00040}
\insertwaterfloodingslide{40}

\subsection{Кейс. Сихорейское месторождение}

\includegraphics[width=\textwidth]{2022-04-01-0_better}

\insertslide{2022-04-01-1_better}

\insertslide{2022-04-01-2_better}

\insertslide{2022-04-01-3_better}

\insertslide{2022-04-01-4_better}

\insertslide{2022-04-01-5_better}

\insertslide{2022-04-01-6_better}

\insertslide{2022-04-01-7_better}

\insertslide{2022-04-01-8_better}

\insertslide{2022-04-01-9_better}

\insertslide{2022-04-01-10_better}

\insertslide{2022-04-01-11_better}

\end{document}
