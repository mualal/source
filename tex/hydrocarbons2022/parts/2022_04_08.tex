\documentclass[main.tex]{subfiles}

\begin{document}
%\section{\textcolor{red}{Семинар 08.04.2022}}
\section{Семинар 08.04.2022}

\subsection{Относительные фазовые проницаемости}

%\includegraphics[width=\textwidth]{slide2_00042}
\includegraphics[width=\textwidth,page=42]{Waterflooding.pdf}

Эволюция моделей вытеснения:
\begin{itemize}
	\item поршневое вытеснение (для очень хороших коллекторов);
	\item модель Баклея-Леверетта (после прохождения фронта воды за фронтом остаётся нефть; вода и нефть мешают друг другу течь)
\end{itemize}

Модель Баклея-Леверетта формулируется для двух типов флюида (например, нефти и воды)

ОФП -- функция от насыщенности данного типа флюида. Является множителем на абсолютную проницаемость.

\textbf{Важна монотонность} зависимости ОФП от насыщенности данного типа флюида.

%\insertslide{slide2_00043}
\insertwaterfloodingslide{43}

Чем выше проницаемость, тем ниже значение остаточной водонасыщенности: в песчанике с крупными порами небольшие капиллярные давления. Так как они небольшие, то и намертво связанной воды тоже немного.



%\insertslide{slide2_00044}
\insertwaterfloodingslide{44}

%\insertslide{slide2_00045}
\insertwaterfloodingslide{45}

%\insertslide{slide2_00046}
\insertwaterfloodingslide{46}

На принципе гистерезиса ОФП построен метод увеличения нефтеотдачи (МУН), называемый циклическим заводнением. В 2008-2010 годах был очень популярен.

%\insertslide{slide2_00047}
\insertwaterfloodingslide{47}

В гидрофобном коллекторе проблемы при заводнении: прорывы, не будет вытеснения. Чисто гидрофобных коллекторов мало, но даже коллектора с промежуточной смачиваемостью -- это уже большая проблема.

%\insertslide{slide2_00048}
\insertwaterfloodingslide{48}

%\insertslide{slide2_00049}
\insertwaterfloodingslide{49}

В модели Баклея-Леверетта давление в законе Дарси для воды и для нефти одно и то же.

Если рассматриваем композиционное моделирование, то для каждой компоненты записываем закон Дарси.

%\insertslide{slide2_00050}
\insertwaterfloodingslide{50}

%\insertslide{slide2_00051}
\insertwaterfloodingslide{51}

%\insertslide{slide2_00052}
\insertwaterfloodingslide{52}

\subsection{Остаточная нефтенасыщенность}

%\includegraphics[width=\textwidth]{slide2_00054}
\includegraphics[width=\textwidth,page=54]{Waterflooding.pdf}

%\insertslide{slide2_00055}
\insertwaterfloodingslide{55}

%\insertslide{slide2_00056}
\insertwaterfloodingslide{56}

%\insertslide{slide2_00057}
\insertwaterfloodingslide{57}

%\insertslide{slide2_00058}
\insertwaterfloodingslide{58}

%\insertslide{slide2_00059}
\insertwaterfloodingslide{59}

%\insertslide{slide2_00060}
\insertwaterfloodingslide{60}

\end{document}