\documentclass[main.tex]{subfiles}

\begin{document}
\textcolor{red}{Семинар 08.04.2022.}

\includegraphics[width=\textwidth]{slide2_00042}
Эволюция моделей вытеснения:
\begin{itemize}
	\item поршневое вытеснение (для очень хороших коллекторов);
	\item модель Баклея-Леверетта (после прохождения фронта воды за фронтом остаётся нефть; вода и нефть мешают друг другу течь)
\end{itemize}

Модель Баклея-Леверетта формулируется для двух типов флюида (например, нефти и воды)

ОФП -- функция от насыщенности данного типа флюида. Является множителем на абсолютную проницаемость.

\textbf{Важна монотонность} зависимости ОФП от насыщенности данного типа флюида.

\includegraphics[width=\textwidth]{slide2_00043}

Чем выше проницаемость, тем ниже значение остаточной водонасыщенности: в песчанике с крупными порами небольшие капиллярные давления. Так как они небольшие, то и намертво связанной воды тоже немного.



\includegraphics[width=\textwidth]{slide2_00044}

\includegraphics[width=\textwidth]{slide2_00045}

\includegraphics[width=\textwidth]{slide2_00046}
На принципе гистерезиса ОФП построен метод увеличения нефтеотдачи (МУН), называемый циклическим заводнением. В 2008-2010 годах был очень популярен.

\includegraphics[width=\textwidth]{slide2_00047}
В гидрофобном коллекторе проблемы при заводнении: прорывы, не будет вытеснения. Чисто гидрофобных коллекторов мало, но даже коллектора с промежуточной смачиваемостью -- это уже большая проблема.

\includegraphics[width=\textwidth]{slide2_00048}

\includegraphics[width=\textwidth]{slide2_00049}

\end{document}