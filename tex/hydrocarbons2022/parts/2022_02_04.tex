\documentclass[main.tex]{subfiles}

\begin{document}
\section{\textcolor{red}{Семинар 04.02.2022}}

1 часть. Заводнение терригенных и карбонатных коллекторов. Практический курс длительностью 4 интенсива по 8 часов. Режимы работы скважины в пласте. Режимы работы пласта. Формулы для производительности скважин, регулярные и нерегулярные системы разработки. Модели вытеснения. Факторы, влияющие на эффективность вытеснения. Разбор кейсов.

И тест по окончании курса.\\

2 часть. Продолжение методов математической физики.

Анизотропия, работа в неоднородных коллекторах, учёт многофазности течения, псевдо-функции Баклея-Леверетта, вязкость зависит от давления, следовательно, нелинейность, и необходимы симплектические численные схемы для решения нелинейных уравнений.

Лабораторная по численным методам. Преобразования Фурье и Лапласа. Криволинейные системы координат. Задача фильтрации в эллиптическую трещину. Уравнение Навье-Стокса. Задача о нагрузке на физический кабель. Трубы. Узловой анализ (лр). Методы понижения размерности. Метод граничных элементов, трубки тока.

%\includegraphics[width=\textwidth]{slide1_00003}


\subsection{Закон фильтрации}
\includegraphics[width=\textwidth]{slide1_00006}

\insertslide{slide1_00007}

\subsection{Стадии разработки месторождения}
\includegraphics[width=\textwidth]{slide1_00009}

\insertslide{slide1_00010}

\insertslide{slide1_00011}

\subsection{Режимы работы пласта}
\includegraphics[width=\textwidth]{slide1_00012}


\subsection{Режимы работы скважины в пласте}
\includegraphics[width=\textwidth]{slide1_00013}

\insertslide{slide1_00014}

\insertslide{slide1_00015}

\insertslide{slide1_00016}

\insertslide{slide1_00017}

Соотношение воронки депрессии скважины к границам области дренирования.

Упражнение (длительность неустановившегося режима)

\begin{listing}[h]
\begin{minted}[frame=single]{Python}
C_t = 3E-4
phi = 0.2
mu = 3
r_b = 1000
k = 100E-3
kappa = k / (phi * mu * C_t)
h = 10

T = 35 * r_b**2 / kappa
print('Transient duration', T / (60 * 60 * 24), 'days')
\end{minted}
\end{listing}

\subsection{Формула Дюпюи}
\includegraphics[width=\textwidth]{slide1_00020}

\insertslide{slide1_00021}

\insertslide{slide1_00022}

\subsection{Общая формула производительности скважины в пласте сложной формы}

\includegraphics[width=\textwidth]{slide1_00023}

\insertslide{slide1_00024}

\insertslide{slide1_00025}

\insertslide{slide1_00026}

\insertslide{slide1_00027}

\insertslide{slide1_00028}

\insertslide{slide1_00029}

\insertslide{slide1_00030}

\insertslide{slide1_00031}

\insertslide{slide1_00032}

\insertslide{slide1_00038}

\end{document}
