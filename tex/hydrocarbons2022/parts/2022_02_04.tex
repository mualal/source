\documentclass[main.tex]{subfiles}

\begin{document}
\section{\textcolor{red}{Семинар 04.02.2022}}

1 часть. Заводнение терригенных и карбонатных коллекторов (практический курс) -- 4 интенсива (по 8 часов). Формулы для производительности скважин, регулярные и нерегулярные системы разработки.

2 часть. Продолжение методов матфизики. Анизотропия, работа в неоднородных коллекторах, учёт многофазности течения, псевдо-функции Баклея-Леверетта, вязкость зависит от давления, следовательно, нелинейность, и необходимы симплектические численные схемы для решения нелинейных уравнений. Лабораторная по численным методам. Преобразования Фурье и Лапласа. Криволинейные системы координат. Задача фильтрации в эллиптическую трещину. Уравнение Навье-Стокса. Задача о нагрузке на физический кабель. Трубы. Узловой анализ (лр). Методы понижения размерности. Метод граничных элементов, трубки тока.

Тест по курсу по заводнению.

\includegraphics[width=\textwidth]{slide1_00003}


\subsection{Закон фильтрации}
\includegraphics[width=\textwidth]{slide1_00006}

\includegraphics[width=\textwidth]{slide1_00007}

\subsection{Стадии разработки месторождения}
\includegraphics[width=\textwidth]{slide1_00009}

\includegraphics[width=\textwidth]{slide1_00010}

\includegraphics[width=\textwidth]{slide1_00011}

\subsection{Режимы работы пласта}
\includegraphics[width=\textwidth]{slide1_00012}


\subsection{Режимы работы скважины в пласте}
\includegraphics[width=\textwidth]{slide1_00013}

\includegraphics[width=\textwidth]{slide1_00014}

\includegraphics[width=\textwidth]{slide1_00015}

\includegraphics[width=\textwidth]{slide1_00016}

\includegraphics[width=\textwidth]{slide1_00017}

Соотношение воронки депрессии скважины к границам области дренирования.

Упражнение (длительность неустановившегося режима)

\begin{python}
C_t = 3E-4
phi = 0.2
mu = 3
r_b = 1000
k = 100E-3
kappa = k/(phi*mu*C_t)
h = 10

T = 35*r_b**2/kappa
print('Длительность неустановившегося режима', T/(60*60*24), 'суток')
\end{python}

\subsection{Формула Дюпюи}
\includegraphics[width=\textwidth]{slide1_00020}

\includegraphics[width=\textwidth]{slide1_00021}

\includegraphics[width=\textwidth]{slide1_00022}

\subsection{Общая формула производительности скважины в пласте сложной формы}

\includegraphics[width=\textwidth]{slide1_00023}

\includegraphics[width=\textwidth]{slide1_00024}

\includegraphics[width=\textwidth]{slide1_00025}

\includegraphics[width=\textwidth]{slide1_00026}

\includegraphics[width=\textwidth]{slide1_00027}

\includegraphics[width=\textwidth]{slide1_00028}

\includegraphics[width=\textwidth]{slide1_00029}

\includegraphics[width=\textwidth]{slide1_00030}

\includegraphics[width=\textwidth]{slide1_00031}

\includegraphics[width=\textwidth]{slide1_00032}

\includegraphics[width=\textwidth]{slide1_00038}

\end{document}
