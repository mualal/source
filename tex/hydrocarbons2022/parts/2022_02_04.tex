\documentclass[main.tex]{subfiles}

\begin{document}
\textcolor{red}{Семинар 04.02.2022.}

1 часть. Заводнение терригенных и карбонатных коллекторов (практический курс) -- 4 интенсива (по 8 часов). Формулы для производиетльности скважин, регулярные и нерегулярные системы разработки.

2 часть. Продолжение методов матфизики. Анизотропия, работа в неоднородных коллекторах, учёт многофазности течения, псевдо-функции Баклея-Леверетта, вязкость зависит от давления, следовательно, нелинейность, и необходимы симплектические численные схемы для решения нелинейных уравнений. Лабораторная по численным методам. Преобразования Фурье и Лапласа. Криволинейные системы координат. Задача фильтрации в эллиптическую трещину. Уравнение Навье-Стокса. Задача о нагрузке на физический кабель. Трубы. Узловой анализ (лр). Методы понижения размерности. Метод граничных элементов, трубки тока.

Тест по курсу по заводнению.

\includegraphics[width=\textwidth]{slide1_00003}

\includegraphics[width=\textwidth]{slide1_00006}

Режимы работы скважины в пласте

Соотношение воронки депрессии скважины к границам области дренирования.

Упражнение (длительность неустановившегося режима)

\begin{python}
C_t = 3E-4
phi = 0.2
mu = 3
r_b = 1000
k = 100E-3
kappa = k/(phi*mu*C_t)
h = 10

T = 35*r_b**2/kappa
print('Длительность неустановившегося режима', T/(60*60*24), 'суток')
\end{python}


\end{document}
