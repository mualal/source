\documentclass[main.tex]{subfiles}

\begin{document}
\section{\textcolor{red}{Семинар 04.02.2022}}

1 часть. Заводнение терригенных и карбонатных коллекторов. Практический курс длительностью 4 интенсива по 8 часов. Режимы работы скважины в пласте. Режимы работы пласта. Формулы для производительности скважин, регулярные и нерегулярные системы разработки. Модели вытеснения. Факторы, влияющие на эффективность вытеснения. Разбор кейсов.

И тест по окончании курса.\\

2 часть. Продолжение методов математической физики.

Анизотропия, работа в неоднородных коллекторах, учёт многофазности течения, псевдо-функции Баклея-Леверетта, вязкость зависит от давления, следовательно, нелинейность, и необходимы симплектические численные схемы для решения нелинейных уравнений.

Лабораторная по численным методам. Преобразования Фурье и Лапласа. Криволинейные системы координат. Задача фильтрации в эллиптическую трещину. Уравнение Навье-Стокса. Задача о нагрузке на физический кабель. Трубы. Узловой анализ (лр). Методы понижения размерности. Метод граничных элементов, трубки тока.

%\includegraphics[width=\textwidth]{slide1_00003}


\subsection{Закон фильтрации}
\includegraphics[width=\textwidth]{slide1_00006}

\insertslide{slide1_00007}

\subsection{Стадии разработки месторождения}
\includegraphics[width=\textwidth]{slide1_00009}

\insertslide{slide1_00010}

\insertslide{slide1_00011}

\subsection{Режимы работы пласта}
\includegraphics[width=\textwidth]{slide1_00012}

Режимы работы пласта отражают энергию, за счёт которой нефть движется к забою скважины.\\

SAGD = парогравитационный дренаж. В пласт битума (ближе к подошве) забуривают горизонтальную скважину. На расстоянии 5-10 м сверху от неё параллельно бурится паронагнетательная скважина. За счёт неё вырабатывается расширяющаяся паровая камера, и битум стекает по краям этой паровой камеры в добывающую скважину.


\subsection{Режимы работы скважины в пласте}
\includegraphics[width=\textwidth]{slide1_00013}

\insertslide{slide1_00014}

\insertslide{slide1_00015}

\insertslide{slide1_00016}

\insertslide{slide1_00017}

Соотношение воронки депрессии скважины к границам области дренирования.

Упражнение (длительность неустановившегося режима)

\begin{listing}[h]
\begin{minted}[frame=single,breaklines]{Python}
C_t = 3E-4  # 1/атм
phi = 0.2  # доли
mu = 3  # сПз
r_b = 1000  # м
k = 100E-3  # Дарси
kappa = k / (phi * mu * C_t)
h = 10

T = 35 * r_b**2 / kappa
print('Длительность неустановившегося режима ', T / (60 * 60 * 24), 'суток')
\end{minted}
\end{listing}

Скважину запускают не сразу. Как правило, её выводят на режим.

Последовательно. Пробурили. Спустили ГИС. Интерпретировали ГИС. Спустили эксплуатационную колонну. Убедились в герметичности (опрессовка). Проперфорировали. Пригнали флот ГРП. Спустили ГНКТ (гибкие насосно-компрессорные трубы) с пакером. Провели ГРП. Промыли пропант из скважины. Заглушили скважину. Далее бригада КРС (капитального ремонта скважины) спускает погружной ЭЦН, который крепится к ГНКТ. Спустили насос (при этом скважина не пустая, в ней есть жидкость глушения, чтобы не допустить газонефтепроявления на устье). Имеется кабель питания для ЭЦН.

Далее кнопочный пуск (начинает работать ЭЦН). С этого момента начинается вывод скважины на режим: частота насоса увеличивается с низких до рабочих частот, дебит жидкости постепенно растёт, а обводнённость падает (постепенно вытягивается вся жидкость глушения).

Процесс от кнопочного пуска до вывода параметров по скважине (как по жидкости, так и по обводнённости) называют выводом скважины на режим.

Среднее время вывода скважины на режим составляет 7 суток.

\subsection{Формула Дюпюи}
\includegraphics[width=\textwidth]{slide1_00020}

\insertslide{slide1_00021}

\insertslide{slide1_00022}

\subsection{Общая формула производительности скважины в пласте сложной формы}

\includegraphics[width=\textwidth]{slide1_00023}

\insertslide{slide1_00024}

\insertslide{slide1_00025}

Классические форм-факторы получены при отсутствии перетока на границе пласта и постоянном дебите на скважине.

Значения классического форм-фактора практически не зависят от формы пласта, если скважина находится в центре, а размеры пласта по длине и ширине при этом практически равны (пласт не вытянут). Если же пласт вытянут или скважина сильно смещена от центра, то форм-фактор существенно зависит от формы пласта.\\

(*) Статья \href[pdfnewwindow=true]{https://mualal.github.io/source/tex/hydrocarbons2022/articles/dietz1965.pdf}{доступна по ссылке}

\insertslide{slide1_00026}

Если на скважине задано постоянное давление, то форм-факторы отличаются от классических.

\insertslide{slide1_00027}

(*) Статья \href[pdfnewwindow=true]{https://mualal.github.io/source/tex/hydrocarbons2022/articles/helmy1998.pdf}{доступна по ссылке}

\insertslide{slide1_00028}

Пунктирная линия = постоянное давление на границе пласта (ППД), которое организуется расстановкой нагнетательных и добывающих скважин в шахматном порядке.

Сплошная линия = условие неперетока, которое организуется расстановкой только добывающих скважин.\\

(*) Статья \href[pdfnewwindow=true]{https://mualal.github.io/source/tex/hydrocarbons2022/articles/larsen1984.pdf}{доступна по ссылке}

\insertslide{slide1_00029}

(*) Статья \href[pdfnewwindow=true]{https://mualal.github.io/source/tex/hydrocarbons2022/articles/larsen1984.pdf}{доступна по ссылке}

\insertslide{slide1_00030}

\insertslide{slide1_00031}

\begin{listing}[h]
\begin{minted}[frame=single,breaklines]{Python}


\end{minted}
\end{listing}


\insertslide{slide1_00032}

\insertslide{slide1_00038}

\end{document}
