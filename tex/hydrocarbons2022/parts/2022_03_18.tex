\documentclass[main.tex]{subfiles}

\begin{document}
\section{\textcolor{red}{Семинар 18.03.2022}}

\subsection{Метод индикаторных диаграмм для многоскважинных систем}

\includegraphics[width=\textwidth]{slide1_00137}

\insertslide{slide1_00138}

\insertslide{slide1_00139}

\insertslide{slide1_00140}

\insertslide{slide1_00141}

\insertslide{slide1_00142}

\insertslide{slide1_00143}

\insertslide{slide1_00144}

\insertslide{slide1_00145}

\insertslide{slide1_00146}

\insertslide{slide1_00147}

\insertslide{slide1_00148}

\insertslide{slide1_00149}

\insertslide{slide1_00151}

\insertslide{slide1_00152}

\subsection{Статический и мгновенный коэффициенты продуктивности}

\includegraphics[width=\textwidth]{slide1_00153}

Коэффициент продуктивности -- это производная дебита по забойному давлению.

\insertslide{slide1_00154}

\insertslide{slide1_00155}

\insertslide{slide1_00156}

\insertslide{slide1_00157}

\insertslide{slide1_00158}

\insertslide{slide1_00159}

\insertslide{slide1_00160}


\end{document}
