\documentclass[main.tex]{subfiles}

\begin{document}
%\section{\textcolor{red}{Семинар 18.03.2022}}
\section{Семинар 18.03.2022 (Юдин Е.В.)}

\subsection{Метод индикаторных диаграмм для многоскважинных систем}

%\includegraphics[width=\textwidth]{slide1_00137}
\includegraphics[width=\textwidth,page=137]{WellProductivity.pdf}

Представлено напоминание о методе индикаторных диаграмм в односкважинном случае. Далее посмотрим, как изменится метод индикаторных диаграмм, когда будем смотреть многоскважинный случай.

Уже прошли: как отличить изменения дебита за счёт изменения забойного давления или коэффициента продуктивности.

Теперь посмотрим, как изменится метод индикаторных диаграмм, когда будем смотреть многоскважинный случай.

%\insertslide{slide1_00138}
\insertproductivityslide{138}

Есть система, работающая на установившемся режиме.\\

Что будет, если снизим забойное давление на добывающей скважине со 130 атм до 80 атм?\\

А что будет, если повысим забойное давление на добывающей скважине со 130 атм до 180 атм?\\

%\insertslide{slide1_00139}
\insertproductivityslide{139}

Здесь опечатка: на самом деле рассматривается понижение забойного от 130 атм до 80 атм.

С точки зрения математики: изменение граничного условия; на месте добывающей скважины включается ещё одна мнимая скважина с депрессией, равной 50 атм.

Если сложим, то появится дополнительное возмущение в пласте, которое будет распространятся от скважины до границы области дренирования (в процессе неустановившегося режима).

До понижения забойного давления среднее пластовое давление:
\beq
\bar{p}=\frac{130+350}{2}=240\text{ атм}
\eeq

После понижения забойного давления со временем должно установится давление:
\beq
\bar{p}=\frac{80+350}{2}=215\text{ атм}
\eeq

Давление будет снижаться до 215 атм в процессе псевдоустановившегося режима.

Таким образом, поле давлений является самосогласованным. Поменяли давление на скважине, изменилось среднепластовое давление, и это изменение в свою очередь тоже повлияло на скважину.

Другими словами, когда переходим от односкважинной системы к многоскважинной, поле давлений становится самосогласованным.

%\insertslide{slide1_00140}
\insertproductivityslide{140}

На индикаторной диаграмме показаны процессы, протекающие в пласте и на скважине при понижении забойного давления от 130 атм до 80 атм.\\

Чёрную прямую можно назвать стационарной индикаторной диаграммой. На этой диаграмме при нулевом дебите отложено такое давление, которое было бы на добывающей скважине, если бы она не работала. В ряссматриваемом случае ясно, почему 350 атм. Так как забойное давление нагнетательной скважины равно 350 атм.\\

Отступление. Есть отдельная тема: нестационарный узловой анализ. Будем проходить на методах матфизики.

%\insertslide{slide1_00141}
\insertproductivityslide{141}

Здесь опечатка: на самом деле рассматривается повышение забойного от 130 атм до 180 атм.

С точки зрения математики: изменение граничного условия; на месте добывающей скважины включается ещё одна мнимая скважина с репрессией, равной 50 атм.

%\insertslide{slide1_00142}
\insertproductivityslide{142}

Аналогично ранее рассмотренному случаю понижения забойного давления.

%\insertslide{slide1_00143}
\insertproductivityslide{143}

Индикаторные диаграммы изображаются аналогично ранее рассмотренному случаю понижения забойного давления.

%\insertslide{slide1_00144}
\insertproductivityslide{144}

Здесь и далее приведены примеры в случае девятиточки.

%\insertslide{slide1_00145}
\insertproductivityslide{145}

%\insertslide{slide1_00146}
\insertproductivityslide{146}

%\insertslide{slide1_00147}
\insertproductivityslide{147}

%\insertslide{slide1_00148}
\insertproductivityslide{148}

%\insertslide{slide1_00149}
\insertproductivityslide{149}

%\insertslide{slide1_00151}
\insertproductivityslide{151}

Отступление. Задача, подобная задаче Льва Толстого.\\

(*) Статья \href[pdfnewwindow=true]{https://mualal.github.io/source/tex/hydrocarbons2022/articles/khasanov.pdf}{доступна по ссылке}

\subsection{Статическая и мгновенная индикаторные диаграммы}

%\includegraphics[width=\textwidth]{slide1_00152}
\includegraphics[width=\textwidth,page=152]{WellProductivity.pdf}

(*) Статья \href[pdfnewwindow=true]{https://mualal.github.io/source/tex/hydrocarbons2022/articles/khasanov.pdf}{доступна по ссылке}

\subsection{Статический и мгновенный коэффициенты продуктивности}

%\includegraphics[width=\textwidth]{slide1_00153}
\includegraphics[width=\textwidth,page=153]{WellProductivity.pdf}

Мгновенный коэффициент продуктивности -- это частная производная дебита по забойному давлению.\\

Статический коэффициент продуктивности -- это полная производная дебита по забойному давлению.\\

Мы понимаем, что при уменьшении забойного давления добывающей скважины в системе разработки, пластовое давление тоже изменится. При расчёте мгновенного коэффициента продуктивности это изменение пластового давления не учитываем. А при расчёте статического коэффициента продуктивности (полная производная) учитываем как изменение забойного, так и пластового давления.

%\insertslide{slide1_00154}
\insertproductivityslide{154}

Необходимо учитывать и снижение дебитов на всех остальных скважинах, так как пластовое давление упало во всей системе разработки.

%\insertslide{slide1_00155}
\insertproductivityslide{155}

За счёт изменения пластового давления ожидаемый прирост дебита в долгосрочной перспективе падает. Коэффициенты падения от ожидаемого прироста в зависимости от соотношения подвижностей представлены на палетке.\\

(*) Статья \href[pdfnewwindow=true]{https://mualal.github.io/source/tex/hydrocarbons2022/articles/khasanov.pdf}{доступна по ссылке}

%\insertslide{slide1_00156}
\insertproductivityslide{156}

%\insertslide{slide1_00157}
\insertproductivityslide{157}

%\insertslide{slide1_00158}
\insertproductivityslide{158}

Пример из практики. Снижение дебита после $T_{pss}$ за счёт снижения среднего пластового давления (псевдоустановившийся режим).

%\insertslide{slide1_00159}
\insertproductivityslide{159}

%\insertslide{slide1_00160}
\insertproductivityslide{160}

\end{document}
