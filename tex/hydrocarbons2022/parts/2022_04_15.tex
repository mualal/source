\documentclass[main.tex]{subfiles}

\begin{document}
%\section{\textcolor{red}{Семинар 15.04.2022}}
\section{Семинар 15.04.2022 (Юдин Е.В.)}

\subsection{Влияние микропараметров на эффективность вытеснения}

В первой части курса обсуждали, что такое скважины, как оценить их продуктивность в различных условиях (система заканчивания скважины; пласт, в котором она работает; системы разработки).

Как параметры на микроуровне определяют вытеснение в целом (малые залежи РВП)?

\includegraphics[width=\textwidth]{add_slide4}

\insertslide{add_slide5}

\insertslide{add_slide6}

\insertslide{add_slide7}

\insertslide{add_slide8}

\insertslide{add_slide9}

\insertslide{add_slide10}

%\insertslide{slide2_00062}
\insertwaterfloodingslide{62}

На эффективность вытеснения (когда говорим про более сложные коллектора, чем гидрофильные песчаники) влияют 2 фактора:
\begin{itemize}
	\item структура пустотного пространства (соотношение пор, трещин, каверн)
	\item смачиваемость (гидрофильность или гидрофобность)
\end{itemize}
В итоге, 8 возможных моделей с разной смачиваемостью и структурой пустотного пространства.

%\insertslide{slide2_00063}
\insertwaterfloodingslide{63}

Рассматриваем 3 возможных режима разработки: естественный, естественный с последующим заводнением и сразу с заводнением.

%\insertslide{slide2_00064}
\insertwaterfloodingslide{64}

\subsection{Вытеснение на макроуровне. Система уравнений двухфазной фильтрации}

%\includegraphics[width=\textwidth]{slide2_00067}
\includegraphics[width=\textwidth,page=67]{Waterflooding.pdf}

На микроуровне проговорили: поверхностное натяжение, смачиваемость, капиллярные силы, капиллярное число, ОФП.\\

Но необходимо попытаться усреднить и упаковать в систему дифференциальных уравнений в частных производных, т.е. построить математическую модель.

%\insertslide{slide2_00068}
\insertwaterfloodingslide{68}

Используем:
\begin{itemize}
	\item закон сохранения массы
	\item закон Дарси в многофазном случае
	\item замыкающие соотношения (связи основных параметров, например, плотности воды и давления)
	\item ГУ и НУ
\end{itemize}

Физический смысл дивергенции: общий поток через бесконечно малую замкнутую поверхность.

По определению система дифференциальных уравнений локальна, так как это соотношения между бесконечно малыми в точке.

Необходимо найти функции насыщенности и давления такие, чтобы они удовлетворяли всем представленным уравнениям, НУ и ГУ.

%\insertslide{slide2_00069}
\insertwaterfloodingslide{69}

Здесь рассматриваем изотермическую фильтрацию, поэтому не выписываем зависимость параметров от температуры.

НУ: начальное распределение давлений и насыщенностей.

ГУ: условие на внешней границе пласта, как правило, кусочное (например, на части границы постоянное давление, а на другой части условие неперетока); на внутренней границе (например, условие неперетока через непроницаемые разломы или зоны выклинивания/замещения); также к внутренним границам относятся скважины.

Отсутствие перетока равносильно $\displaystyle{}\frac{\partial p}{\partial \vec{n}}=0$ по закону Дарси.

В общем виде решить полученную систему не представляется возможным. Можно решить численно, но тогда не сможем понять основные влияющие на эффективность заводнения факторы.

Но можем упростить постановку: например, подход Баклея-Леверетта.

\subsection{Поршневое вытеснение}

%\includegraphics[width=\textwidth]{slide2_00071}
\includegraphics[width=\textwidth,page=71]{Waterflooding.pdf}

Исторически первой появилась не модель Баклея-Леверетта, а модель поршневого вытеснения.

Модель поршневого вытеснения далеко не совершенна и может быть использована для очень приближённо оценочных расчётов: для верхней оценки времени прихода фронта (фронт достигнет определённого положения не дольше, чем за время поршневого вытеснения).

Чтобы понять эффективность заводнения, выделяется маленький участок: объёмы закачиваемой жидкости должны быть разумными, чтобы время прихода фронта к добывающей скважине не было слишком большим.

%\insertslide{slide2_00072}
\insertwaterfloodingslide{72}

\subsection{Модель Баклея-Леверетта. За фронтом вытеснения остаётся ещё много нефти}

%\includegraphics[width=\textwidth]{slide2_00074}
\includegraphics[width=\textwidth,page=74]{Waterflooding.pdf}

Достаточно быстро заметили, что с приходом фронта воды обводнились не сразу до 100\%, а продолжаем добывать нефть. Это говорит о том, что несмотря на прорыв фронта, за фронтом осталось достаточно много запасов (не только остаточная нефтенасыщенность). И эти запасы постепенно вымываются. Чтобы учесть подобное поведение, была разработана модель Баклея-Леверетта.\\

Рассматривается двухфазная фильтрация (система вода-нефть) в пористой среде. Выписываем обобщённый закон Дарси для каждой из фаз (ставим множитель в виде относительной фазовой проницаемости, зависящей от насыщенности).\\

Функция Баклея-Леверетта представляет собой отношение подвижности воды к общей подвижности по жидкости.\\

Числитель и знаменатель функции Баклея-Леверетта можем умножить на абсолютную проницаемость и градиент давления, тогда получим отношение дебита воды к суммарному дебиту жидкости (т.е. обводнённость).\\

По своей природе функция Баклея-Леверетта представляет из себя зависимость обводнённости в потоке от насыщенности.

%\insertslide{slide2_00075}
\insertwaterfloodingslide{75}

\begin{center}
\includegraphics[width=\textwidth/2]{Figure_1}
\end{center}

Зная обводнённость, можем получить насыщенность из представленного графика. Насыщенность знать полезно, тогда можем оценить, например, остаточные запасы.

Все решения по производительности скважины основаны на решении уравнения пьезопроводности:
\beq
\frac{\varphi\mu C_t}{k}\frac{\partial p}{\partial t}=\frac{\partial^2p}{\partial x^2}+\frac{\partial^2p}{\partial y^2}+\frac{\partial^2p}{\partial z^2}
\eeq

Но это уравнение описывает однофазную фильтрацию. Эффекты многофазной фильтрации учитываются за счёт знания насыщенности. Именно от объёмной доли содержания воды и нефти зависят эффективные свойства (например, эффективная вязкость) рассматриваемой смеси.

После пересчёта всех многофазных артефактов в эффективные величины можем подставить эти величины в решение уравнения пьезопроводности.


%\insertslide{slide2_00076}
\insertwaterfloodingslide{76}

По аналогии с поршневым вытеснением перейдём к модели Баклея-Леверетта. Если жидкости несжимаемые, то в системе уравнений двухфазной фильтрации можем обнулить сжимаемости. При отсутствии капиллярных давлений в системе уравнений двухфазной фильтрации $p_2=p_1$.

Постулат модели Баклея-Леверетта: жидкости движутся в соответствии с фазовыми проницаемостями, другими словами, принимается обобщённый (на многофазный случай) вид закона Дарси.

В законе сохранения массы из под знака частной производной вынесли $\varphi\rho$, т.к. рассматриваем несжимаемые жидкости.

В красной рамке представлено основное уравнение, описывающее модель двухфазной фильтрации по подходу Баклея-Леверетта. В уравнении $f_w$ есть функция Баклея-Леверетта.

Полученное уравнение похоже на уравнение теплопереноса. Решаются уравнения подобного типа с помощью метода характеристик. Говорим, что у задачи есть какая-то встроенная симметрия (гиперповерхность), относительно которой задача выглядит в более упрощённом виде. 

%\insertslide{slide2_00077}
\insertwaterfloodingslide{77}

Можно показать, что скорость движения фронта с данной насыщенностью зависит от производной функции Баклея-Леверетта по насыщенности.

Важно! Разные насыщенности движутся с разной скоростью.

\subsection{Движение фронта вытеснения}

%\includegraphics[width=\textwidth]{slide2_00078}
\includegraphics[width=\textwidth,page=78]{Waterflooding.pdf}

Скорость движения фронта с данной насыщенностью есть производная функции Баклея-Леверетта на скорость закачки.

Случай 1. Самый плохой.

%\insertslide{slide2_00079}
\insertwaterfloodingslide{79}

Скорость движения фронта с данной насыщенностью пропорциональна производной функции Баклея-Леверетта по насыщенности.\\

В рассматриваемом случае фронты с маленькой насыщенностью двигаются быстро, с большой насыщенностью -- медленно.

Фронт со временем размазывается, и мы рано поймаем обводнённость на добывающей скважине, но значение этой обводнённости будет небольшим. Дальше со временем эта обводнённость будет расти.

%\insertslide{slide2_00080}
\insertwaterfloodingslide{80}

Случай 2. Хороший случай. Функция Баклея-Леверетта выгнута вниз.

В рассматриваемом случае фронты с большими насыщенностями движутся быстрее, чем фронты с маленькой насыщенностью.

%\insertslide{slide2_00081}
\insertwaterfloodingslide{81}

В данном случае даже если фронт изначально размазан, то постепенно со временем фронт будет выправляться, и будет происходить поршневое вытеснение.

В данном случае скважина обводнится резко, когда выровненный фронт её достигнет.

%\insertslide{slide2_00082}
\insertwaterfloodingslide{82}

Пример размазанного и поршневого вытеснений.

%\insertslide{slide2_00083}
\insertwaterfloodingslide{83}

Случай 3. S-образная форма функции Баклея-Леверетта.

Проблема: сначала производная растёт, а затем снижается, т.е. фронты с большой и с маленькой насыщенностью двигаются с одинаковыми скоростями.

%\insertslide{slide2_00084}
\insertwaterfloodingslide{84}

В данном случае с приходом фронта к добывающей скважине будет скачок обводнённости (величина скачка может быть как большой, так и маленькой).

%\insertslide{slide2_00085}
\insertwaterfloodingslide{85}

%\insertslide{slide2_00086}
\insertwaterfloodingslide{86}

Обрезаем неопределённый фронт так, чтобы выполнился закон сохранения массы: для этого находим границу, где $A_1=A_2$.

%\insertslide{slide2_00087}
\insertwaterfloodingslide{87}

\end{document}