\documentclass[main.tex]{subfiles}

\begin{document}
\section{\textcolor{red}{Семинар 08.04.2022}}

\subsection{Влияние микропараметров на эффективность вытеснения}

\includegraphics[width=\textwidth]{slide2_00062}

На эффективность вытеснения (когда говорим про более сложные коллектора, чем гидрофильные песчаники) влияют 2 фактора:
\begin{itemize}
	\item структура пустотного пространства (соотношение пор, трещин, каверн)
	\item смачиваемость (гидрофильность или гидрофобность)
\end{itemize}
В итоге, 8 возможных моделей с разной смачиваемостью и структурой пустотного пространства.

\insertslide{slide2_00063}

Рассматриваем 3 возможных режима разработки: естественный, естественный с последующим заводнением и сразу с заводнением.

\insertslide{slide2_00064}

\subsection{Вытеснение на макроуровне. Система уравнений двухфазной фильтрации}

\includegraphics[width=\textwidth]{slide2_00067}

На микроуровне проговорили: поверхностное натяжение, смачиваемость, капиллярные силы, капиллярное число, ОФП.\\

Но необходимо попытаться усреднить и упаковать в систему дифференциальных уравнений в частных производных, т.е. построить математическую модель.

\insertslide{slide2_00068}

Используем:
\begin{itemize}
	\item закон сохранения массы
	\item закон Дарси в многофазном случае
	\item замыкающие соотношения (связи основных параметров, например, плотности воды и давления)
	\item ГУ и НУ
\end{itemize}

Физический смысл дивергенции: общий поток через бесконечно малую замкнутую поверхность.

По определению система дифференциальных уравнений локальна, так как это соотношения между бесконечно малыми в точке.

Необходимо найти функции насыщенности и давления такие, чтобы они удовлетворяли всем представленным уравнениям, НУ и ГУ.

\insertslide{slide2_00069}

Здесь рассматриваем изотермическую фильтрацию, поэтому не выписываем зависимость параметров от температуры.

НУ: начальное распределение давлений и насыщенностей.

ГУ: условие на внешней границе пласта, как правило, кусочное (например, на части границы постоянное давление, а на другой части условие неперетока); на внутренней границе (например, условие неперетока через непроницаемые разломы или зоны выклинивания/замещения); также к внутренним границам относятся скважины.

Отсутствие перетока равносильно $\displaystyle{}\frac{\partial p}{\partial \vec{n}}=0$ по закону Дарси.

В общем виде решить полученную систему не представляется возможным. Можно решить численно, но тогда не сможем понять основные влияющие на эффективность заводнения факторы.

Но можем упростить постановку: например, подход Баклея-Леверетта.

\subsection{Поршневое вытеснение}

\includegraphics[width=\textwidth]{slide2_00071}

Исторически первой появилась не модель Баклея-Леверетта, а модель поршневого вытеснения.

Модель поршневого вытеснения далеко не совершенна и может быть использована для очень приближённо оценочных расчётов: для верхней оценки времени прихода фронта (фронт достигнет определённого положения не дольше, чем за время поршневого вытеснения).

Чтобы понять эффективность заводнения, выделяется маленький участок: объёмы закачиваемой жидкости должны быть разумными, чтобы время прихода фронта к добывающей скважине не было слишком большим.

\insertslide{slide2_00072}

\subsection{Модель Баклея-Леверетта. За фронтом вытеснения остаётся ещё много нефти}

\includegraphics[width=\textwidth]{slide2_00074}

Достаточно быстро заметили, что с приходом фронта воды обводнились не сразу до 100\%, а продолжаем добывать нефть. Это говорит о том, что несмотря на прорыв фронта, за фронтом осталось достаточно много запасов (не только остаточная нефтенасыщенность). И эти запасы постепенно вымываются. Чтобы учесть подобное поведение, была разработана модель Баклея-Леверетта.\\

Рассматривается двухфазная фильтрация (система вода-нефть) в пористой среде. Выписываем обобщённый закон Дарси для каждой из фаз (ставим множитель в виде относительной фазовой проницаемости, зависящей от насыщенности).\\

Функция Баклея-Леверетта представляет собой отношение подвижности воды к общей подвижности по жидкости.\\

Числитель и знаменатель функции Баклея-Леверетта можем умножить на абсолютную проницаемость и градиент давления, тогда получим отношение дебита воды к суммарному дебиту жидкости (т.е. обводнённость).\\

По своей природе функция Баклея-Леверетта представляет из себя зависимость обводнённости в потоке от насыщенности.

\insertslide{slide2_00075}

\begin{center}
\includegraphics[width=\textwidth/2]{Figure_1}
\end{center}

Зная обводнённость, можем получить насыщенность из представленного графика. Насыщенность знать полезно, тогда можем оценить, например, остаточные запасы.

Все решения по производительности скважины основаны на решении уравнения пьезопроводности:
\beq
\frac{k}{\varphi\mu C_t}\frac{\partial p}{\partial t}=\frac{\partial^2p}{\partial x^2}+\frac{\partial^2p}{\partial y^2}+\frac{\partial^2p}{\partial z^2}
\eeq

Но это уравнение описывает однофазную фильтрацию. Эффекты многофазной фильтрации учитываются за счёт знания насыщенности. Именно от объёмной доли содержания воды и нефти зависят эффективные свойства (например, эффективная вязкость) рассматриваемой смеси.

После пересчёта всех многофазных артефактов в эффективные величины можем подставить эти величины в решение уравнения пьезопроводности.


\insertslide{slide2_00076}


\insertslide{slide2_00077}

\subsection{Движение фронта вытеснения}

\includegraphics[width=\textwidth]{slide2_00078}

\insertslide{slide2_00079}

Скорость движения фронта с данной насыщенностью пропорциональна производной функции Баклея-Леверетта по насыщенности.

\insertslide{slide2_00080}

\insertslide{slide2_00081}

\insertslide{slide2_00082}

\insertslide{slide2_00083}

\insertslide{slide2_00084}

\insertslide{slide2_00085}

\insertslide{slide2_00086}

\insertslide{slide2_00087}

\end{document}