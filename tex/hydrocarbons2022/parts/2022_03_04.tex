\documentclass[main.tex]{subfiles}

\begin{document}
\section{\textcolor{red}{Семинар 04.03.2022}}

\includegraphics[width=\textwidth]{slide1_00068}

Была написана программа для анализа работы скважины сложного заканчивания. Представлено поле распределения давления: для трещины равнопритока, трещины бесконечной проводимости и трещины конечной проводимости.

В рассматриваемых условиях распределение давления от трещины конечной проводимости похоже на распределение от вертикальной скважины. Почему?\\

Основной приток к трещине конечной проводимости идёт с кончиков. При низкой безрамерной проводимости трещины её кончики перестают работать, работает только прискважинная зона. Поэтому и распределение давления похоже на распределение от вертикальной скважины.

Можем увеличить проводимость проппанта: пытаться запустить в трещину более крупный проппант. Для этого нужно обеспечивать большую скорость, использовать более дорогие гели (с лучшей несущей способностью). Но не всегда есть такая возможность.

Именно от безразмерной проводимости трещины зависит её продуктивность.

\includegraphics[width=\textwidth]{slide1_00069}

Показаны распределения давления для трещины с различной безразмерной проводимостью. Видим, что в случае большей безразмерной проводимости эффективность трещины выше и распределение давления напоминает форму глаза, а не круга.\\

Именно поэтому трещины наиболее эффективны на низкопроницаемых коллекторах: чем ниже проницаемость пласта, тем больше $C_{fD}$ (безразмерная проводимость трещины) и тем лучше работают кончики трещины, к которым идёт основной приток.

\subsection{Производительность наклонно-направленной скважины}

\includegraphics[width=\textwidth]{slide1_00071}

Вертикальные скважины в России практически сейчас не бурим. Как правило, бурим наклонно-направленные скважины от куста, к которому подведены дороги и около которого организована инфраструктура.

Таким образом, важно знать формулу для производительности наклонно-направленной скважины.

Как рассчитать по формуле в рамке? Сложно!\\

(*) Статья \href{https://mualal.github.io/source/tex/hydrocarbons2022/articles/cinco1975.pdf}{доступна по ссылке}

\insertslide{slide1_00072}

Важно! Полученная формула для скин-фактора
\beq
S_\theta=-\left(\frac{\theta_w'}{41}\right)^{2.06}-\left(\frac{\theta_w'}{56}\right)^{1.865}\cdot\text{lg}\left(\frac{h_D}{100}\right)
\eeq
верна при полном вскрытии пласта скважиной.\\

(*) Статья \href{https://mualal.github.io/source/tex/hydrocarbons2022/articles/cinco1975.pdf}{доступна по ссылке}

\insertslide{slide1_00073}

Besson вывел формулу из более физических соображений, но также как у Синко Ли у него был численный симулятор, основанный на методе источников, и он сравнил как множество полученных численных решений отличается от решения для вертикальной скважины.

Besson получил более физически обоснованную корреляцию. Формулы относительно длины скважины в продуктивном пласте и относительно угла наклона представлены в красных рамках.

(*) Статья \href{https://mualal.github.io/source/tex/hydrocarbons2022/articles/besson1990.pdf}{доступна по ссылке}

\insertslide{slide1_00074}

Видим, что полученные формулы для скин-фактора наклонно-направленной скважины, полностью вскрывающей пласт, очень хорошо совпадают.

Формула Бессона работает условно до 89 градусов.

Формула Синко Ли работает только до 75 градусов.

\subsection{Учёт вертикальной анизотропии}

\includegraphics[width=\textwidth]{slide1_00075}

Когда говорим о наклонно-направленной скважине необходимо учитывать наличие вертикальной составляющей потока. Тогда необходимо учесть и анизотропию пласта.\\

Используется замена координат так, чтобы привести данный анизотропный случай к эффективному изотропному.

При переходе от одной системе координат к другой должны задать, что поток через произвольную площадку в исходной и новой системах координат один и тот же. При выполнении этого условия дебит не изменится при переходе к новой системе координат.

\insertslide{slide1_00076}



\insertslide{slide1_00077}

Другой подход к определению производительности наклонно-направленной скважины.

(*) Статья \href[pdfnewwindow=true]{https://mualal.github.io/source/tex/hydrocarbons2022/articles/ozkan2000.pdf}{доступна по ссылке}

\insertslide{slide1_00078}

(*) Статья \href[pdfnewwindow=true]{https://mualal.github.io/source/tex/hydrocarbons2022/articles/besson1990.pdf}{доступна по ссылке}

\subsection{Производительность горизонтальной скважины}

\includegraphics[width=\textwidth]{slide1_00081}

(*) Статья \href[pdfnewwindow=true]{https://mualal.github.io/source/tex/hydrocarbons2022/articles/joshi1988.pdf}{доступна по ссылке}

\insertslide{slide1_00082}

(*) Статья \href[pdfnewwindow=true]{https://mualal.github.io/source/tex/hydrocarbons2022/articles/joshi1988.pdf}{доступна по ссылке}

\insertslide{slide1_00083}

\insertslide{slide1_00084}

\insertslide{slide1_00085}

\insertslide{slide1_00086}

\insertslide{slide1_00087}

\insertslide{slide1_00088}

\insertslide{slide1_00089}

\subsection{Подходы к планированию продуктивности скважины произвольного заканчивания}

\includegraphics[width=\textwidth]{slide1_00091}

\insertslide{slide1_00092}

\insertslide{slide1_00093}

\insertslide{slide1_00094}

\insertslide{slide1_00095}

\subsection{Производительность скважины в системе разработки}

\includegraphics[width=\textwidth]{slide1_00098}

\insertslide{slide1_00099}

\insertslide{slide1_00100}

\insertslide{slide1_00101}

\insertslide{slide1_00102}

\insertslide{slide1_00103}

\insertslide{slide1_00104}

\insertslide{add_slide1}

\insertslide{add_slide2}

\insertslide{add_slide3}

\insertslide{slide1_00105}

\insertslide{slide1_00106}


\end{document}