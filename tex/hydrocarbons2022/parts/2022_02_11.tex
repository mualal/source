\documentclass[main.tex]{subfiles}

\begin{document}
\section{\textcolor{red}{Семинар 11.02.2022}}

\subsection{Аналитический расчёт форм-факторов}

Далее формулы для форм-факторов в различных ситуациях.

\includegraphics[width=\textwidth]{slide1_00034}

\insertslide{slide1_00035}

\insertslide{slide1_00036}

\insertslide{slide1_00037}

\subsection{Скин-фактор}

\includegraphics[width=\textwidth]{slide1_00040}

Скин-фактор исторически вводился для того, чтобы учесть зону с изменённой проницаемостью в призабойной зоне. Буровой раствор = раствор воды с взвешенными частицами или раствор на углеводородной основе (когда нужно бурить на невысокой депрессии). Эта жидкость с частицами начинает фильтроваться и оседает в призабойной зоне пласта, снижая её пористость и проницаемость. Чтобы учесть, в 40-х годах ввели понятие скин-фактор. Скин-фактор = безразмерный перепад давления на стенке скважины.

\insertslide{slide1_00041}

Скин-фактор можно обобщить и использовать не только в случае загрязнённой призабойной зоны, но и в других случаях. Горизонтальность скважины тоже можно учесть в скин-факторе и использовать формулу Дюпюи.
$\Delta p_{add}$ -- разность забойных давлений в гидравлически идеальном случае и по факту.

\insertslide{slide1_00042}

Задача на скин-фактор в случае зонально неоднородного пласта.

\insertslide{slide1_00043}

Что делать в более сложных случаях? Например, когда скважина частично перфорирована (в этом случае скин-фактор может быть как положительным, так и отрицательным). Палетки Щурова. Принцип электромеханической аналогии: уравнения электростатики похожи на уравнения фильтрации жидкости на установившемся режиме.

Физические эксперименты на насыпной модели. Спрессовываем грунт, песок. Необходимо правильно установить все датчики. Вводим много коэффициентов подобия, чтобы правильно перенести результаты физического эксперимента на реальный кейс.

А в электростатике всё проще и точнее при проведении эксперимента. Силу тока и напряжения можем измерить мультиметром.

Электромеханическая аналогия полезна. Так Щуров сделал и получил палетки для скин-факторов.

\insertslide{slide1_00044}

Метод электромеханической аналогии был изобретён не в 50-х годах. В конце XIX-го века метод архитектора Антонио Гауди, для повышения эффективности использования материалов при проектировании.

\insertslide{slide1_00045}

\insertslide{slide1_00046}

\insertslide{slide1_00047}

\insertslide{slide1_00048}

\insertslide{slide1_00049}

\insertslide{slide1_00050}

\insertslide{slide1_00051}

\insertslide{slide1_00052}

\insertslide{slide1_00053}

\insertslide{slide1_00054}

\insertslide{slide1_00055}


\subsection{Производительность трещины ГРП}

\includegraphics[width=\textwidth]{slide1_00057}

\insertslide{slide1_00058}

\insertslide{slide1_00059}

\insertslide{slide1_00060}

\insertslide{slide1_00061}

\insertslide{slide1_00062}

\insertslide{slide1_00063}

\insertslide{slide1_00064}

\insertslide{slide1_00065}

\insertslide{slide1_00066}

\insertslide{slide1_00067}

\end{document}