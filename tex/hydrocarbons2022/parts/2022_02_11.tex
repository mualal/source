\documentclass[main.tex]{subfiles}

\begin{document}
\section{\textcolor{red}{Семинар 11.02.2022}}

Резюме с прошлого семинара: когда говорим о производительности скважины сложного заканчивания в пласте со сложными границами на псевдоустановившемся режиме, то формула Дюпюи имеет более общий вид:
\beq
q=\dfrac{kh\left(\bar{p}-p_w\right)}{18.4\mu B\left(\dfrac{1}{2}\ln{\dfrac{4A}{e^\gamma C_A r_w^2}}+S\right)}
\eeq

За форму пласта отвечает форм-фактор $C_A$.

За заканчивание скважины отвечает скин-фактор $S$.

Далее рассмотрим собранные воедино формулы для форм-факторов в различных ситуациях и поговорим про скин-фактор.

В итоге, получим инженерные формулы, позволяющие вычислить производительность скважины в разных ситуациях.

\subsection{Аналитический расчёт форм-факторов}

\includegraphics[width=\textwidth]{slide1_00034}

(*) Статья \href[pdfnewwindow=true]{https://mualal.github.io/source/tex/hydrocarbons2022/articles/gringarten1978.pdf}{доступна по ссылке}

\insertslide{slide1_00035}

Приведены формулы для форм-фактора в случае прямоугольной формы пласта с условием неперетока.

\insertslide{slide1_00036}

Приведены формулы для форм-фактора в случае прямоугольной формы пласта с наличием границы, на которой поддерживается постоянное давление.\\

Данные формулы получены суммированием бесконечных рядов (используется метод отражений / мнимых источников).

\insertslide{slide1_00037}

Представлена формула форм-фактора для скважины, находящейся в углу двух пересекающихся непроницаемых разломов.

Ограничение формулы: угол должен целое количество раз укладываться в 360 градусов (для того, чтобы можно было применить метод отражений / мнимых источников).

\subsection{Скин-фактор}

\includegraphics[width=\textwidth]{slide1_00040}

Скин-фактор исторически вводился для того, чтобы учесть зону с изменённой проницаемостью в призабойной зоне. Эта зона возникает при бурении скважины за счёт наличия бурового раствора (= раствор воды со взвешенными частицами или раствор на углеводородной основе, если нужно бурить на невысокой депрессии).

Буровой раствор начинает фильтроваться и оседает в призабойной зоне пласта, снижая её пористость и проницаемость. Чтобы учесть это изменение, в 40-х годах ввели понятие скин-фактор.

Скин-фактор = безразмерный перепад давления на стенке скважины (возникает вследствие загрязнённой призабойной зоны).

Аналогично в призабойной зоне может быть увеличение проницаемости (например, за счёт появления протравленных каналов при обработке призабойной зоны кислотным раствором: для терригенных  -- глинокислота, для карбонатных -- соляная кислота).

Минимальное значение скин-фактора можем оценить из формулы Дюпюи.

\insertslide{slide1_00041}

Скин-фактор универсален. Его можно обобщить и использовать не только в случае загрязнённой призабойной зоны, но и в других случаях. Горизонтальность скважины тоже можно учесть в скин-факторе и использовать формулу Дюпюи.

$\Delta p_{add}$ -- разность забойных давлений в гидравлически идеальном случае и по факту.

\insertslide{slide1_00042}

Задача на скин-фактор в случае зонально неоднородного пласта.

\insertslide{slide1_00043}

Что делать в более сложных случаях? Например, когда скважина частично перфорирована (в этом случае скин-фактор может быть как положительным, так и отрицательным).

Палетки Щурова. Принцип электромеханической аналогии: уравнения электростатики похожи на уравнения фильтрации жидкости на установившемся режиме.

В лоб: физические эксперименты на насыпной модели. Спрессовываем грунт, песок. Необходимо правильно установить все датчики давления и расхода. Вводим много коэффициентов подобия, чтобы правильно перенести результаты физического эксперимента на реальный кейс.

С использованием электромеханической аналогии: в электростатике всё проще и точнее при проведении эксперимента. Силу тока и напряжения можем измерить мультиметром.

Электромеханическая аналогия полезна и упрощает проведение экспериментов. Щуров воспользовался этой аналогией и получил палетки для скин-факторов.

\insertslide{slide1_00050}

Представлена схожесть уравнений Максвелла с уравнениями фильтрации жидкости на установившемся режиме.

\insertslide{slide1_00044}

Метод электромеханической аналогии был изобретён не в 50-х годах. В конце XIX-го века метод аналогий архитектора Антонио Гауди, для повышения эффективности использования материалов при проектировании.\\

Метод топологической оптимизации. Вариационные методы.

\insertslide{slide1_00045}

Чтобы получить выражение для скин-фактора в данном случае, сравниваем полученную формулу для производительности с формулой Дюпюи и вычисляем $S$.

Такие скважины бурят, когда необходимо избежать конуса воды. Или когда бурим газовую скважину, то нужно избежать подтягивание конденсата или подошвенной воды.

Но сейчас все эти трудности лучше решают горизонтальные скважины (площадь соприкосновения больше $\Rightarrow$ выработка лучше).

\insertslide{slide1_00046}

Если рассматриваем плоскорадиальное симметричное течение, то это двумерная задача. И для неё есть решение.

Рекомендуется использовать формулу в красной рамке.

$h$ -- мощность пласта;

$h_w$ -- мощность открытого интервала;

$z_w$ -- расстояние от подошвы до середины интервала перфорации;

$h_d$ -- безразмерная мощность пласта (нормирована на радиус скважины); в случае анизотропного пласта со множителем (корень из отношения горизонтальной и вертикальной проницаемостей).\\

(*) Статья \href[pdfnewwindow=true]{https://mualal.github.io/source/tex/hydrocarbons2022/articles/papatzacos1987.pdf}{доступна по ссылке}

(**) Статья \href[pdfnewwindow=true]{https://mualal.github.io/source/tex/hydrocarbons2022/articles/vrbik1991.pdf}{доступна по ссылке}

\insertslide{slide1_00047}

Результаты моделей Papatzacos и Vrbik практически совпадают. Но у Papatzacos формула выглядит проще.

Представлены графики для скин-фактора в случае расположения открытого интервала посередине пласта. Из графиков видим, что при уменьшении открытого интервала увеличивается скин-фактор.\\

(*) Статья \href[pdfnewwindow=true]{https://mualal.github.io/source/tex/hydrocarbons2022/articles/papatzacos1987.pdf}{доступна по ссылке}

(**) Статья \href[pdfnewwindow=true]{https://mualal.github.io/source/tex/hydrocarbons2022/articles/vrbik1991.pdf}{доступна по ссылке}

\insertslide{slide1_00048}

Помним, что скин-фактор за счёт перфорации может быть как положительным, так и отрицательным (это зависит от параметров перфорации).\\

Кейс из Тимано-Печоры. Бурим наклонно-направленную скважину в карбонатном коллекторе (хорошо консолидированный камень, часто его даже не надо обсаживать). Залили кислоту. Но приток вызвать не получается. Как будто бесконечный скин-фактор. В чём проблема?

Кислота не доходит до забоя (ей мешает вода). Что делали?

Спустили перфоратор (несколько зарядов). Это позволило создать фильтрацию. Задавили кислоту в пласт, и скважина начала работать (100-150 т/сут).

В этом кейсе из Тимано-Печоры перфорация позволила скважину в принципе запустить.\\

Когда говорим про перфорацию, то это существенно трёхмерная фильтрация, т.е. необходимо учесть множество факторов.

Karakas и Tariq взяли результаты численного моделирования и попытались под эти результаты сделать корреляционную модель, которая с приемлемой точностью аппроксимирует результаты численного моделирования. В итоге, разделили скин-фактор за счёт перфорации на 3 составляющие.\\

(*) Статья \href[pdfnewwindow=true]{https://mualal.github.io/source/tex/hydrocarbons2022/articles/karakas1991.pdf}{доступна по ссылке}

\insertslide{slide1_00049}

Представлена таблица для набора корреляционных параметров $\alpha_\varphi$, $a_1$, $a_2$, $b_1$, $b_2$, $c_1$ и $c_2$.

Фазировка $\varphi$ определяет расположение зарядов (а именно угол между перфорационными зарядами).

Плотность перфорации определяет частоту (количество зарядов на единицу длины), с которой производится перфорация.

При одинаковой плотности перфорации скин-фактор существенно меняется при переходе от $\varphi=0^{\circ}$ к $\varphi=180^{\circ}$, при других изменениях фазировки (например от 60 градусов к 90 градусам) скин фактор изменяется незначительно.\\

(*) Статья \href[pdfnewwindow=true]{https://mualal.github.io/source/tex/hydrocarbons2022/articles/papatzacos1987.pdf}{доступна по ссылке}

(**) Статья \href[pdfnewwindow=true]{https://mualal.github.io/source/tex/hydrocarbons2022/articles/vrbik1991.pdf}{доступна по ссылке}

\insertslide{slide1_00051}

Скин-фактор не является аддитивной величиной. Ранее говорили о сумме скин-факторов в обобщённом смысле (есть вклад каждого из слагаемых, но это не арифметическая сумма).

Хотелось бы иметь формулы: как из отдельных известных составляющих получить общий скин-фактор.

Решение упражнения представлено на следующей странице.\\

(*) Статья \href[pdfnewwindow=true]{https://mualal.github.io/source/tex/hydrocarbons2022/articles/yildiz2006.pdf}{доступна по ссылке}

\insertslide{slide1_00052}

\insertslide{slide1_00053}

\insertslide{slide1_00054}

У наклонно-направленной скважины, полностью вскрывающей пласт, скин-фактор отрицательный (т.к. площадь контакта с пластом заметно больше, чем у вертикальной скважины)\\

(*) Статья \href[pdfnewwindow=true]{https://mualal.github.io/source/tex/hydrocarbons2022/articles/yildiz2006.pdf}{доступна по ссылке}

\insertslide{slide1_00055}

(*) Статья \href[pdfnewwindow=true]{https://mualal.github.io/source/tex/hydrocarbons2022/articles/yildiz2006.pdf}{доступна по ссылке}

\subsection{Производительность трещины ГРП}

\includegraphics[width=\textwidth]{slide1_00057}

Когда не помогают кислотные обработки, можем рвать пласт. Создаём избыточное давление на забое, будет образовываться трещина (пласт будет пытаться фильтровать закачиваемую жидкость, но если не получается, то образуется трещина, чтобы увеличить площадь притока). Как только снизим давление, трещина обратно сомкнётся. Необходимо закрепить: закачать расклинивающий агент (раньше был песок, сейчас проппант -- керамические шарики).\\

Проппант закачивается в виде геля (если с помощью воды, то керамические шарики где-то осядут и даже не зайдут в трещину). Через определённое время гель разлагается, а проппант остаётся в трещине. Трещина теперь закреплена.\\

Как смоделировать работу скважины с такой трещиной ГРП?

Исторически есть 3 подхода (модели):

1) модель равнопритока (предполагаем, что к каждой единице длины трещины идёт одна и та же плотность притока); проста в математической реализации, но плохо предсказывает;

2) модель бесконечной проводимости (один разрез с одним и тем же давлением по всей длине трещины, равным забойному давлению на скважине), но тоже плохо предсказывает;

3) модель конечной проводимости (есть распределение как давления, так и притока вдоль трещины); на рисунке небольшая опечатка, синей линией изображена депрессия, а не давление; видим, что к кончикам трещины идёт наибольший приток; если представить трещину в виде набора скважин, то видим, что для кончиков будет относительно невысокая интерференция; форма может изменяться, например, при снижении проницаемости трещины приток с кончиков уменьшается и больший вклад вносит основной приток к скважине (кончики экранируются); достаточно точная модель.

В модели конечной проводимости основными параметрами являются размер трещины (полудлина и средняя раскрытость) и проницаемость трещины.

\insertslide{slide1_00058}

(*) Статья \href[pdfnewwindow=true]{https://mualal.github.io/source/tex/hydrocarbons2022/articles/gringarten1974.pdf}{доступна по ссылке}

\insertslide{slide1_00059}

Инженерные подходы к расчёту продуктивности трещины ГРП.

Используется формула Дюпюи с поправкой на эффективный радиус. Должны найти такой эффективный радиус вертикальной скважины, чтобы при его подстановке в формулу Дюпюи дебит этой скважины совпадал с продуктивностью трещины с полудлиной $x_f$.

Для трещины равнопритока:
\beq
\frac{r_w'}{x_f}=\frac{1}{e}
\eeq

Для трещины бесконечной проводимости:
\beq
\frac{r_w'}{x_f}=\frac{1}{2}
\eeq

(*) Статья \href[pdfnewwindow=true]{https://mualal.github.io/source/tex/hydrocarbons2022/articles/gringarten1974.pdf}{доступна по ссылке}

\insertslide{slide1_00060}

Чуть сложнее, когда хотим определить продуктивность скважины конечной проводимости. Здесь необходимо учитывать геометрические особенности трещины и проницаемость внутри трещины.

\textbf{Важно!} Практически вся Западная Сибирь не потечёт, если не сделать ГРП. Поэтому крайне важно уметь считать продуктивность трещины ГРП.\\

Синко Ли: продуктивность трещины конечной проводимости зависит от безразмерной комбинации (безразмерной проводимости трещины):
\beq
C_{fD}=\frac{k_fw_f}{kx_f},
\eeq
где $k_f$ -- проницаемость трещины, $w_f$ -- средняя раскрытость трещины, $k$ -- проницаемость пласта, $x_f$ -- полудлина трещины.\\

Тоже используется формула Дюпюи, но чтобы найти эффективный радиус скважины, необходимо сделать промежуточные выкладки.

Economides упаковал численные результаты Синко Ли и нашёл удобную аппроксимацию для нахождения эффективного радиуса скважины.\\

Видим, что при увеличении безразмерной проводимости трещины, зависимость асимптотически приближается к значению $0.5$. Так и должно быть, ведь модель трещины конечной проводимости переходит в модель бесконечной проводимости при очень больших значениях безразмерной проницаемости.\\

(*) Статья \href[pdfnewwindow=true]{https://mualal.github.io/source/tex/hydrocarbons2022/articles/gringarten1974.pdf}{доступна по ссылке}

\insertslide{slide1_00061}

Приведены 2 реальных примера трещин ГРП.

Видим, что у трещины есть некоторое распределение раскрытости как по вертикали, так и по горизонтали; ещё есть распределение проводимости (трещина неоднородная), но все параметры, которые есть в модели Economides можно найти в отчёте о ГРП. Как правило, в папке скважин есть 2 таких отчёта: первый отчёт о дизайне, второй отчёт о проведённом ГРП (Post Frac Report).

В отчёте о ГРП можем найти: псевдорадиальный скин-фактор, геометрию трещины (закреплённые высоту, ширину и полудлину), безразмерную проводимость.\\

Но если возьмём скин-фактор, представленный в отчёте, подставим его в формулу Дюпюи, то получим примерно в 2 раза завышенное значение дебита. Это связано с тем, что коллеги, формулирующие Post Frac Report используют паспортные характеристики проппанта, а на самом деле во время освоения делаем много неточностей: есть неразложившийся гель, есть приток частиц/глин из пласта, есть глушение скважины. Всё это существенно снижает проницаемость трещины (если по паспорту от 100 до 300 Дарси, то по факту средняя проводимость трещины примерно 30 Дарси).

Поэтому необходимо делать собственные поправки к отчёту о проведённом ГРП.\\

(*) Книга \href[pdfnewwindow=true]{https://mualal.github.io/source/tex/hydrocarbons2022/articles/economides_fracture_design.pdf}{доступна по ссылке}

\insertslide{slide1_00062}

\insertslide{slide1_00063}

\insertslide{slide1_00064}

\insertslide{slide1_00065}

\insertslide{slide1_00066}

\insertslide{slide1_00067}

\end{document}