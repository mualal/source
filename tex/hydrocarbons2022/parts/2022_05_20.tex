\documentclass[main.tex]{subfiles}

\begin{document}
%\section{\textcolor{red}{Семинар 20.05.2022}}
\section{Семинар 20.05.2022 (Юдин Е.В.)}

\subsection{Анизотропия по проницаемости}
\includegraphics[width=\textwidth]{2022-05-20-anisotropy-1}

\subsubsection{Анизотропия. Производительность наклонно-направленной скважины}
\includegraphics[width=\textwidth]{2022-05-20-anisotropy-2}

\insertslide{2022-05-20-anisotropy-3}

\insertslide{2022-05-20-anisotropy-4}

\insertslide{2022-05-20-anisotropy-5}

\subsubsection{Учёт вертикальной анизотропии: простой подход}
\includegraphics[width=\textwidth]{2022-05-20-anisotropy-6}

\subsubsection{Учёт вертикальной анизотропии: общий случай}
\includegraphics[width=\textwidth]{2022-05-20-anisotropy-7}

\insertslide{2022-05-20-anisotropy-8}

\insertslide{2022-05-20-anisotropy-9}

\subsubsection{Расчёт анизотропии для горизонтальной скважины с наклоном относительно двух осей координат}
\includegraphics[width=\textwidth]{2022-05-20-anisotropy-10}

\insertslide{2022-05-20-anisotropy-11}

\insertslide{2022-05-20-anisotropy-12}

\insertslide{2022-05-20-anisotropy-13}

\subsubsection{Трещина ГРП}
\includegraphics[width=\textwidth]{2022-05-20-anisotropy-14}

\insertslide{2022-05-20-anisotropy-15}

\insertslide{2022-05-20-anisotropy-16}

\subsubsection{Упражнение. Вычислить тензор проницаемости для среды}
\includegraphics[width=\textwidth]{2022-05-20-anisotropy-17}

\insertslide{2022-05-20-anisotropy-18}

\insertslide{2022-05-20-anisotropy-19}

\insertslide{2022-05-20-anisotropy-20}

\insertslide{2022-05-20-anisotropy-21}

\insertslide{2022-05-20-anisotropy-22}

\insertslide{2022-05-20-anisotropy-23}

\insertslide{2022-05-20-anisotropy-24}

\insertslide{2022-05-20-anisotropy-25}

\insertslide{2022-05-20-anisotropy-26}

\insertslide{2022-05-20-anisotropy-27}

\insertslide{2022-05-20-anisotropy-28}


\subsection{Псевдофункции. Фильтрация сжимаемой жидкости}
\includegraphics[width=\textwidth]{2022-05-20-1}

\subsubsection{Расчётный пайплайн}
\includegraphics[width=\textwidth]{2022-05-20-2}

\subsubsection{Линеаризация при неустановившейся фильтрации}
\includegraphics[width=\textwidth]{2022-05-20-3}

\subsubsection{Расчёт продуктивности для сжимаемого флюида}
\includegraphics[width=\textwidth]{2022-05-20-4}

\subsubsection{Вычислить функцию псевдодавления для газа}
\includegraphics[width=\textwidth]{2022-05-20-5}

\subsubsection{Корректировка модели Вогеля}
\includegraphics[width=\textwidth]{2022-05-20-6}

\insertslide{2022-05-20-7}

\insertslide{2022-05-20-8}

\insertslide{2022-05-20-9}

\subsubsection{Сравнение для условий ВУ ОНГКМ}
\includegraphics[width=\textwidth]{2022-05-20-10}


\subsection{Криволинейные координаты}
\includegraphics[width=\textwidth]{2022-05-20-11}

\insertslide{2022-05-20-12}

\insertslide{2022-05-20-13}

\insertslide{2022-05-20-14}

\insertslide{2022-05-20-15}

\subsubsection{Градиент и дивергенция в криволинейных координатах}
\includegraphics[width=\textwidth]{2022-05-20-16}

\subsubsection{Ротор в криволинейных координатах}
\includegraphics[width=\textwidth]{2022-05-20-17}

\subsubsection{Упражнение. Оператор Лапласа в криволинейных координатах}
\includegraphics[width=\textwidth]{2022-05-20-18}

\subsubsection{Вывод выражения для оператора Лапласа в криволинейных координатах через коэффициенты Ляме}
\includegraphics[width=\textwidth]{2022-05-20-19}

\subsubsection{Течение в трещине}
\includegraphics[width=\textwidth]{2022-05-20-20}

\insertslide{2022-05-20-21}

\insertslide{2022-05-20-22}

\insertslide{2022-05-20-23}

\insertslide{2022-05-20-24}

\insertslide{2022-05-20-25}

\insertslide{2022-05-20-26}

\insertslide{2022-05-20-27}

\insertslide{2022-05-20-28}

\insertslide{2022-05-20-29}

\insertslide{2022-05-20-30}

\subsection{Численные методы}
\includegraphics[width=\textwidth]{2022-05-20-31}

\insertslide{2022-05-20-32}

\insertslide{2022-05-20-33}

\insertslide{2022-05-20-34}

\insertslide{2022-05-20-35}

\insertslide{2022-05-20-36}

\insertslide{2022-05-20-37}

\insertslide{2022-05-20-38}

\insertslide{2022-05-20-39}

\insertslide{2022-05-20-40}

\insertslide{2022-05-20-41}

\insertslide{2022-05-20-42}

\insertslide{2022-05-20-43}

\insertslide{2022-05-20-44}

\insertslide{2022-05-20-45}

\insertslide{2022-05-20-46}

\insertslide{2022-05-20-47}

\insertslide{2022-05-20-48}

\insertslide{2022-05-20-49}

\insertslide{2022-05-20-50}

\insertslide{2022-05-20-51}

\insertslide{2022-05-20-52}

\insertslide{2022-05-20-53}

\insertslide{2022-05-20-54}

\insertslide{2022-05-20-55}


\end{document}
