\documentclass[main.tex]{subfiles}

\begin{document}
\section{\textcolor{red}{Семинар 25.03.2022}}

\subsection{Работа скважины в произвольной многоскважинной системе. Матрица взаимной продуктивности}

\includegraphics[width=\textwidth]{slide1_00162}

Как найти продуктивность системы, в которой скважины расположены беспорядочно?

\insertslide{slide1_00163}

Матрица взаимных продуктивностей (MPI) связывает вектор дебитов (элементы этого вектора есть дебиты для каждой из скважин) и вектор депрессий (элементы этого вектора есть депрессии для каждой из скважин).\\

На диагонали MPI стоят классические коэффициенты продуктивности.

\insertslide{slide1_00164}

\insertslide{slide1_00165}

\insertslide{slide1_00166}

\insertslide{slide1_00167}

\insertslide{slide1_00168}

\insertslide{slide1_00169}

\insertslide{slide1_00170}

\insertslide{slide1_00171}

\insertslide{slide1_00172}

\insertslide{slide1_00173}

\insertslide{slide1_00174}

\insertslide{slide1_00175}

\subsection{Основные факторы, влияющие на эффективность заводнения}

\includegraphics[width=\textwidth]{slide2_00004}

\insertslide{slide2_00005}

\insertslide{slide2_00006}


\end{document}