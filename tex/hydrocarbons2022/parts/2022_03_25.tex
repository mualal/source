\documentclass[main.tex]{subfiles}

\begin{document}
%\section{\textcolor{red}{Семинар 25.03.2022}}
\section{Семинар 25.03.2022}

\subsection{Работа скважины в произвольной многоскважинной системе. Матрица взаимной продуктивности}

%\includegraphics[width=\textwidth]{slide1_00162}
\includegraphics[width=\textwidth,page=162]{WellProductivity.pdf}

Как найти продуктивность системы, в которой скважины расположены в произвольном нерегулярном порядке?

%\insertslide{slide1_00163}
\insertproductivityslide{163}

%\insertslide{slide1_00164}
\insertproductivityslide{164}

Представляет из себя расширение понятия индекса (коэффициента) продуктивности.

Матрица взаимных продуктивностей (MPI) связывает вектор дебитов (элементы этого вектора есть дебиты для каждой из скважин) и вектор депрессий (элементы этого вектора есть депрессии для каждой из скважин).\\

На диагонали MPI стоят классические коэффициенты продуктивности.

Внедиагональные элементы отражают взаимовлияние скважин.

Другими словами, диагональные элементы показывают влияние на дебит скважины при изменении депрессии на этой скважине, а внедиагональные -- влияние на дебит скважины при изменении депрессии на соседних скважинах.\\

Если добавим ещё расчёт среднего пластового давления по материальному балансу, то с помощью представленного подхода можем прогнозировать производительность многоскважинной системы на псевдоустановившемся режиме в нерегулярной системе разработки.

В однородном изотропном пласте: чем дальше находятся скважины друг от друга, тем ниже их взаимовлияние.

%\insertslide{slide1_00165}
\insertproductivityslide{165}

Помним, что поле давлений в многоскважинной системе является самосогласованным.

%\insertslide{slide1_00166}
\insertproductivityslide{166}

В общем случае MPI несимметрична, т.к. есть влияние способов заканчивания каждой из скважин. Симметрична, когда примерно одинаковое заканчивание на каждой из скважин.

Когда запускаем скважину в работу, все коэффициенты в матрице взаимных продуктивностей начинают меняться: диагональные индексы продуктивности снижаются, а внедиагональные увеличиваются.

Диагональные элементы положительны, а внедиагональные -- отрицательны.

По модулю диагональные элементы больше, так как логично, что любая скважина сама на себя влияет больше, чем на соседние скважины.

MPI характеризует неоднородность в рассматриваемой системе.

%\insertslide{slide1_00167}
\insertproductivityslide{167}

%\insertslide{slide1_00168}
\insertproductivityslide{168}

%\insertslide{slide1_00169}
\insertproductivityslide{169}

%\insertslide{slide1_00170}
\insertproductivityslide{170}

%\insertslide{slide1_00171}
\insertproductivityslide{171}

\insertslide{2022-03-25_missing1}

%\insertslide{slide1_00172}
\insertproductivityslide{172}

%\insertslide{slide1_00173}
\insertproductivityslide{173}

%\insertslide{slide1_00174}
\insertproductivityslide{174}

%\insertslide{slide1_00175}
\insertproductivityslide{175}

\subsection{Основные факторы, влияющие на эффективность заводнения}

%\includegraphics[width=\textwidth]{slide2_00004}
\includegraphics[width=\textwidth,page=4]{Waterflooding.pdf}

Ранее в курсе не проговаривали вопросы вытеснения, но они являются критическими при планировании эффективности заводнения. Поэтому поговорим о вытеснении.\\

Вспомним, что заводнение -- это мощный инструмент инженера-разработчика, позволяющий во многих случаях кратно увеличить коэффициент извлечения нефти (КИН). \\

Эффективность заводнения зависит от:

1) смачиваемости (в гидрофильном коллекторе всё прекрасно; в гидрофобном пиши пропало: воду не затолкнуть в пласт);

2) внутреннего строения порового пространства: поры (много запасов, но низкая проницаемость), трещины (высокая проницаемость, но мало запасов), каверны (много запасов и высокая проницаемость, но их обычно в пласте очень мало);

3) свойства закачиваемой воды (будет кейс с несовместимостью пластовых вод, которая приводит к выпадению солей);

4) свойства минералов, составляющих породу продуктивных интервалов;

5) проницаемость породы;

6) неоднородность породы

%\insertslide{slide2_00005}
\insertwaterfloodingslide{5}

\beq
KIN=K_{\text{выт}}\cdot K_{\text{охв}}\cdot K_{\text{дрен}}\cdot K_{\text{коммерция}}
\eeq

Коэффициент вытеснения показывает эффективность вытеснения на микроуровне (говорит о том, какое количество запасов нефти останется за фронтом вытеснения). Обычно равен 60-80\%. 

Коэффициент охвата заводнением есть отношение площади, охваченной заводением, к общей площади. Может быть представлен в виде произведения коэффициента охвата по площади и коэффициента охвата по разрезу.

Коэффициент дренирования есть отношение запасов, вовлечённых в разработку, к общему количеству запасов. Например, запасы, находящиеся между двумя непроницаемыми разломами, не вовлечены в разработку. Или если запасы ещё даже не покрыты сеткой скважин, то они тоже не вовлечены в разработку.\\

Ещё есть коэффициент коммерческой обрезки, когда нефть ещё можем добывать, но жидкость будет на 98-99\% обводнена. Не выгодно, так как тратим ресурсы (электроэнергию и т.д.) на закачку, но добываем очень мало нефти. В проектном документе обозначен год экономического предела (ГЭП), когда промысел становится неэффективно эксплуатировать.\\ 

КИН обычно очень низкий: около 15-25\%. Но есть очень редкие исключения.\\

Интересный кейс: месторождение в Канаде (карбонаты), максимально известный КИН (86\%), использовали смешивающее вытеснение: закачку газа и растворителей, т.е. фактически вытеснили всю имеющуюся нефть газом; получилась только газонасыщенность, т.е. подземное хранилище газа (ПХГ).\\

В среднем коэффициент извлечения газовых месторождений (около 70\%) выше, чем нефтяных. Но есть проблемы с конденсатом (ретроградная конденсация).

%\insertslide{slide2_00006}
\insertwaterfloodingslide{6}

\end{document}

