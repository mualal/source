\chapter*{Введение} % * не проставляет номер
\addcontentsline{toc}{chapter}{Введение} % вносим в содержание

На протяжении всего цикла разработки месторождения, начиная с момента создания гидродинамической связи пласта с поверхностью посредством бурения скважин и до момента ликвидации всех скважин, горная порода подвергается непрерывному воздействию за счёт изменения механических условий её существования (изменение давления, температуры, удаление части горной породы при бурении).
В большинстве случаев изменения, которые происходят с горной породой, существенно влияют на разработку месторождения, поэтому изучение влияния механических изменений на поведение горной породы является важной задачей для успешной разработки месторождений нефти и газа.

В настоящее время вопрос изучения поведения горной породы при изменении механического состояния встаёт наиболее остро, так как для экономически выгодной разработки сложных месторождений с трудноизвлекаемыми запасами (ТрИЗ) необходимо применение технологий, значительно изменяющих первоначальное механическое состояние горных пород (например, необходимо применять несколько стадий ГРП, бурить протяжённые горизонтальные скважины).

Наиболее простым и наглядным способом анализа геомеханического состояния горной породы является построение геомеханической модели, которая представляет собой пространственное распределение напряжений и геомеханических свойств горных пород \cite{aliev_book, zobak_book, konoshonkin_book}.
Поскольку пространство является трёхмерным, наиболее адекватной геомеханической моделью является 3D модель, описывающая напряжения и механические свойства горных пород в трёхмерном пространстве.
Однако множество задач геомеханики можно решить, используя более простую 1D модель, представляющую собой распределение напряжений и геомеханических свойств горных пород в зависимости от одной пространственной координаты (глубины).

В данной работе будет построена одномерная геомеханическая модель вдоль ствола скважины на основе данных геофизических исследований (ГИС) и данных по керну, полученных в лаборатории.




