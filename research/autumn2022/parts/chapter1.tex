\chapter{Общие положения} \label{ch1}


К необходимым геомеханическим свойствам горных пород для построения адекватной геомеханической модели относятся модуль Юнга, коэффициент Пуассона, угол внутреннего трения горных пород, а также прочность горных пород на одноосное сжатие.

Для описания напряжённого состояния горных пород нет необходимости определять все шесть компонент тензора напряжений, достаточно определить 4 параметра напряжения (вертикальное напряжение, максимальное горизонтальное напряжение, минимальное горизонтальное напряжение, направление горизонтальных напряжений) и поровое давление.

По данным ГИС (а именно акустического каротажа) определяются динамические модуль Юнга и коэффициент Пуассона.
Для определения вертикального напряжения используется плотностной каротаж, а для определения минимального горизонтального напряжения используется, например, мини-ГРП.

Исследования, направленные на определение значений параметров, необходимых для построения геомеханической модели представлены в таблице \ref{tab:long:invest}.

\small %выставляем шрифт в 12bp
\begin{longtable}[c]{|p{8cm}|p{8cm}|}
	\caption{Исходные данные для построения геомеханической модели}%
	\label{tab:long:invest}% label всегда желательно идти после caption
	\\
	\hline
	\multicolumn{1}{|c|}{\textbf{Параметр}}&\multicolumn{1}{|c|}{\textbf{Исследование}}\\ \hline
	\endfirsthead%
	\captionsetup{format=tablenocaption,labelformat=continued} % до caption!
	\caption[]{}\\ % печать слов о продолжении таблицы
	\hline
	\multicolumn{1}{|c|}{\textbf{Параметр}}&\multicolumn{1}{|c|}{\textbf{Исследование}}\\ \hline
	\endhead
	\hline
	\endfoot
	\hline
	\endlastfoot
	Динамические модули Юнга ($E_{dyn}$) и коэффициенты Пуассона ($\nu_{dyn}$) & Акустический каротаж ($V_p$ и $V_s$)\\ \hline \hline
	Статические модули Юнга ($E_{sta}$) и коэффициенты Пуассона ($\nu_{sta}$) & Пересчёт через динамические модули с помощью корреляций, полученных на основе керновых данных\\ \hline
	Угол внутреннего трения ($\beta$) и прочность горной породы на одноосное сжатие (UCS) & Испытания на керне (например, тест на одноосное сжатие образца)\\ \hline
	Вертикальное напряжение ($S_v$) & Плотностной каротаж\\ \hline
	Минимальное горизонтальное напряжение ($S_h$) & мини-ГРП\\ \hline
	Направление горизонтальных напряжений & Микроимиджер или акустический каротаж ($V_p$ и $V_s$)\\ \hline
	Максимальное горизонтальное напряжение ($S_H$) & Плотностной каротаж, акустический каротаж\\ \hline
	Поровое давление & Делаем допущение о гидростатическом режиме\\ \hline

\end{longtable}
\normalsize% возвращаем шрифт к нормальному

Наиболее наглядным способом представления прочностных свойств горных пород является построение паспортов прочности, отражающих функциональную зависимость возникающих в породе касательных напряжений от нормальных напряжений.
Также паспорта прочности наглядно показывают, при каких напряжениях сохраняется прочное состояние породы.

Обычно построение геомеханических моделей осуществляется с помощью специализированного ПО, например, Schlumberger Techlog.
В данной работе построение 1D геомеханической модели вдоль вертикальной скважины (ось скважины направлена по одному из главных напряжений) будет выполнено с помощью языка программирования Python.
В дальнейшем (уже за рамками данной работы) планируется сделать расчёт для скважины с произвольной инклинометрией (ствол скважины не обязательно направлен вдоль одного из главных напряжений) и сравнить полученные результаты с результатами коммерческого ПО.

Этапы построения геомеханической модели:
\begin{enumerate}[1)]
	\item расчёт вертикального напряжения и пластового давления;
	\item расчёт динамических и статических упругих модулей;
	\item импорт прочностных свойств и построение паспортов прочности;
	\item расчёт горизонтальных напряжений по пороупругой модели;
	\item расчёт устойчивости вдоль главных напряжений и построение итогового планшета градиентов и напряжений.
\end{enumerate}


