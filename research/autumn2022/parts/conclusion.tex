\chapter*{Заключение} \label{ch-conclusion}
\addcontentsline{toc}{chapter}{Заключение}

В данной работе на основе данных ГИС и керновых данных построена одномерная геомеханическая модель вертикальной скважины с помощью языка программирования Python.
Проанализирована устойчивость стенок ствола скважины при разных плотностях бурового раствора.
Сделаны выводы об устойчивости при бурении вдоль максимального и минимального главных горизонтальных напряжений.

В дальнейшем планирую сделать код расчёта устойчивости стенок для скважины с произвольной инклинометрией (ствол скважины не обязательно направлен вдоль одного из главных напряжений) и сравнить полученные результаты с результатами коммерческого ПО Schlumberger Techlog.
