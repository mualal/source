\chapter{Анализ результатов} \label{ch3}

Из построенной одномерной геомеханической модели можем сделать вывод, что в рассматриваемом простейшем случае вертикальной скважины (ось скважины направлена вдоль главного вертикального напряжения) ствол скважины будет устойчивым (градиент обвала существенно ниже градиента пластового давления).

При плотности бурового раствора $1.6\text{ г/см}^3$ начнётся поглощение раствора в породу и при достаточно высоких плотностях бурового раствора (около $2\text{ г/см}^3$) произойдёт гидроразрыв породы.

По полученным планшетам напряжений можем сделать вывод, что наиболее устойчивую скважину удастся пробурить вдоль минимального горизонтального напряжения (так как в этом случае разница между главными напряжениями, действующими перпендикулярно оси скважины, будет минимальна), а устойчивость скважины, пробуренной вдоль максимального горизонтального напряжения, будет ниже (так как в этом случае возрастёт разница между главными напряжениями, действующими перпендикулярно оси скважины).
