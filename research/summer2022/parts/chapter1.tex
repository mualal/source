\chapter{Описание методики проведения модального анализа} \label{ch1}

Линейное дифференциальное уравнение движения рассматриваемого тела:
\begin{equation}
\left[M\right]\left\{\ddot{u}\right\}+\left[K\right]\left\{u\right\}=\left\{0\right\},
\end{equation}
где $\left[M\right]$ - матрица масс, $\left[K\right]$ - матрица жёсткости.

Предположим, что движение происходит по гармоническому закону, тогда
\begin{equation}
\left\{u\right\}=\left\{A\right\}_i\sin{\left(\omega_i t+\varphi_i\right)}
\end{equation}
\begin{equation}
\left\{\ddot{u}\right\}=-\omega_i^2\left\{A\right\}_i\sin{\left(\omega_i t+\varphi_i\right)}
\end{equation}

Подставляя в уравнение движения, получаем задачу на собственные значения:
\begin{equation}
\left(\left[K\right]-\omega_i^2\left[M\right]\right)\left\{A\right\}_i=0
\end{equation}

У реальных физических деталей в силу их сложного строения могут быть тысячи и миллионы собственных частот и форм. Но в большинстве случаев собственные формы на очень высоких собственных частотах могут быть отброшены. И не каждая собственная форма вносит одинаковый вклад в деформацию рассматриваемой структуры под динамической нагрузкой.

Чтобы количественно охарактеризовать вклад каждой собственной формы, вводится параметр вклада (mode participation factor):
\begin{equation}
\gamma_i=\left\{A\right\}_i^{T}\left[M\right]\left\{D\right\},
\end{equation}
где $\left\{D\right\}$ - единичный вектор, в направлении которого хотим найти параметр вклада.

Квадрат параметра вклада есть эффективная масса:
\begin{equation}
M_{\text{eff, i}}=\gamma_i^2
\end{equation}

Параметр вклада и эффективная масса показывают количество массы изделия, которое движется в заданном направлении для каждой из собственных форм. Высокие значения эффективной массы в заданном направлении указывают на то, что рассматриваемая собственная форма будет возбуждаться силами именно в этом направлении.

При проведении модального анализа необходимо исследовать все собственные формы, вносящие существенный вклад в деформацию структуры. Для этого используют отношение между эффективными массами и полной массой рассматриваемого изделия. Если в заданном направлении сумма (по всем собственным формам) отношений эффективных масс к полной массе близка к единице, то наиболее существенные собственные формы уже найдены. В противном случае, необходимо найти дополнительные собственные частоты и формы рассматриваемого изделия.

Далее в данной работе методом конечных элементов с помощью пакета Ansys \cite{fea_in_engin} будут найдены первые 10 собственных частот и форм заданного изделия, а также будут рассчитаны значения эффективных масс для каждой степени свободы (6 степеней свободы: 3 поступательные и 3 вращательные) и каждой собственной формы. После проведённого модального анализа будет сделано замечание о полноте извлечения существенных собственных форм. 

