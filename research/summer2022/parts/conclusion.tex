\chapter*{Заключение} \label{ch-conclusion}
\addcontentsline{toc}{chapter}{Заключение}

В данной работе с помощью Ansys Workbench 2022R1 методом конечных элементов проведён модальный анализ вилки, состоящей из хомута, булавки и U-образной формы. Найдены первые 5 собственных частот и форм рассматриваемого изделия.

Так как подобные изделия часто применяются для закрепления и обеспечения надёжности мест соединения в различных инженерных системах, то рассмотренный в данной работе набросок последовательности действий (построение геометрии изделия, построение конечно-элементной сетки, задание граничного условия и поиск собственных частот) является важным для дальнейшего изучения подобных изделий с целью минимизации рисков разрушения в процессе эксплуатации и увеличения срока службы изделия.



