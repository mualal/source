\chapter*{Введение} % * не проставляет номер
\addcontentsline{toc}{chapter}{Введение} % вносим в содержание

Проектирование любой современной сложной конструкции предполагает проведение большого количества компьютерных экспериментов с целью выбора подходящих материалов компонентов конструкции, сокращения затрат во время производства, увеличения срока службы изделия, а также раннего выявления инженерных ошибок, которые могут привести к разрушению.

Чтобы предотвратить поломку сложной конструкции, для каждого её компонента необходимо провести тщательный анализ физического состояния (напряжённо-деформированного состояния и т.д.), в котором будет находиться рассматриваемая деталь. Другими словами, провести моделирование реакции рассматриваемой детали на заданные внешние воздействия.

В частности, вибрация (механические колебания машин и механизмов) может вызвать неисправность и поломку конструкции, при проектировании детали которой нарушена динамическая целостность и баланс прилагаемых нагрузок. Это может привести к массовым техногенным авариям, таким как обрушение моста.

Модальное представление — один из возможных способов рассмотрения вибрации конструкций. Вибрация и деформации конструкций при механическом возбуждении на собственных частотах характеризуются конкретными формами, которые называются собственными формами колебаний (колебательными модами). В типовых условиях эксплуатации характер вибрации будет сложным, включающим все собственные формы. Но если изучить каждую собственную форму отдельно, то с помощью этих знаний можно анализировать все имеющиеся типы вибрации. Определение собственных частот, коэффициентов демпфирования и форм колебаний конструкции по результатам измерений частотной передаточной функции (ЧПФ) называется модальным анализом.

Для прогнозирования вибрационных характеристик проектируемой конструкции используется динамический модальный анализ методом конечных элементов. При таком анализе всю конструкцию представляют теоретически в виде набора пружин и масс, после чего составляют систему матричных уравнений, описывающих конструкцию. Затем к полученным матрицам применяется математический алгоритм для определения собственных частот и форм колебаний конструкции. С помощью этого метода прогнозируют модальные параметры конструкции до ее изготовления, чтобы заблаговременно выявить возможные проблемы и устранить их на ранних стадиях процесса проектирования.

Для достижения цели были поставлены следующие задачи:
\begin{itemize}
	\item изучить способы построения известных моделей машинного обучения с помощью языка программирования Python \cite{brownlee_alg, sklearn_doc};
	\item оформить визуализацию диаграмм ошибок прогноза и сравнить их с аналогичными диаграммами из работы \cite{muravtsev:metamodel}
\end{itemize}
