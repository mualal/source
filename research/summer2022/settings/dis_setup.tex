%%%%%%%%%%%%%%%%%%%%%%%%%%%%%%%%%%%%%%%%%%%%%%%%%%%%%%
%%%% Файл упрощённых настроек шаблона диссертации %%%%
%%%%%%%%%%%%%%%%%%%%%%%%%%%%%%%%%%%%%%%%%%%%%%%%%%%%%%

%%% Инициализирование переменных, не трогать!  %%%
\newcounter{intvl}
\newcounter{otstup}
\newcounter{contnumeq}
\newcounter{contnumfig}
\newcounter{contnumtab}
\newcounter{pgnum}
\newcounter{chapstyle}
\newcounter{headingdelim}
\newcounter{headingalign}
\newcounter{headingsize}
\newcounter{tabcap}
\newcounter{tablaba}
\newcounter{tabtita}
\newcounter{docType} 		% тип документа
\newcounter{tskPrint} 		% печать Задания на ВКР двух(одно)сторонняя
\newcounter{tskPages}       % для учёта количества страниц в Задании
\newcounter{tskPageFirst}   % для учёта количества страниц в Задании
\newcounter{tskPageLast}    % для учёта количества страниц в Задании 
\newcounter{sumPrint} 		% печать Реферата на ВКР двух(одно)сторонняя
\newcounter{sumPages}       % для учёта количества страниц в Реферате
\newcounter{sumPageFirst}   % для учёта количества страниц в Реферате
\newcounter{sumPageLast}    % для учёта количества страниц в Реферате 
\newcommand{\Single}{0.78}  % пропорция для одинароного отступа в \Spacing
%%%%%%%%%%%%%%%%%%%%%%%%%%%%%%%%%%%%%%%%%%%%%%%%%%

%%% Область упрощённого управления оформлением %%%

% Управление перенесено в главые файлы компиляции ВКР, Задания, Реферата
\setcounter{tskPrint}{0} %по умолчанию односторонняя печать              
%\setcounter{sumPrint}{0} %по умолчанию односторонняя печать 

%% Интервал между заголовками и между заголовком и текстом
% Заголовки отделяют от текста сверху и снизу тремя интервалами (ГОСТ Р 7.0.11-2011, 5.3.5)
\setcounter{intvl}{3}               % Коэффициент кратности к размеру шрифта

% Заголовки отделяют от текста сверху и снизу тремя интервалами 
\newcommand{\intvlS}{1.5}               % Коэффициент кратности к размеру шрифта SPbPU-student-templates

\newcommand{\intervalS}{\vspace{\intvlS\curtextsize}}

% печать списка источников в Задании
\newcommand{\printbibliographyTask}{\vspace{-0.28\curtextsize}
	\printbibliography[env=tsk] % печать списка литературы в исходных данных
	\vspace{-0.28\curtextsize}}


%% Отступы у заголовков в тексте
\setcounter{otstup}{0}              % 0 --- без отступа; 1 --- абзацный отступ

%% Нумерация формул, таблиц и рисунков
\setcounter{contnumeq}{0}           % Нумерация формул: 0 --- пораздельно (во введении подряд, без номера раздела); 1 --- сквозная нумерация по всей диссертации
\setcounter{contnumfig}{0}          % Нумерация рисунков: 0 --- пораздельно (во введении подряд, без номера раздела); 1 --- сквозная нумерация по всей диссертации
\setcounter{contnumtab}{0}          % Нумерация таблиц: 0 --- пораздельно (во введении подряд, без номера раздела); 1 --- сквозная нумерация по всей диссертации


%% Нумерация подстраничных сносок (ссылок)
%сквозная
\counterwithout{footnote}{chapter} %сквозная нумерация подразделов (во всех главах)


%% Нумерация подразделов
%убрать номер главы в секции
%\counterwithout{section}{chapter} %сквозная нумерация подразделов (во всех главах)
%\renewcommand\thesection{\arabic{section}} %в каждой главе нумерация заново

%\renewcommand\thesection{\arabic{section}}
%\renewcommand\thefigure{\fbox{\arabic{figure}}}
%\renewcommand\thetable{\arabic{table}}
%\renewcommand\theequation{\arabic{equation}}



%\counterwithout{section}{chapter}
%\counterwithout{figure}{chapter}
%\counterwithout{table}{chapter}
%\counterwithout{equation}{chapter}

%\counterwithin{section}{chapter}
%\counterwithin{figure}{chapter}
%\counterwithin{table}{chapter}

%% Оглавление

\setcounter{pgnum}{1}               %NB УДАЛЕНО ФИЗИЧЕСКИ 0 --- номера страниц никак не обозначены; 1 --- Стр. над номерами страниц (дважды компилировать после изменения)  
\settocdepth{subsection} %             до какого уровня подразделов выносить в оглавление
\setsecnumdepth{subsubsection}         % до какого уровня нумеровать подразделы
\setcounter{tocdepth}{2}

%% Текст и форматирование заголовков
\setcounter{chapstyle}{0}           % 0 --- разделы только под номером; 1 --- разделы с названием "Глава" перед номером
\setcounter{headingdelim}{2}        % 0 --- номер отделен пропуском в 1em или \quad; 1 --- номера разделов и ений отделены точкой с пробелом, подразделы пропуском без точки; 2 --- номера разделов, подразделов и приложений отделены точкой с пробелом.

%% Выравнивание заголовков в тексте
\setcounter{headingalign}{0}        % 0 --- по центру; 1 --- по левому краю

%% Размеры заголовков в тексте
\setcounter{headingsize}{0}         % 0 --- SPbPU style, все всегда 14 пт; 1 --- пропорционально изменяющийся размер в зависимости от базового шрифта;

%% Подпись таблиц
\setcounter{tabcap}{1}              % 0 --- по ГОСТ, номер таблицы и название разделены тире, выровнены по левому краю, при необходимости на нескольких строках; 1 --- подпись таблицы не по ГОСТ, на двух и более строках, дальнейшие настройки: 
%Выравнивание первой строки, с подписью и номером
\setcounter{tablaba}{2}             % 0 --- по левому краю; 1 --- по центру; 2 --- по правому краю
%Выравнивание строк с самим названием таблицы
\setcounter{tabtita}{1}             % 0 --- по левому краю; 1 --- по центру; 2 --- по правому краю
%Разделитель записи «Таблица #» и названия таблицы
\newcommand{\tablabelsep}{space}   % space = пробел, period =  (определены в подключенных пакетах)

%% Подпись рисунков
%Разделитель записи «Рисунок #» и названия рисунка
\newcommand{\figlabelsep}{period}   % emdash = тире, определён в common/styles; period = точка определён в подключенных пакетах; space
%\newcommand{\figlabelsep}{emdash}   % emdash = тире, определён в common/styles; period = точка определён в подключенных пакетах


%%% Цвета гиперссылок %%%
% Latex color definitions: http://latexcolor.com/

%\definecolor{linkcolor}{rgb}{0.9,0,0}
%\definecolor{citecolor}{rgb}{0,0.6,0}
%\definecolor{urlcolor}{rgb}{0,0,1}


%\definecolor{linkbordercolor}{rgb}{0,0,1}

\definecolor{linkcolor}{HTML}{FF0000} %very light red from the SPbPU brandbook (2nd level)
\definecolor{citecolor}{HTML}{6CF479} %very light green from the SPbPU brandbook (2nd level)
\definecolor{urlcolor}{HTML}{4481BA} %very light blue from the SPbPU brandbook (2nd level)

%\definecolor{linkcolor}{rgb}{0,0,0} %black
%\definecolor{citecolor}{rgb}{0,0,0} %black
%\definecolor{urlcolor}{rgb}{0,0,0} %black