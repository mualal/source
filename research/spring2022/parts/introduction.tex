\chapter*{Введение} % * не проставляет номер
\addcontentsline{toc}{chapter}{Введение} % вносим в содержание

В процессе своей работы исследователи часто используют упрощённые аппроксимации для описания сложных физических процессов. Разработано большое количество способов построения таких аппроксимаций.
Большинство способов основаны на данных, полученных в результате конечного числа экспериментов, проведённых над рассматриваемой моделью.

В случае компьютерных моделей аппроксимации могут быть использованы для увеличения скорости расчёта результата новых экспериментов (с ранее не рассматриваемой комбинацией параметров).

В работе \cite{muravtsev:metamodel} подробно описано построение метамоделей с помощью искусственных нейронных сетей и градиентного бустинга на основе деревьев регрессии в случае фиксированной обучающей выборки.
В рамках работы \cite{muravtsev:metamodel} было рассмотрено 2 датасета, полученные при моделировании притока жидкости к вертикальной скважине в гидродинамическом симуляторе ECLIPSE Blackoil. Первый датасет был получен при рассмотрении сценария разработки с поддержанием пластового давления (ППД), начиная с первого месяца. А при генерации данных второго датасета до существенного падения значений дебитов рассматривался естественный режим и только потом включалась система ППД. Для обоих датасетов варьировались 4 входных параметра геологии пласта: пористость матрицы, проницаемость матрицы, пористость трещин и проницаемость трещин; на выходе отслеживались дебиты нефти в течение пяти лет.

Целью данной работы является построение других моделей машинного обучения на основе тех же входных данных и сравнение полученных результатов с результатами работы \cite{muravtsev:metamodel}.

В качестве моделей машинного обучения в данной работе выбраны широко известные модели регрессии: линейной (linear regression), ближайших соседей (the nearest neighbours regression), на основе метода опорных векторов (support vector machine regression) и на основе чрезвычайно рандомизированных деревьев (extra trees regressor). Также рассмотрены расширенные с помощью базисных функций линейные модели.

Для достижения цели были поставлены следующие задачи:
\begin{itemize}
	\item изучить способы построения известных моделей машинного обучения с помощью языка программирования Python;
	\item оформить визуализацию диаграмм ошибок прогноза и сравнить их с аналогичными диаграммами из работы \cite{muravtsev:metamodel}
\end{itemize}
